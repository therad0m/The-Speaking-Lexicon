\begin{glossarymc}[Cambridge 11]
    \begin{test}{TEST 1}
    \noindent
    \textbf{Part 1. Food and Cooking}
    \begin{qa}{What sorts of food do you like eating most? [Why?]}
    I guess I cannot \textbf{pick out} which food is the most \textbf{delectable} one. In my opinion, as long as my wife cooks something for me, it is always my favorite food as I feel her strong \textbf{affection} for me displayed in the cooking process. In other words, I can \textbf{tuck into} anything she prepares for me until I’m \textbf{full up}.
    \end{qa}

    \begin{qa}{Who normally does the cooking in your home? [Why?/Why not?]}
    My wife often \textbf{rustles up} a meal for me. She is in charge of doing household chores when I am \textbf{the head of the household} and \textbf{bring home the bacon}. I can only \textbf{lend her a helping hand} in the kitchen at weekends.
    \end{qa}

    \begin{qa}{Do you watch cookery programmes on TV? [Why?/Why not?]}
    No, I don’t. During the whole week, I am frequently \textbf{overwhelmed with} the workload and as a result, I rarely watch anything on TV, not to \textbf{mention} cookery programmes. I can only spare a few hours at weekends on watching football matches only.
    \end{qa}

    \begin{qa}{In general, do you prefer eating out or eating at home? [Why?]}
    Whatever my wife cooks at home is always my favorite so it \textbf{takes precedence over} what is available at any \textbf{gourmet} restaurants. She is definitely \textbf{a dab hand at} cooking so a \textbf{home-cooked} meal is always my first choice. I can \textbf{feast on} a \textbf{slap-up} meal only at home while I feel rather reluctant to eat out.
    \end{qa}

        \begin{VocabExplain}[Part 1]
            \begin{ExplainCard}{pick out}[phr.v][B2]
            \EN{to choose or recognize something/someone from a group.}
            \VI{chọn ra, nhận ra.}
            \SY{select; single out; identify}
            \EX{She tried to pick out the ripest apples from the basket.}
            \EX{In data analysis, researchers must pick out significant patterns from the noise.}
            \CO{pick out details; pick out a tune; pick out from}
            \end{ExplainCard}

            \begin{ExplainCard}{delectable}[adj][C1]
            \EN{(of food or drink) extremely pleasant to taste or smell; delightful.}
            \VI{ngon tuyệt, hấp dẫn.}
            \SY{delicious; appetizing; scrumptious}
            \EX{The pie was absolutely delectable.}
            \EX{The chef presented a range of delectable desserts at the banquet.}
            \CO{delectable meal/dish/treat}
            \end{ExplainCard}

            \begin{ExplainCard}{affection}[n][B2]
            \EN{a gentle feeling of fondness, liking, or love.}
            \VI{tình cảm, sự yêu mến.}
            \SY{fondness; attachment; devotion}
            \EX{The puppy showed great affection towards its owner.}
            \EX{Children need affection to develop emotionally and socially.}
            \CO{show affection; deep affection; affection for/towards}
            \end{ExplainCard}

            \begin{ExplainCard}{tuck into}[phr.v][C1]
            \EN{to eat something with enthusiasm and in large amounts.}
            \VI{ăn ngon lành, ăn ngấu nghiến.}
            \SY{devour; gobble; feast on}
            \EX{The kids tucked into the pizza as soon as it arrived.}
            \EX{After the long hike, the group tucked into a hearty stew.}
            \CO{tuck into a meal; tuck into dinner}
            \end{ExplainCard}

            \begin{ExplainCard}{full up}[adj][B1]
            \EN{having eaten so much that you cannot eat any more.}
            \VI{no bụng, không ăn thêm được.}
            \SY{satiated; stuffed; replete}
            \EX{I can’t have another bite, I’m full up.}
            \EX{After the buffet, everyone felt full up and sleepy.}
            \CO{be full up; feel full up}
            \end{ExplainCard}

            \begin{ExplainCard}{rustle up}[phr.v][C1]
            \EN{to quickly prepare a meal with the available ingredients.}
            \VI{chế biến nhanh, làm bữa ăn đơn giản.}
            \SY{whip up; throw together; prepare quickly}
            \EX{She rustled up some pasta for the kids in no time.}
            \EX{The chef rustled up a quick snack for the late-night crew.}
            \CO{rustle up a meal/dinner/snack}
            \end{ExplainCard}

            \begin{ExplainCard}{head of the household}[phrase][B2]
            \EN{the person who manages and provides for the family.}
            \VI{chủ gia đình, người trụ cột.}
            \SY{breadwinner; patriarch; provider}
            \EX{He is the head of the household and takes care of financial matters.}
            \EX{Tax regulations often treat the head of the household differently.}
            \CO{be the head of the household}
            \end{ExplainCard}

            \begin{ExplainCard}{bring home the bacon}[idiom][C1]
            \EN{to earn money to support the family.}
            \VI{kiếm tiền nuôi sống gia đình.}
            \SY{earn a living; provide for; support financially}
            \EX{He works two jobs to bring home the bacon.}
            \EX{Traditionally, men were expected to bring home the bacon, but this has changed.}
            \CO{struggle to bring home the bacon; manage to bring home the bacon}
            \end{ExplainCard}

            \begin{ExplainCard}{lend a helping hand}[idiom][B2]
            \EN{to help someone with a task.}
            \VI{giúp một tay.}
            \SY{assist; support; aid}
            \EX{She lent a helping hand with the cleaning.}
            \EX{Community members often lend a helping hand in times of crisis.}
            \CO{lend a helping hand with/to}
            \end{ExplainCard}

            \begin{ExplainCard}{overwhelmed with}[phrase][B2]
            \EN{to be affected by something (work, feelings, problems) in a way that is too strong to manage.}
            \VI{choáng ngợp, ngập tràn (công việc, cảm xúc).}
            \SY{overloaded; swamped; inundated}
            \EX{I was overwhelmed with work last week.}
            \EX{The charity was overwhelmed with offers of support.}
            \CO{overwhelmed with work/emotion/responsibilities}
            \end{ExplainCard}

            \begin{ExplainCard}{mention}[v][B1]
            \EN{to speak about something quickly, giving little detail.}
            \VI{đề cập đến.}
            \SY{refer to; allude to; bring up}
            \EX{She didn’t mention her plans to anyone.}
            \EX{The report mentions the need for further research.}
            \CO{mention briefly; mention casually; mention in passing}
            \end{ExplainCard}

            \begin{ExplainCard}{take precedence over}[phrase][C1]
            \EN{to be more important or prioritized than something else.}
            \VI{được ưu tiên hơn, quan trọng hơn.}
            \SY{prevail over; outrank; supersede}
            \EX{Safety should take precedence over speed.}
            \EX{In academic publishing, originality often takes precedence over length.}
            \CO{take precedence over something}
            \end{ExplainCard}

            \begin{ExplainCard}{gourmet}[adj][C1]
            \EN{(of food) of very high quality and often expensive; suitable for connoisseurs.}
            \VI{cao cấp, tinh tế.}
            \SY{fine; deluxe; high-quality}
            \EX{They served a gourmet meal at the wedding.}
            \EX{Gourmet coffee has become popular in recent years.}
            \CO{gourmet food/restaurant/meal}
            \end{ExplainCard}

            \begin{ExplainCard}{dab hand at}[phrase][C2]
            \EN{someone who is very good or skilled at a particular activity.}
            \VI{rất giỏi, thành thạo trong việc gì.}
            \SY{expert; adept; skilled}
            \EX{She’s a dab hand at painting.}
            \EX{He proved to be a dab hand at negotiating complex contracts.}
            \CO{a dab hand at cooking/painting/negotiating}
            \end{ExplainCard}

            \begin{ExplainCard}{home-cooked}[adj][B1]
            \EN{(of food) prepared at home, not bought ready-made.}
            \VI{nấu tại nhà.}
            \SY{homemade; traditional; domestic}
            \EX{Nothing beats a home-cooked meal.}
            \EX{Home-cooked dishes are often healthier than fast food.}
            \CO{home-cooked meal/food/dinner}
            \end{ExplainCard}

            \begin{ExplainCard}{feast on}[phr.v][C1]
            \EN{to enjoy eating a lot of a particular food.}
            \VI{ăn thỏa thích.}
            \SY{devour; indulge in; gorge on}
            \EX{We feasted on fresh seafood at the beach.}
            \EX{Scholars feasted on the newly discovered manuscripts.}
            \CO{feast on delicacies; feast on a banquet}
            \end{ExplainCard}

            \begin{ExplainCard}{slap-up}[adj][C2]
            \EN{(informal, of a meal) large and of good quality.}
            \VI{(bữa ăn) thịnh soạn, ngon miệng.}
            \SY{sumptuous; lavish; hearty}
            \EX{They had a slap-up meal to celebrate their anniversary.}
            \EX{The hotel offered a slap-up breakfast buffet for all guests.}
            \CO{slap-up dinner/meal/breakfast}
            \end{ExplainCard}
        \end{VocabExplain}

    \noindent
        \textbf{Part 2.}
    \begin{qa}{Describe a house/apartment that someone you know lives in. You should say:}
    \begin{itemize}
        \item Whose house/apartment this is
        \item Where the house/apartment is
        \item What it looks like inside
        \item and explain what you like or dislike about this person’s house/apartment.
    \end{itemize}

    I would like to talk about a house that I had the chance to visit a couple of days ago. It is also the ideal house I want to own in the near future. With regard to its owner, it was bought by a student of mine who \textbf{aced the IELTS test} a few months ago.  

    It was situated \textbf{in the heart of} Hoang Mai business district, which is in the suburban area so I guess the house owners could buy it \textbf{without breaking the bank}. It is quite not far from my accommodation, so it only took me 20 minutes to reach it. Talking about when I visited this house, my student \textbf{passed IELTS with flying colors} one week ago, 7.5 overall band score particularly, so she wanted to \textbf{throw a small party} to express her gratitude to me for supporting her.  

    On arriving at that house, I realized that I was totally wrong. It must \textbf{have cost the earth} to build such a house like that. What struck me is that the house was extremely spacious with 10 rooms, a garden and a swimming pool. I \textbf{was lost for words} when I first saw it because of its wonderful inside decoration that \textbf{looks like a million dollars}. I was mesmerized by stylish wooden furniture, expensive colorful tiles and startling chandeliers.  

    There are some lavish sofas in the drawing room and extraordinarily designed beds in every living room. Most \textbf{eye-catching} areas are living rooms, kitchen with wooden figure and fine baths. To the best of my knowledge, this house was erected by a legendary architect, who designed Lotte landmark a few years ago. The layout of the house is perfect, so there is nothing I could complain about.
    \end{qa}

        \begin{VocabExplain}[Part 2]
            \begin{ExplainCard}{ace the test}[idiom][C1]
            \EN{to do very well in an exam or test; to get an excellent result.}
            \VI{thi rất tốt, đạt điểm cao.}
            \SY{excel; succeed; pass with distinction}
            \EX{She aced the driving test on her first attempt.}
            \EX{Many candidates struggle, but he aced the physics exam with ease.}
            \CO{ace the exam/test/interview}
            \end{ExplainCard}

            \begin{ExplainCard}{in the heart of}[phrase][B2]
            \EN{in the central or most important part of a place.}
            \VI{ngay trung tâm của.}
            \SY{in the center of; in the middle of}
            \EX{The hotel is located in the heart of the city.}
            \EX{This museum lies in the heart of the cultural district.}
            \CO{in the heart of the city/district/forest}
            \end{ExplainCard}

            \begin{ExplainCard}{without breaking the bank}[idiom][C1]
            \EN{to manage to buy/do something without spending too much money.}
            \VI{không tốn quá nhiều tiền.}
            \SY{affordably; inexpensively; cheaply}
            \EX{You can travel abroad without breaking the bank.}
            \EX{This furniture looks stylish yet it won’t break the bank.}
            \CO{buy/get/afford without breaking the bank}
            \end{ExplainCard}

            \begin{ExplainCard}{pass with flying colors}[idiom][C1]
            \EN{to pass a test or exam with very high marks.}
            \VI{đỗ với kết quả xuất sắc.}
            \SY{excel; triumph; succeed brilliantly}
            \EX{He passed his final exams with flying colors.}
            \EX{The proposal passed the review with flying colors.}
            \CO{pass an exam/test with flying colors}
            \end{ExplainCard}

            \begin{ExplainCard}{throw a party}[phrase][B1]
            \EN{to organize and host a social gathering.}
            \VI{tổ chức tiệc.}
            \SY{host; arrange; hold}
            \EX{They threw a party for her birthday.}
            \EX{The company threw a party to celebrate its 50th anniversary.}
            \CO{throw/host/hold a party}
            \end{ExplainCard}

            \begin{ExplainCard}{cost the earth}[idiom][C2]
            \EN{to be extremely expensive.}
            \VI{cực kỳ đắt đỏ.}
            \SY{cost a fortune; cost an arm and a leg; exorbitant}
            \EX{That watch must have cost the earth.}
            \EX{Maintaining such a large building can cost the earth.}
            \CO{must have cost the earth}
            \end{ExplainCard}

            \begin{ExplainCard}{lost for words}[idiom][C1]
            \EN{unable to say anything because of surprise, shock, or strong emotion.}
            \VI{cạn lời, không nói nên lời.}
            \SY{speechless; dumbstruck; stunned}
            \EX{I was lost for words when I heard the news.}
            \EX{The beauty of the cathedral left the visitors lost for words.}
            \CO{be lost for words; leave someone lost for words}
            \end{ExplainCard}

            \begin{ExplainCard}{look like a million dollars}[idiom][C2]
            \EN{to look extremely attractive, impressive, or expensive.}
            \VI{trông rất sang trọng, lộng lẫy.}
            \SY{stunning; gorgeous; glamorous}
            \EX{She looked like a million dollars in that evening gown.}
            \EX{The renovated hall looks like a million dollars.}
            \CO{look/feel like a million dollars}
            \end{ExplainCard}

            \begin{ExplainCard}{eye-catching}[adj][B2]
            \EN{immediately noticeable because it is attractive or unusual.}
            \VI{bắt mắt, thu hút.}
            \SY{striking; noticeable; attractive}
            \EX{She wore an eye-catching red dress.}
            \EX{The report was printed in an eye-catching layout.}
            \CO{eye-catching design/advertisement/style}
            \end{ExplainCard}
        \end{VocabExplain}

    \noindent
    \textbf{Part 3.}
    \begin{qa}{What kinds of home are most popular in your country? Why is this?}
    Personally, I have never ever thought of buying my own house as I am still on \textbf{a tight budget}, but housing \textbf{preference} varies according to \textbf{disparate} groups of individuals, I assume. For young married couples, an apartment seems to be a reasonable choice as it is \textbf{economical}. A detached house or a villa is also gaining in popularity but it can \textbf{cost a fortune} and may be suitable for those who are \textbf{well-heeled}.
    \end{qa}

    \begin{qa}{What do you think are the advantages of living in a house rather than an apartment?}
    Obviously, those living in their own house could create a sense of ownership, which means they can come up with any architectural design, old-fashioned or \textbf{contemporary} one for example, and make alternations without having to ask for anyone’s permission. However, \textbf{residing in} an apartment means the owner should \textbf{comply with} community rules, and everything such as the playground and parking is \textbf{communal}. An apartment owner is subject to other service charges like \textbf{maintenance} and parking fees, electricity, water and Internet \textbf{bills} exclusive, whereas a house counterpart need not pay such fees.
    \end{qa}

    \begin{qa}{Do you think that everyone would like to live in a larger home? Why is that?}
    I guess living in a small house might not \textbf{disturb} couples without children, and a small apartment can satisfy their fundamental needs. Nonetheless, a \textbf{nuclear family} or even an \textbf{extended family} would find it difficult to cram into a \textbf{humble house} because this type of house is not spacious enough for family members to have their own privacy.
    \end{qa}

    \begin{qa}{How easy is it to find a place to live in your country?}
    It depends on the location that one decides to move to. In my country, the property market is quite \textbf{problematic} due to the \textbf{exorbitant} price of houses, especially in downtown area. This means looking for a place to live is \textbf{laborious} and time-consuming. Recently, however, there are governmental policies that provide \textbf{social housing} for residents, especially the poor. The process to apply for this type of accommodation is transparent and \textbf{as easy as pie}.
    \end{qa}

    \begin{qa}{Do you think it’s better to rent or to buy a place to live in? Why?}
    Well, choosing where to live is basically \textbf{predicated on} people’s socioeconomic conditions. Of course, rented accommodations allow people to pay monthly and this will suit the working class. The downside is that despite being bound by contract, the rent conditions can be altered by \textbf{landlords}, which is pretty \textbf{precarious}. On the other hand, not many people are \textbf{well-off} enough to invest in a \textbf{residence}, but it is always worthwhile to get a permanent \textbf{shelter}.
    \end{qa}

    \begin{qa}{Do you agree that there is a right age for young adults to stop living with their parents? Why is that?}
    That’s an interesting question. I do not think there’s a right age as it depends on the contexts. In Western countries, young adults seem to live independently once they have graduated or got \textbf{hitched}, I suppose. But an opposite \textbf{inclination} could be seen in Asian countries as children still \textbf{live off} their parents even once they reach maturity. This stems from the social \textbf{stereotype} which dictates the firstborn child should stay with their parents to support them when the offspring is \textbf{no spring chicken}.
    \end{qa}

        \begin{VocabExplain}[Part 3]
            \begin{ExplainCard}{a tight budget}[phrase][B2]
            \EN{having very little money available to spend.}
            \VI{ngân sách eo hẹp.}
            \SY{financially constrained; short of money}
            \EX{As a student, I’m always on a tight budget.}
            \EX{Many research teams operate on a tight budget.}
            \CO{be on a tight budget; manage a tight budget}
            \end{ExplainCard}

            \begin{ExplainCard}{preference}[n][B2]
            \EN{a greater interest in one thing over another.}
            \VI{sự ưa thích, sự lựa chọn ưu tiên.}
            \SY{liking; choice; inclination}
            \EX{I have a preference for tea over coffee.}
            \EX{Cultural preferences often shape consumer demand.}
            \CO{have a preference for; show a preference}
            \end{ExplainCard}

            \begin{ExplainCard}{disparate}[adj][C1]
            \EN{completely different in kind; not allowing comparison.}
            \VI{khác biệt hoàn toàn.}
            \SY{diverse; contrasting; distinct}
            \EX{They hold disparate opinions on education.}
            \EX{The report combined data from disparate sources.}
            \CO{disparate groups; disparate elements}
            \end{ExplainCard}

            \begin{ExplainCard}{economical}[adj][B2]
            \EN{using money or resources carefully and without waste.}
            \VI{tiết kiệm, kinh tế.}
            \SY{cost-effective; efficient; thrifty}
            \EX{Cooking at home is usually more economical.}
            \EX{An economical solution can reduce project costs.}
            \CO{economical choice; economical use}
            \end{ExplainCard}

            \begin{ExplainCard}{cost a fortune}[idiom][B2]
            \EN{to be very expensive.}
            \VI{tốn kém rất nhiều tiền.}
            \SY{cost an arm and a leg; be pricey}
            \EX{That car must have cost a fortune.}
            \EX{Publishing in color can cost a fortune in academic journals.}
            \CO{cost a fortune to buy/run}
            \end{ExplainCard}

            \begin{ExplainCard}{well-heeled}[adj][C2]
            \EN{wealthy and affluent.}
            \VI{giàu có, sang trọng.}
            \SY{prosperous; rich; affluent}
            \EX{She came from a well-heeled family.}
            \EX{Well-heeled investors dominate the housing market.}
            \CO{well-heeled customers; well-heeled tourists}
            \end{ExplainCard}

            \begin{ExplainCard}{contemporary}[adj][B2]
            \EN{belonging to the present time; modern.}
            \VI{đương đại, hiện đại.}
            \SY{modern; present-day}
            \EX{He prefers contemporary music.}
            \EX{Contemporary issues require innovative solutions.}
            \CO{contemporary art; contemporary style}
            \end{ExplainCard}

            \begin{ExplainCard}{reside in}[phr.v][C1]
            \EN{to live in a place.}
            \VI{cư trú, sinh sống.}
            \SY{dwell in; inhabit; live in}
            \EX{She resides in Paris with her family.}
            \EX{Authority often resides in the central government.}
            \CO{reside in the city/country}
            \end{ExplainCard}

            \begin{ExplainCard}{comply with}[phr.v][B2]
            \EN{to obey a rule, order, or request.}
            \VI{tuân thủ, chấp hành.}
            \SY{adhere to; obey; follow}
            \EX{Employees must comply with company rules.}
            \EX{Firms are required to comply with safety standards.}
            \CO{comply with regulations/laws}
            \end{ExplainCard}

            \begin{ExplainCard}{communal}[adj][C1]
            \EN{shared by all members of a community.}
            \VI{chung, tập thể.}
            \SY{shared; collective; public}
            \EX{They share a communal kitchen in the dormitory.}
            \EX{Communal land ownership exists in some regions.}
            \CO{communal space; communal facilities}
            \end{ExplainCard}

            \begin{ExplainCard}{maintenance}[n][B2]
            \EN{the process of preserving something in good condition.}
            \VI{bảo trì, bảo dưỡng.}
            \SY{upkeep; servicing; repair}
            \EX{Car maintenance can be expensive.}
            \EX{Building maintenance is required annually.}
            \CO{maintenance cost; regular maintenance}
            \end{ExplainCard}

            \begin{ExplainCard}{disturb}[v][B2]
            \EN{to interrupt or bother.}
            \VI{làm phiền, quấy rầy.}
            \SY{bother; disrupt; upset}
            \EX{Don’t disturb him while he’s sleeping.}
            \EX{Noise can disturb the ecosystem balance.}
            \CO{disturb peace; disturb someone’s sleep}
            \end{ExplainCard}

            \begin{ExplainCard}{nuclear family}[n][B2]
            \EN{a family consisting of two parents and their children.}
            \VI{gia đình hạt nhân.}
            \SY{small family unit; immediate family}
            \EX{They live as a nuclear family in the city.}
            \EX{Sociologists often compare nuclear families with extended families.}
            \CO{nuclear family structure; live in a nuclear family}
            \end{ExplainCard}

            \begin{ExplainCard}{extended family}[n][B2]
            \EN{a family including grandparents, aunts, uncles, and cousins.}
            \VI{gia đình mở rộng.}
            \SY{relatives; kin; larger family unit}
            \EX{She grew up in an extended family household.}
            \EX{Extended family support is crucial in some cultures.}
            \CO{extended family ties; extended family members}
            \end{ExplainCard}

            \begin{ExplainCard}{humble house}[phrase][C1]
            \EN{a modest or simple dwelling.}
            \VI{ngôi nhà khiêm tốn.}
            \SY{modest home; simple house}
            \EX{He was raised in a humble house in the countryside.}
            \EX{Despite living in a humble house, they were content.}
            \CO{live in a humble house}
            \end{ExplainCard}

            \begin{ExplainCard}{problematic}[adj][C1]
            \EN{difficult to solve or deal with.}
            \VI{khó khăn, rắc rối.}
            \SY{troublesome; challenging}
            \EX{His absence is problematic for the team.}
            \EX{Climate change is a problematic issue for governments.}
            \CO{problematic situation; problematic area}
            \end{ExplainCard}

            \begin{ExplainCard}{exorbitant}[adj][C2]
            \EN{unreasonably high (price or cost).}
            \VI{giá cả quá cao, cắt cổ.}
            \SY{excessive; steep; unreasonable}
            \EX{They paid an exorbitant fee for the tickets.}
            \EX{The exorbitant cost of housing excludes many buyers.}
            \CO{exorbitant prices/fees}
            \end{ExplainCard}

            \begin{ExplainCard}{laborious}[adj][C1]
            \EN{requiring much work and effort.}
            \VI{nặng nhọc, khó khăn.}
            \SY{arduous; demanding; toilsome}
            \EX{Filling out all the forms was laborious.}
            \EX{Compiling this dataset was a laborious process.}
            \CO{laborious task; laborious journey}
            \end{ExplainCard}

            \begin{ExplainCard}{social housing}[n][B2]
            \EN{housing provided by the government at low cost.}
            \VI{nhà ở xã hội.}
            \SY{public housing; subsidized housing}
            \EX{The city council invested in social housing.}
            \EX{Social housing policies aim to help low-income families.}
            \CO{provide social housing; apply for social housing}
            \end{ExplainCard}

            \begin{ExplainCard}{as easy as pie}[idiom][B2]
            \EN{extremely easy.}
            \VI{dễ như ăn bánh.}
            \SY{simple; effortless}
            \EX{The exam was as easy as pie.}
            \EX{For trained experts, diagnosing the problem was as easy as pie.}
            \CO{be as easy as pie}
            \end{ExplainCard}

            \begin{ExplainCard}{predicated on}[phrase][C2]
            \EN{based on or dependent on something.}
            \VI{dựa trên, phụ thuộc vào.}
            \SY{contingent on; founded on}
            \EX{Their success is predicated on teamwork.}
            \EX{This theory is predicated on several assumptions.}
            \CO{predicated on the fact that}
            \end{ExplainCard}

            \begin{ExplainCard}{landlord}[n][B1]
            \EN{a person who rents out land, a building, or accommodation.}
            \VI{chủ nhà, chủ đất cho thuê.}
            \SY{owner; lessor}
            \EX{Our landlord fixed the heating last week.}
            \EX{Landlords must follow legal housing standards.}
            \CO{private landlord; landlord obligations}
            \end{ExplainCard}

            \begin{ExplainCard}{precarious}[adj][C2]
            \EN{uncertain and likely to change for the worse.}
            \VI{bấp bênh, không ổn định.}
            \SY{unstable; risky; insecure}
            \EX{Their financial situation is precarious.}
            \EX{Precarious jobs affect economic stability.}
            \CO{precarious situation; precarious balance}
            \end{ExplainCard}

            \begin{ExplainCard}{well-off}[adj][B2]
            \EN{wealthy; having enough money to live comfortably.}
            \VI{khá giả, giàu có.}
            \SY{affluent; prosperous; rich}
            \EX{She comes from a well-off family.}
            \EX{Well-off households dominate private education.}
            \CO{well-off family; well-off background}
            \end{ExplainCard}

            \begin{ExplainCard}{residence}[n][C1]
            \EN{a place where someone lives.}
            \VI{nơi cư trú, nhà ở.}
            \SY{dwelling; home; abode}
            \EX{This is his official residence.}
            \EX{Residences in the city are densely packed.}
            \CO{permanent residence; private residence}
            \end{ExplainCard}

            \begin{ExplainCard}{shelter}[n][B1]
            \EN{a place that gives protection.}
            \VI{nơi trú ẩn, chỗ nương náu.}
            \SY{refuge; haven; asylum}
            \EX{We found shelter from the storm under a tree.}
            \EX{Shelter is one of the basic human needs.}
            \CO{seek shelter; provide shelter}
            \end{ExplainCard}

            \begin{ExplainCard}{hitched}[adj][C1][informal]
            \EN{(informal) married.}
            \VI{kết hôn.}
            \SY{married; wedded; tied the knot}
            \EX{They got hitched last summer.}
            \EX{Many couples choose to get hitched abroad.}
            \CO{get hitched; be newly hitched}
            \end{ExplainCard}

            \begin{ExplainCard}{inclination}[n][C1]
            \EN{a tendency or preference for something.}
            \VI{xu hướng, khuynh hướng.}
            \SY{tendency; disposition; propensity}
            \EX{He has an inclination to overspend.}
            \EX{There is an inclination in politics towards reform.}
            \CO{inclination to do sth; inclination towards sth}
            \end{ExplainCard}

            \begin{ExplainCard}{live off}[phr.v][B2]
            \EN{to depend on someone/something for support.}
            \VI{sống nhờ, phụ thuộc vào.}
            \SY{depend on; rely on}
            \EX{He still lives off his parents.}
            \EX{Some tribes live off the land entirely.}
            \CO{live off parents; live off benefits}
            \end{ExplainCard}

            \begin{ExplainCard}{stereotype}[n][C1]
            \EN{a fixed, oversimplified idea about people or groups.}
            \VI{định kiến, khuôn mẫu.}
            \SY{generalization; cliché; label}
            \EX{The stereotype that women are bad drivers is unfair.}
            \EX{Cultural stereotypes influence hiring decisions.}
            \CO{racial stereotype; gender stereotype}
            \end{ExplainCard}

            \begin{ExplainCard}{no spring chicken}[idiom][C2]
            \EN{(informal) not young anymore.}
            \VI{không còn trẻ trung nữa.}
            \SY{aged; elderly; getting old}
            \EX{He’s no spring chicken, but he still runs marathons.}
            \EX{The professor may be no spring chicken, yet he remains active in research.}
            \CO{be no spring chicken}
            \end{ExplainCard}
        \end{VocabExplain}

    \begin{VocabHighlights}
        \VH{to pick out}{(phr.v) to choose somebody/something carefully from a group of people or things}{(cụm động từ) chọn lựa, lọc ra}
        \VH{delectable}{(adj) extremely pleasant to taste, smell or look at}{(tính từ) cực ngon}
        \VH{affection}{(n) the feeling of liking or loving somebody/something very much and caring about them}{(danh từ) tình cảm}
        \VH{to tuck into}{(phr.v) to eat vigorously}{(cụm động từ) ăn liên hồi}
        \VH{to be full up}{(idiom) unable to accommodate any more}{(thành ngữ) không nuốt nổi nữa}
        \VH{to rustle up}{(phr.v) to prepare (food, a meal, etc.) quickly}{(cụm động từ) nấu nhanh}
        \VH{the head of the household}{(idiom) the master of the house, usually the father}{(thành ngữ) chủ nhà}
        \VH{to bring home the bacon}{(idiom) supply material provision or support; earn a living}{(thành ngữ) mang lại thu nhập cho gia đình}
        \VH{to lend somebody a helping hand}{(idiom) to help somebody}{(thành ngữ) giúp ai đó}
        \VH{to be overwhelmed with}{(adj) be busy with}{(tính từ) bận rộn làm gì}
        \VH{to take precedence over}{(phrase) the condition of being more important than somebody else and therefore coming or being dealt with first}{(cụm từ) ưu tiên hơn}
        \VH{gourmet}{(adj) of high quality and often expensive; connected with food or wine of this type}{(tính từ) thượng hạng, đắt đỏ}
        \VH{to be a dab hand at}{(idiom) a person who is very good at doing something or using something}{(thành ngữ) 1 người rất khéo làm gì}
        \VH{home-cooked}{(adj) made and eaten at home}{(tính từ) làm/ăn ở nhà}
        \VH{to feast on}{(phr.v) to eat large quantities of something, usually with pleasure}{(cụm động từ) ăn nhiều}
        \VH{slap-up}{(adj) lavish; excellent; first-class}{(tính từ) xuất sắc, hàng đầu, đầy đặn}
        \VH{to ace the test}{(idiom) to do very well in a test}{(thành ngữ) giành điểm cao kỳ thi}
        \VH{in the heart of}{(phrase) in the center of}{(cụm từ) trung tâm}
        \VH{to break the bank}{(idiom) to cost too much}{(thành ngữ) đắt tiền}
        \VH{pass with flying colors}{(idiom) pass with very high scores}{(thành ngữ) thi đỗ điểm số cao}
        \VH{to throw a small party}{(idiom) to have a party}{(thành ngữ) tổ chức một bữa tiệc}
        \VH{to cost the earth}{(idiom) cost too much}{(thành ngữ) tốn kém, đắt đỏ}
        \VH{to be lost for words}{(idiom) to be so shocked, surprised, full of admiration, etc. that you cannot speak}{(thành ngữ) quá bất ngờ, không nói nên lời}
        \VH{to look like a million dollars}{(idiom) to look exceptionally attractive or in very robust health}{(thành ngữ) trông cực kì hấp dẫn}
        \VH{extraordinarily}{(adj) very unusual, special, unexpected, or strange}{(tính từ) phi thường}
        \VH{lavish}{(adj) spending, giving, or using more than is necessary or reasonable; more than enough}{(tính từ) xa hoa}
        \VH{eye-catching}{(adj) very attractive or noticeable}{(tính từ) bắt mắt}
        \VH{on a tight budget}{(phrase) involving a relatively small amount of money for planned spending}{(cụm từ) chi tiêu tiết kiệm; ngân sách có hạn}
        \VH{preference}{(n) the fact that you like something or someone more than another thing or person}{(danh từ) sở thích}
        \VH{disparate}{(adj) made up of parts or people that are very different from each other}{(tính từ) khác biệt}
        \VH{economical}{(adj) providing good service or value in relation to the amount of time or money spent}{(tính từ) tiết kiệm}
        \VH{to cost a fortune}{(phrase) to cost a lot of money}{(cụm từ) rất tốn tiền}
        \VH{contemporary}{(adj) belonging to the same time}{(tính từ) đương đại}
        \VH{to reside}{(v) to live in a particular place}{(động từ) sống, cư trú}
        \VH{to comply with}{(v) to obey a rule, an order, etc.}{(động từ) tuân theo}
        \VH{communal}{(adj) shared by, or for the use of, a number of people, especially people who live together}{(tính từ) thuộc về của chung}
        \VH{to disturb}{(v) to interrupt somebody when they are trying to work, sleep, etc}{(động từ) làm phiền}
        \VH{a nuclear family}{(phrase) a couple and their dependent children, regarded as a basic social unit}{(cụm từ) gia đình hạt nhân, gia đình nhỏ}
        \VH{an extended family}{(phrase) a family that extends beyond the nuclear family, including grandparents, aunts, uncles, and other relatives, who all live nearby or in one household}{(cụm từ) gia đình lớn gồm nhiều thành viên}
        \VH{to cram}{(v) to push or force somebody/something into a small space; to move into a small space with the result that it is full}{(động từ) chen chúc}
        \VH{humble}{(adj) showing you do not think that you are as important as other people}{(tính từ) nhỏ bé, khiêm tốn}
        \VH{problematic}{(adj) difficult to deal with or to understand; full of problems; not certain to be successful}{(tính từ) có vấn đề}
        \VH{exorbitant}{(adj) (of a price) much too high}{(tính từ) giá cao}
        \VH{laborious}{(adj) taking a lot of time and effort}{(tính từ) siêng năng, chăm chỉ}
        \VH{social housing}{(phrase) housing which is provided for rent or sale at a fairly low cost by housing associations and local councils}{(cụm từ) nhà ở xã hội}
        \VH{as easy as pie}{(idiom) very easy}{(thành ngữ) rất dễ dàng}
        \VH{landlord}{(n) a person or company from whom you rent a room, a house, an office, etc}{(danh từ) chủ nhà; chủ đất}
        \VH{precarious}{(adj) not safe or certain; dangerous}{(tính từ) bấp bênh; nguy hiểm}
        \VH{residence}{(n) a house, especially a large or impressive one}{(danh từ) chỗ ở}
        \VH{shelter}{(n) the fact of having a place to live or stay, considered as a basic human need}{(danh từ) nơi ở, nơi ẩn náu}
        \VH{to get hitched}{(idiom) to marry}{(thành ngữ) kết hôn}
        \VH{inclination}{(n) a feeling that makes you want to do something}{(danh từ) khuynh hướng, thiên hướng}
        \VH{to live off}{(phr.v) to depend on}{(cụm động từ) sống dựa vào, ăn bám}
        \VH{stereotype}{(n) a set idea that people have about what someone or something is like, especially an idea that is wrong}{(danh từ) khuôn mẫu}
        \VH{no spring chicken}{(idiom) somebody who is no longer young}{(thành ngữ) ai đó không còn trẻ nữa}
    \end{VocabHighlights}
    \end{test}

    \begin{test}{TEST 2}
    \noindent
    \textbf{Part 1. Friends}
    \begin{qa}{How often do you go out with friends? [Why?/Why not?]}
    As I’m an introvert, I’m not accustomed to \textbf{hanging out with} friends. Perhaps once a month, I guess, because I’m fully \textbf{occupied} with taking care of my family. Prioritizing his family is what every father should do, isn’t it?
    \end{qa}

    \begin{qa}{Tell me about your best friend at school?}
    The last time I went to school was more than a decade ago but I still remember Minh vividly. I \textbf{hit it off with} him when we first met. We are \textbf{on the same wavelength} and I consider him my close \textbf{confidant} as he has never revealed any secrets of mine so far.
    \end{qa}

    \begin{qa}{How friendly are you with your neighbours? [Why?/Why not?]}
    Although I’m an introvert, I am still \textbf{easy-going} and \textbf{approachable} so I can \textbf{get on well with} most of my neighbors. Those who \textbf{blacken} my name are the ones I \textbf{distance myself from}. \textbf{What goes around comes around}, right?
    \end{qa}

    \begin{qa}{Which is more important to you, friends or family? [Why?]}
    Definitely the latter. There’s a proverb that “\textbf{Blood is thicker than water}”. No matter how close a friend may be, he is still \textbf{no match for} a family member in my opinion. It is partly because lots of buddies can \textbf{stay by my side} if they can \textbf{reap} some benefits from that. However, the assistance from a family member, especially one from an \textbf{immediate family}, is often unconditional.
    \end{qa}

        \begin{VocabExplain}[Part 1]
            \begin{ExplainCard}{hang out with}[phr.v][B1]
            \EN{to spend a lot of time with someone in a relaxed way.}
            \VI{đi chơi, dành thời gian với ai.}
            \SY{spend time with; socialize with}
            \EX{I usually hang out with my classmates after school.}
            \EX{Adolescents tend to hang out with peers of similar interests.}
            \CO{hang out with friends; hang out downtown}
            \end{ExplainCard}

            \begin{ExplainCard}{occupied}[adj][B2]
            \EN{busy doing something; not free.}
            \VI{bận rộn, không rảnh.}
            \SY{busy; engaged; preoccupied}
            \EX{She was occupied with preparing dinner.}
            \EX{The professor is currently occupied with research commitments.}
            \CO{occupied with work/study; fully occupied}
            \end{ExplainCard}

            \begin{ExplainCard}{hit it off with}[idiom][B2]
            \EN{to quickly become friendly with someone.}
            \VI{hợp nhau ngay từ đầu.}
            \SY{get along; connect; bond}
            \EX{We hit it off with our new neighbors immediately.}
            \EX{The interview panel hit it off with the candidate.}
            \CO{hit it off with someone}
            \end{ExplainCard}

            \begin{ExplainCard}{on the same wavelength}[idiom][C1]
            \EN{to think in a similar way and understand each other well.}
            \VI{cùng suy nghĩ, hiểu ý nhau.}
            \SY{in tune; like-minded; in harmony}
            \EX{I’m glad we’re on the same wavelength about this project.}
            \EX{Colleagues must be on the same wavelength to collaborate effectively.}
            \CO{be on the same wavelength with}
            \end{ExplainCard}

            \begin{ExplainCard}{confidant}[n][C1]
            \EN{a person you trust and share secrets with.}
            \VI{người tâm sự, bạn tri kỷ.}
            \SY{trusted friend; companion; adviser}
            \EX{She was my closest confidant during high school.}
            \EX{Leaders often have a small circle of confidants.}
            \CO{close confidant; trusted confidant}
            \end{ExplainCard}

            \begin{ExplainCard}{easy-going}[adj][B2]
            \EN{relaxed and not easily upset.}
            \VI{dễ chịu, thoải mái.}
            \SY{laid-back; relaxed; tolerant}
            \EX{He is easy-going and never gets angry.}
            \EX{An easy-going attitude helps in team cooperation.}
            \CO{easy-going personality; easy-going nature}
            \end{ExplainCard}

            \begin{ExplainCard}{approachable}[adj][B2]
            \EN{friendly and easy to talk to.}
            \VI{dễ gần, thân thiện.}
            \SY{friendly; open; welcoming}
            \EX{The teacher is very approachable.}
            \EX{Approachable leaders foster better communication.}
            \CO{seem approachable; approachable manner}
            \end{ExplainCard}

            \begin{ExplainCard}{get on well with}[phr.v][B1]
            \EN{to have a friendly relationship with someone.}
            \VI{hòa hợp, có quan hệ tốt với ai.}
            \SY{get along; be friendly with}
            \EX{I get on well with my colleagues.}
            \EX{Managers should get on well with their staff to improve morale.}
            \CO{get on well with someone}
            \end{ExplainCard}

            \begin{ExplainCard}{blacken}[v][C2]
            \EN{to damage someone’s reputation.}
            \VI{bôi nhọ, làm xấu danh tiếng.}
            \SY{defame; slander; malign}
            \EX{They tried to blacken his name in the media.}
            \EX{Political rivals often blacken each other’s reputations.}
            \CO{blacken one’s name/reputation}
            \end{ExplainCard}

            \begin{ExplainCard}{distance oneself from}[phrase][C1]
            \EN{to avoid being connected with someone or something.}
            \VI{giữ khoảng cách, tránh liên quan.}
            \SY{dissociate; detach; separate}
            \EX{He distanced himself from the scandal.}
            \EX{Scholars may distance themselves from controversial views.}
            \CO{distance oneself from sth/sb}
            \end{ExplainCard}

            \begin{ExplainCard}{What goes around comes around}[idiom][C1]
            \EN{the way you treat others will eventually return to you.}
            \VI{gieo nhân nào gặt quả nấy.}
            \SY{karma; reap what you sow}
            \EX{Be kind—what goes around comes around.}
            \EX{The theory of reciprocity echoes the idea that what goes around comes around.}
            \CO{say/quote what goes around comes around}
            \end{ExplainCard}

            \begin{ExplainCard}{Blood is thicker than water}[idiom][C1]
            \EN{family ties are stronger than other relationships.}
            \VI{máu mủ ruột rà quan trọng hơn bạn bè.}
            \SY{family first; kinship matters most}
            \EX{I chose to help my brother—blood is thicker than water.}
            \EX{The proverb “blood is thicker than water” highlights familial loyalty.}
            \CO{say/quote blood is thicker than water}
            \end{ExplainCard}

            \begin{ExplainCard}{no match for}[phrase][B2]
            \EN{not as good or strong as someone or something.}
            \VI{không bằng, không sánh kịp.}
            \SY{inferior to; weaker than}
            \EX{He is no match for his opponent in chess.}
            \EX{This evidence is no match for scientific proof.}
            \CO{be no match for sth/sb}
            \end{ExplainCard}

            \begin{ExplainCard}{stay by my side}[phrase][B2]
            \EN{to remain loyal and supportive.}
            \VI{ở bên cạnh, ủng hộ.}
            \SY{support; stand by; remain with}
            \EX{She promised to stay by my side forever.}
            \EX{Allies agreed to stay by each other’s side during the conflict.}
            \CO{stay by sb’s side; remain by one’s side}
            \end{ExplainCard}

            \begin{ExplainCard}{reap}[v][C1]
            \EN{to obtain something, especially as a result of effort.}
            \VI{gặt hái, thu được.}
            \SY{gain; obtain; harvest}
            \EX{He reaped the benefits of his hard work.}
            \EX{Nations reap advantages from international trade agreements.}
            \CO{reap benefits/rewards/profits}
            \end{ExplainCard}

            \begin{ExplainCard}{immediate family}[n][B2]
            \EN{a person’s closest relatives, such as parents, siblings, spouse, and children.}
            \VI{gia đình ruột thịt gần nhất.}
            \SY{close family; nuclear family}
            \EX{Only immediate family members attended the funeral.}
            \EX{Visa applications often require details of one’s immediate family.}
            \CO{member of immediate family; immediate family only}
            \end{ExplainCard}
        \end{VocabExplain}

    \noindent
    \textbf{Part 2.}
    \begin{qa}{Describe a writer you would like to meet. You should say:}
    \begin{itemize}
        \item Who the writer is
        \item What you know about this writer already
        \item What you would like to find out about him/her
        \item and explain why you would like to meet this writer.
    \end{itemize}

    Alright then, in response to the first question of who the writer is, my choice is going to be Nguyen Nhat Anh, who is a \textbf{distinguished} writer in Vietnam, but I haven’t seen him in real life. So I hope that in the near future, I can meet him. Now I believe that he \textbf{resides in} Ho Chi Minh city though I’m not pretty sure. It is due to the fact that he usually posts some pictures with fans in Ho Chi Minh.  

    He was born in 1955. He is working as a teacher, a poet and a writer. He \textbf{is skilled in} composing stories in \textbf{genres} like horror, mystery, science fiction, supernatural fiction and others for teenagers and adults. Some of his books are bestsellers, the \textbf{remaining} of which have been adapted for films as well.  

    So far, he has composed around 54 short stories, 2 novel series and a few collections of poems under his authorship and he is \textbf{arguably} the most \textbf{prolific} writer of his generation. His best-known composition, “Dreamy Eyes” in English or “Mắt Biếc” in Vietnamese, became a \textbf{box-office smash} after having been adapted to the big screen.  

    Progressing to the subject of why I’d like to meet him, I suppose I should underline that I aspire to take a photo with him and get an autograph from him as well. He is my idol, so I \textbf{thirst for} an opportunity to learn about his inspiration to compose stories. More importantly, there is a rumor that he has an attractive voice and a sense of humor, so it would be great to \textbf{have a chin-wag} to find out more about his life, passion and outlook.
    \end{qa}

        \begin{VocabExplain}[Part 2]
            \begin{ExplainCard}{distinguished}[adj][C1]
            \EN{very successful, respected, and admired.}
            \VI{lỗi lạc, xuất chúng, được kính trọng.}
            \SY{renowned; eminent; notable}
            \EX{She had a long and distinguished career in medicine.}
            \EX{Distinguished scholars attended the international conference.}
            \CO{a distinguished career; distinguished writer/professor}
            \end{ExplainCard}

            \begin{ExplainCard}{reside in}[phr.v][C1]
            \EN{to live in a particular place.}
            \VI{cư trú, sinh sống ở.}
            \SY{dwell in; inhabit; live in}
            \EX{They currently reside in the countryside.}
            \EX{Authority often resides in the hands of a few.}
            \CO{reside in the city/country; reside permanently}
            \end{ExplainCard}

            \begin{ExplainCard}{skilled in}[phrase][B2]
            \EN{having the ability or training to do something well.}
            \VI{thành thạo, có kỹ năng về.}
            \SY{proficient in; adept at; talented in}
            \EX{She is skilled in negotiating contracts.}
            \EX{Graduates skilled in computer science are in high demand.}
            \CO{skilled in doing sth; skilled in a field}
            \end{ExplainCard}

            \begin{ExplainCard}{genre}[n][B2]
            \EN{a particular style or category of art, literature, or music.}
            \VI{thể loại (văn học, nghệ thuật).}
            \SY{category; type; style}
            \EX{My favorite genre of film is comedy.}
            \EX{The study compared different literary genres across cultures.}
            \CO{literary genre; film genre; music genre}
            \end{ExplainCard}

            \begin{ExplainCard}{remaining}[adj][B2]
            \EN{still existing, left after others are gone or used.}
            \VI{còn lại.}
            \SY{leftover; surviving; unused}
            \EX{Only a few minutes are remaining.}
            \EX{The remaining data points were excluded from the analysis.}
            \CO{remaining time; remaining part; remaining members}
            \end{ExplainCard}

            \begin{ExplainCard}{arguably}[adv][C1]
            \EN{used when stating something that may be open to doubt but can be argued as true.}
            \VI{có thể cho là, được cho là.}
            \SY{possibly; debatably; conceivably}
            \EX{He is arguably the best player in the team.}
            \EX{This is arguably the most important discovery in physics.}
            \CO{arguably the best/most important}
            \end{ExplainCard}

            \begin{ExplainCard}{prolific}[adj][C1]
            \EN{producing a large amount of something, especially work.}
            \VI{sáng tác nhiều, có năng suất cao.}
            \SY{productive; creative; fertile}
            \EX{She is a prolific writer of novels.}
            \EX{The scientist was a prolific contributor to the field.}
            \CO{prolific writer/artist; prolific output}
            \end{ExplainCard}

            \begin{ExplainCard}{box-office smash}[idiom][C1]
            \EN{a film or show that is extremely successful financially.}
            \VI{bom tấn phòng vé.}
            \SY{blockbuster; hit; success}
            \EX{The film was a box-office smash worldwide.}
            \EX{Her latest release became a box-office smash, breaking records.}
            \CO{be a box-office smash; box-office smash hit}
            \end{ExplainCard}

            \begin{ExplainCard}{thirst for}[phrase][C1]
            \EN{a strong desire to have or do something.}
            \VI{khao khát, mong muốn mãnh liệt.}
            \SY{long for; yearn for; crave}
            \EX{She has a thirst for knowledge.}
            \EX{A thirst for justice often drives social movements.}
            \CO{thirst for knowledge/power/success}
            \end{ExplainCard}

            \begin{ExplainCard}{have a chin-wag}[idiom][informal][C1]
            \EN{to have a friendly conversation with someone.}
            \VI{tán gẫu, trò chuyện thân mật.}
            \SY{chat; gossip; talk}
            \EX{We had a chin-wag over coffee.}
            \EX{Colleagues had a chin-wag about new policies.}
            \CO{have a chin-wag with sb}
            \end{ExplainCard}
        \end{VocabExplain}

    \noindent
    \textbf{Part 3.}
    \begin{qa}{What kinds of book are most popular with children in your country? Why do you think that is?}
    I think most children would be \textbf{fanatical} about comic books which have been in great demand these days. For a long time, while comic books get a pretty \textbf{bad rap} as an unwanted distraction for children, its apparent benefit is to encourage their \textbf{fertile} imagination. These books usually offer simple \textbf{plots} and easy-to-read sentences, alongside other visual cues which promote the prediction of children.
    \end{qa}

    \begin{qa}{Why do you think some children do not read books very often?}
    I guess this is ascribed to the enlargement of entertainment industry which \textbf{inundates} children with many forms of entertainment and a lot of games, both indoor and outdoor ones. Unlike \textbf{static} images in books, video games with \textbf{animations} and pleasing sounds are likely to spark keen interests of children. Besides, reading books often requires patience because this is a \textbf{stand-alone} activity, so children, especially energetic ones, can feel bored stiff and want to give up in a flash.
    \end{qa}

    \begin{qa}{How do you think children can be encouraged to read more?}
    Parental \textbf{orientation} should play an active role in developing reading habits of children. To exemplify, parents could limit the time children spend on playing video games and replace these games by many genres of books such as folktale or detective stories. Moreover, the availability of audio book apps is another \textbf{remedy}. Thanks to vibrant and lively sounds, kids may be \textbf{emboldened} to \textbf{peruse} their books.
    \end{qa}

    \begin{qa}{Are there any occasions when reading at speed is a useful skill to have? What are they?}
    Judging from my own experience, speed reading is a real tool to have in my personal \textbf{arsenal} when it comes to examination due to time limit. Moreover, learning how to \textbf{skim through} a handful of pages, if only for a brief moment, is incredibly useful when I want to \textbf{scour for} a piece of information to strengthen the arguments in my assignment and \textbf{dissertation}, for example. However, it takes time to train my ability to read at speed without affecting reading \textbf{comprehension}, otherwise, I might fail to perceive important information.
    \end{qa}

    \begin{qa}{Are there any jobs where people need to read a lot? What are they?}
    It is safe to say that reading is a basic skill for most jobs, but there are specific jobs that focus on building proficient reading skills. For example, \textbf{archivists} are those who work with written \textbf{relics} to sort out and \textbf{appraise} documents for museums or educational institutions. There are also content editors who are vital to any publisher and company in the advertising industry. I believe they are an authority on spelling and \textbf{punctuation}, and work to maintain \textbf{consistency} in tone and style for their publication.
    \end{qa}

    \begin{qa}{Do you think that reading novels is more interesting than reading factual books? Why is that?}
    Both types of book can be equally \textbf{gripping} in my opinion. Novels or fiction books could depict legendary heroes and reveal myths which were used to be \textbf{enigmas} for readers, and novels can make up stories based on people’s imagination. By contrast, factual books arm readers with \textbf{plain} facts about anything from psychology to space exploration. Some factual books on the biography of some celebrities are inspiring, successfully depicting how the main character started from scratch and managed to \textbf{do all right for himself or herself}.
    \end{qa}

        \begin{VocabExplain}[Part 3]
            \begin{ExplainCard}{fanatical}[adj][C1]
            \EN{extremely interested or enthusiastic about something.}
            \VI{cuồng nhiệt, mê mẩn.}
            \SY{obsessive; passionate; fervent}
            \EX{He’s fanatical about video games.}
            \EX{Some fanatical supporters attended every single match.}
            \CO{fanatical about sth; fanatical supporter}
            \end{ExplainCard}

            \begin{ExplainCard}{bad rap}[idiom][C1]
            \EN{an unfairly bad reputation or criticism.}
            \VI{tiếng xấu, sự chỉ trích không công bằng.}
            \SY{unjust blame; unfair reputation}
            \EX{Video games often get a bad rap from parents.}
            \EX{The industry has a bad rap for being polluting.}
            \CO{get/catch a bad rap}
            \end{ExplainCard}

            \begin{ExplainCard}{fertile}[adj][C1]
            \EN{(of imagination) producing many new ideas.}
            \VI{phong phú, sáng tạo.}
            \SY{creative; imaginative; productive}
            \EX{Children have fertile imaginations.}
            \EX{A fertile ground for innovation lies in research labs.}
            \CO{fertile imagination; fertile ideas}
            \end{ExplainCard}

            \begin{ExplainCard}{plot}[n][B1]
            \EN{the main story of a play, film, or book.}
            \VI{cốt truyện.}
            \SY{storyline; narrative}
            \EX{The film had a complicated plot.}
            \EX{Literary critics analyze the plot structure of novels.}
            \CO{plot twist; plot development}
            \end{ExplainCard}

            \begin{ExplainCard}{inundate}[v][C1]
            \EN{to overwhelm someone with a lot of things.}
            \VI{tràn ngập, ngập lụt.}
            \SY{overwhelm; flood; swamp}
            \EX{I was inundated with emails this morning.}
            \EX{Children are inundated with choices of entertainment.}
            \CO{inundate with requests/work}
            \end{ExplainCard}

            \begin{ExplainCard}{static}[adj][C1]
            \EN{not moving or changing.}
            \VI{tĩnh, không thay đổi.}
            \SY{motionless; unchanging}
            \EX{The painting was static and lifeless.}
            \EX{Static models cannot explain economic fluctuations.}
            \CO{static image; static position}
            \end{ExplainCard}

            \begin{ExplainCard}{animation}[n][B2]
            \EN{moving images created for films, games or media.}
            \VI{hoạt hình, hình ảnh động.}
            \SY{cartoon; motion picture}
            \EX{Kids love Disney animations.}
            \EX{Animations are widely used in advertising.}
            \CO{3D animation; animated movie}
            \end{ExplainCard}

            \begin{ExplainCard}{stand-alone}[adj][C1]
            \EN{independent; not requiring connection to others.}
            \VI{độc lập, riêng lẻ.}
            \SY{independent; self-contained}
            \EX{It is a stand-alone course.}
            \EX{Reading is often a stand-alone activity.}
            \CO{stand-alone product/activity}
            \end{ExplainCard}

            \begin{ExplainCard}{orientation}[n][C1]
            \EN{guidance or training in a particular direction.}
            \VI{sự định hướng.}
            \SY{guidance; direction}
            \EX{The company gives orientation to new employees.}
            \EX{Parental orientation influences children’s behavior.}
            \CO{career orientation; parental orientation}
            \end{ExplainCard}

            \begin{ExplainCard}{remedy}[n][B2]
            \EN{a solution to a problem.}
            \VI{biện pháp khắc phục.}
            \SY{solution; cure; fix}
            \EX{Meditation is a remedy for stress.}
            \EX{Policy reform is a remedy for systemic issues.}
            \CO{remedy for sth; provide a remedy}
            \end{ExplainCard}

            \begin{ExplainCard}{embolden}[v][C2]
            \EN{to give courage or confidence.}
            \VI{làm mạnh dạn, khuyến khích.}
            \SY{encourage; inspire}
            \EX{Her success emboldened him to try.}
            \EX{Social media emboldens people to speak out.}
            \CO{embolden sb to do sth}
            \end{ExplainCard}

            \begin{ExplainCard}{peruse}[v][C2]
            \EN{to read carefully.}
            \VI{đọc kỹ lưỡng.}
            \SY{examine; scrutinize}
            \EX{She perused the letter slowly.}
            \EX{Academics peruse manuscripts for details.}
            \CO{peruse documents; peruse books}
            \end{ExplainCard}

            \begin{ExplainCard}{arsenal}[n][C1]
            \EN{a collection of resources or skills available.}
            \VI{kho vũ khí (nghĩa bóng: kho công cụ/kỹ năng).}
            \SY{collection; repertoire}
            \EX{He has an arsenal of jokes.}
            \EX{Data scientists use an arsenal of tools to analyze data.}
            \CO{an arsenal of weapons/skills}
            \end{ExplainCard}

            \begin{ExplainCard}{skim through}[phr.v][B2]
            \EN{to read quickly without focusing on details.}
            \VI{đọc lướt.}
            \SY{scan; glance through}
            \EX{He skimmed through the article before class.}
            \EX{Researchers skim through abstracts to find relevant studies.}
            \CO{skim through pages/books}
            \end{ExplainCard}

            \begin{ExplainCard}{scour for}[phr.v][C1]
            \EN{to search very carefully for something.}
            \VI{lục lọi, tìm kiếm kỹ lưỡng.}
            \SY{search; hunt; comb}
            \EX{She scoured for her missing keys.}
            \EX{Historians scour for evidence in archives.}
            \CO{scour for information/evidence}
            \end{ExplainCard}

            \begin{ExplainCard}{dissertation}[n][C1]
            \EN{a long piece of writing on a subject for a degree.}
            \VI{luận văn, luận án.}
            \SY{thesis; essay}
            \EX{She is writing her dissertation on linguistics.}
            \EX{PhD students must defend their dissertations publicly.}
            \CO{write a dissertation; doctoral dissertation}
            \end{ExplainCard}

            \begin{ExplainCard}{comprehension}[n][B2]
            \EN{the ability to understand something.}
            \VI{sự hiểu.}
            \SY{understanding; grasp}
            \EX{Listening comprehension is essential in exams.}
            \EX{Reading comprehension is tested in standardized tests.}
            \CO{reading comprehension; comprehension skills}
            \end{ExplainCard}

            \begin{ExplainCard}{archivist}[n][C1]
            \EN{a person who maintains and organizes historical records.}
            \VI{nhà lưu trữ.}
            \SY{record keeper; curator}
            \EX{The archivist catalogued old letters.}
            \EX{Archivists preserve historical relics in museums.}
            \CO{professional archivist; work as an archivist}
            \end{ExplainCard}

            \begin{ExplainCard}{relic}[n][C1]
            \EN{an object from the past that has survived.}
            \VI{di tích, tàn tích.}
            \SY{artifact; remains}
            \EX{The relics of the old temple were found.}
            \EX{Relics provide evidence of ancient cultures.}
            \CO{cultural relics; historical relics}
            \end{ExplainCard}

            \begin{ExplainCard}{appraise}[v][C1]
            \EN{to assess the value or quality of something.}
            \VI{đánh giá, thẩm định.}
            \SY{evaluate; assess}
            \EX{They appraised the painting before selling it.}
            \EX{Scholars appraise manuscripts to verify authenticity.}
            \CO{appraise value/quality}
            \end{ExplainCard}

            \begin{ExplainCard}{punctuation}[n][B2]
            \EN{the marks in writing that separate sentences and clarify meaning.}
            \VI{dấu câu.}
            \SY{syntax marks; writing symbols}
            \EX{Check your punctuation before submitting the essay.}
            \EX{Punctuation plays a role in clarity of research papers.}
            \CO{correct punctuation; punctuation errors}
            \end{ExplainCard}

            \begin{ExplainCard}{consistency}[n][C1]
            \EN{the quality of always behaving or performing in a similar way.}
            \VI{tính nhất quán.}
            \SY{uniformity; steadiness}
            \EX{She shows consistency in her work.}
            \EX{Consistency in style is important in publications.}
            \CO{maintain consistency; consistency of tone}
            \end{ExplainCard}

            \begin{ExplainCard}{gripping}[adj][C1]
            \EN{very exciting and holding your attention.}
            \VI{hấp dẫn, cuốn hút.}
            \SY{thrilling; captivating}
            \EX{It’s a gripping film.}
            \EX{The gripping evidence captured researchers’ interest.}
            \CO{gripping story; gripping performance}
            \end{ExplainCard}

            \begin{ExplainCard}{enigma}[n][C1]
            \EN{something mysterious or difficult to understand.}
            \VI{điều bí ẩn, khó hiểu.}
            \SY{mystery; puzzle}
            \EX{His past remains an enigma.}
            \EX{Dark matter is an enigma in astrophysics.}
            \CO{remain an enigma; pose an enigma}
            \end{ExplainCard}

            \begin{ExplainCard}{plain}[adj][B1]
            \EN{simple and not complicated.}
            \VI{đơn giản, rõ ràng.}
            \SY{simple; straightforward}
            \EX{She wore a plain dress.}
            \EX{Plain facts are needed in scientific arguments.}
            \CO{plain language; plain truth; plain facts}
            \end{ExplainCard}

            \begin{ExplainCard}{do all right for himself/herself}[idiom][C1]
            \EN{to be successful in life or career.}
            \VI{thành công, ổn định cuộc sống.}
            \SY{succeed; prosper}
            \EX{He has done all right for himself since leaving school.}
            \EX{The researcher did all right for herself with many publications.}
            \CO{do all right for oneself}
            \end{ExplainCard}
        \end{VocabExplain}

    \begin{VocabHighlights}
        \VH{to hang out with}{(phr.v) to engage in aimless recreation or frivolous time-wasting; to fool around}{(cụm động từ) đi chơi}
        \VH{to be occupied V-ing}{(phrase) to be busy V-ing}{(cụm từ) bận làm gì}
        \VH{to hit it off with somebody}{(idiom) to have a good friendly relationship with somebody}{(thành ngữ) có mối quan hệ tốt với ai}
        \VH{to be on the same wavelength}{(idiom) to have the same way of thinking or the same ideas or feelings as somebody else}{(thành ngữ) cùng cách suy nghĩ, tư duy}
        \VH{confidant}{(n) a person that you trust and who you talk to about private or secret things}{(danh từ) người đáng tin cậy}
        \VH{easy-going}{(adj) relaxed and happy to accept things without worrying or getting angry}{(tính từ) dễ tính}
        \VH{approachable}{(adj) friendly and easy to talk to; easy to understand}{(tính từ) dễ gần}
        \VH{to get on (well) with}{(phr.v) to have a friendly relationship with somebody}{(cụm động từ) có quan hệ tốt với ai}
        \VH{to blacken somebody’s name}{(phrase) to say unpleasant things that give people a bad opinion of somebody}{(cụm từ) nói xấu ai}
        \VH{to distance oneself from}{(v) to become, or to make somebody/something become, less involved or connected with somebody/something}{(động từ) tránh xa khỏi}
        \VH{what goes around comes around}{(proverb) used to say that if someone treats other people badly he or she will eventually be treated badly by someone else}{(tục ngữ) gieo nhân nào gặt quả nấy}
        \VH{blood is thicker than water}{(proverb) relationships and loyalties within a family are the strongest \& most important ones}{(tục ngữ) 1 giọt máu đào hơn ao nước lã}
        \VH{to be no match for}{(idiom) a person who is not equal to somebody else in strength, skill, intelligence, etc}{(thành ngữ) không phải đối thủ}
        \VH{to stay by one’s side}{(idiom) to be with someone, and take care of them or support them in difficult situations}{(thành ngữ) luôn ủng hộ ai}
        \VH{to reap}{(v) to obtain something, especially something good, as a direct result of something that you have done}{(động từ) gặt hái, đạt được}
        \VH{immediate family}{(phrase) people who are very closely related to you, such as your parents, children, brothers, and sisters}{(cụm từ) người thân gia đình}
        \VH{fanatical}{(adj) extremely enthusiastic}{(tính từ) phát cuồng}
        \VH{bad rap}{(n) a negative reputation}{(danh từ) danh tiếng xấu}
        \VH{fertile}{(adj) (of land or soil) that plants grow well in}{(tính từ) phì nhiêu}
        \VH{plot}{(n) the series of events that form the story of a novel, play, film/movie, etc.}{(danh từ) cốt truyện}
        \VH{to inundate}{(v) to give or send somebody so many things that they cannot deal with them all}{(động từ) khiến ai đó quá tải}
        \VH{static}{(adj) staying in one place without moving, or not changing for a long time}{(tính từ) tĩnh, không chuyển động}
        \VH{animation}{(n) moving images created from drawings, models, etc. that are photographed or created by a computer}{(danh từ) phim hoạt hình}
        \VH{stand-alone}{(adj) able to operate independently of other hardware or software}{(tính từ) độc lập, một mình}
        \VH{orientation}{(n) the type of aims or interests that a person or an organization has; the act of directing your aims towards a particular thing}{(danh từ) sự định hướng}
        \VH{remedy}{(n) a way of dealing with or improving an unpleasant or difficult situation}{(danh từ) phương pháp}
        \VH{to be emboldened to V-inf}{(p2) to be encouraged to V-inf}{(phần từ 2) được khuyến khích}
        \VH{to peruse}{(v) to read something, especially in a careful way}{(động từ) xem kỹ, duyệt kỹ}
        \VH{arsenal}{(n) a collection of weapons such as guns and explosive}{(danh từ) vũ khí, vật có ích}
        \VH{to skim through}{(phrase) to read something hurriedly without being attached to details}{(cụm từ) lướt qua}
        \VH{to scour for}{(phr.v) to search a place or thing carefully and completely in order to find somebody/something}{(cụm động từ) tìm kiếm, sục sạo, lùng sục}
        \VH{dissertation}{(n) a long piece of writing on a particular subject, especially one written for a university degree}{(danh từ) luận văn}
        \VH{comprehension}{(n) the ability to understand}{(danh từ) khả năng hiểu}
        \VH{archivists}{(n) a person whose job is to develop and manage an archive}{(danh từ) người lưu trữ}
        \VH{relics}{(n) an object, a tradition, a system, etc. that has survived from the past}{(danh từ) di tích}
        \VH{to appraise}{(v) consider or examine somebody/something and form an opinion about them or it}{(động từ) thẩm định}
        \VH{punctuation}{(n) the marks used in writing that divide sentences and phrases; the system of using these marks}{(danh từ) dấu chấm câu}
        \VH{consistency}{(n) the quality of always behaving in the same way or of having the same opinions, standard, etc.; the quality of being consistent}{(danh từ) tính nhất quán}
        \VH{gripping}{(adj) exciting or interesting in a way that keeps your attention}{(tính từ) làm say sưa}
        \VH{enigma}{(n) a person, thing or situation that is mysterious and difficult to understand}{(danh từ) điều bí ẩn}
        \VH{to do all right for oneself}{(idiom) to be successful in somebody’s life/job}{(thành ngữ) thành công trong cuộc sống, công việc}
    \end{VocabHighlights}
    \end{test}

    \begin{test}{TEST 3}
    \noindent
    \textbf{Part 1. Photograph}
    \begin{qa}{What type of photos do you like taking? [Why?/Why not?]}
    Although taking selfies and posting on social networks are \textbf{in the mainstream in today’s world}, I still fancy taking \textbf{snapshots} of life around me. I am a realistic person so I prefer depicting daily life as it is in my photos.
    \end{qa}

    \begin{qa}{What do you do with photos you take? [Why?/Why not?]}
    I don’t normally \textbf{pose for} pictures. Maybe what I often do most before taking a picture is raising my hand to display the “Let’s rock” gesture. It is not only a typical symbol of a metal fan but also able to indicate how I want to \textbf{live my life to the fullest}. Then, I may upload some photos to Facebook to show my \textbf{friends} what I’ve done and where I’ve been to. It’s also a good way to \textbf{back} my photos \textbf{up} as well.
    \end{qa}

    \begin{qa}{When you visit other places, do you take photos or buy postcards? [Why?/Why not?]}
    Taking personal photos from my own angle is what I opt for. Sometimes I feel like shooting a photo with a view to checking in a place with my face in it, which would be totally impossible if I buy a postcard. It’s \textbf{a piece of cake} to buy postcards of a \textbf{landscape} only. In my opinion, having a photo of myself in a landscape is more worthwhile.
    \end{qa}

    \begin{qa}{Do you like people taking photos of you? [Why?/Why not?]}
    It depends. If they notify me in advance and do this with my permission, then they can \textbf{go ahead}. However, if they take unwanted photos of me without my permission, I may be \textbf{offended}. No one wants their right to privacy to be \textbf{violated}.
    \end{qa}

        \begin{VocabExplain}[Part 1]
            \begin{ExplainCard}{in the mainstream}[phrase][C1]
            \EN{considered normal and accepted by most people.}
            \VI{theo xu hướng phổ biến, chính thống.}
            \SY{popular; conventional; widespread}
            \EX{Online shopping is now in the mainstream.}
            \EX{In the mainstream media, these views are often promoted.}
            \CO{in the mainstream of society; enter the mainstream}
            \end{ExplainCard}

            \begin{ExplainCard}{snapshot}[n][B2]
            \EN{a quick photo taken without preparation.}
            \VI{ảnh chụp nhanh.}
            \SY{photo; shot; picture}
            \EX{She took a quick snapshot of the sunset.}
            \EX{Snapshots can reveal valuable insights in research studies.}
            \CO{take a snapshot; family snapshots}
            \end{ExplainCard}

            \begin{ExplainCard}{pose for}[phr.v][B2]
            \EN{to stand or sit in a particular position for a photo.}
            \VI{tạo dáng chụp hình.}
            \SY{model; posture; position}
            \EX{They posed for a wedding photo.}
            \EX{Participants were asked to pose for official documentation.}
            \CO{pose for pictures/photos}
            \end{ExplainCard}

            \begin{ExplainCard}{live life to the fullest}[idiom][B2]
            \EN{to enjoy life as much as possible.}
            \VI{sống hết mình.}
            \SY{enjoy life; seize the day}
            \EX{She travels the world to live her life to the fullest.}
            \EX{Many students strive to live life to the fullest during their youth.}
            \CO{live life to the fullest}
            \end{ExplainCard}

            \begin{ExplainCard}{back up}[phr.v][B2]
            \EN{to make a copy of information to keep it safe.}
            \VI{sao lưu dữ liệu.}
            \SY{save; duplicate; store}
            \EX{Don’t forget to back up your files.}
            \EX{Researchers always back up data before analysis.}
            \CO{back up files/photos}
            \end{ExplainCard}

            \begin{ExplainCard}{a piece of cake}[idiom][B1]
            \EN{something very easy to do.}
            \VI{dễ như ăn bánh.}
            \SY{simple; effortless; straightforward}
            \EX{The exam was a piece of cake.}
            \EX{For experts, coding this program is a piece of cake.}
            \CO{be a piece of cake}
            \end{ExplainCard}

            \begin{ExplainCard}{landscape}[n][B1]
            \EN{an area of countryside or scenery that you can see.}
            \VI{phong cảnh.}
            \SY{scenery; view; countryside}
            \EX{The mountain landscape is breathtaking.}
            \EX{Landscape paintings are studied in art history.}
            \CO{beautiful landscape; urban landscape; rural landscape}
            \end{ExplainCard}

            \begin{ExplainCard}{go ahead}[phr.v][B2]
            \EN{to begin or continue with something after permission.}
            \VI{cứ tiến hành, tiếp tục.}
            \SY{proceed; continue}
            \EX{You can go ahead with your plans.}
            \EX{After approval, the project will go ahead next year.}
            \CO{go ahead with a plan/project}
            \end{ExplainCard}

            \begin{ExplainCard}{offended}[adj][B2]
            \EN{upset or hurt by someone’s words or actions.}
            \VI{cảm thấy bị xúc phạm.}
            \SY{hurt; insulted; upset}
            \EX{She was offended by his rude remarks.}
            \EX{Students felt offended by the unfair comments.}
            \CO{feel offended; deeply offended}
            \end{ExplainCard}

            \begin{ExplainCard}{violate}[v][C1]
            \EN{to break a law or someone’s rights.}
            \VI{xâm phạm, vi phạm.}
            \SY{infringe; breach; disobey}
            \EX{He violated the speed limit.}
            \EX{Publishing private photos violates personal rights.}
            \CO{violate rights/laws/privacy}
            \end{ExplainCard}
        \end{VocabExplain}

    \noindent
    \textbf{Part 2.}
    \begin{qa}{Describe a day when you thought the weather was perfect. You should say:}
    \begin{itemize}
        \item Where you were on this day
        \item What the weather was like on this day
        \item What you did during the day
        \item and explain why you thought the weather was perfect on this day.
    \end{itemize}

    Today I would like to tell you about a day when the weather was \textbf{superb}. A month ago, my company hosted a big party in Ha Long to celebrate its 20\textsuperscript{th} birthday. It was also a golden opportunity for people to \textbf{let off steam} during the \textbf{sweltering heat} of the summer.  

    We departed from Hanoi, and it took us 2 hours to reach Ha Long. When we set out, the weather was dull, and I thought it might rain. But my idea was totally wrong. When we reached Ha Long, the \textbf{azure} sky was clear with sunshine. The temperature was not too high, ranging from 26 to 29 Celsius degrees, milder than that a few days ago. Besides, there were some breezes, so it was an ideal time for outdoor activities.  

    There were \textbf{heaps of} activities that we took part in. We went for a swim as soon as we arrived because all of us \textbf{got a kick out of it}. In the afternoon, my company organized teambuilding activities with many games. We had \textbf{piles of fun}, and thanks to these games, we had memorable \textbf{bonding moments}. After games, we returned to the hotel, and \textbf{dressed up} for an outdoor party at 6:30.  

    In the evening, the weather was \textbf{pleasant}, and a sea breeze was blowing mildly. We \textbf{enjoyed ourselves very much}!
    \end{qa}

        \begin{VocabExplain}[Part 2]
            \begin{ExplainCard}{superb}[adj][C1]
            \EN{excellent; of the highest quality.}
            \VI{tuyệt vời, xuất sắc.}
            \SY{outstanding; excellent; magnificent}
            \EX{The food was superb.}
            \EX{They gave a superb performance at the conference.}
            \CO{superb weather; superb view; superb performance}
            \end{ExplainCard}

            \begin{ExplainCard}{let off steam}[idiom][C1]
            \EN{to get rid of strong feelings or energy by doing an activity.}
            \VI{xả hơi, giải tỏa căng thẳng.}
            \SY{unwind; relax; release tension}
            \EX{He jogs every morning to let off steam.}
            \EX{Sports allow students to let off steam after exams.}
            \CO{let off steam by doing sth}
            \end{ExplainCard}

            \begin{ExplainCard}{sweltering heat}[phrase][C1]
            \EN{extremely hot and uncomfortable weather.}
            \VI{cái nóng ngột ngạt, oi bức.}
            \SY{boiling heat; scorching heat}
            \EX{We struggled in the sweltering heat of the desert.}
            \EX{Many cities face sweltering heat waves in summer.}
            \CO{sweltering heat; sweltering weather}
            \end{ExplainCard}

            \begin{ExplainCard}{azure}[adj][C2]
            \EN{bright blue in color, like a clear sky.}
            \VI{xanh da trời.}
            \SY{sky-blue; cerulean}
            \EX{The water was a deep azure.}
            \EX{Painters often depict azure skies in landscapes.}
            \CO{azure sky; azure sea}
            \end{ExplainCard}

            \begin{ExplainCard}{heaps of}[phrase][B2]
            \EN{a large amount of something.}
            \VI{rất nhiều.}
            \SY{lots of; loads of; plenty of}
            \EX{There were heaps of people at the concert.}
            \EX{Students face heaps of assignments at university.}
            \CO{heaps of fun/work/activities}
            \end{ExplainCard}

            \begin{ExplainCard}{get a kick out of}[idiom][C1]
            \EN{to enjoy something very much.}
            \VI{rất thích thú với.}
            \SY{enjoy; take pleasure in; relish}
            \EX{She gets a kick out of playing the guitar.}
            \EX{Many people get a kick out of solving puzzles.}
            \CO{get a kick out of sth}
            \end{ExplainCard}

            \begin{ExplainCard}{piles of fun}[phrase][B2]
            \EN{a large amount of enjoyment.}
            \VI{rất vui, cực nhiều niềm vui.}
            \SY{loads of fun; great fun}
            \EX{The kids had piles of fun at the party.}
            \EX{The experiment was piles of fun for the participants.}
            \CO{piles of fun}
            \end{ExplainCard}

            \begin{ExplainCard}{bonding moments}[phrase][B2]
            \EN{special times that strengthen relationships.}
            \VI{khoảnh khắc gắn kết.}
            \SY{shared moments; connecting experiences}
            \EX{Family dinners create bonding moments.}
            \EX{Team projects can provide bonding moments for students.}
            \CO{memorable bonding moments; bonding time}
            \end{ExplainCard}

            \begin{ExplainCard}{dress up}[phr.v][B1]
            \EN{to put on formal or special clothes.}
            \VI{ăn mặc đẹp, diện.}
            \SY{wear formal clothes; attire}
            \EX{She dressed up for the wedding.}
            \EX{Scientists dressed up for the award ceremony.}
            \CO{dress up for a party/event}
            \end{ExplainCard}

            \begin{ExplainCard}{pleasant}[adj][B1]
            \EN{enjoyable, giving a sense of happiness.}
            \VI{dễ chịu, thoải mái.}
            \SY{enjoyable; agreeable; delightful}
            \EX{It was pleasant to sit in the garden.}
            \EX{We had a pleasant discussion after the lecture.}
            \CO{pleasant weather; pleasant surprise}
            \end{ExplainCard}

            \begin{ExplainCard}{enjoy oneself very much}[phrase][B2]
            \EN{to have a really good time.}
            \VI{rất vui vẻ, tận hưởng.}
            \SY{have fun; have a great time}
            \EX{They enjoyed themselves very much at the concert.}
            \EX{Participants enjoyed themselves very much during the event.}
            \CO{enjoy oneself very much}
            \end{ExplainCard}
        \end{VocabExplain}

    \noindent
    \textbf{Part 3.}
    \begin{qa}{What types of weather do people in your country dislike most? Why is that?}
    The weather in my country is quite unpredictable as seasonal variations in tropical climate are dominated by changes in \textbf{precipitation}. For the most part, people can easily get \textbf{frustrated} with \textbf{mist} and \textbf{drizzle} because of muddy roads, not to mention that this type of weather can develop some diseases such as \textbf{arthritis} of the elderly.
    \end{qa}

    \begin{qa}{What jobs can be affected by different weather conditions? Why?}
    It seems to me that different weather conditions can affect most social activities. For example, it is very dangerous for pilots to operate their aircraft \textbf{at the mercy of} bad weather, because this may lead to unexpected \textbf{incidents}. Severe weather conditions also affect fishing patterns, especially in stormy seasons when thunderstorms or lightnings can destroy many fishing boats and jeopardize the fishermen’s lives.
    \end{qa}

    \begin{qa}{Are there any important festivals in your country that celebrate a season or type of weather?}
    Actually, there are lots of seasonal festivals in my country, most of which are linked with local customs so they are only renowned in the area. However, Tet festival or Lunar New Year from early spring is the time when I find homes and streets come alive with a \textbf{jubilant mood}. Spring is generally the season of festivals because Vietnamese people consider this season the beginning of all things. People, therefore, will \textbf{immerse} themselves in dancing and partying, and some would head off for temples to pay \textbf{tribute} to ancestors in the hope of a \textbf{fruitful} year.
    \end{qa}

    \begin{qa}{How important do you think it is for everyone to check what the next day’s weather will be? Why?}
    Of course, the weather can greatly affect the daily routine and schedule of each individual, so knowing what the weather will be the next day can \textbf{safeguard} people from uncomfortable situations. For example, watching weather forecast allows people to make a better choice of seasonal clothes and transportation. On a rainy day, carrying an umbrella or catching a \textbf{cab} seem to be more preferable so as to avoid \textbf{filthiness}, compared to riding a motorbike in damp weather. In contrast, it is advisable to \textbf{bundle up} on bitterly cold days.
    \end{qa}

    \begin{qa}{What is the best way to get accurate information about the weather?}
    In my opinion, the best way is using weather applications on cell phones that update themselves hourly. Besides, there are plenty of weather forecast programs broadcasting weather reports on TV or the Internet for people to update. On a national scale, a long-range forecast is available. Likewise, a local weather forecast will specify the weather pattern in a particular area. All weather reports have to go through \textbf{calibration} and \textbf{verification} before being aired, so I think it is reliable enough for people to check weather forecast.
    \end{qa}

    \begin{qa}{How easy or difficult is it to predict the weather in your country? Why is that?}
    \textbf{By virtue of} technological innovations, \textbf{hydrometeorological} engineers nowadays can produce weather forecast news more accurately than ever before. This flying \textbf{leap} has saved millions of people and protected their properties from natural \textbf{calamities}. Having said that, there are still major disadvantages in predicting the precise information about some disasters such as \textbf{landslides} or earthquakes due to the lack of \textbf{topographic} features, for example.
    \end{qa}

        \begin{VocabExplain}[Part 3]
            \begin{ExplainCard}{precipitation}[n][C1]
            \EN{rain, snow, or other forms of moisture that fall from the sky.}
            \VI{lượng mưa (hoặc tuyết).}
            \SY{rainfall; moisture}
            \EX{Heavy precipitation is expected tomorrow.}
            \EX{Climate models estimate annual precipitation levels.}
            \CO{heavy precipitation; annual precipitation}
            \end{ExplainCard}

            \begin{ExplainCard}{frustrated}[adj][B2]
            \EN{feeling annoyed because you cannot achieve what you want.}
            \VI{bực bội, thất vọng.}
            \SY{annoyed; irritated}
            \EX{He felt frustrated by the delay.}
            \EX{Citizens are frustrated with the lack of public services.}
            \CO{feel frustrated; frustrated with sth}
            \end{ExplainCard}

            \begin{ExplainCard}{drizzle}[n][B1]
            \EN{light rain falling in fine drops.}
            \VI{mưa phùn.}
            \SY{light rain; misty rain}
            \EX{We walked home in the drizzle.}
            \EX{Drizzle reduces visibility for drivers.}
            \CO{light drizzle; a constant drizzle}
            \end{ExplainCard}

            \begin{ExplainCard}{arthritis}[n][C1]
            \EN{a disease that causes painful swelling in the joints.}
            \VI{bệnh viêm khớp.}
            \SY{joint inflammation}
            \EX{Elderly people often suffer from arthritis.}
            \EX{Research aims to find treatments for arthritis.}
            \CO{arthritis pain; treatment for arthritis}
            \end{ExplainCard}

            \begin{ExplainCard}{at the mercy of}[phrase][C1]
            \EN{not able to protect yourself from something.}
            \VI{phó mặc, chịu sự chi phối.}
            \SY{vulnerable to; powerless against}
            \EX{They were at the mercy of the storm.}
            \EX{Small economies are at the mercy of global markets.}
            \CO{at the mercy of the weather/fate}
            \end{ExplainCard}

            \begin{ExplainCard}{incident}[n][B2]
            \EN{an unexpected event, usually unpleasant.}
            \VI{sự cố, vụ việc.}
            \SY{event; occurrence}
            \EX{There was a minor incident at the station.}
            \EX{Historical incidents shape national identity.}
            \CO{serious incident; minor incident}
            \end{ExplainCard}

            \begin{ExplainCard}{jubilant}[adj][C1]
            \EN{feeling or expressing great happiness.}
            \VI{hân hoan, vui mừng.}
            \SY{joyful; elated; overjoyed}
            \EX{Fans were jubilant after the victory.}
            \EX{Villagers were in a jubilant mood during the festival.}
            \CO{jubilant mood; jubilant crowd}
            \end{ExplainCard}

            \begin{ExplainCard}{immerse}[v][C1]
            \EN{to become completely involved in an activity.}
            \VI{đắm chìm vào, hoà mình vào.}
            \SY{absorb; engage}
            \EX{He immersed himself in study.}
            \EX{Tourists immerse themselves in local culture.}
            \CO{immerse in culture/work; immerse oneself}
            \end{ExplainCard}

            \begin{ExplainCard}{tribute}[n][C1]
            \EN{an act, statement, or gift that shows respect.}
            \VI{sự tri ân, cống phẩm.}
            \SY{homage; respect}
            \EX{They paid tribute to the fallen soldiers.}
            \EX{The award is a tribute to her dedication.}
            \CO{pay tribute to; a tribute concert}
            \end{ExplainCard}

            \begin{ExplainCard}{fruitful}[adj][C1]
            \EN{producing good or useful results.}
            \VI{có kết quả, thành công.}
            \SY{productive; beneficial}
            \EX{The discussion was very fruitful.}
            \EX{Fruitful research has led to new discoveries.}
            \CO{fruitful discussion; fruitful year}
            \end{ExplainCard}

            \begin{ExplainCard}{safeguard}[v][C1]
            \EN{to protect something from harm.}
            \VI{bảo vệ, giữ an toàn.}
            \SY{protect; defend; secure}
            \EX{New laws safeguard endangered species.}
            \EX{Citizens’ rights must be safeguarded by the constitution.}
            \CO{safeguard rights; safeguard health}
            \end{ExplainCard}

            \begin{ExplainCard}{cab}[n][B1]
            \EN{a taxi.}
            \VI{xe taxi.}
            \SY{taxi}
            \EX{We took a cab to the station.}
            \EX{Cabs are cheaper in some cities than others.}
            \CO{hail a cab; take a cab}
            \end{ExplainCard}

            \begin{ExplainCard}{filthiness}[n][C1]
            \EN{the state of being extremely dirty.}
            \VI{sự dơ bẩn.}
            \SY{dirtiness; uncleanliness}
            \EX{The filthiness of the streets shocked us.}
            \EX{Researchers studied the filthiness of polluted rivers.}
            \CO{filthiness of sth}
            \end{ExplainCard}

            \begin{ExplainCard}{bundle up}[phr.v][B2]
            \EN{to dress warmly.}
            \VI{mặc ấm.}
            \SY{wrap up; dress warmly}
            \EX{We bundled up before going out in the snow.}
            \EX{Children should bundle up in winter to avoid colds.}
            \CO{bundle up in coats/clothes}
            \end{ExplainCard}

            \begin{ExplainCard}{calibration}[n][C1]
            \EN{the process of adjusting equipment for accuracy.}
            \VI{hiệu chỉnh.}
            \SY{adjustment; standardization}
            \EX{Calibration of the instrument is required.}
            \EX{Weather stations rely on calibration for accurate data.}
            \CO{device calibration; calibration process}
            \end{ExplainCard}

            \begin{ExplainCard}{verification}[n][C1]
            \EN{the process of checking that something is true or correct.}
            \VI{xác minh, kiểm chứng.}
            \SY{confirmation; validation}
            \EX{You need verification of your identity.}
            \EX{Verification ensures data reliability in research.}
            \CO{verification process; require verification}
            \end{ExplainCard}

            \begin{ExplainCard}{by virtue of}[phrase][C2]
            \EN{because of or as a result of something.}
            \VI{nhờ vào, bởi vì.}
            \SY{due to; owing to; thanks to}
            \EX{She got the job by virtue of her experience.}
            \EX{By virtue of innovation, productivity has increased.}
            \CO{by virtue of sth}
            \end{ExplainCard}

            \begin{ExplainCard}{hydrometeorological}[adj][C2]
            \EN{relating to the study of water and atmospheric phenomena.}
            \VI{thuộc khí tượng thủy văn.}
            \SY{climatic; meteorological}
            \EX{Hydrometeorological data is used for weather forecasts.}
            \EX{They built a hydrometeorological station near the river.}
            \CO{hydrometeorological data/station}
            \end{ExplainCard}

            \begin{ExplainCard}{leap}[n][C1]
            \EN{a big jump or sudden improvement.}
            \VI{bước nhảy vọt.}
            \SY{advance; breakthrough}
            \EX{The invention was a great leap forward in science.}
            \EX{Technology has taken a huge leap in the past decade.}
            \CO{a leap forward; a flying leap}
            \end{ExplainCard}

            \begin{ExplainCard}{calamity}[n][C1]
            \EN{a serious event causing damage or suffering.}
            \VI{thảm họa.}
            \SY{disaster; catastrophe}
            \EX{The earthquake was a national calamity.}
            \EX{Economic calamities can destroy countries.}
            \CO{natural calamity; economic calamity}
            \end{ExplainCard}

            \begin{ExplainCard}{landslide}[n][C1]
            \EN{a mass of rock and earth sliding down a mountain.}
            \VI{lở đất.}
            \SY{earthfall; landslip}
            \EX{A landslide destroyed the road.}
            \EX{Landslides are common in mountainous areas.}
            \CO{cause a landslide; landslide disaster}
            \end{ExplainCard}

            \begin{ExplainCard}{topographic}[adj][C1]
            \EN{relating to the shape and features of land surfaces.}
            \VI{thuộc địa hình.}
            \SY{geographic; geomorphologic}
            \EX{The topographic map shows mountain ranges.}
            \EX{Topographic features influence rainfall distribution.}
            \CO{topographic map; topographic feature}
            \end{ExplainCard}
        \end{VocabExplain}

    \begin{VocabHighlights}
        \VH{to be in the mainstream}{(idiom) following the current trends or styles that are popular}{(thành ngữ) phổ biến}
        \VH{in today’s world}{(phrase) nowadays}{(cụm từ) ngày nay}
        \VH{snapshot}{(n) a photograph taken quickly and often not very skilfully}{(danh từ) ảnh chụp nhanh, không trau chuốt kỹ thuật nhiều}
        \VH{to pose for}{(v) to sit or stand in a particular position in order to be photographed or painted}{(động từ) tạo dáng chụp ảnh}
        \VH{to live one’s life to the fullest}{(idiom) to fully enjoy one’s life}{(thành ngữ) sống hết mình để tận hưởng cuộc sống}
        \VH{to back something up}{(phr.v) to prepare a second copy of something to use if the main one fails or needs extra support}{(cụm động từ) sao lưu dữ liệu}
        \VH{a piece of cake}{(idiom) very easy}{(thành ngữ) dễ ợt}
        \VH{to go ahead}{(phr.v) to begin to do something, especially when somebody has given permission or has expressed doubts or opposition}{(cụm động từ) tiếp tục làm sau khi được cho phép}
        \VH{to violate}{(v) to go against or refuse to obey a law, an agreement, etc}{(động từ) xâm phạm, chống lại}
        \VH{superb}{(adj) excellent quality; very great}{(tính từ) tuyệt vời}
        \VH{to let off steam}{(idiom) relax}{(thành ngữ) xả hơi}
        \VH{the sweltering heat}{(phrase) intense heat}{(cụm từ) nóng nực}
        \VH{azure}{(adj) having the bright blue colour of the sky on a clear day}{(tính từ) màu xanh da trời}
        \VH{heaps of}{(phrase) a lot}{(cụm từ) đông, nhiều}
        \VH{to get a kick out of something}{(idiom) to enjoy something very much}{(thành ngữ) cực thích cái gì}
        \VH{piles of}{(phrase) a lot of something}{(cụm từ) nhiều}
        \VH{to dress up}{(phrase) to put on special clothes in order to change your appearance}{(cụm động từ) ăn mặc đẹp}
        \VH{to enjoy ourselves very much}{(idiom) have fun}{(thành ngữ) rất vui}
        \VH{precipitation}{(n) water that falls from the clouds towards the ground, especially as rain or snow}{(danh từ) lượng mưa}
        \VH{frustrated}{(adj) feeling annoyed or less confident because you cannot achieve what you want}{(tính từ) khó chịu}
        \VH{mist}{(n) thin fog produced by very small drops of water collecting in the air just above an area of ground or water}{(danh từ) sương mù}
        \VH{drizzle}{(n) rain in very small, light drops}{(danh từ) mưa phùn}
        \VH{arthritis}{(n) a disease that causes pain and swelling in one or more joints of the body}{(danh từ) bệnh đau khớp; viêm khớp}
        \VH{at the mercy of}{(phrase) completely in the power or under the control of}{(cụm từ) chịu sự ảnh hưởng của cái gì}
        \VH{incident}{(n) an event or occurrence}{(danh từ) sự cố}
        \VH{jubilant}{(adj) feeling or expressing great happiness and triumph}{(tính từ) hân hoan, vui sướng}
        \VH{to immerse}{(v) involve oneself deeply in a particular activity or interest}{(động từ) đắm chìm vào}
        \VH{to pay tribute to}{(phrase) something that you say, write, or give that shows your respect and admiration for someone, especially on a formal occasion}{(cụm từ) tôn kính, kính trọng}
        \VH{fruitful}{(adj) producing good results}{(tính từ) nhiều thành quả}
        \VH{to safeguard}{(v) to protect something/somebody from loss, harm or damage; to keep something/somebody safe}{(động từ) bảo vệ}
        \VH{a cab}{(n) a taxi}{(danh từ) taxi}
        \VH{filthiness}{(n) the quality of being very dirty or unpleasant}{(danh từ) bẩn thỉu}
        \VH{to bundle up}{(v) to wear enough clothing to keep very warm}{(động từ) mặc thêm nhiều}
        \VH{calibration}{(n) the units of measurement marked on a thermometer or other instrument}{(danh từ) hiệu chuẩn}
        \VH{verification}{(n) the act of showing or checking that something is true or accurate}{(danh từ) kiểm nghiệm}
        \VH{by virtue of}{(phrase) because or as a result of}{(cụm từ) bởi vì}
        \VH{hydrometeorological}{(adj) related to a branch of meteorology and hydrology that studies the transfer of water and energy between the land surface and the lower atmosphere}{(tính từ) thuộc khí tượng thủy văn}
        \VH{leap}{(n) a forceful jump or quick movement}{(danh từ) bước nhảy vọt, bước phát triển}
        \VH{calamity}{(n) an event that causes great damage to people’s lives, property, etc}{(danh từ) thiên tai}
        \VH{landslide}{(n) a mass of earth, rock, etc. that falls down the slope of a mountain or a cliff}{(danh từ) sạt lở đất}
        \VH{topographic}{(adj) connected with the physical features of an area of land, especially the position of its rivers, mountains, etc}{(tính từ) thuộc địa hình}
    \end{VocabHighlights}
    \end{test}

    \begin{test}{TEST 4}
    \noindent
    \textbf{Part 1. Names}
    \begin{qa}{How did your parents choose your name(s)?}
    I \textbf{heard through the grapevine} that my parents finalized my name after \textbf{taking almost every name into consideration}. My name must meet with at least two requirements. Firstly, it must indicate \textbf{masculinity}. Secondly, no one in my genealogy records had \textbf{borne the same name}. Carrying similar names to any of the ancestors in Vietnam is strictly forbidden.
    \end{qa}

    \begin{qa}{Does your name have any special meaning?}
    I don’t know whether English names hold any specific meanings but in Vietnam, parents, including mine, tend to \textbf{establish expectations} for their offspring via naming them. My middle name is “Thanh”, which is equivalent to “Success and Solidity” in English. My name is Son, which means “Mountain”. By combining the two elements, my parents wish I could be a \textbf{tower of strength} in the future.
    \end{qa}

    \begin{qa}{Is your name common or unusual in your country?}
    It is neither an unusual nor a familiar name. At least my parents did not give me a strange name. Having a weird name which sounds unfamiliar may \textbf{do more harm than good} to one. \textbf{To the best of my recollection}, a friend of mine having a funny name, say “Tien Tung” or “Lack of Money” in English, was often \textbf{made sport of} when he was young.
    \end{qa}

    \begin{qa}{If you could change your name, would you? [Why?/Why not?]}
    If my name meant something bad, I would not hesitate to rename myself although I know changing names can exert a lot of serious effects on my life later on. Nevertheless, I’m \textbf{utterly content} with my name. I’m \textbf{immensely grateful} to my parents for choosing a name that has underlying symbolic meanings like this.
    \end{qa}

        \begin{VocabExplain}[Part 1]
            \begin{ExplainCard}{hear through the grapevine}[idiom][C1]
            \EN{to hear news or information informally, often through gossip.}
            \VI{nghe tin đồn, nghe phong thanh.}
            \SY{hear rumors; hear by word of mouth}
            \EX{I heard through the grapevine that she was leaving the company.}
            \EX{Researchers heard through the grapevine about upcoming policy changes.}
            \CO{hear through the grapevine that...}
            \end{ExplainCard}

            \begin{ExplainCard}{take into consideration}[phrase][B2]
            \EN{to think carefully about something before making a decision.}
            \VI{xem xét, cân nhắc.}
            \SY{consider; think about}
            \EX{We must take all factors into consideration.}
            \EX{Judges took into consideration the student’s effort.}
            \CO{take sth into consideration}
            \end{ExplainCard}

            \begin{ExplainCard}{masculinity}[n][C1]
            \EN{qualities traditionally associated with men.}
            \VI{tính nam tính, sự nam tính.}
            \SY{manliness; virility}
            \EX{He tried to prove his masculinity by being tough.}
            \EX{Sociologists often study norms of masculinity in cultures.}
            \CO{toxic masculinity; traditional masculinity}
            \end{ExplainCard}

            \begin{ExplainCard}{borne}[v][C1]
            \EN{past participle of “bear”; to have carried or held something.}
            \VI{mang, chịu, đã mang (trong quá khứ).}
            \SY{carried; held}
            \EX{She had borne the name proudly for years.}
            \EX{No one in the genealogy had borne the same title.}
            \CO{borne the name; borne responsibility}
            \end{ExplainCard}

            \begin{ExplainCard}{establish expectations}[phrase][C1]
            \EN{to set hopes or standards for someone to meet.}
            \VI{đặt ra kỳ vọng.}
            \SY{set standards; impose hopes}
            \EX{Parents often establish expectations for their children’s success.}
            \EX{Teachers establish expectations for academic performance.}
            \CO{establish expectations for sb}
            \end{ExplainCard}

            \begin{ExplainCard}{tower of strength}[idiom][C1]
            \EN{a person who is very strong and supportive in difficult times.}
            \VI{chỗ dựa vững chắc.}
            \SY{pillar; strong support}
            \EX{He was a tower of strength during the crisis.}
            \EX{Parents should be a tower of strength for children.}
            \CO{be a tower of strength to sb}
            \end{ExplainCard}

            \begin{ExplainCard}{do more harm than good}[idiom][C1]
            \EN{to have a worse rather than a better effect.}
            \VI{lợi bất cập hại.}
            \SY{be counterproductive}
            \EX{Arguing may do more harm than good.}
            \EX{Some policies do more harm than good for the poor.}
            \CO{do more harm than good}
            \end{ExplainCard}

            \begin{ExplainCard}{to the best of my recollection}[phrase][C1]
            \EN{as far as I can remember.}
            \VI{theo như tôi nhớ.}
            \SY{as far as I remember; to my memory}
            \EX{To the best of my recollection, we met in 2005.}
            \EX{The witness said, to the best of his recollection, the suspect wore black.}
            \CO{to the best of sb’s recollection}
            \end{ExplainCard}

            \begin{ExplainCard}{make sport of}[idiom][C2]
            \EN{to laugh at someone; to mock or ridicule.}
            \VI{chế giễu, giễu cợt.}
            \SY{mock; ridicule}
            \EX{They made sport of his accent.}
            \EX{In history, outsiders were often made sport of in societies.}
            \CO{make sport of sb}
            \end{ExplainCard}

            \begin{ExplainCard}{utterly content}[phrase][C1]
            \EN{completely satisfied.}
            \VI{hoàn toàn hài lòng.}
            \SY{totally satisfied; fully pleased}
            \EX{She felt utterly content with her life.}
            \EX{Workers reported being utterly content with new policies.}
            \CO{utterly content with sth}
            \end{ExplainCard}

            \begin{ExplainCard}{immensely grateful}[phrase][C1]
            \EN{extremely thankful.}
            \VI{vô cùng biết ơn.}
            \SY{deeply thankful; extremely appreciative}
            \EX{I’m immensely grateful for your help.}
            \EX{Scholars are immensely grateful for funding support.}
            \CO{immensely grateful to sb for sth}
            \end{ExplainCard}
        \end{VocabExplain}

    \noindent
    \textbf{Part 2.}
    \begin{qa}{Describe a TV documentary you watched that was particularly interesting. You should say:}
    \begin{itemize}
        \item What the documentary was about
        \item Why you decided to watch it
        \item What you learnt during the documentary
        \item and explain why the TV documentary was particularly interesting.
    \end{itemize}

    Thank you very much for the opportunity to let me talk about an interesting TV documentary that I have watched. The most intriguing TV documentaries I have ever watched is Truth and I would like to talk about it today. I would like to begin by highlighting the fact that this film was directed and written by Manh Cuong. He is a \textbf{legendary figure} in film-making industry and he \textbf{gains immense prestige worldwide} for making historical films.  

    Speaking about this film, this documentary portrayed the horrifying and \textbf{catastrophic} effects of Vietnam War in 1945. It was awarded the Academy Award for the Best Documentary in 2018. This was approximately a 50-minute black-and-white documentary that I watched \textbf{all of a sudden}. It was the evening time and it was raining \textbf{cats and dogs} outside. I sat on a sofa and was surfing different satellite TV channels. I happened to tune to VTV1, and the title of the film \textbf{captured my attention}. Another reason is that my favorite actor, Ninh Ngoc took the lead role in the film.  

    The documentary was so absorbing, shocking, obsessive that I \textbf{was glued to the screen}. While watching this documentary, I realised the \textbf{brutality} of wars, how people become the victims of war, how a single wrong decision can cost the lives of millions, and how cities and civilisations can be \textbf{wiped out} in few minutes. As soon as I finished watching it, tears started to \textbf{well up} in my eyes. I realised it was worth watching TV documentary and I would recommend it to my close friends. Thanks for listening!
    \end{qa}

        \begin{VocabExplain}[Part 2]
            \begin{ExplainCard}{legendary figure}[phrase][C1]
            \EN{a person who is very famous and admired, often in a particular field.}
            \VI{nhân vật huyền thoại.}
            \SY{icon; celebrated person}
            \EX{Einstein is a legendary figure in physics.}
            \EX{He became a legendary figure in the film industry.}
            \CO{legendary figure in sth}
            \end{ExplainCard}

            \begin{ExplainCard}{gain immense prestige worldwide}[phrase][C1]
            \EN{to achieve great respect and admiration globally.}
            \VI{đạt được uy tín lớn trên toàn thế giới.}
            \SY{earn respect; achieve reputation}
            \EX{The company gained immense prestige worldwide after its innovation.}
            \EX{He gained immense prestige worldwide for his humanitarian work.}
            \CO{gain prestige; immense prestige worldwide}
            \end{ExplainCard}

            \begin{ExplainCard}{catastrophic}[adj][C1]
            \EN{causing a lot of sudden damage or suffering.}
            \VI{thảm khốc, thê thảm.}
            \SY{disastrous; devastating}
            \EX{The flood was catastrophic.}
            \EX{Catastrophic consequences followed the accident.}
            \CO{catastrophic event; catastrophic impact}
            \end{ExplainCard}

            \begin{ExplainCard}{all of a sudden}[idiom][B2]
            \EN{very quickly and unexpectedly.}
            \VI{bất ngờ, đột ngột.}
            \SY{suddenly; abruptly}
            \EX{All of a sudden, the lights went out.}
            \EX{The results changed all of a sudden during the experiment.}
            \CO{all of a sudden}
            \end{ExplainCard}

            \begin{ExplainCard}{rain cats and dogs}[idiom][B2]
            \EN{to rain very heavily.}
            \VI{mưa như trút nước.}
            \SY{pour; rain heavily}
            \EX{It was raining cats and dogs all afternoon.}
            \EX{During the storm, it rained cats and dogs for hours.}
            \CO{it rains cats and dogs}
            \end{ExplainCard}

            \begin{ExplainCard}{capture sb’s attention}[phrase][B2]
            \EN{to make someone notice and be interested in something.}
            \VI{thu hút sự chú ý.}
            \SY{attract; engage; fascinate}
            \EX{The painting captured my attention.}
            \EX{Headlines are designed to capture readers’ attention.}
            \CO{capture the attention of sb}
            \end{ExplainCard}

            \begin{ExplainCard}{be glued to the screen}[idiom][C1]
            \EN{to watch something with great attention without looking away.}
            \VI{chăm chú dán mắt vào màn hình.}
            \SY{watch intently; engrossed}
            \EX{He was glued to the screen during the match.}
            \EX{Children are glued to the screen playing games.}
            \CO{glued to the screen/TV}
            \end{ExplainCard}

            \begin{ExplainCard}{brutality}[n][C1]
            \EN{violent and cruel behaviour.}
            \VI{sự tàn bạo.}
            \SY{cruelty; savagery; violence}
            \EX{The brutality of the attack shocked everyone.}
            \EX{War exposes the brutality of human nature.}
            \CO{acts of brutality; sheer brutality}
            \end{ExplainCard}

            \begin{ExplainCard}{wipe out}[phr.v][C1]
            \EN{to destroy completely.}
            \VI{xóa sổ, hủy diệt.}
            \SY{eradicate; annihilate; eliminate}
            \EX{Whole villages were wiped out by the tsunami.}
            \EX{The disease wiped out large parts of the population.}
            \CO{wipe out completely; be wiped out by}
            \end{ExplainCard}

            \begin{ExplainCard}{well up}[phr.v][C1]
            \EN{(of emotions, especially tears) to build up and start to show.}
            \VI{dâng trào (nước mắt, cảm xúc).}
            \SY{overflow; rise; surge}
            \EX{Tears welled up in her eyes.}
            \EX{Emotions well up when recalling past tragedies.}
            \CO{well up with tears/emotion}
            \end{ExplainCard}
        \end{VocabExplain}

    \noindent
    \textbf{Part 3.}
    \begin{qa}{What are the most popular kinds of TV programmes in your country? Why is this?}
    Given the fact that life is becoming more hectic, \textbf{reality shows} and \textbf{comedy shows} are among the most popular TV programs in my country. This is simply because these programs can serve audiences of all ages. The reality shows documents \textbf{purportedly} \textbf{unscripted} real-life situations, which bring practical experience to the audience. This kind of information might \textbf{pander to} humans’ \textbf{inquisitive} nature as viewers are given a glimpse of lives of those who are somehow \textbf{on par with} themselves. In addition, television comedy entertains people with \textbf{gags} told by famous comedians. Viewers might \textbf{split their sides} after watching these programs, which relieves their stresses after a stressful day at work.
    \end{qa}

    \begin{qa}{Do you think there are too many game shows on TV nowadays? Why?}
    Yes, when it comes to game shows on TV, I am \textbf{spoilt for choice} and I think the number of shows is \textbf{on the rise}. Understandably, the ever-increasing number of shows is to fit the different tastes of audiences. Athletic people are really huge on sports shows to \textbf{root for} their super stars. By contrast, quiz shows can gain tremendous followers being concerned about knowledge \textbf{acquisition}. Another reason that accounts for the rising number of shows is advertisement. The more game shows a TV station airs, the more likely the revenue from advertisements can be gained.
    \end{qa}

    \begin{qa}{Do you think TV is the main way for people to get the news in your country? What other ways are there?}
    Well, a few decades ago, when everything was still \textbf{rudimentary}, watching TV is the main source of information. Most households would possess a black and white TV with a few satellite channels, and the broadcasted programs were \textbf{momentous} events. However, technological advances have \textbf{enlightened} people about the way they absorb information. Cable TV and Internet-connected devices \textbf{predominantly} serve the audience with updated information whenever they want. In summary, the TV is no longer the major source of information in my country.
    \end{qa}

    \begin{qa}{What types of products are advertised most often on TV?}
    I do not often \textbf{slack off}. Instead, I am always \textbf{up to my ears in} work, so I do not normally spare for watching TV, what a pity. Having said that, the only time I watch the telly is dinner time when my family gather and \textbf{savour} the relaxation. At that time, I would say the \textbf{prime time}, there are countless advertisements from cosmetics to medicine. In general, basic necessities such as food, clothes and digital items would \textbf{constitute} a majority of advertisements.
    \end{qa}

    \begin{qa}{Do you think that people pay attention to adverts on TV? Why do you think that?}
    It is undeniable that we are all \textbf{captivated by} TV adverts to some extent, otherwise companies would not \textbf{pay through the nose} to have them shown. To be honest, advertisers are \textbf{accomplished} artists who can change our shopping habits by \textbf{flowery words} and \textbf{exaggeration}. Today, advertisements are as important as other shows, and can be seen as food for the mind. While some adults prefer skipping advertisements, young children are apt to being \textbf{engrossed in} fun and catchy melodies of some commercials and cannot take their eyes off the screen.
    \end{qa}

    \begin{qa}{How important are regulations on TV advertising?}
    It stands to reason that governmental \textbf{supervision} is of paramount importance to regulate TV advertising. For one thing, the number of \textbf{commercial breaks} should be restricted in order not to disturb the concentration of audiences on their programs. Almost everyone would be bothered by ads \textbf{popping up} too often. Secondly, the all \textbf{falsified} information of advertisements needs to be removed before the broadcasting time to protect customers from fraud. Any marketing content can be written to attract viewers as long as it neither distorts the facts nor \textbf{defames} competitors whose products are of the same category in the same market.
    \end{qa}

        \begin{VocabExplain}[Part 3]
            \begin{ExplainCard}{purportedly}[adv][C2]
            \EN{as stated or claimed to be true, though not proven.}
            \VI{được cho là, được tuyên bố là.}
            \SY{allegedly; supposedly}
            \EX{He is purportedly the richest man in town.}
            \EX{The purportedly leaked document was denied by officials.}
            \CO{purportedly true/authorised}
            \end{ExplainCard}

            \begin{ExplainCard}{unscripted}[adj][C1]
            \EN{not written or planned in advance.}
            \VI{không có kịch bản sẵn.}
            \SY{improvised; spontaneous}
            \EX{The actor gave an unscripted speech.}
            \EX{Reality shows are often unscripted.}
            \CO{unscripted remarks; unscripted shows}
            \end{ExplainCard}

            \begin{ExplainCard}{pander to}[phr.v][C2]
            \EN{to do or say what people want, even if it is not good.}
            \VI{chiều theo, nuông chiều.}
            \SY{indulge; gratify}
            \EX{The film panders to popular taste.}
            \EX{Politicians sometimes pander to voters’ fears.}
            \CO{pander to desires/fears/taste}
            \end{ExplainCard}

            \begin{ExplainCard}{inquisitive}[adj][C1]
            \EN{wanting to know many things; curious.}
            \VI{tò mò, hiếu kỳ.}
            \SY{curious; questioning}
            \EX{Children are naturally inquisitive.}
            \EX{An inquisitive student asked about every detail.}
            \CO{inquisitive mind; inquisitive nature}
            \end{ExplainCard}

            \begin{ExplainCard}{on par with}[phrase][C1]
            \EN{equal to something in quality or standard.}
            \VI{ngang bằng với.}
            \SY{equal; equivalent}
            \EX{His skills are on par with professionals.}
            \EX{The service is on par with international standards.}
            \CO{on par with sb/sth}
            \end{ExplainCard}

            \begin{ExplainCard}{gag}[n][C1]
            \EN{a joke or funny story.}
            \VI{truyện cười, câu đùa.}
            \SY{joke; quip}
            \EX{The comedian told some old gags.}
            \EX{Students often laugh at silly gags.}
            \CO{tell a gag; running gag}
            \end{ExplainCard}

            \begin{ExplainCard}{split one’s sides}[idiom][C1]
            \EN{to laugh a lot.}
            \VI{cười vỡ bụng.}
            \SY{laugh heartily; burst out laughing}
            \EX{The audience split their sides at the joke.}
            \EX{He split his sides laughing at the cartoon.}
            \CO{split one’s sides with laughter}
            \end{ExplainCard}

            \begin{ExplainCard}{spoilt for choice}[idiom][C1]
            \EN{having so many options that it is hard to choose.}
            \VI{có quá nhiều lựa chọn.}
            \SY{overwhelmed with options}
            \EX{Consumers are spoilt for choice in supermarkets.}
            \EX{Tourists are spoilt for choice of hotels.}
            \CO{be spoilt for choice}
            \end{ExplainCard}

            \begin{ExplainCard}{on the rise}[phrase][B2]
            \EN{increasing.}
            \VI{đang gia tăng.}
            \SY{increasing; growing}
            \EX{Inflation is on the rise.}
            \EX{The use of AI tools is on the rise in education.}
            \CO{be on the rise}
            \end{ExplainCard}

            \begin{ExplainCard}{root for}[phr.v][B2]
            \EN{to support someone in a competition.}
            \VI{cổ vũ cho.}
            \SY{support; cheer for}
            \EX{Fans rooted for their team.}
            \EX{She rooted for her favourite singer.}
            \CO{root for a team/player}
            \end{ExplainCard}

            \begin{ExplainCard}{acquisition}[n][C1]
            \EN{the act of getting or gaining something.}
            \VI{sự giành được, thu được.}
            \SY{gain; attainment; obtaining}
            \EX{Language acquisition takes time.}
            \EX{The company completed the acquisition of a rival.}
            \CO{knowledge acquisition; business acquisition}
            \end{ExplainCard}

            \begin{ExplainCard}{rudimentary}[adj][C1]
            \EN{basic, simple, not developed.}
            \VI{thô sơ, sơ đẳng.}
            \SY{basic; elementary}
            \EX{He had only rudimentary knowledge of math.}
            \EX{Villagers lived with rudimentary tools.}
            \CO{rudimentary skills; rudimentary system}
            \end{ExplainCard}

            \begin{ExplainCard}{momentous}[adj][C1]
            \EN{very important, especially in its effects.}
            \VI{trọng đại, quan trọng.}
            \SY{significant; historic}
            \EX{It was a momentous event in history.}
            \EX{The discovery was momentous for science.}
            \CO{momentous occasion; momentous decision}
            \end{ExplainCard}

            \begin{ExplainCard}{predominantly}[adv][C1]
            \EN{mainly; mostly.}
            \VI{phần lớn, chủ yếu.}
            \SY{mainly; primarily}
            \EX{The region is predominantly rural.}
            \EX{The festival is predominantly attended by young people.}
            \CO{predominantly male/female; predominantly rural/urban}
            \end{ExplainCard}

            \begin{ExplainCard}{slack off}[phr.v][C1]
            \EN{to work less hard than usual.}
            \VI{lười biếng, chểnh mảng.}
            \SY{loaf; be idle}
            \EX{Workers tend to slack off on Fridays.}
            \EX{Students slack off before holidays.}
            \CO{slack off at work/school}
            \end{ExplainCard}

            \begin{ExplainCard}{up to one’s ears in}[idiom][C1]
            \EN{extremely busy with something.}
            \VI{bận ngập đầu.}
            \SY{overwhelmed; swamped}
            \EX{I’m up to my ears in deadlines.}
            \EX{She was up to her ears in work.}
            \CO{up to one’s ears in sth}
            \end{ExplainCard}

            \begin{ExplainCard}{prime time}[n][B2]
            \EN{the time when the largest number of people are watching TV.}
            \VI{giờ vàng.}
            \SY{peak viewing time}
            \EX{The show is broadcast in prime time.}
            \EX{Prime time ads are more expensive.}
            \CO{prime time slot; prime time show}
            \end{ExplainCard}

            \begin{ExplainCard}{constitute}[v][C1]
            \EN{to form or make up something.}
            \VI{cấu thành, chiếm.}
            \SY{make up; comprise}
            \EX{Women constitute 50\% of the workforce.}
            \EX{Data breaches constitute a major risk.}
            \CO{constitute a majority; constitute a problem}
            \end{ExplainCard}

            \begin{ExplainCard}{captivated by}[phrase][C1]
            \EN{very interested in or attracted by.}
            \VI{bị cuốn hút, say mê.}
            \SY{fascinated by; enthralled by}
            \EX{The audience was captivated by her voice.}
            \EX{Readers are captivated by the thrilling story.}
            \CO{captivated by sth/sb}
            \end{ExplainCard}

            \begin{ExplainCard}{pay through the nose}[idiom][C1]
            \EN{to pay too much for something.}
            \VI{trả giá đắt.}
            \SY{overpay; spend excessively}
            \EX{We paid through the nose for concert tickets.}
            \EX{Businesses pay through the nose for prime ads.}
            \CO{pay through the nose for sth}
            \end{ExplainCard}

            \begin{ExplainCard}{accomplished}[adj][C1]
            \EN{skilled and successful at doing something.}
            \VI{giỏi, tài năng.}
            \SY{skilled; proficient}
            \EX{She is an accomplished pianist.}
            \EX{He is an accomplished writer.}
            \CO{accomplished artist/musician}
            \end{ExplainCard}

            \begin{ExplainCard}{flowery words}[phrase][C2]
            \EN{words that are very elaborate and decorative, often more than necessary.}
            \VI{lời văn hoa mỹ.}
            \SY{ornate language; grandiloquent words}
            \EX{He impressed her with flowery words.}
            \EX{Flowery words are often used in political speeches.}
            \CO{use flowery words}
            \end{ExplainCard}

            \begin{ExplainCard}{exaggeration}[n][B2]
            \EN{a statement that makes something seem better or worse than it really is.}
            \VI{phóng đại.}
            \SY{overstatement; hyperbole}
            \EX{It’s not an exaggeration to say the film was a success.}
            \EX{Politicians often rely on exaggeration.}
            \CO{gross exaggeration; wild exaggeration}
            \end{ExplainCard}

            \begin{ExplainCard}{engrossed in}[phrase][C1]
            \EN{giving all your attention to something.}
            \VI{mải mê, chăm chú.}
            \SY{absorbed in; preoccupied with}
            \EX{He was engrossed in a book.}
            \EX{Children were engrossed in watching cartoons.}
            \CO{engrossed in sth}
            \end{ExplainCard}

            \begin{ExplainCard}{supervision}[n][B2]
            \EN{the act of managing or directing people.}
            \VI{sự giám sát.}
            \SY{oversight; monitoring}
            \EX{The work is done under strict supervision.}
            \EX{Teachers provide supervision during exams.}
            \CO{strict supervision; supervision of sth}
            \end{ExplainCard}

            \begin{ExplainCard}{commercial break}[n][B2]
            \EN{a short interruption in a TV program for adverts.}
            \VI{quảng cáo chen ngang.}
            \SY{ad break; interval}
            \EX{The movie was interrupted by commercial breaks.}
            \EX{Commercial breaks generate huge revenue for TV.}
            \CO{during the commercial break}
            \end{ExplainCard}

            \begin{ExplainCard}{pop up}[phr.v][B2]
            \EN{to appear suddenly or unexpectedly.}
            \VI{xuất hiện bất ngờ.}
            \SY{appear; arise}
            \EX{Problems keep popping up at work.}
            \EX{Ads pop up on my screen constantly.}
            \CO{pop up suddenly; pop up too often}
            \end{ExplainCard}

            \begin{ExplainCard}{falsified}[adj][C1]
            \EN{changed to make people believe something that is not true.}
            \VI{bị làm giả, xuyên tạc.}
            \SY{forged; fake}
            \EX{He was accused of using falsified documents.}
            \EX{Falsified information misleads customers.}
            \CO{falsified data; falsified report}
            \end{ExplainCard}

            \begin{ExplainCard}{defame}[v][C2]
            \EN{to damage someone’s reputation by saying untrue things.}
            \VI{bôi nhọ, phỉ báng.}
            \SY{slander; libel}
            \EX{The article defamed the politician.}
            \EX{He sued the magazine for defaming him.}
            \CO{defame sb; defame publicly}
            \end{ExplainCard}
        \end{VocabExplain}

    \begin{VocabHighlights}
        \VH{to hear through the grapevine}{(idiom) to hear or learn of something through an informal means of communication, especially gossip}{(thành ngữ) nghe đồn}
        \VH{to take something into consideration}{(idiom) to take something into account}{(thành ngữ) cân nhắc}
        \VH{masculinity}{(n) the quality of being masculine}{(danh từ) sự nam tính}
        \VH{to bear the same name}{(phrase) to have a similar name to somebody}{(cụm từ) trùng tên ai}
        \VH{to establish expectations}{(phrase) to show how you expect}{(cụm từ) thể hiện sự mong mỏi}
        \VH{a tower of strength}{(idiom) a person that you can rely on to help, protect and comfort you when you are in trouble}{(thành ngữ) 1 người đáng tin cậy, có thể gánh vác mọi việc khi khó khăn}
        \VH{to do more harm than good}{(idiom) inadvertently make a situation worse rather than better}{(thành ngữ) gây hại nhiều hơn lợi}
        \VH{to the best of my recollection}{(phrase) from what my memory tells me}{(cụm từ) từ những gì tôi nhớ}
        \VH{to make sport of}{(idiom) to make fun of}{(thành ngữ) trêu, chọc, chế nhạo}
        \VH{utterly}{(adv) completely}{(trạng từ) hoàn toàn}
        \VH{to be content with}{(adj) happy, satisfied with what you have}{(tính từ) mãn nguyện về}
        \VH{to be immensely grateful to}{(phrase) feeling or showing deep gratitude because somebody has done something good for you}{(cụm từ) vô cùng cảm ơn}
        \VH{legendary figure}{(phrase) a legendary character}{(cụm từ) nhân vật huyền thoại}
        \VH{to gain immense prestige worldwide}{(phrase) to become famous internationally}{(cụm từ) trở nên nổi tiếng toàn cầu}
        \VH{catastrophic}{(adj) causing very great trouble or destruction}{(tính từ) thảm khốc}
        \VH{all of a sudden}{(idiom) quickly and without warning}{(thành ngữ) đột nhiên}
        \VH{to rain cats and dogs}{(idiom) to rain very heavily}{(thành ngữ) mưa nặng hạt}
        \VH{to happen to V-inf}{(v) accidentally do something}{(động từ) tình cờ làm gì}
        \VH{to capture somebody’s attention}{(phrase) to attract somebody’s attention}{(cụm từ) thu hút sự chú ý của ai}
        \VH{be glued to the screen}{(phrase) to be unable to stop watching something}{(cụm từ) dán mắt vào cái gì đó}
        \VH{brutality}{(n) behaviour that is very cruel or violent and showing no feelings for others}{(danh từ) sự khốc liệt, tàn bạo}
        \VH{to wipe out}{(phr.v) destroy completely}{(cụm động từ) phá hủy sạch}
        \VH{to well up}{(phr.v) to gradually or steadily flow upwards or outwards}{(cụm động từ) trào ra}
        \VH{reality show}{(phrase) a television program in which ordinary people are continuously filmed, designed to be entertaining rather than informative}{(cụm từ) chương trình truyền hình thực tế}
        \VH{comedy show}{(phrase) a (type of) film, play, or book that is intentionally funny}{(cụm từ) chương trình hài giải trí}
        \VH{purportedly}{(adv) used to say that something has been stated to have happened or to be true, but this might not be the case}{(trạng từ) cố ý}
        \VH{unscripted}{(adj) (of a speech, broadcast, etc.) not written or prepared in detail in advance}{(tính từ) không được chuẩn bị trước kịch bản}
        \VH{to pander to}{(phr.v) to gratify or indulge}{(cụm động từ) chiều theo}
        \VH{to be on par with}{(idiom) equal or similar to someone or something}{(thành ngữ) tương đương}
        \VH{gag}{(n) a joke or a funny story, especially one told by a professional comedian}{(danh từ) chuyện cười}
        \VH{to split somebody’s sides}{(idiom) to laugh uproariously or hysterically}{(thành ngữ) cười như nắc nẻ}
        \VH{to be spoiled/spoilt for choice}{(idiom) to be unable to choose because there are so many possible good choices}{(thành ngữ) có quá nhiều sự lựa chọn}
        \VH{on the rise}{(phrase) becoming greater or more numerous; increasing}{(cụm từ) đang tăng lên}
        \VH{huge on}{(v) to like something a lot}{(động từ) rất thích}
        \VH{root for}{(v) support or hope for the success of (a person or group entering a contest or undertaking a challenge)}{(động từ) cổ vũ, ủng hộ}
        \VH{acquisition}{(n) the act of getting something, especially knowledge, a skill, etc}{(danh từ) sự học tập; tiếp nhận}
        \VH{rudimentary}{(adj) not highly or fully developed}{(tính từ) thô sơ}
        \VH{momentous}{(adj) very important or serious, especially because there may be important results}{(tính từ) quan trọng}
        \VH{enlighten}{(v) having or showing an understanding of people’s needs, a situation, etc. that is not based on old-fashioned attitudes and prejudice}{(động từ) giác ngộ}
        \VH{predominantly}{(adv) mostly, mainly}{(trạng từ) chủ yếu}
        \VH{slack off}{(v) to do something with less effort or energy than before}{(động từ) lười biếng}
        \VH{up to somebody’s ears in}{(idiom) very busy}{(thành ngữ) bận ngập đầu}
        \VH{savour}{(v) to enjoy the full taste or flavour of something, especially by eating or drinking it slowly}{(động từ) thưởng thức}
        \VH{prime time}{(phrase) the regularly occurring time at which a television or radio audience is expected to be greatest}{(cụm từ) khung giờ vàng phát sóng}
        \VH{constitute}{(v) to be considered to be something}{(động từ) cấu tạo, tạo thành}
        \VH{to captivate}{(v) to keep somebody’s attention by being interesting, attractive, etc}{(động từ) làm say mê; quyến rũ}
        \VH{to pay through the nose for something}{(idiom) to pay so much for something}{(thành ngữ) trả quá nhiều tiền cho cái gì}
        \VH{accomplished}{(adj) very good at a particular thing; having a lot of skills}{(tính từ) giỏi giang, thành thạo}
        \VH{flowery words}{(phrase) too complicated; not expressed in a clear and simple word}{(cụm từ) những từ hoa mỹ}
        \VH{exaggeration}{(n) a statement or description that makes something seem larger, better, worse or more important than it really is; the act of making a statement like this}{(danh từ) phóng đại}
        \VH{to be engrossed in}{(p2) giving all your attention to something}{(phân từ 2) bị chìm đắm, mê mẩn gì}
        \VH{supervision}{(n) the work or activity involved in being in charge of somebody/something and making sure that everything is done correctly, safely, etc}{(danh từ) sự giám sát}
        \VH{commercial break}{(phrase) an interruption in the transmission of broadcast programming during which advertisements are broadas}{(cụm từ) quảng cáo giữa các chương trình}
        \VH{to pop up}{(phr.v) to appear or occur suddenly and unexpectedly}{(cụm động từ) xuất hiện bất thình lình}
        \VH{to defame}{(v) damage the good reputation of (someone); slander or libel}{(động từ) bôi xấu, làm hạ uy tín của ai}
    \end{VocabHighlights}
    \end{test}
\end{glossarymc}