\begin{glossarymc}[Cambridge 12]
    \begin{test}{TEST 1}
    \noindent
    \textbf{Part 1. Health}
    \begin{qa}{Is it important to you to eat healthy food? [Why?/Why not?]}
    It is crucial for one to have \textbf{nourishing} meals. In other words, absorbing nutritious foods is \textbf{essential} if one wants to \textbf{keep fit} and stay healthy. There is a proverb like ‘\textbf{You are what you eat}’ which highlights the importance of eating the right food. Maintaining a high level of \textbf{fitness} would be impossible were we not to consume \textbf{healthful} foods.
    \end{qa}

    \begin{qa}{If you catch a cold, what do you do to help you feel better? [Why?]}
    In that case, I’ll \textbf{lie down} and \textbf{take a nap}. Then, I may ask for the advice of my neighbor who works as a doctor. If I am \textbf{dead beat}, I might ask my wife to drop by a pharmacist to get me some medicine. If I am still strong enough, I will \textbf{fend for myself}.
    \end{qa}

    \begin{qa}{Do you pay attention to public information about health? [Why?/Why not?]}
    Yes, I do. Taking notice of such information also raises my awareness of protecting my health. Every day, when I am on the way to work, I can see a big sign board on the pavement that says “Good health is gold”, and “A good health is above wealth” in English.
    \end{qa}

    \begin{qa}{What could you do to have a healthier lifestyle?}
    I need to change my lifestyle a little bit. Instead of \textbf{hitting the hay} at midnight as usual, I need to \textbf{sack out} earlier, say 10 p.m or so. That will make it easier for me to \textbf{roll out of bed} earlier than I often do. “Early to bed and early to rise makes a man healthy, wealthy and wise”. I guess I might experience memory loss as I’ve been quite \textbf{absent-minded} these days.
    \end{qa}

        \begin{VocabExplain}[Part 1]
            \begin{ExplainCard}{nourishing}[adj][C1]
            \EN{containing substances needed to live, grow, and stay healthy.}
            \VI{bổ dưỡng.}
            \SY{nutritious; healthy}
            \EX{A nourishing breakfast helps you start the day well.}
            \EX{Nourishing food is essential for children’s development.}
            \CO{nourishing meal; nourishing food}
            \end{ExplainCard}

            \begin{ExplainCard}{essential}[adj][B2]
            \EN{completely necessary; extremely important.}
            \VI{thiết yếu.}
            \SY{vital; crucial}
            \EX{Water is essential for life.}
            \EX{It is essential to wear a helmet when cycling.}
            \CO{essential need; absolutely essential}
            \end{ExplainCard}

            \begin{ExplainCard}{keep fit}[phrase][B1]
            \EN{to stay healthy and in good physical condition.}
            \VI{giữ dáng, giữ sức khỏe.}
            \SY{stay healthy; stay in shape}
            \EX{She runs to keep fit.}
            \EX{Keeping fit reduces health risks.}
            \CO{exercise to keep fit}
            \end{ExplainCard}

            \begin{ExplainCard}{You are what you eat}[proverb][C1]
            \EN{your health and body condition reflect the food you eat.}
            \VI{ăn gì bổ nấy (sức khỏe phản ánh chế độ ăn uống).}
            \SY{diet defines health}
            \EX{Doctors often remind patients: You are what you eat.}
            \EX{The proverb “You are what you eat” emphasizes good nutrition.}
            \CO{proverb: You are what you eat}
            \end{ExplainCard}

            \begin{ExplainCard}{healthful}[adj][C1]
            \EN{helping to produce good health.}
            \VI{có lợi cho sức khỏe.}
            \SY{wholesome; beneficial}
            \EX{A healthful environment is necessary for well-being.}
            \EX{They adopted a healthful lifestyle with regular exercise.}
            \CO{healthful diet; healthful lifestyle}
            \end{ExplainCard}

            \begin{ExplainCard}{lie down}[phr.v][B1]
            \EN{to put your body flat on a bed or ground to rest.}
            \VI{nằm xuống nghỉ ngơi.}
            \SY{recline; rest}
            \EX{He lay down for a short while.}
            \EX{Patients are asked to lie down after treatment.}
            \CO{lie down on the bed/sofa}
            \end{ExplainCard}

            \begin{ExplainCard}{take a nap}[phrase][A2]
            \EN{to sleep for a short time, especially during the day.}
            \VI{ngủ trưa, chợp mắt.}
            \SY{doze; short sleep}
            \EX{I often take a nap after lunch.}
            \EX{Research shows naps improve memory.}
            \CO{take a short nap}
            \end{ExplainCard}

            \begin{ExplainCard}{dead beat}[adj][C1]
            \EN{extremely tired.}
            \VI{kiệt sức.}
            \SY{exhausted; worn out}
            \EX{I’m dead beat after the long trip.}
            \EX{He was dead beat following the exam.}
            \CO{feel dead beat; be dead beat}
            \end{ExplainCard}

            \begin{ExplainCard}{fend for myself}[idiom][C1]
            \EN{to take care of yourself without help.}
            \VI{tự lo liệu cho bản thân.}
            \SY{look after oneself; be independent}
            \EX{Children must learn to fend for themselves.}
            \EX{He fended for himself after moving abroad.}
            \CO{fend for myself/himself}
            \end{ExplainCard}

            \begin{ExplainCard}{hit the hay}[idiom][B2]
            \EN{to go to bed.}
            \VI{đi ngủ.}
            \SY{go to bed; turn in}
            \EX{I’m tired; I’ll hit the hay early tonight.}
            \EX{Farmers usually hit the hay before midnight.}
            \CO{hit the hay early/late}
            \end{ExplainCard}

            \begin{ExplainCard}{sack out}[phr.v][C1]
            \EN{to fall asleep, usually quickly.}
            \VI{ngủ thiếp đi.}
            \SY{doze off; crash}
            \EX{He sacked out on the sofa.}
            \EX{Kids often sack out after playing.}
            \CO{sack out on the couch/bed}
            \end{ExplainCard}

            \begin{ExplainCard}{roll out of bed}[idiom][B2]
            \EN{to get out of bed, often with difficulty.}
            \VI{ra khỏi giường.}
            \SY{get up; crawl out of bed}
            \EX{I rolled out of bed at noon.}
            \EX{Scientists often roll out of bed early for experiments.}
            \CO{roll out of bed early/late}
            \end{ExplainCard}

            \begin{ExplainCard}{absent-minded}[adj][C1]
            \EN{forgetful, often lost in thought.}
            \VI{đãng trí.}
            \SY{forgetful; distracted}
            \EX{He became absent-minded after retirement.}
            \EX{Students can be absent-minded under stress.}
            \CO{absent-minded professor; quite absent-minded}
            \end{ExplainCard}
        \end{VocabExplain}

    \noindent
    \textbf{Part 2.}
    \begin{qa}{Describe an occasion when you had to wait a long time for someone or something to arrive. You should say:}
    \begin{itemize}
        \item Who or what you were waiting for
        \item How long you had to wait
        \item Why you had to wait a long time
        \item and explain how you felt about waiting a long time
    \end{itemize}

    In life, there are times when what one desires to have does not arrive at once. In my life, there have been several occasions on which I have to wait \textbf{for donkey’s years} for something to \textbf{come into reality}. In that sense, the most significant event that comes to my mind is the \textbf{once-in-a-lifetime} opportunity I managed to seize to attend the live concert of my favorite rock band, X Japan.  

    In fact, though this band is not \textbf{all the go} in Vietnam, they are still iconic legends in Asia. I first overheard a song of theirs \textbf{at random} back in 2004 and I \textbf{took an immediate liking to} this band. Since the first time I \textbf{was mesmerized by} their catchy songs, I had to wait for 7 years to attend their live concert held in Bangkok, Thailand in 2011. Truth be told, they disbanded in 1997 and it took them 11 years to reunite and start their \textbf{first tour} in Japan again in 2008.  

    It was not until 2011 that they \textbf{kicked off} their world tour and I \textbf{jumped at the chance} to see my idols \textbf{in their presence} in Thailand. Travelling to this country is always a \textbf{walk in the park} as I did not need to apply for a visa, let alone the airfares is \textbf{dirt cheap}. That day, November 8th 2011, finally came, though their show was supposedly staged at 8 p.m, I arrived at the venue, Impact Arena, at around 3 p.m to join the queue.  

    Waiting for 5 hours \textbf{on end} did not matter and when their show started, I could not \textbf{hold back} my tears filled with excitement. It was like I was \textbf{over the moon} when listening to their melody and watching them right in front of my eyes. The memory of that day is still vivid as I’m describing now and this event will surely remain unforgettable in my mind.
    \end{qa}

        \begin{VocabExplain}[Part 2]
            \begin{ExplainCard}{for donkey’s years}[idiom][C1]
            \EN{a very long time.}
            \VI{một khoảng thời gian rất dài.}
            \SY{for ages; for eternity}
            \EX{I haven’t seen him for donkey’s years.}
            \EX{That law has existed for donkey’s years.}
            \CO{wait for donkey’s years; last for donkey’s years}
            \end{ExplainCard}

            \begin{ExplainCard}{come into reality}[phrase][B2]
            \EN{to become true or real.}
            \VI{trở thành hiện thực.}
            \SY{come true; materialize}
            \EX{Her dream finally came into reality.}
            \EX{Scientific theories can come into reality through experiments.}
            \CO{come into reality eventually/finally}
            \end{ExplainCard}

            \begin{ExplainCard}{once-in-a-lifetime}[adj][C1]
            \EN{very special, not likely to happen again.}
            \VI{cơ hội có một không hai.}
            \SY{unique; rare}
            \EX{Winning the prize was a once-in-a-lifetime experience.}
            \EX{The trip offers a once-in-a-lifetime adventure.}
            \CO{once-in-a-lifetime opportunity/experience}
            \end{ExplainCard}

            \begin{ExplainCard}{at random}[phrase][B2]
            \EN{without a plan or method; by chance.}
            \VI{ngẫu nhiên.}
            \SY{by chance; randomly}
            \EX{Names were chosen at random.}
            \EX{The sample was selected at random.}
            \CO{pick at random; choose at random}
            \end{ExplainCard}

            \begin{ExplainCard}{take an immediate liking to}[idiom][C1]
            \EN{to like someone or something as soon as you first see or meet them.}
            \VI{lập tức thích ngay từ đầu.}
            \SY{be attracted to; instantly like}
            \EX{She took an immediate liking to the new student.}
            \EX{I took an immediate liking to this book.}
            \CO{take an immediate liking to sb/sth}
            \end{ExplainCard}

            \begin{ExplainCard}{be mesmerized by}[phrase][C1]
            \EN{to be completely attracted or fascinated by something.}
            \VI{bị cuốn hút, mê hoặc.}
            \SY{fascinated by; captivated by}
            \EX{The audience was mesmerized by her voice.}
            \EX{Children are mesmerized by magic shows.}
            \CO{mesmerized by music/beauty/performance}
            \end{ExplainCard}

            \begin{ExplainCard}{kick off}[phr.v][B2]
            \EN{to start an event or activity.}
            \VI{bắt đầu, khai mạc.}
            \SY{begin; launch}
            \EX{The festival kicked off last week.}
            \EX{The company kicked off the campaign with a party.}
            \CO{kick off a tour/campaign/event}
            \end{ExplainCard}

            \begin{ExplainCard}{jump at the chance}[idiom][C1]
            \EN{to accept an opportunity eagerly.}
            \VI{chộp lấy cơ hội.}
            \SY{seize the chance; grasp the opportunity}
            \EX{She jumped at the chance to study abroad.}
            \EX{He jumped at the chance to meet his idol.}
            \CO{jump at the chance/opportunity}
            \end{ExplainCard}

            \begin{ExplainCard}{in one’s presence}[phrase][B2]
            \EN{while someone is there with you.}
            \VI{trước mặt, khi có mặt ai đó.}
            \SY{in front of; before}
            \EX{She was nervous in his presence.}
            \EX{The truth was revealed in the presence of witnesses.}
            \CO{in the presence of sb}
            \end{ExplainCard}

            \begin{ExplainCard}{a walk in the park}[idiom][C1]
            \EN{something very easy to do.}
            \VI{chuyện dễ như ăn bánh.}
            \SY{piece of cake; effortless}
            \EX{The test was a walk in the park.}
            \EX{For him, public speaking is a walk in the park.}
            \CO{be a walk in the park}
            \end{ExplainCard}

            \begin{ExplainCard}{dirt cheap}[idiom][B2]
            \EN{extremely cheap.}
            \VI{rẻ mạt.}
            \SY{very cheap; bargain}
            \EX{The tickets were dirt cheap.}
            \EX{Food here is dirt cheap compared to the city.}
            \CO{be dirt cheap}
            \end{ExplainCard}

            \begin{ExplainCard}{on end}[phrase][C1]
            \EN{continuously, without stopping.}
            \VI{liên tục.}
            \SY{continuously; endlessly}
            \EX{She worked for hours on end.}
            \EX{It rained for days on end.}
            \CO{for hours/days on end}
            \end{ExplainCard}

            \begin{ExplainCard}{hold back}[phr.v][B2]
            \EN{to stop yourself from showing emotion.}
            \VI{kìm nén.}
            \SY{restrain; suppress}
            \EX{She held back her tears.}
            \EX{He held back his anger.}
            \CO{hold back tears/emotions}
            \end{ExplainCard}

            \begin{ExplainCard}{over the moon}[idiom][B2]
            \EN{extremely happy.}
            \VI{cực kỳ sung sướng.}
            \SY{delighted; thrilled}
            \EX{She was over the moon about her exam results.}
            \EX{Fans were over the moon when their team won.}
            \CO{be over the moon about sth}
            \end{ExplainCard}
        \end{VocabExplain}

    \noindent
    \textbf{Part 3.}
    \begin{qa}{In what kinds of situations should people always arrive early?}
    In my opinion, \textbf{punctuality} is one of the most important \textbf{attributes} that people should have. Even though there are people, especially youngsters who miss the point of the meaning of punctuality, I believe people must not be \textbf{tardy} in terms of special occasions such as job interviews or important conferences. I would \textbf{be red in the face} if I was late on these occasions except for emergency cases.
    \end{qa}

    \begin{qa}{How important it is to arrive early in your country?}
    There are certain benefits of showing up on time whenever you have an appointment in my country or anywhere in the world. First and foremost, being punctual is a demonstration of great respect to your partners, especially when it comes to business meetings. It is socially unacceptable for people who usually turn up late with \textbf{excuses} as this might result in the \textbf{breakdown} in relationships. Besides, arriving early can give me extra time for better preparation, such as to make sure I look \textbf{sharp} enough.
    \end{qa}

    \begin{qa}{How can modern technology help people to arrive early?}
    Nowadays, modern technology has facilitated the way people organize events in a \textbf{blink of an eye}. There are plenty of apps which allow them to draw up a daily schedule and set a \textbf{reminder} to each appointment such as the default calendar app on smart devices. More importantly, the integration of digital map and in-vehicle positioning like Google maps, HERE maps, etc. are \textbf{life-savers} since they can \textbf{steer} direction and keep users away from \textbf{bumper-to-bumper} traffic.
    \end{qa}

    \begin{qa}{What kinds of job require the most patience?}
    For the most part, keeping calm is an important quality to everybody irrespective of their job. In that sense, call center representatives are those who always have to \textbf{keep cool} when answering the phone and consulting the customers whose \textbf{insatiable} demands need to be \textbf{fulfilled} outright via the telephone. And, I could not forget to mention \textbf{surgeons} who are under \textbf{ceaseless} stress but cannot \textbf{lose their temper} when performing an operation as it will definitely do harm to the patient.
    \end{qa}

    \begin{qa}{Is it always better to be patient in work (or studies)?}
    Obviously, patience is key to success in life and I do not see the point of \textbf{getting all riled up} for the things that are out of my control. Being patient shows your \textbf{respectful} and thoughtful attitudes to others, and to avoid minor \textbf{disputes} or \textbf{squabbles}. However, if somebody \textbf{pulls the wool over my eyes} about something important, this would easily trigger my impatience.
    \end{qa}

    \begin{qa}{Do you agree or disagree that the older people are, the more patient they are?}
    In my opinion, people tend to lose patience regardless of their age. For example, many people become impatient due to physical factors such as hunger or \textbf{fatigue}. In general, people complain when there is a delay or something annoying happens. I feel both young and old people express their anger either verbally or their body language shows how tense and upset they are. Impatient people are often seen as arrogant and \textbf{impulsive}. In order to be patient, people need to work out the causes of being impatient and find ways to practice this \textbf{virtue}.
    \end{qa}

        \begin{VocabExplain}[Part 3]
            \begin{ExplainCard}{punctuality}[n][C1]
            \EN{the habit of arriving on time.}
            \VI{tính đúng giờ.}
            \SY{timeliness; promptness}
            \EX{Punctuality is important in business.}
            \EX{The company values employees’ punctuality.}
            \CO{show punctuality; value punctuality}
            \end{ExplainCard}

            \begin{ExplainCard}{attribute}[n][B2]
            \EN{a quality or characteristic that someone has.}
            \VI{phẩm chất, đặc điểm.}
            \SY{trait; quality; feature}
            \EX{Honesty is an essential attribute for a leader.}
            \EX{Patience is a positive attribute in teachers.}
            \CO{personal attribute; important attribute}
            \end{ExplainCard}

            \begin{ExplainCard}{tardy}[adj][C1]
            \EN{arriving late or delayed.}
            \VI{chậm trễ.}
            \SY{late; delayed}
            \EX{Students were punished for being tardy.}
            \EX{She gave a tardy reply to the invitation.}
            \CO{tardy response; tardy arrival}
            \end{ExplainCard}

            \begin{ExplainCard}{be red in the face}[idiom][C1]
            \EN{to feel embarrassed or ashamed.}
            \VI{đỏ mặt vì xấu hổ.}
            \SY{embarrassed; ashamed}
            \EX{He was red in the face after making the mistake.}
            \EX{I was red in the face when I arrived late.}
            \CO{be red in the face with embarrassment}
            \end{ExplainCard}

            \begin{ExplainCard}{breakdown}[n][B2]
            \EN{a failure in communication, system, or relationship.}
            \VI{sự sụp đổ, tan vỡ.}
            \SY{collapse; failure}
            \EX{Misunderstandings caused a breakdown in communication.}
            \EX{Stress can lead to a mental breakdown.}
            \CO{breakdown in sth; complete breakdown}
            \end{ExplainCard}

            \begin{ExplainCard}{sharp}[adj][B2]
            \EN{looking neat and stylish.}
            \VI{bảnh bao, gọn gàng.}
            \SY{stylish; neat}
            \EX{He looked sharp in his new suit.}
            \EX{You need to dress sharp for the interview.}
            \CO{look sharp; dress sharp}
            \end{ExplainCard}

            \begin{ExplainCard}{blink of an eye}[idiom][C1]
            \EN{very quickly, almost instantly.}
            \VI{trong nháy mắt.}
            \SY{instantly; immediately}
            \EX{The accident happened in a blink of an eye.}
            \EX{She finished the task in a blink of an eye.}
            \CO{in a blink of an eye}
            \end{ExplainCard}

            \begin{ExplainCard}{reminder}[n][B1]
            \EN{something that helps you to remember something.}
            \VI{lời nhắc.}
            \SY{prompt; cue}
            \EX{He set a reminder on his phone.}
            \EX{The teacher’s note was a reminder of the deadline.}
            \CO{set a reminder; gentle reminder}
            \end{ExplainCard}

            \begin{ExplainCard}{life-saver}[n][B2]
            \EN{something that helps you in a difficult situation.}
            \VI{cứu cánh.}
            \SY{rescue; blessing}
            \EX{This guidebook was a real life-saver on our trip.}
            \EX{Mobile maps are life-savers for drivers.}
            \CO{be a life-saver}
            \end{ExplainCard}

            \begin{ExplainCard}{bumper-to-bumper}[adj][C1]
            \EN{used to describe heavy traffic where cars are very close together.}
            \VI{kẹt xe nối đuôi.}
            \SY{congested; jammed}
            \EX{We were stuck in bumper-to-bumper traffic.}
            \EX{The highway is bumper-to-bumper every morning.}
            \CO{bumper-to-bumper traffic}
            \end{ExplainCard}

            \begin{ExplainCard}{keep cool}[phrase][B2]
            \EN{to remain calm even in difficult situations.}
            \VI{giữ bình tĩnh.}
            \SY{stay calm; remain composed}
            \EX{He managed to keep cool under pressure.}
            \EX{Doctors must keep cool during surgery.}
            \CO{keep cool under pressure}
            \end{ExplainCard}

            \begin{ExplainCard}{insatiable}[adj][C2]
            \EN{always wanting more; not able to be satisfied.}
            \VI{không bao giờ thỏa mãn.}
            \SY{unquenchable; greedy}
            \EX{He has an insatiable curiosity.}
            \EX{There is insatiable demand for luxury goods.}
            \CO{insatiable hunger; insatiable demand}
            \end{ExplainCard}

            \begin{ExplainCard}{fulfilled}[adj][C1]
            \EN{having all your needs satisfied; completed.}
            \VI{được hoàn thành, thỏa mãn.}
            \SY{satisfied; achieved}
            \EX{She felt fulfilled after reaching her goals.}
            \EX{Orders were fulfilled on time.}
            \CO{fulfilled order; fulfilled life}
            \end{ExplainCard}

            \begin{ExplainCard}{ceaseless}[adj][C2]
            \EN{never ending; constant.}
            \VI{không ngừng nghỉ.}
            \SY{constant; unending}
            \EX{He worked with ceaseless energy.}
            \EX{Ceaseless rain flooded the city.}
            \CO{ceaseless efforts; ceaseless stress}
            \end{ExplainCard}

            \begin{ExplainCard}{lose one’s temper}[idiom][B2]
            \EN{to suddenly become angry.}
            \VI{mất bình tĩnh.}
            \SY{get angry; blow up}
            \EX{He lost his temper when the kids were noisy.}
            \EX{Surgeons cannot afford to lose their temper.}
            \CO{lose temper easily}
            \end{ExplainCard}

            \begin{ExplainCard}{get all riled up}[idiom][C1]
            \EN{to become very annoyed or upset.}
            \VI{trở nên bực tức.}
            \SY{get irritated; get angry}
            \EX{He got all riled up over nothing.}
            \EX{Don’t get all riled up about traffic.}
            \CO{get riled up about sth}
            \end{ExplainCard}

            \begin{ExplainCard}{dispute}[n][C1]
            \EN{an argument or disagreement.}
            \VI{sự tranh cãi.}
            \SY{conflict; argument}
            \EX{The dispute between workers and management lasted weeks.}
            \EX{They resolved the dispute peacefully.}
            \CO{legal dispute; dispute over sth}
            \end{ExplainCard}

            \begin{ExplainCard}{squabble}[n][C1]
            \EN{a noisy quarrel about something small.}
            \VI{tranh cãi nhỏ nhặt.}
            \SY{quarrel; bickering}
            \EX{Children often have squabbles.}
            \EX{A squabble broke out among the committee members.}
            \CO{minor squabble; squabble with sb}
            \end{ExplainCard}

            \begin{ExplainCard}{pull the wool over sb’s eyes}[idiom][C2]
            \EN{to deceive or trick someone.}
            \VI{lừa gạt ai đó.}
            \SY{deceive; trick}
            \EX{He tried to pull the wool over my eyes with fake documents.}
            \EX{Consumers were angry that the ad pulled the wool over their eyes.}
            \CO{pull the wool over sb’s eyes}
            \end{ExplainCard}

            \begin{ExplainCard}{impulsive}[adj][C1]
            \EN{acting suddenly without thinking carefully.}
            \VI{bốc đồng.}
            \SY{rash; spontaneous}
            \EX{She made an impulsive decision to quit her job.}
            \EX{Young people can be impulsive in love.}
            \CO{impulsive action; impulsive behaviour}
            \end{ExplainCard}

            \begin{ExplainCard}{virtue}[n][C1]
            \EN{a good moral quality.}
            \VI{đức hạnh, đức tính tốt.}
            \SY{goodness; morality}
            \EX{Patience is a great virtue.}
            \EX{They taught their children the virtue of honesty.}
            \CO{cardinal virtue; practise a virtue}
            \end{ExplainCard}
        \end{VocabExplain}

    \begin{VocabHighlights}
        \VH{nourishing}{(adj) helping a person, an animal or a plant to grow and be healthy}{(tính từ) giúp cho khỏe mạnh}
        \VH{you are what you eat}{(proverb) to be fit \& healthy you must eat good food}{(tục ngữ) ăn gì bổ nấy}
        \VH{healthful}{(adj) good for your health}{(tính từ) tốt cho sức khỏe}
        \VH{to lie down}{(phr.v) to be or get into a flat position, especially in bed, in order to sleep or rest}{(cụm động từ) nằm xuống}
        \VH{dead beat}{(idiom) very tired}{(thành ngữ) hết hơi}
        \VH{to fend for oneself}{(phr.v) to take care of yourself without help from anyone else}{(cụm động từ) tự lực cánh sinh}
        \VH{to hit the hay}{(idiom) to go to bed}{(thành ngữ) đi ngủ}
        \VH{to sack out}{(phr.v) to go to bed}{(cụm động từ) đi ngủ}
        \VH{to roll out of bed}{(idiom) to wake up}{(thành ngữ) tỉnh dậy}
        \VH{absent-minded}{(adj) tending to forget things, perhaps because you are not thinking about what is around you, but about something else}{(tính từ) đãng trí}
        \VH{for donkey’s years}{(idiom) for a very long time}{(thành ngữ) trong thời gian dài}
        \VH{to come into reality}{(phrase) come true}{(cụm từ) trở thành hiện thực}
        \VH{once-in-a-lifetime}{(phrase) probably only have it once}{(cụm từ) cơ hội 1 lần duy nhất trong đời}
        \VH{all the go}{(idiom) very popular}{(thành ngữ) được ưa chuộng}
        \VH{at random}{(phrase) randomly}{(cụm từ) bất kỳ}
        \VH{to take an immediate liking to}{(phrase) to begin to like something/somebody quickly}{(cụm từ) bắt đầu thích rất nhanh}
        \VH{be mesmerized by}{(v2) someone’s attention is completely captivated so that they cannot think of anything else}{(phần từ 2) bị mê mẩn bởi}
        \VH{to kick off}{(phr.v) to begin}{(cụm động từ) bắt đầu}
        \VH{to jump at the chance}{(phrase) to grab a chance}{(cụm từ) nắm bắt cơ hội}
        \VH{in somebody’s presence}{(phrase) the fact that someone or something is in a place}{(cụm từ) trong sự hiện diện của}
        \VH{punctuality}{(n) the fact of happening or doing something at the agreed or correct time and not being late}{(danh từ) đúng giờ}
        \VH{an attribute}{(n) a quality or characteristic that someone or something has}{(danh từ) đức tính tốt}
        \VH{tardy}{(adj) late}{(tính từ) muộn}
        \VH{to be red in the face}{(idiom) to suffer embarrassment or shame}{(thành ngữ) xấu hổ}
        \VH{excuse}{(n) a reason, either true or invented, that you give to explain or defend your behaviour}{(danh từ) lý do ngụy biện}
        \VH{squabble}{(n) a noisy argument about something that is not very important}{(danh từ) cuộc cãi nhau}
        \VH{to pull the wool over my eyes}{(phrase) somebody is trying to deceive you, in order to have an advantage over you}{(cụm từ) lừa dối}
        \VH{fatigue}{(n) extreme tiredness}{(danh từ) cực kì mệt mỏi}
        \VH{impulsive}{(adj) showing behaviour in which you do things suddenly without any planning and without considering the effects they may have}{(tính từ) bộc phát}
        \VH{virtue}{(n) a good behavior, quality}{(danh từ) đức tính tốt}
    \end{VocabHighlights}
    \end{test}

    \begin{test}{TEST 2}
    \noindent
    \textbf{Part 1. Songs and Singing}
    \begin{qa}{Did you enjoy singing when you were younger? [Why?/Why not?]}
    Yes, I did. Until now, I still sing \textbf{as much from habit as from desire}. When I was a high school student, I did not mind singing at any events that my class held. The \textbf{soothing} beat of pop music has a positive effect of helping me \textbf{keep a cool head} when I am in a bad mood. In rock music, its aggressive rhythm can also boost my mood when I need a source of inspiration.
    \end{qa}

    \begin{qa}{How often do you sing now? [Why?]}
    I still \textbf{have a crack at} singing on a daily basis. In other words, I sing whenever there’s no one around. I am \textbf{eerily familiar with} locking myself in my own room, turning on the speaker’s max volume and singing and dancing to the music if possible. It lasts for only a quarter an hour so luckily my neighbors have not complained anything yet.
    \end{qa}

    \begin{qa}{Do you have a favourite song you like listening to? [Why?/Why not?]}
    My favorite song is “Dear God”, a metal ballad by Avenged Sevenfold. It is about a man’s praying that God would protect his beloved one when he was far away. It reminds me of the time when I studied in the U.K and my fiance remained in Vietnam. At that time, I sometimes felt helpless as what I could do was only lending a sympathetic ear to my girlfriend without being able to do anything to help her at all.  

    This song, thanks to its meaningful and touching lyrics stating the \textbf{fervent} desire of a man for the God’s protection of his \textbf{significant other}, made me stronger and believe in a happy ending upon my return. Finally, thank God, I did it.
    \end{qa}

    \begin{qa}{How important is singing in your culture? [Why?]}
    Singing is an \textbf{integral} part of my culture. When I was an infant then a toddler, my mother’s singing in the form of a lullaby to get me into sleep characterized my childhood. Then, as I grew older, I realized music has a significant effect on others’ mind. I do \textbf{take pride in} seeing other patriots such as sportsmen \textbf{tightening their fists} in front of their chests and singing the national anthem of Vietnam.
    \end{qa}

        \begin{VocabExplain}[Part 1]
            \begin{ExplainCard}{as much from habit as from desire}[phrase][C1]
            \EN{doing something partly out of routine and partly because of liking it.}
            \VI{làm việc gì vừa vì thói quen vừa vì mong muốn.}
            \SY{partly routine, partly passion}
            \EX{He writes daily as much from habit as from desire.}
            \EX{I read books as much from habit as from desire.}
            \CO{do sth as much from habit as from desire}
            \end{ExplainCard}

            \begin{ExplainCard}{soothing}[adj][B2]
            \EN{making you feel calm and less worried.}
            \VI{êm dịu, làm dịu.}
            \SY{calming; relaxing}
            \EX{She spoke in a soothing voice.}
            \EX{Soothing music helps reduce stress.}
            \CO{soothing effect; soothing music}
            \end{ExplainCard}

            \begin{ExplainCard}{keep a cool head}[idiom][C1]
            \EN{to stay calm in a difficult situation.}
            \VI{giữ bình tĩnh.}
            \SY{stay calm; remain composed}
            \EX{She kept a cool head during the crisis.}
            \EX{Doctors must keep a cool head in emergencies.}
            \CO{keep a cool head under pressure}
            \end{ExplainCard}

            \begin{ExplainCard}{have a crack at}[idiom][C1]
            \EN{to try something.}
            \VI{thử làm việc gì.}
            \SY{try; attempt}
            \EX{I’ll have a crack at fixing the car.}
            \EX{She had a crack at solving the puzzle.}
            \CO{have a crack at doing sth}
            \end{ExplainCard}

            \begin{ExplainCard}{eerily familiar with}[phrase][C1]
            \EN{strangely or unnaturally familiar with something.}
            \VI{quen thuộc một cách kỳ lạ.}
            \SY{uncannily familiar; oddly familiar}
            \EX{The story was eerily familiar with my dream.}
            \EX{He was eerily familiar with the abandoned house.}
            \CO{eerily familiar with sth}
            \end{ExplainCard}

            \begin{ExplainCard}{fervent}[adj][C1]
            \EN{showing strong and sincere feelings.}
            \VI{nồng nhiệt, tha thiết.}
            \SY{ardent; passionate}
            \EX{He is a fervent supporter of the team.}
            \EX{They made a fervent plea for peace.}
            \CO{fervent desire; fervent supporter}
            \end{ExplainCard}

            \begin{ExplainCard}{significant other}[n][C1]
            \EN{a person you have a romantic relationship with.}
            \VI{người thương, nửa kia.}
            \SY{partner; loved one}
            \EX{She came to the party with her significant other.}
            \EX{He bought a gift for his significant other.}
            \CO{support from significant other}
            \end{ExplainCard}

            \begin{ExplainCard}{integral}[adj][C1]
            \EN{necessary and important as part of a whole.}
            \VI{thiết yếu, không thể thiếu.}
            \SY{essential; crucial}
            \EX{Music is an integral part of education.}
            \EX{Trust is integral to a good marriage.}
            \CO{integral part; integral to sth}
            \end{ExplainCard}

            \begin{ExplainCard}{take pride in}[idiom][B2]
            \EN{to be proud of something.}
            \VI{tự hào về.}
            \SY{be proud of; value}
            \EX{He takes pride in his work.}
            \EX{Citizens take pride in their heritage.}
            \CO{take pride in sth}
            \end{ExplainCard}

            \begin{ExplainCard}{tighten one’s fist}[phrase][C1]
            \EN{to close the hand firmly, often as a sign of determination or emotion.}
            \VI{siết chặt nắm tay (thể hiện quyết tâm/cảm xúc).}
            \SY{clench fist; grip}
            \EX{He tightened his fist in anger.}
            \EX{They tightened their fists during the anthem.}
            \CO{tighten fist in determination}
            \end{ExplainCard}
        \end{VocabExplain}

    \noindent
    \textbf{Part 2.}
    \begin{qa}{Describe a film/movie actor from your country who is very popular. You should say:}
    \begin{itemize}
        \item Who this actor is
        \item What kinds of films/movies he/she acts in
        \item What you know about this actor’s life
        \item and explain why this actor is so popular.
    \end{itemize}

    \textbf{To tell you the truth}, I don’t know much about film actors in general, let alone those from my own country, but I’ll do my best to talk about Hong Dang, who is a \textbf{rising movie star} in Vietnam. To the best of my knowledge, he is 37 years old, but he is \textbf{youthful-looking}. It is due to the fact that he is quite optimistic about life, so I guess, if you \textbf{meet him} in person, you will think he is in his 20s. He is \textbf{of athletic build} because he works out in the gym every day.  

    What \textbf{took me by surprise} is that he \textbf{tied the knot} when he came \textbf{fresh out} of university, so now he is a father of two girls. I know him by chance when he is my sister’s high-school friend. Let me tell you about his career. He began his acting career in short films \textbf{as a supporting actor} around 15 years ago. Recently, he \textbf{has come to prominence} as an \textbf{accomplished actor} thanks to his \textbf{lead role} in several most-watched TV films such as “Forever Young”, “The Arbitrator”, “The Maze”, etc.  

    Those films got record high viewership ratings when they were broadcast on TV a few months ago, so he \textbf{gained wider recognition} for his talent. The primary reason why he \textbf{makes a name for himself} is that he has inspired young people to fulfill their long-held dreams. \textbf{Believe it or not}, he graduated from Foreign Trade University, but he decided not to \textbf{follow in his father’s footsteps} as an entrepreneur. Instead, he \textbf{pursued his passion} for acting although he \textbf{came in for} strong opposition from his family. That’s why many young people regard him as \textbf{a shining example} to follow.
    \end{qa}

        \begin{VocabExplain}[Part 2]
            \begin{ExplainCard}{to tell you the truth}[phrase][B2]
            \EN{used to emphasize honesty about something.}
            \VI{nói thật lòng.}
            \SY{honestly; frankly}
            \EX{To tell you the truth, I don’t like the movie.}
            \EX{To tell you the truth, I was nervous.}
            \CO{to tell you the truth}
            \end{ExplainCard}

            \begin{ExplainCard}{rising movie star}[phrase][B2]
            \EN{an actor becoming increasingly famous.}
            \VI{ngôi sao điện ảnh đang lên.}
            \SY{up-and-coming actor; emerging star}
            \EX{He is a rising movie star in Hollywood.}
            \EX{She became a rising movie star after her first film.}
            \CO{rising star in sth}
            \end{ExplainCard}

            \begin{ExplainCard}{youthful-looking}[adj][B2]
            \EN{appearing younger than one’s actual age.}
            \VI{trông trẻ trung hơn tuổi.}
            \SY{young-looking; fresh-faced}
            \EX{She is youthful-looking despite being 50.}
            \EX{His youthful-looking face impressed everyone.}
            \CO{remain youthful-looking}
            \end{ExplainCard}

            \begin{ExplainCard}{of athletic build}[phrase][B2]
            \EN{having a strong and fit body.}
            \VI{có vóc dáng thể thao, săn chắc.}
            \SY{muscular; fit}
            \EX{He is of athletic build from years of training.}
            \EX{She is of athletic build thanks to daily exercise.}
            \CO{be of athletic build}
            \end{ExplainCard}

            \begin{ExplainCard}{take sb by surprise}[idiom][B2]
            \EN{to shock or amaze someone unexpectedly.}
            \VI{khiến ai đó bất ngờ.}
            \SY{astonish; startle}
            \EX{The ending took me by surprise.}
            \EX{His proposal took her by surprise.}
            \CO{take sb by surprise completely}
            \end{ExplainCard}

            \begin{ExplainCard}{tie the knot}[idiom][B2]
            \EN{to get married.}
            \VI{kết hôn.}
            \SY{get married; wed}
            \EX{They decided to tie the knot last year.}
            \EX{She tied the knot with her longtime boyfriend.}
            \CO{tie the knot with sb}
            \end{ExplainCard}

            \begin{ExplainCard}{fresh out of}[phrase][C1]
            \EN{having just finished something, especially school/university.}
            \VI{mới tốt nghiệp, mới ra trường.}
            \SY{just graduated; newly finished}
            \EX{She was fresh out of college.}
            \EX{He joined the company fresh out of university.}
            \CO{fresh out of school/university}
            \end{ExplainCard}

            \begin{ExplainCard}{supporting actor}[n][B2]
            \EN{an actor who plays a secondary role.}
            \VI{diễn viên phụ.}
            \SY{secondary actor; co-star}
            \EX{He won Best Supporting Actor award.}
            \EX{She started as a supporting actor in TV dramas.}
            \CO{Best Supporting Actor}
            \end{ExplainCard}

            \begin{ExplainCard}{come to prominence}[phrase][C1]
            \EN{to become important or well-known.}
            \VI{trở nên nổi bật, nổi tiếng.}
            \SY{rise to fame; gain attention}
            \EX{He came to prominence in the 1990s.}
            \EX{The issue came to prominence last year.}
            \CO{come to prominence quickly}
            \end{ExplainCard}

            \begin{ExplainCard}{accomplished actor}[n][C1]
            \EN{a skilled and successful actor.}
            \VI{diễn viên tài năng, thành công.}
            \SY{skilled actor; experienced actor}
            \EX{She is an accomplished actor with many awards.}
            \EX{The play featured an accomplished actor.}
            \CO{accomplished artist/musician/actor}
            \end{ExplainCard}

            \begin{ExplainCard}{lead role}[n][B2]
            \EN{the most important part in a play or movie.}
            \VI{vai chính.}
            \SY{main role; protagonist role}
            \EX{He played the lead role in the film.}
            \EX{The lead role was given to a young actress.}
            \CO{play a lead role}
            \end{ExplainCard}

            \begin{ExplainCard}{gain wider recognition}[phrase][C1]
            \EN{to be acknowledged and respected by more people.}
            \VI{được công nhận rộng rãi.}
            \SY{achieve acknowledgment; earn respect}
            \EX{She gained wider recognition after her album.}
            \EX{The project gained wider recognition in the community.}
            \CO{gain recognition/wider recognition}
            \end{ExplainCard}

            \begin{ExplainCard}{make a name for oneself}[idiom][C1]
            \EN{to become famous or respected.}
            \VI{tạo dựng tên tuổi.}
            \SY{establish reputation; become known}
            \EX{He made a name for himself in politics.}
            \EX{She made a name for herself as a designer.}
            \CO{make a name for oneself as}
            \end{ExplainCard}

            \begin{ExplainCard}{believe it or not}[phrase][B2]
            \EN{used to express surprise that something is true.}
            \VI{tin hay không thì tùy.}
            \SY{surprisingly; incredibly}
            \EX{Believe it or not, he is 60 years old.}
            \EX{Believe it or not, she has never flown before.}
            \CO{believe it or not}
            \end{ExplainCard}

            \begin{ExplainCard}{follow in one’s footsteps}[idiom][C1]
            \EN{to do the same work or live in the same way as someone else, usually in your family.}
            \VI{theo gương, nối gót.}
            \SY{imitate; emulate}
            \EX{She followed in her father’s footsteps as a doctor.}
            \EX{He refused to follow in his parents’ footsteps.}
            \CO{follow in sb’s footsteps}
            \end{ExplainCard}

            \begin{ExplainCard}{pursue passion}[phrase][B2]
            \EN{to follow something you strongly want to do.}
            \VI{theo đuổi đam mê.}
            \SY{follow passion; chase dream}
            \EX{She pursued her passion for painting.}
            \EX{He gave up his job to pursue his passion.}
            \CO{pursue passion for sth}
            \end{ExplainCard}

            \begin{ExplainCard}{come in for}[phr.v][C1]
            \EN{to receive something, especially criticism or blame.}
            \VI{hứng chịu, nhận lấy.}
            \SY{receive; attract}
            \EX{The policy came in for criticism.}
            \EX{The actor came in for strong opposition.}
            \CO{come in for criticism/opposition}
            \end{ExplainCard}

            \begin{ExplainCard}{a shining example}[idiom][C1]
            \EN{a perfect model for others to follow.}
            \VI{tấm gương sáng.}
            \SY{role model; good example}
            \EX{She is a shining example of hard work.}
            \EX{He is a shining example to other students.}
            \CO{a shining example of sth}
            \end{ExplainCard}
        \end{VocabExplain}

    \noindent
    \textbf{Part 3.}
    \begin{qa}{What are the most popular types of films in your country?}
    It \textbf{comes as no surprise} that action movies are \textbf{all the rage} in not only my countries but also many parts of the world. In fact, most of \textbf{action-packed} movies are highly recommended because of the \textbf{creepy}, thrilling or heartbreaking stories. This is why all movie lovers often \textbf{jump for joy} whenever a \textbf{blockbuster} is to go \textbf{on general release}.
    \end{qa}

    \begin{qa}{What is the difference between watching a film in the cinema and watching a film at home?}
    Going to the movies is a common hobby, but there are different opinions about watching movies at different places. Obviously, \textbf{catching a flick} will be the perfect choice for individuals who want to experience lively scenes of \textbf{marvelous} movies through the big screen, while watching TV at home seems \textbf{tedious}. However, going to the theater is sometimes pretty pricey due to accompanied services, so watching a film at home is more economical.
    \end{qa}

    \begin{qa}{Do you think cinemas will close in the future?}
    From my own perspective, cinemas will hardly become \textbf{out of fashion} despite the fact that home entertainment is becoming more and more popular due to a \textbf{streaming service} like Netflix and other forms of watching films at home. Firstly, \textbf{cinema}, thanks to its dramatic sound effects and giant screens, fills the audiences with a sense of \textbf{awe} and wonder, which are suitable for those who seek \textbf{thrill} and adventure or simply enhance movie experience. Provided that cinema managers try to upgrade their services, I think it still attracts throngs of people in the future.
    \end{qa}

    \begin{qa}{How important is the theater in your country's history?}
    I am proud to say that the theater plays a significant role in my country as it has highlighted periods of \textbf{turbulence} in the rich history. In the past, going to the theater was mainly for \textbf{nobles} or an \textbf{affluent class} as it could \textbf{cost an arm and a leg} for a single ticket to go there. But things have changed a lot. Nowadays, most people, even \textbf{blue-collar workers} can gain access to the cinema to watch their favourite movie.
    \end{qa}

    \begin{qa}{How strong a tradition is it today in your country to go to the theater?}
    Pretty strong I believe. With the \textbf{robust} development of entertainment industry, an increasing number of audience are swinging by the theater more regularly than ever before. Apart from entertaining movies, documentaries and \textbf{theatrical performances} have \textbf{risen to prominence}, especially among old people because these genres are \textbf{remakes} of classic movies.
    \end{qa}

    \begin{qa}{Do you think the theater should be run as a business or as a public service?}
    As far as I am concerned, I believe the theater should run for both commercial purposes and public purposes. On the one hand, most cinemas are owned by private companies, and they need to make decent profits to \textbf{finance overhead expenses} such as salary pay or maintenance fees. However, if the cinema ticket \textbf{charges people through their nose}, there will be fewer people go to the cinema as a result. It should quote the public a reasonable price for a ticket to stimulate everyone to come over and enjoy the films.
    \end{qa}

        \begin{VocabExplain}[Part 3]
            \begin{ExplainCard}{it comes as no surprise}[phrase][B2]
            \EN{used to say that something is expected.}
            \VI{không có gì ngạc nhiên.}
            \SY{not unexpected; as expected}
            \EX{It comes as no surprise that she won the award.}
            \EX{It comes as no surprise that action films are popular.}
            \CO{it comes as no surprise that + clause}
            \end{ExplainCard}

            \begin{ExplainCard}{all the rage}[idiom][C1]
            \EN{very popular at a particular time.}
            \VI{rất thịnh hành.}
            \SY{fashionable; trendy}
            \EX{TikTok videos are all the rage nowadays.}
            \EX{K-pop is all the rage among teenagers.}
            \CO{be all the rage}
            \end{ExplainCard}

            \begin{ExplainCard}{action-packed}[adj][C1]
            \EN{full of exciting events and activities.}
            \VI{nhiều cảnh hành động.}
            \SY{full of action; thrilling}
            \EX{The film was action-packed and exciting.}
            \EX{He loves action-packed adventures.}
            \CO{action-packed movie/film}
            \end{ExplainCard}

            \begin{ExplainCard}{creepy}[adj][B2]
            \EN{causing an unpleasant feeling of fear or unease.}
            \VI{rùng rợn, ghê rợn.}
            \SY{scary; eerie}
            \EX{The old house looked creepy.}
            \EX{She felt creepy walking alone at night.}
            \CO{creepy story; creepy feeling}
            \end{ExplainCard}

            \begin{ExplainCard}{jump for joy}[idiom][B2]
            \EN{to be extremely happy.}
            \VI{nhảy cẫng lên vì sung sướng.}
            \SY{be overjoyed; be delighted}
            \EX{She jumped for joy when she heard the news.}
            \EX{Fans jumped for joy at the victory.}
            \CO{jump for joy when + clause}
            \end{ExplainCard}

            \begin{ExplainCard}{blockbuster}[n][B2]
            \EN{a very successful and popular film.}
            \VI{bom tấn.}
            \SY{hit movie; smash}
            \EX{This film is expected to be a blockbuster.}
            \EX{The movie became a summer blockbuster.}
            \CO{summer blockbuster; box-office blockbuster}
            \end{ExplainCard}

            \begin{ExplainCard}{on general release}[phrase][B2]
            \EN{when a film is made available to the public in cinemas.}
            \VI{ra rạp công chiếu rộng rãi.}
            \SY{in theaters; released}
            \EX{The film goes on general release next week.}
            \EX{It was first shown at Cannes before going on general release.}
            \CO{film go on general release}
            \end{ExplainCard}

            \begin{ExplainCard}{catch a flick}[idiom][B2]
            \EN{to watch a movie. (informal)}
            \VI{đi xem phim.}
            \SY{watch a movie; see a film}
            \EX{We decided to catch a flick after dinner.}
            \EX{Let’s catch a flick at the cinema tonight.}
            \CO{catch a flick at the cinema}
            \end{ExplainCard}

            \begin{ExplainCard}{marvelous}[adj][B2]
            \EN{extremely good; wonderful.}
            \VI{tuyệt vời.}
            \SY{wonderful; fantastic}
            \EX{We had a marvelous time at the party.}
            \EX{The film was absolutely marvelous.}
            \CO{marvelous time; marvelous movie}
            \end{ExplainCard}

            \begin{ExplainCard}{tedious}[adj][C1]
            \EN{boring and too slow or long.}
            \VI{tẻ nhạt.}
            \SY{boring; dull}
            \EX{The lecture was tedious and long.}
            \EX{Watching that film was tedious.}
            \CO{tedious job; tedious process}
            \end{ExplainCard}

            \begin{ExplainCard}{out of fashion}[phrase][B2]
            \EN{no longer popular or trendy.}
            \VI{lỗi thời.}
            \SY{old-fashioned; outdated}
            \EX{Bell-bottom jeans went out of fashion.}
            \EX{This hairstyle is out of fashion now.}
            \CO{go out of fashion}
            \end{ExplainCard}

            \begin{ExplainCard}{streaming service}[n][B2]
            \EN{an online platform that delivers media to users.}
            \VI{dịch vụ xem phim trực tuyến.}
            \SY{online media service; OTT platform}
            \EX{Netflix is the most popular streaming service.}
            \EX{He subscribed to a streaming service.}
            \CO{subscribe to streaming service}
            \end{ExplainCard}

            \begin{ExplainCard}{awe}[n][C1]
            \EN{a feeling of great respect or wonder.}
            \VI{sự kính phục, thán phục.}
            \SY{wonder; admiration}
            \EX{She looked at the mountains in awe.}
            \EX{The audience watched in awe.}
            \CO{in awe of; fill sb with awe}
            \end{ExplainCard}

            \begin{ExplainCard}{thrill}[n][B2]
            \EN{a sudden feeling of excitement and pleasure.}
            \VI{sự hồi hộp, phấn khích.}
            \SY{excitement; exhilaration}
            \EX{It was a thrill to meet the singer.}
            \EX{The roller coaster gave me a thrill.}
            \CO{thrill of sth; feel the thrill}
            \end{ExplainCard}

            \begin{ExplainCard}{turbulence}[n][C1]
            \EN{a state of conflict or confusion.}
            \VI{sự hỗn loạn, biến động.}
            \SY{chaos; upheaval}
            \EX{The country was in political turbulence.}
            \EX{There was turbulence during the protest.}
            \CO{political turbulence; economic turbulence}
            \end{ExplainCard}

            \begin{ExplainCard}{affluent}[adj][C1]
            \EN{having a lot of money.}
            \VI{giàu có, thịnh vượng.}
            \SY{wealthy; prosperous}
            \EX{They live in an affluent neighborhood.}
            \EX{He comes from an affluent family.}
            \CO{affluent class; affluent area}
            \end{ExplainCard}

            \begin{ExplainCard}{cost an arm and a leg}[idiom][B2]
            \EN{to be very expensive.}
            \VI{rất đắt đỏ.}
            \SY{be pricey; exorbitant}
            \EX{The phone cost me an arm and a leg.}
            \EX{Tickets to the show cost an arm and a leg.}
            \CO{cost sb an arm and a leg}
            \end{ExplainCard}

            \begin{ExplainCard}{blue-collar worker}[n][B2]
            \EN{a worker who does manual work.}
            \VI{công nhân lao động tay chân.}
            \SY{manual worker; laborer}
            \EX{Most blue-collar workers are in manufacturing.}
            \EX{Blue-collar workers now earn higher wages.}
            \CO{blue-collar jobs; blue-collar community}
            \end{ExplainCard}

            \begin{ExplainCard}{robust}[adj][C1]
            \EN{strong and successful.}
            \VI{mạnh mẽ, phát triển bền vững.}
            \SY{strong; vigorous}
            \EX{The company is in robust health.}
            \EX{He gave a robust performance.}
            \CO{robust economy; robust growth}
            \end{ExplainCard}

            \begin{ExplainCard}{theatrical performance}[n][B2]
            \EN{a play or drama performed on stage.}
            \VI{màn biểu diễn kịch.}
            \SY{stage play; drama}
            \EX{The school gave a theatrical performance.}
            \EX{They enjoyed the theatrical performance.}
            \CO{classical theatrical performance}
            \end{ExplainCard}

            \begin{ExplainCard}{rise to prominence}[phrase][C1]
            \EN{to become well-known or important.}
            \VI{trở nên nổi bật, nổi tiếng.}
            \SY{come to prominence; gain importance}
            \EX{He rose to prominence in the 1990s.}
            \EX{The singer rose to prominence after her debut.}
            \CO{rise to prominence quickly}
            \end{ExplainCard}

            \begin{ExplainCard}{remake}[n][B2]
            \EN{a new version of an old film or TV show.}
            \VI{phiên bản làm lại.}
            \SY{new version; adaptation}
            \EX{They made a remake of the classic film.}
            \EX{The remake was less successful than the original.}
            \CO{remake of a film}
            \end{ExplainCard}

            \begin{ExplainCard}{finance overhead expenses}[phrase][C1]
            \EN{to provide money for ongoing costs of running a business.}
            \VI{tài trợ chi phí vận hành.}
            \SY{cover costs; fund expenses}
            \EX{The company must finance overhead expenses.}
            \EX{They raised money to finance overhead expenses.}
            \CO{finance overhead costs/expenses}
            \end{ExplainCard}

            \begin{ExplainCard}{charge people through the nose}[idiom][C1]
            \EN{to charge excessively high prices.}
            \VI{chặt chém, bắt trả giá quá cao.}
            \SY{overcharge; exploit}
            \EX{The hotel charged us through the nose.}
            \EX{They charge customers through the nose during holidays.}
            \CO{charge sb through the nose for sth}
            \end{ExplainCard}
        \end{VocabExplain}

    \begin{VocabHighlights}
        \VH{as much from habit as from desire}{(phrase) both a habit and desire}{(cụm từ) vừa là thói quen vừa là đam mê}
        \VH{soothing}{(adj) that makes somebody who is anxious, upset, etc. feel calmer}{(tính từ) làm êm dịu}
        \VH{to keep a cool head}{(idiom) to keep calm}{(thành ngữ) giữ bình tĩnh}
        \VH{to have a crack at V-ing}{(idiom) to try V-ing}{(thành ngữ) thử làm gì}
        \VH{eerily}{(adv) in a strange, mysterious and frightening way}{(trạng từ) kì quặc}
        \VH{fervent}{(adj) having or showing very strong and sincere feelings about something}{(tính từ) cháy bỏng}
        \VH{significant other}{(phrase) your husband, wife, partner or somebody that you have a special relationship with}{(cụm từ) người chồng, vợ}
        \VH{integral}{(adj) being an essential part of something}{(tính từ) là phần quan trọng, không thể thiếu được của}
        \VH{to take pride in}{(phrase) to be proud of}{(cụm từ) tự hào về}
        \VH{to tighten one’s fists}{(phrase) to make fists become tight or tighter}{(cụm từ) nắm chặt tay}
        \VH{to tell you the truth}{(idiom) to be honest}{(thành ngữ) thành thực mà nói}
        \VH{rising movie star}{(phrase) a movie star is attracting much attention from the public}{(cụm từ) ngôi sao điện ảnh đang nổi}
        \VH{to take somebody by surprise}{(idiom) make somebody surprised}{(thành ngữ) làm ai đó bất ngờ}
        \VH{to tie the knot}{(idiom) to get married}{(thành ngữ) kết hôn}
        \VH{fresh out of}{(idiom) having just finished education or training, and not having a lot of experience}{(thành ngữ) vừa mới xong, chân ướt chân ráo}
        \VH{to come to prominence}{(idiom) become notable; to become renowned}{(thành ngữ) bắt đầu nổi tiếng}
        \VH{lead role}{(phrase) the important part of something}{(cụm từ) vai chính}
        \VH{to gain wider recognition}{(phrase) people show admiration and respect for your achievements}{(cụm từ) nhận được sự ghi nhận}
        \VH{to make a name for}{(idiom) be well-known for}{(thành ngữ) nổi tiếng vì cái gì đó}
        \VH{believe it or not}{(idiom) said when telling someone about something that is true, although it seems unlikely}{(thành ngữ) khó tin nhưng sự thật là}
        \VH{to follow in his father’s footsteps}{(idiom) to do the same thing as someone else did previously}{(thành ngữ) theo chân ai đó, nối nghiệp ai đó}
        \VH{to pursue his passion}{(phrase) follow passion with the eagerness}{(cụm từ) theo đuổi đam mê}
        \VH{to come in for strong opposition}{(phrase) to receive blame or criticism}{(cụm từ) nhận trách nhiệm hoặc chỉ trích}
        \VH{a shining example}{(phrase) an excellent example}{(thành ngữ) ví dụ điển hình}
        \VH{it comes as no surprise}{(phrase) to be completely unsurprising}{(cụm từ) hoàn toàn không bất ngờ}
        \VH{to be all the range}{(phrase) very popular}{(cụm từ) rất nổi tiếng, thịnh hành}
        \VH{action-packed movies}{(phrase) films that filled with action, danger, and excitement}{(cụm từ) phim hành động}
        \VH{creepy}{(adj) causing an unpleasant feeling of fear or slight horror}{(tính từ) làm sợ hãi, làm rợn tóc gáy}
        \VH{to jump for joy}{(phrase) to be extremely happy}{(cụm từ) rất hạnh phúc}
        \VH{a blockbuster}{(n) something very successful, especially a very successful book or film/movie}{(danh từ) phim bom tấn}
        \VH{on general release}{(idiom) available to be seen in cinemas}{(thành ngữ) có chiếu ở rạp}
        \VH{to catch a flick}{(phrase) to go to the movie}{(cụm từ) đi xem phim}
        \VH{marvelous}{(adj) extremely good; wonderful}{(tính từ) tuyệt phẩm, cực kỳ hay}
        \VH{tedious}{(adj) lasting or taking too long and not interesting}{(tính từ) tẻ nhạt}
        \VH{out of fashion}{(phrase) to be outdated}{(cụm từ) lỗi thời}
        \VH{streaming}{(n) a method of sending or receiving data, especially video, over a computer network}{(danh từ) phát trực tiếp}
        \VH{awe}{(n) a feeling of great respect sometimes mixed with fear or surprise}{(danh từ) kinh ngạc}
        \VH{thrill}{(n) a feeling of extreme excitement, usually caused by something pleasant}{(danh từ) hồi hộp}
        \VH{turbulence}{(n) a situation in which there is a lot of sudden change, confusion, disagreement and sometimes violence}{(danh từ) thăng trầm}
        \VH{noble}{(n) a person who comes from a family of high social rank; a member of the nobility}{(danh từ) quý tộc}
        \VH{affluent class}{(phrase) having a lot of money and a good standard of living}{(cụm từ) giới giàu có}
        \VH{to cost an arm and a leg}{(idiom) very expensive}{(thành ngữ) cực kỳ đắt tiền}
        \VH{blue-collar workers}{(phrase) a working class person}{(cụm từ) tầng lớp lao động}
        \VH{robust}{(adj) strong and healthy development}{(tính từ) khỏe mạnh}
        \VH{to swing by}{(phrasal verb) to make a short visit to a person or place}{(cụm động từ) ghé qua thăm}
        \VH{theatrical}{(adj) belonging or relating to the theatre, or to the performance or writing of plays, opera}{(tính từ) liên quan đến phim, kịch}
        \VH{risen to prominence}{(phrase) the state of being prominent; conspicuousness}{(cụm từ) được chú ý, thu hút}
        \VH{remake}{(n) a new or different version of an old film/movie or song}{(danh từ) sự làm lại}
        \VH{overhead expenses}{(phrase) all costs on the income statement except for direct labor, direct materials, and direct expenses}{(cụm từ) chi phí vận hành}
        \VH{to charge somebody through somebody’s nose}{(idiom) very expensive}{(thành ngữ) đắt đỏ}
    \end{VocabHighlights}
    \end{test}

    \begin{test}{TEST 3}
    \noindent
    \textbf{Part 1. CLothes}
    \begin{qa}{Where do you buy most of your clothes? [Why?]}
    I admit I \textbf{do not have an eye for} fashion. I will never ever \textbf{splash out on} designers’ clothes at any fancy stores. I also do not want to waste time \textbf{scouring} every clothing store to \textbf{be dressed to the nines} like some of my friends. Instead, if I happen to see advertisements on any clothing items at online shops on Facebook that \textbf{garner my interests}, I won’t hesitate to order one for me. That saves my time as well.
    \end{qa}

    \begin{qa}{How often do you buy new clothes for yourself? [Why?]}
    Maybe once or twice every season. Provided that my clothes are in \textbf{pristine} condition, it’s needless for me to replace them with new ones. It is also necessary to \textbf{economize on} clothes to accomplish other meaningful purposes such as running the family \textbf{in lieu of shelling out money} for a basic necessity like this.
    \end{qa}

    \begin{qa}{How do you decide which clothes to buy? [Why?]}
    I have a number of criteria when it comes to choosing appropriate clothes. Firstly, the color leaves others with the first but important impression. I am \textbf{dead keen on} dark or navy blue as this shade of blue is said to suit my \textbf{feng shui element of wood in Chinese zodiac}. I’m quite \textbf{superstitious} so I feel this choice of color might boost my confidence and bring me luck as well. Secondly, the size of any outfit also matters. I prefer something loose or regular fit because wearing tight clothes, especially skinny jeans, is not comfortable at all.
    \end{qa}

    \begin{qa}{Have the kinds of clothes you like changed in recent years? [Why?/Why not?]}
    No, they have not. As long as two main criteria are the same, my choice of clothes have remained unchanged. The only change here lies in what I wear at home. In the past, I had a tendency to wear T-shirts or polos but now, \textbf{tank-top} ones are my first choice.
    \end{qa}

        \begin{VocabExplain}[Part 1]
            \begin{ExplainCard}{not have an eye for}[phrase][C1]
            \EN{to lack the ability to judge or appreciate something well.}
            \VI{không có mắt thẩm mỹ.}
            \SY{lack taste; lack judgment}
            \EX{He does not have an eye for fashion.}
            \EX{I don’t have an eye for art.}
            \CO{not have an eye for detail/fashion/design}
            \end{ExplainCard}

            \begin{ExplainCard}{splash out on}[phr.v][B2]
            \EN{to spend a lot of money on something.}
            \VI{vung tiền vào.}
            \SY{spend lavishly; splurge}
            \EX{She splashed out on a new dress.}
            \EX{He splashed out on luxury goods.}
            \CO{splash out on sth}
            \end{ExplainCard}

            \begin{ExplainCard}{scour}[v][C1]
            \EN{to search thoroughly for something.}
            \VI{lùng sục, tìm kiếm kỹ lưỡng.}
            \SY{search; comb}
            \EX{I scoured the shops for a gift.}
            \EX{Researchers scoured the archives for data.}
            \CO{scour shops/archives/internet}
            \end{ExplainCard}

            \begin{ExplainCard}{be dressed to the nines}[idiom][C1]
            \EN{to be wearing very fashionable or elegant clothes.}
            \VI{ăn mặc bảnh bao, lộng lẫy.}
            \SY{be well-dressed; be decked out}
            \EX{She was dressed to the nines for the party.}
            \EX{He always arrives dressed to the nines.}
            \CO{be dressed to the nines for sth}
            \end{ExplainCard}

            \begin{ExplainCard}{garner interest}[phrase][C1]
            \EN{to attract or gain attention or curiosity.}
            \VI{thu hút sự chú ý.}
            \SY{attract; capture}
            \EX{The film garnered interest from critics.}
            \EX{His proposal garnered interest among investors.}
            \CO{garner interest/support/attention}
            \end{ExplainCard}

            \begin{ExplainCard}{pristine}[adj][C1]
            \EN{in perfect condition; completely new or clean.}
            \VI{còn nguyên vẹn, như mới.}
            \SY{immaculate; spotless}
            \EX{The shoes are still in pristine condition.}
            \EX{The island is famous for its pristine beaches.}
            \CO{pristine condition/environment}
            \end{ExplainCard}

            \begin{ExplainCard}{economize on}[phr.v][C1]
            \EN{to save money by spending less on something.}
            \VI{tiết kiệm chi tiêu vào.}
            \SY{cut back on; save on}
            \EX{We economized on fuel costs.}
            \EX{Families try to economize on food.}
            \CO{economize on sth}
            \end{ExplainCard}

            \begin{ExplainCard}{in lieu of}[phrase][C1]
            \EN{instead of.}
            \VI{thay vì.}
            \SY{instead of; in place of}
            \EX{He gave money in lieu of flowers.}
            \EX{The new system was adopted in lieu of the old one.}
            \CO{in lieu of sth}
            \end{ExplainCard}

            \begin{ExplainCard}{shell out money}[phr.v][B2]
            \EN{to spend money, especially reluctantly.}
            \VI{bỏ tiền ra (miễn cưỡng).}
            \SY{pay out; fork out}
            \EX{I had to shell out money for repairs.}
            \EX{They shelled out a fortune on the car.}
            \CO{shell out money for sth}
            \end{ExplainCard}

            \begin{ExplainCard}{dead keen on}[phrase][B2]
            \EN{extremely interested in or enthusiastic about something.}
            \VI{rất thích, cực kỳ đam mê.}
            \SY{be crazy about; be passionate about}
            \EX{She is dead keen on football.}
            \EX{He’s dead keen on photography.}
            \CO{dead keen on sth}
            \end{ExplainCard}

            \begin{ExplainCard}{feng shui element}[n][C1]
            \EN{a natural element (wood, fire, earth, metal, water) associated with feng shui.}
            \VI{ngũ hành phong thủy.}
            \SY{feng shui principle; zodiac element}
            \EX{Her feng shui element is water.}
            \EX{This color suits my feng shui element.}
            \CO{feng shui element of wood/fire/etc.}
            \end{ExplainCard}

            \begin{ExplainCard}{superstitious}[adj][B2]
            \EN{believing in and influenced by old traditions or magical ideas.}
            \VI{mê tín.}
            \SY{irrational; credulous}
            \EX{He is very superstitious about Friday 13th.}
            \EX{Some athletes are superstitious before matches.}
            \CO{superstitious belief/custom}
            \end{ExplainCard}

            \begin{ExplainCard}{tank-top}[n][A2]
            \EN{a sleeveless shirt with no collar.}
            \VI{áo ba lỗ.}
            \SY{sleeveless shirt; vest}
            \EX{He wore a tank-top to the gym.}
            \EX{She prefers tank-tops in the summer.}
            \CO{wear tank-tops}
            \end{ExplainCard}
        \end{VocabExplain}

    \noindent
    \textbf{Part 2.}
    \begin{qa}{Describe an interesting discussion you had about how you spend your money. You should say:}
    \begin{itemize}
        \item Who you had the discussion with
        \item Why you discussed this topic
        \item What the result of the discussion was
        \item and explain why this discussion was interesting for you.
    \end{itemize}

    Saving money is a good habit and I got some ideas about it. But the recent discussion about spending money had changed my former concept. I had a \textbf{fruitful discussion} with Mr Trung, who is a financial consultant in a \textbf{multinational} company and my Chinese teacher as well.  

    The \textbf{bottom line} was that I was always short of money although I had a decent job. We had a 2-hour conversation, and he asked me to \textbf{keep a tally of} all the expenses and income in a period of one week. After having a closer look at my personal \textbf{financial statement}, he analyzed that I did not set my priorities, so sometimes I \textbf{splashed out on} unnecessary things.  

    In addition, I made mistakes in money management, so sometimes my bank account is \textbf{badly in arrears}. Concerning the result of the discussion, he offered me some suggestions. First, it would be better if I could deposit money in the bank so that after a while, I would get a more significant sum thanks to the interest rates. Second, he advised me to limit the use of credit cards and \textbf{resort to} cash instead, which might prevent me from \textbf{running up a credit card bill}.  

    The discussion attracted me for its nature. I was unfamiliar with \textbf{putting money aside} and it is clear that I was in the wrong direction. In other words, I need to \textbf{save money for a rainy day}. Accordingly, I started following his advice and find everything right with me. Thanks to his advice, I did not \textbf{rack up a debt} any more. Thank you for listening!
    \end{qa}

        \begin{VocabExplain}[Part 2]
            \begin{ExplainCard}{fruitful discussion}[phrase][C1]
            \EN{a conversation that produces good results or useful ideas.}
            \VI{cuộc thảo luận hữu ích, hiệu quả.}
            \SY{productive; rewarding}
            \EX{We had a fruitful discussion about the project.}
            \EX{The meeting with my mentor was truly fruitful.}
            \CO{fruitful discussion/meeting/talk}
            \end{ExplainCard}

            \begin{ExplainCard}{multinational}[adj][B2]
            \EN{involving or operating in several countries.}
            \VI{đa quốc gia.}
            \SY{international; global}
            \EX{She works for a multinational corporation.}
            \EX{The multinational company expanded rapidly.}
            \CO{multinational company/corporation}
            \end{ExplainCard}

            \begin{ExplainCard}{the bottom line}[idiom][B2]
            \EN{the most important fact in a situation.}
            \VI{điểm mấu chốt.}
            \SY{the essence; the core}
            \EX{The bottom line is that we must save costs.}
            \EX{The bottom line is he was not qualified.}
            \CO{the bottom line is that + clause}
            \end{ExplainCard}

            \begin{ExplainCard}{keep a tally of}[phrase][C1]
            \EN{to keep a continuous count of something.}
            \VI{ghi lại, theo dõi số lượng.}
            \SY{track; record}
            \EX{He kept a tally of expenses during the trip.}
            \EX{You should keep a tally of your spending.}
            \CO{keep a tally of expenses/votes}
            \end{ExplainCard}

            \begin{ExplainCard}{financial statement}[n][C1]
            \EN{a record that shows the financial activities of a person or company.}
            \VI{báo cáo tài chính.}
            \SY{balance sheet; account statement}
            \EX{The company published its annual financial statement.}
            \EX{I reviewed my financial statement last week.}
            \CO{annual/quarterly financial statement}
            \end{ExplainCard}

            \begin{ExplainCard}{splash out on}[phr.v][B2]
            \EN{to spend a lot of money on something you do not really need.}
            \VI{vung tiền vào.}
            \SY{splurge on; squander}
            \EX{She splashed out on a luxury bag.}
            \EX{He splashed out on the latest phone.}
            \CO{splash out on sth}
            \end{ExplainCard}

            \begin{ExplainCard}{badly in arrears}[phrase][C1]
            \EN{to be very late in paying money that is owed.}
            \VI{nợ đọng nặng.}
            \SY{overdue; defaulting}
            \EX{His rent is badly in arrears.}
            \EX{The company is badly in arrears with payments.}
            \CO{be badly in arrears with sth}
            \end{ExplainCard}

            \begin{ExplainCard}{resort to}[phr.v][C1]
            \EN{to make use of something, especially something bad, because nothing else is possible.}
            \VI{phải dùng đến, viện đến.}
            \SY{fall back on; turn to}
            \EX{They resorted to violence during the protest.}
            \EX{I had to resort to borrowing money.}
            \CO{resort to violence/borrowing/cash}
            \end{ExplainCard}

            \begin{ExplainCard}{run up a credit card bill}[idiom][C1]
            \EN{to use a credit card so much that you owe a lot of money.}
            \VI{nợ chồng chất thẻ tín dụng.}
            \SY{accumulate debt; overspend}
            \EX{She ran up a huge credit card bill.}
            \EX{Don’t run up a credit card bill you can’t pay.}
            \CO{run up a bill/debt}
            \end{ExplainCard}

            \begin{ExplainCard}{put money aside}[phrase][B2]
            \EN{to save money regularly.}
            \VI{để dành tiền.}
            \SY{save; set aside}
            \EX{I put money aside each month for travel.}
            \EX{They put money aside for their child’s education.}
            \CO{put money aside for sth}
            \end{ExplainCard}

            \begin{ExplainCard}{save money for a rainy day}[idiom][B2]
            \EN{to save money for a time when it might be needed unexpectedly.}
            \VI{dành tiền phòng khi cần.}
            \SY{save for emergencies; keep savings}
            \EX{It’s wise to save money for a rainy day.}
            \EX{She saved money for a rainy day in case of illness.}
            \CO{save money for a rainy day}
            \end{ExplainCard}

            \begin{ExplainCard}{rack up a debt}[idiom][C1]
            \EN{to accumulate a lot of debt.}
            \VI{chất đống nợ nần.}
            \SY{pile up; accumulate}
            \EX{He racked up a huge debt at the casino.}
            \EX{They racked up debts on credit cards.}
            \CO{rack up a debt/deficit}
            \end{ExplainCard}
        \end{VocabExplain}

    \noindent
    \textbf{Part 3.}
    \begin{qa}{Why do some parents give their children money to spend each week?}
    When I was small, my parents seldom gave me any money, but things are different nowadays. Today, many families allow the children to \textbf{set aside} extra money for a \textbf{whole host of} reasons. For one, I guess the parents want their children to \textbf{get into the swing of} money management and how to spend it on a daily basis. Moreover, many students need pocket money to buy something like snacks whenever they \textbf{get the munchies} at school.
    \end{qa}

    \begin{qa}{Do you agree that schools should teach children how to manage money?}
    There is little \textbf{consensus} on the time that children should be taught money management at the early stage. In my opinion, schools could provide lessons of financial management skills as these skills are as essential as specialized knowledge. In fact, children who do not \textbf{get a handle on} the value of money would not cherish sustained efforts of their parents to \textbf{make ends’ meet} and, by and by, become \textbf{frugal} and fail to know how to \textbf{stand on their own feet}.
    \end{qa}

    \begin{qa}{Do you think it is a good idea for students to earn money while studying?}
    I think every coin has two sides. On the one hand, I guess a part-time job enables to learn \textbf{transferable skills} such as design and communication skills which are not officially taught at school. Besides, those students can live more independently and \textbf{pamper themselves} with small gifts sometimes. However, regardless of how much money students can earn, they cannot \textbf{disregard} their study. Too much time spent on generating extra income might be \textbf{at the expense of} school \textbf{underachievement}.
    \end{qa}

    \begin{qa}{Do you think it is true that in today’s society money cannot buy happiness?}
    No, not really. It is undeniable that \textbf{financial security} can protect people from emotional \textbf{turmoils}. In other words, money \textbf{literally} cannot buy happiness, but it bring cheers to people. All that said, there are ways people can spend their money which is more likely to lead to happiness and well-being. For example, buying new clothes or booking a holiday can all bring people joy. In general, money does not necessarily bring out happiness but it would be hard to achieve happiness in the absence of money.
    \end{qa}

    \begin{qa}{What disadvantages are there in a society where the gap between rich and poor is very large?}
    Unequal distribution of wealth is one of the common problems faced by developing nations. Economic \textbf{disparity} not only affects the lives of the people but also \textbf{hinders} the overall economic development of a nation. For example, not all people can \textbf{gain access to} adequate education, which would prevent the development of a \textbf{versatile} workforce and directly influence the economic well-being.
    \end{qa}

    \begin{qa}{Do you think richer countries have a responsibility to help poorer countries?}
    I completely agree that the governments in prosperous countries have an initial role to keep their citizens stay healthy and well-educated. Any social \textbf{chaos} in leading countries can lead to \textbf{domino effects} in \textbf{deprived} countries. On the other hand, there are still ways the authorities in richer countries should provide external assistance for the poorer ones without affecting their countries. For example, developed countries might \textbf{pitch in} to provide free vaccine to save the people’s lives in African countries. “There’s no such thing as a free lunch”. By lending a helping hand in the form of donating money to the poorer countries in need, the richer nations, especially their enterprises, might be given favorable conditions to invest in the poorer ones and yield \textbf{substantial} profit later on.
    \end{qa}

        \begin{VocabExplain}[Part 3]
            \begin{ExplainCard}{set aside}[phr.v][B2]
            \EN{to save money or time for a particular purpose.}
            \VI{để dành, dành dụm.}
            \SY{save; reserve}
            \EX{She set aside some money for her children's education.}
            \EX{The government set aside funds for disaster relief.}
            \CO{set aside money/time for sth}
            \end{ExplainCard}

            \begin{ExplainCard}{a whole host of}[phrase][C1]
            \EN{a large number of something.}
            \VI{rất nhiều, hàng loạt.}
            \SY{a multitude of; a wide range of}
            \EX{There are a whole host of reasons for this problem.}
            \EX{The new law faces a whole host of criticisms.}
            \CO{a whole host of reasons/proposals}
            \end{ExplainCard}

            \begin{ExplainCard}{get into the swing of}[idiom][C1]
            \EN{to become familiar with an activity and start enjoying it.}
            \VI{bắt nhịp, quen dần.}
            \SY{get used to; adapt to}
            \EX{He got into the swing of college life quickly.}
            \EX{After a few weeks, she got into the swing of her new job.}
            \CO{get into the swing of sth}
            \end{ExplainCard}

            \begin{ExplainCard}{get the munchies}[idiom][B2]
            \EN{to suddenly feel hungry, especially for snacks.}
            \VI{thèm ăn vặt.}
            \SY{feel peckish; crave snacks}
            \EX{I always get the munchies late at night.}
            \EX{Children often get the munchies after school.}
            \CO{get the munchies for sth}
            \end{ExplainCard}

            \begin{ExplainCard}{consensus}[n][C1]
            \EN{a general agreement among a group of people.}
            \VI{sự đồng thuận, nhất trí.}
            \SY{agreement; harmony}
            \EX{There is a consensus among experts on this issue.}
            \EX{We reached a broad consensus about the plan.}
            \CO{reach consensus; general consensus}
            \end{ExplainCard}

            \begin{ExplainCard}{get a handle on}[idiom][C1]
            \EN{to understand or control something.}
            \VI{nắm bắt, kiểm soát.}
            \SY{grasp; gain control of}
            \EX{I can’t get a handle on this new software.}
            \EX{Parents must get a handle on their children’s spending.}
            \CO{get a handle on sth}
            \end{ExplainCard}

            \begin{ExplainCard}{make ends meet}[idiom][B2]
            \EN{to earn just enough money to live on.}
            \VI{đủ sống, xoay sở đủ sống.}
            \SY{get by; survive financially}
            \EX{He had to work two jobs to make ends meet.}
            \EX{With rising costs, many families struggle to make ends meet.}
            \CO{struggle to make ends meet}
            \end{ExplainCard}

            \begin{ExplainCard}{frugal}[adj][C1]
            \EN{careful to use only as much money, food, etc. as necessary.}
            \VI{tiết kiệm, thanh đạm.}
            \SY{economical; thrifty}
            \EX{They lead a frugal lifestyle.}
            \EX{He was frugal with his money.}
            \CO{frugal lifestyle/habits}
            \end{ExplainCard}

            \begin{ExplainCard}{stand on one’s own feet}[idiom][C1]
            \EN{to be independent and not rely on others.}
            \VI{tự lập, độc lập.}
            \SY{be self-reliant; be independent}
            \EX{Children must learn to stand on their own feet.}
            \EX{She stood on her own feet after moving abroad.}
            \CO{learn to stand on one’s own feet}
            \end{ExplainCard}

            \begin{ExplainCard}{transferable skills}[n][C1]
            \EN{skills that can be applied to different jobs or situations.}
            \VI{kỹ năng có thể chuyển giao.}
            \SY{portable skills; adaptable skills}
            \EX{Communication is one of the most valuable transferable skills.}
            \EX{He developed transferable skills in his part-time job.}
            \CO{gain/develop transferable skills}
            \end{ExplainCard}

            \begin{ExplainCard}{pamper oneself}[phrase][C1]
            \EN{to treat oneself indulgently and luxuriously.}
            \VI{chiều chuộng bản thân.}
            \SY{indulge; spoil}
            \EX{She pampered herself with a spa day.}
            \EX{He likes to pamper himself after exams.}
            \CO{pamper oneself with sth}
            \end{ExplainCard}

            \begin{ExplainCard}{disregard}[v][C1]
            \EN{to ignore something important.}
            \VI{phớt lờ, bỏ qua.}
            \SY{neglect; overlook}
            \EX{He disregarded the doctor’s advice.}
            \EX{The company disregarded safety standards.}
            \CO{disregard rules/warnings}
            \end{ExplainCard}

            \begin{ExplainCard}{at the expense of}[idiom][C1]
            \EN{resulting in the loss of something else.}
            \VI{đánh đổi, gây thiệt hại cho.}
            \SY{to the detriment of; at the cost of}
            \EX{He succeeded at the expense of his health.}
            \EX{Economic growth came at the expense of the environment.}
            \CO{at the expense of sth}
            \end{ExplainCard}

            \begin{ExplainCard}{underachievement}[n][C1]
            \EN{less success than expected.}
            \VI{sự kém thành tích, chưa đạt được kỳ vọng.}
            \SY{low performance; failure}
            \EX{There is evidence of underachievement among students.}
            \EX{The team suffered from consistent underachievement.}
            \CO{academic underachievement}
            \end{ExplainCard}

            \begin{ExplainCard}{financial security}[n][C1]
            \EN{the state of having enough money to cover needs and emergencies.}
            \VI{sự an toàn tài chính.}
            \SY{financial stability; economic safety}
            \EX{Savings give families financial security.}
            \EX{He invested for long-term financial security.}
            \CO{achieve/ensure financial security}
            \end{ExplainCard}

            \begin{ExplainCard}{turmoil}[n][C1]
            \EN{a state of confusion or disorder.}
            \VI{sự hỗn loạn, bất ổn.}
            \SY{chaos; unrest}
            \EX{The country is in political turmoil.}
            \EX{She was in emotional turmoil after the breakup.}
            \CO{political/economic turmoil}
            \end{ExplainCard}

            \begin{ExplainCard}{disparity}[n][C1]
            \EN{a great difference or inequality.}
            \VI{sự chênh lệch.}
            \SY{inequality; imbalance}
            \EX{There is a disparity between rich and poor.}
            \EX{The study showed regional disparities in income.}
            \CO{economic disparity; income disparity}
            \end{ExplainCard}

            \begin{ExplainCard}{versatile}[adj][C1]
            \EN{able to do many things; adaptable.}
            \VI{đa năng, linh hoạt.}
            \SY{flexible; all-round}
            \EX{She is a versatile actress.}
            \EX{Versatile workers are more employable.}
            \CO{versatile skills/worker}
            \end{ExplainCard}

            \begin{ExplainCard}{chaos}[n][C1]
            \EN{a state of complete disorder.}
            \VI{hỗn loạn.}
            \SY{confusion; turmoil}
            \EX{The protest ended in chaos.}
            \EX{Economic chaos followed the war.}
            \CO{political/social/economic chaos}
            \end{ExplainCard}

            \begin{ExplainCard}{domino effect}[idiom][C1]
            \EN{a situation where one event causes a chain of similar events.}
            \VI{hiệu ứng dây chuyền.}
            \SY{chain reaction; ripple effect}
            \EX{The strike had a domino effect on production.}
            \EX{Economic crises can create domino effects globally.}
            \CO{a domino effect on sth}
            \end{ExplainCard}

            \begin{ExplainCard}{deprived}[adj][C1]
            \EN{lacking the basic necessities of life.}
            \VI{túng thiếu, nghèo khó.}
            \SY{poor; disadvantaged}
            \EX{The charity helps deprived children.}
            \EX{He grew up in a deprived neighborhood.}
            \CO{deprived area/community}
            \end{ExplainCard}

            \begin{ExplainCard}{pitch in}[phr.v][C1]
            \EN{to join others in giving help.}
            \VI{chung tay giúp sức.}
            \SY{contribute; lend a hand}
            \EX{Everyone pitched in to clean the park.}
            \EX{Countries should pitch in to fight climate change.}
            \CO{pitch in to do sth}
            \end{ExplainCard}

            \begin{ExplainCard}{substantial}[adj][C1]
            \EN{large in amount, value, or importance.}
            \VI{đáng kể, lớn lao.}
            \SY{considerable; significant}
            \EX{They made a substantial profit.}
            \EX{Substantial evidence supports the claim.}
            \CO{substantial profit/increase/amount}
            \end{ExplainCard}
        \end{VocabExplain}

    \begin{VocabHighlights}
        \VH{to have an eye for}{(idiom) be able to recognize, appreciate, and make good judgments about}{(thành ngữ) có con mắt tinh đời về việc gì}
        \VH{to splash out on}{(phr.v) to spend a lot of money on something}{(cụm động từ) tiêu nhiều tiền vào việc gì}
        \VH{to scour}{(v) to search a place or thing thoroughly in order to find somebody/something}{(động từ) sục sạo, tìm kiếm}
        \VH{to be dressed to the nines}{(idiom) to be dressed up}{(thành ngữ) ăn mặc tử tế}
        \VH{to garner one’s interests}{(phrase) to draw one’s attention}{(cụm từ) thu hút sự chú ý của ai}
        \VH{pristine}{(adj) fresh and clean, as if new}{(tính từ) như mới}
        \VH{to economize on}{(v) to use less money, time, etc. than you normally use}{(động từ) tiết kiệm}
        \VH{in lieu of}{(phrase) instead of}{(cụm từ) thay vì}
        \VH{to shell out money for}{(phr.v) to pay a lot of money for something}{(cụm động từ) trả nhiều tiền cho}
        \VH{to be dead keen on}{(phrase) to be very keen on}{(cụm từ) cực kì mê}
        \VH{feng shui element of wood in Chinese zodiac}{(phrase) an expression in the form of Wood of the Chi energy that governs every activity on Earth}{(cụm từ) mang Mộc}
        \VH{superstitious}{(adj) believing in superstitions}{(tính từ) mê tín}
        \VH{tank-top}{(n) a piece of clothing like a T-shirt without sleeves}{(danh từ) áo ba lỗ}
        \VH{fruitful}{(adj) producing good results}{(tính từ) hiệu quả}
        \VH{multinational}{(adj) involving several different countries}{(tính từ) đa quốc gia}
        \VH{the bottom line}{(idiom) the main issue}{(thành ngữ) vấn đề cốt lõi}
        \VH{to keep a tally of}{(phrase) keep a record of something}{(cụm từ) giữ một bản ghi chép lại}
        \VH{financial statement}{(phrase) a report provided by a company for its shareholders and investors}{(cụm từ) bản báo cáo tài chính}
        \VH{in arrears}{(idiom) money that is owed and should already have been paid}{(thành ngữ) nợ tiền}
        \VH{to resort to}{(v) to do something that you do not want to do because you cannot find any other way}{(động từ) buộc phải dùng}
        \VH{to run up}{(phr.v) to increase a debt by spending more}{(cụm động từ) tăng nợ vì chi lắm}
        \VH{to put money aside}{(phr.v) to save something, usually time or money, for a special purpose}{(cụm động từ) tiết kiệm, để dành tiền}
        \VH{to rack up a debt}{(phrase) increase your debt}{(cụm từ) tăng nợ nần}
        \VH{to set aside}{(v) to save for a particular purpose}{(động từ) tiết kiệm}
        \VH{to get into the swing of}{(phrase) to start to understand, enjoy, and be active in something}{(cụm từ) bắt đầu hiểu, thích, quen}
        \VH{to get the munchies}{(idiom) to become insatiably hungry so that one has the urge to continuously eat, especially snack foods}{(thành ngữ) đói bụng}
        \VH{consensus}{(n) an opinion that all members of a group agree with}{(danh từ) sự đồng thuận}
        \VH{to get a handle on}{(idiom) to understand something well}{(thành ngữ) hiểu rõ cái gì}
        \VH{to make ends meet}{(idiom) earn enough money to live without getting into debt}{(thành ngữ) kiếm sống}
        \VH{frugal}{(adj) using only as much money or food as is necessary}{(tính từ) biết tính toán, chi tiêu}
        \VH{to stand on somebody’s own feet}{(phrase) be or become self-reliant or independent}{(cụm từ) sống độc lập}
        \VH{transferable}{(adj) able to be moved from one place or situation to another}{(tính từ) có thể dịch chuyển}
        \VH{to pamper somebody with something}{(phrase) to treat with extreme or excessive care and attention}{(cụm từ) nuông chiều}
        \VH{to disregard}{(v) to not consider something; to treat something as unimportant}{(động từ) xem thường, không quan tâm}
        \VH{at the expense of}{(phrase) the second thing suffers or is not done properly because of the first}{(cụm từ) đánh đổi bằng}
        \VH{underachievement}{(n) the fact of doing less well than expected, especially in schoolwork}{(danh từ) sự học sút kém đi}
        \VH{financial security}{(phrase) refers to the peace of mind you feel when you aren’t worried about your income being enough to cover your expenses}{(cụm từ) an toàn về tài chính}
        \VH{turmoil}{(n) a state of confusion, uncertainty, or disorder}{(danh từ) rối loạn}
        \VH{literally}{(adv) used to emphasize the truth of something that may seem surprising}{(trạng từ) thật sự, rõ ràng}
        \VH{disparity}{(n) a lack of equality or similarity, especially in a way that is not fair}{(danh từ) thiếu bình đẳng}
        \VH{to hinder}{(v) to make it difficult for somebody to do something or for something to happen}{(động từ) cản trở}
        \VH{adequate}{(adj) enough in quantity, or good enough in quality, for a particular purpose or need}{(tính từ) đầy đủ}
        \VH{versatile}{(adj) able to do many different things}{(tính từ) giỏi giang, đa năng}
        \VH{prosperous}{(adj) wealthy, successful}{(tính từ) thịnh vượng}
        \VH{chaos}{(n) a state of total confusion with no order}{(danh từ) hỗn loạn}
        \VH{domino effect}{(n) the situation in which something, usually something bad, happens, causing other similar events to happen}{(danh từ) hiệu ứng domino}
        \VH{deprived}{(adj) not having the things that are necessary for a pleasant life, such as enough money, food, or good living conditions}{(tính từ) túng thiếu, nghèo đói}
        \VH{to pitch in}{(phr.v) to vigorously join in to help with a task or activity}{(cụm động từ) cùng chung tay làm gì}
    \end{VocabHighlights}
    \end{test}

    \begin{test}{TEST 4}
    \noindent
    \textbf{Part 1. Art}
    \begin{qa}{Did you enjoy doing art lessons when you were a child?}
    I have to admit that I was not in favor of joining art lessons when I attended school a long time ago. It was just that I couldn’t draw what I liked, namely manga characters, because the whole lesson is \textbf{confined to} traditional forms only. For instance, drawing something \textbf{immobile} like a vase was not my interest at all.
    \end{qa}

    \begin{qa}{Do you ever draw or paint pictures now?}
    No, indeed. The last time I drew something dated back to more than 12 years ago. I’m not \textbf{dexterous} enough to paint anything. In other words, drawing a picture is never my \textbf{speciality}.
    \end{qa}

    \begin{qa}{When was the last time you went to an art gallery or exhibition?}
    In 2015, I happened to drop by the Louvre museum when I was on a trip around Europe. Although I was not particularly interested in any art-related items, the Louvre museum was definitely on top of my list as it is always considered a \textbf{must-see place} of interests in Paris. It was so huge a museum that I didn’t manage to walk to \textbf{every nook and cranny} of it in a single afternoon. I could only \textbf{scratch the surface of} it before it was closed later.
    \end{qa}

    \begin{qa}{What kind of pictures do you like having in your home?}
    Although I am not in favor of artistic works, I still hang some pictures around my house. They are canvas pictures with motivational slogans like “Home is where the love is”, “Love makes a house a home” and “Inhale the future, exhale the past”. These slogans succeed in turning my house into a cozy and loving home for me.
    \end{qa}

        \begin{VocabExplain}[Part 1]
            \begin{ExplainCard}{confined to}[phr.v][C1]
            \EN{restricted to a particular group or activity.}
            \VI{giới hạn, bó hẹp.}
            \SY{restricted to; limited to}
            \EX{The illness is not confined to any one group.}
            \EX{Her interest is confined to classical music.}
            \CO{confined to sth}
            \end{ExplainCard}

            \begin{ExplainCard}{immobile}[adj][C1]
            \EN{not moving; unable to move.}
            \VI{bất động.}
            \SY{motionless; still}
            \EX{The accident left him immobile.}
            \EX{An immobile figure stood in the doorway.}
            \CO{remain immobile; completely immobile}
            \end{ExplainCard}

            \begin{ExplainCard}{dexterous}[adj][C2]
            \EN{showing skill, especially with hands.}
            \VI{khéo léo, lành nghề.}
            \SY{skillful; adept}
            \EX{He was dexterous at using chopsticks.}
            \EX{The dexterous artist painted with both hands.}
            \CO{dexterous fingers; dexterous movement}
            \end{ExplainCard}

            \begin{ExplainCard}{speciality}[n][B2]
            \EN{a subject or skill you know a lot about or have responsibility for.}
            \VI{chuyên môn, điểm mạnh.}
            \SY{expertise; forte}
            \EX{Her speciality is portrait painting.}
            \EX{Cooking seafood is his speciality.}
            \CO{area of speciality; medical speciality}
            \end{ExplainCard}

            \begin{ExplainCard}{must-see}[adj][B2]
            \EN{something that should not be missed.}
            \VI{đáng xem, không thể bỏ qua.}
            \SY{unmissable; essential}
            \EX{The Eiffel Tower is a must-see for visitors to Paris.}
            \EX{This film is a must-see for action lovers.}
            \CO{must-see attraction/event}
            \end{ExplainCard}

            \begin{ExplainCard}{every nook and cranny}[idiom][C1]
            \EN{every part of a place; every small detail.}
            \VI{mọi ngóc ngách.}
            \SY{every corner; every inch}
            \EX{Dust was in every nook and cranny of the house.}
            \EX{They searched every nook and cranny for the missing key.}
            \CO{explore every nook and cranny}
            \end{ExplainCard}

            \begin{ExplainCard}{scratch the surface of}[idiom][C1]
            \EN{to deal with only a small part of a subject or problem.}
            \VI{mới chỉ động chạm sơ qua, chưa tìm hiểu sâu.}
            \SY{touch upon; barely address}
            \EX{We only scratched the surface of this complex issue.}
            \EX{His knowledge only scratches the surface of the subject.}
            \CO{scratch the surface of sth}
            \end{ExplainCard}
        \end{VocabExplain}

    \noindent
    \textbf{Part 2.}
    \begin{qa}{Describe a time when you visited a friend or family member at their workplace. You should say:}
    \begin{itemize}
        \item Who you visited
        \item Where this person worked
        \item Why you visited this person’s workplace
        \item and explain how you felt about visiting this person’s workplace.
    \end{itemize}

    Today, I’m going to tell you one of my unforgettable memories in my life. It is a time when I attended a party at my father’s company. Talking about my father, he is 46 years old, and he is in charge of accounting and finance in a multinational company.  

    To be honest, I don’t know why its name \textbf{slips my mind} now, but what I can tell you is that there are over 300 employees and its office is located on Nguyen Dinh Chieu Street. It mostly \textbf{deals with} import and export of a wide variety of goods, including cosmetics and medicine.  

    So far he has worked for this company for more than 4 years. Last year, he \textbf{was promoted to} a senior position. To be exact, it was the financial manager thanks to his great contribution to the company. To celebrate his milestone, his company \textbf{pushed the boat out}, and my family was invited to attend it for a surprise.  

    This was the first time I had \textbf{paid a visit to} his company, and I was \textbf{overwhelmed with} eagerness. The people there were hospitable, and they were hilarious as well. I was lucky to have an opportunity to have a person \textbf{show me around} the office. I was also fascinated by people’s enthusiasm, so I believe that it was a \textbf{dynamic working environment}. I hope that I would have a chance to work for this company when I \textbf{come fresh from} university.
    \end{qa}

        \begin{VocabExplain}[Part 2]
            \begin{ExplainCard}{slip one’s mind}[idiom][C1]
            \EN{to be forgotten or not remembered.}
            \VI{quên mất, thoát khỏi trí nhớ.}
            \SY{forget; overlook}
            \EX{Her birthday completely slipped my mind.}
            \EX{I meant to call you, but it slipped my mind.}
            \CO{completely slip one’s mind; suddenly slip one’s mind}
            \end{ExplainCard}

            \begin{ExplainCard}{deal with}[phr.v][B2]
            \EN{to take action to do something, especially to solve a problem or manage work.}
            \VI{xử lý, giải quyết, liên quan đến.}
            \SY{handle; manage}
            \EX{He deals with customer complaints daily.}
            \EX{This department deals with exports.}
            \CO{deal with a problem/situation}
            \end{ExplainCard}

            \begin{ExplainCard}{be promoted to}[phr.v][B2]
            \EN{to be given a higher and more important position.}
            \VI{được thăng chức.}
            \SY{advance; elevate}
            \EX{She was promoted to manager after two years.}
            \EX{He was promoted to director at a young age.}
            \CO{be promoted to manager/director}
            \end{ExplainCard}

            \begin{ExplainCard}{push the boat out}[idiom][C2]
            \EN{to spend a lot of money on celebrating something.}
            \VI{ăn mừng linh đình, chi tiêu lớn cho dịp đặc biệt.}
            \SY{celebrate lavishly; splash out}
            \EX{They really pushed the boat out for their wedding.}
            \EX{The company pushed the boat out for its 50th anniversary.}
            \CO{push the boat out for sth}
            \end{ExplainCard}

            \begin{ExplainCard}{pay a visit to}[phrase][B2]
            \EN{to visit someone or somewhere.}
            \VI{ghé thăm.}
            \SY{visit; drop by}
            \EX{We paid a visit to our grandparents last weekend.}
            \EX{She paid a visit to the local museum.}
            \CO{pay a visit to sb/sth}
            \end{ExplainCard}

            \begin{ExplainCard}{overwhelmed with}[adj][C1]
            \EN{feeling sudden and intense emotion.}
            \VI{choáng ngợp bởi (cảm xúc).}
            \SY{stunned; overcome}
            \EX{He was overwhelmed with gratitude.}
            \EX{I felt overwhelmed with excitement.}
            \CO{overwhelmed with joy/emotion}
            \end{ExplainCard}

            \begin{ExplainCard}{show sb around}[phr.v][B1]
            \EN{to take someone on a tour of a place.}
            \VI{dẫn đi tham quan.}
            \SY{guide; take on a tour}
            \EX{He showed us around the new office.}
            \EX{She showed me around the campus.}
            \CO{show around the city/school/office}
            \end{ExplainCard}

            \begin{ExplainCard}{dynamic working environment}[phrase][C1]
            \EN{a workplace full of energy, new ideas, and constant change.}
            \VI{môi trường làm việc năng động.}
            \SY{energetic workplace; innovative environment}
            \EX{A dynamic working environment fosters creativity.}
            \EX{Graduates seek jobs in dynamic working environments.}
            \CO{work in a dynamic environment}
            \end{ExplainCard}

            \begin{ExplainCard}{come fresh from}[phrase][C1]
            \EN{to arrive directly from an experience or event, usually new or inexperienced.}
            \VI{vừa mới rời khỏi, mới tốt nghiệp, còn non kinh nghiệm.}
            \SY{newly graduated; inexperienced}
            \EX{She came fresh from university into her first job.}
            \EX{The actor came fresh from drama school.}
            \CO{come fresh from university/school}
            \end{ExplainCard}
        \end{VocabExplain}

    \noindent
    \textbf{Part 3.}
    \begin{qa}{What things make an office comfortable to work in?}
    There are a lot of things that make a \textbf{congenial} workplace. Firstly, \textbf{job satisfaction} depends on the colleague that people cooperate with. I believe an \textbf{armchair critic} or a \textbf{big mouth} are those who people find it hard to accommodate. More importantly, workplace utility is another judging factor. A lack of lighting or computers can \textbf{hamper} the productivity of employees.
    \end{qa}

    \begin{qa}{Why do some people prefer to work outdoors?}
    I would like to start off by saying that outdoor workplace could can \textbf{nourish} creativity. Working indoors cannot \textbf{hold a candle to} working outside which creates more \textbf{laid-back} atmosphere and makes people feel more relaxed while they are working.
    \end{qa}

    \begin{qa}{Do you agree that the building people work in is more important than the colleagues they work with?}
    Personally, I would value the colleagues above the workplace. This does not mean a \textbf{stuffy and compact} office is what I am talking about, but it must be a \textbf{well-furnished} one. On top of that, co-workers and office relationships will have greater effects on the way we make decisions. Supportive and \textbf{down-to-earth} people are factors which boost our proficiency. By contrast, a \textbf{conservative} partner can adversely affect our projects.
    \end{qa}

    \begin{qa}{What would life be like if people didn’t have to work?}
    Well, I cannot imagine the day people stop working would be like, whether it might be the day when robotics can be seen as a sort of \textbf{surrogate} human beings or not. If this is the case, I guess we can become \textbf{couch potatoes}, and everything could be handled automatically. This scenario is too \textbf{implausible}, and human cannot make headway in that sense.
    \end{qa}

    \begin{qa}{Are all jobs of equal importance?}
    For the most part, every job is equally beneficial for the society as a whole but not all. Firstly, all jobs could \textbf{theoretically} be seen as equal. After all, they all \textbf{contribute} in some way to our life based on its function. However, some jobs \textbf{ultimately} require more training or experience than others, which means that they are not only \textbf{tough going} but the level of expertise required to perform these tasks makes specialised labour less common and increases its value.
    \end{qa}

    \begin{qa}{Why do some people become workaholics?}
    Basically, workaholics are those who \textbf{exclusively} devote their time to working and working again. Most workaholics will spend hours working tirelessly to ensure they receive recognition and high esteem above others. Moreover, long time dedication is often associated with other welfares such as generous \textbf{remuneration} or promotion. Financial gain thanks to working overtime is worthy of being mentioned. The downside is that workaholics usually refuse \textbf{holiday entitlements}, which can destroy their health \textbf{within moments}.
    \end{qa}

        \begin{VocabExplain}[Part 3]
            \begin{ExplainCard}{congenial}[adj][C1]
            \EN{pleasant and friendly; making you feel comfortable.}
            \VI{dễ chịu, thoải mái, hợp ý.}
            \SY{agreeable; pleasant}
            \EX{She found the work congenial.}
            \EX{A congenial office boosts productivity.}
            \CO{congenial atmosphere/workplace}
            \end{ExplainCard}

            \begin{ExplainCard}{job satisfaction}[n][C1]
            \EN{the good feeling you get when you enjoy your job and feel it is worth doing.}
            \VI{sự hài lòng trong công việc.}
            \SY{fulfillment; gratification}
            \EX{He left the company because of low job satisfaction.}
            \EX{High job satisfaction reduces turnover rates.}
            \CO{increase job satisfaction; job satisfaction level}
            \end{ExplainCard}

            \begin{ExplainCard}{armchair critic}[idiom][C1]
            \EN{someone who knows or says they know a lot about something without having experience of it.}
            \VI{người “chém gió”, chỉ trích mà không có kinh nghiệm thực tế.}
            \SY{theorist; commentator}
            \EX{He’s just an armchair critic who has never worked in the field.}
            \EX{Armchair critics rarely contribute solutions.}
            \CO{mere armchair critic}
            \end{ExplainCard}

            \begin{ExplainCard}{big mouth}[idiom][C1]
            \EN{someone who talks too much and cannot keep secrets.}
            \VI{người nhiều chuyện, không giữ được bí mật.}
            \SY{blabbermouth; chatterbox}
            \EX{Don’t tell her anything—she’s got a big mouth.}
            \EX{Having a big mouth may damage trust at work.}
            \CO{have a big mouth; be such a big mouth}
            \end{ExplainCard}

            \begin{ExplainCard}{hamper}[v][C1]
            \EN{to make it difficult for someone to do something.}
            \VI{cản trở, gây khó khăn.}
            \SY{hinder; obstruct}
            \EX{High winds hampered the rescue attempt.}
            \EX{Lack of funding hampered the project’s progress.}
            \CO{hamper progress/development}
            \end{ExplainCard}

            \begin{ExplainCard}{nourish}[v][C1]
            \EN{to help an idea, feeling, or skill to develop.}
            \VI{nuôi dưỡng (ý tưởng, sáng tạo).}
            \SY{foster; nurture}
            \EX{Reading nourishes imagination.}
            \EX{Outdoor work nourishes creativity.}
            \CO{nourish talent/creativity}
            \end{ExplainCard}

            \begin{ExplainCard}{hold a candle to}[idiom][C1]
            \EN{to be equal to someone or something else in quality.}
            \VI{so sánh được, sánh ngang với.}
            \SY{measure up to; rival}
            \EX{No one can hold a candle to her in singing.}
            \EX{Working indoors cannot hold a candle to outdoor creativity.}
            \CO{not hold a candle to}
            \end{ExplainCard}

            \begin{ExplainCard}{laid-back}[adj][C1]
            \EN{relaxed and not easily worried.}
            \VI{thư giãn, thoải mái.}
            \SY{easy-going; carefree}
            \EX{He’s a laid-back boss who rarely gets angry.}
            \EX{A laid-back workplace atmosphere reduces stress.}
            \CO{laid-back attitude/lifestyle}
            \end{ExplainCard}

            \begin{ExplainCard}{stuffy}[adj][C1]
            \EN{(of a room) lacking fresh air; unpleasant to breathe in.}
            \VI{ngột ngạt, bí bách.}
            \SY{airless; suffocating}
            \EX{The room was hot and stuffy.}
            \EX{A stuffy office lowers productivity.}
            \CO{stuffy atmosphere/office}
            \end{ExplainCard}

            \begin{ExplainCard}{compact}[adj][C1]
            \EN{small but arranged well or efficiently.}
            \VI{nhỏ gọn, tiện lợi.}
            \SY{concise; efficient}
            \EX{She lives in a compact apartment.}
            \EX{The office is compact but comfortable.}
            \CO{compact office/structure}
            \end{ExplainCard}

            \begin{ExplainCard}{well-furnished}[adj][B2]
            \EN{containing good or a lot of furniture.}
            \VI{được trang bị đầy đủ nội thất.}
            \SY{well-equipped; well-appointed}
            \EX{It was a spacious, well-furnished office.}
            \EX{Well-furnished rooms improve comfort.}
            \CO{well-furnished office/room}
            \end{ExplainCard}

            \begin{ExplainCard}{down-to-earth}[adj][C1]
            \EN{practical, reasonable, and friendly.}
            \VI{thực tế, thân thiện.}
            \SY{realistic; sensible}
            \EX{She’s very down-to-earth and easy to talk to.}
            \EX{Down-to-earth managers gain employees’ trust.}
            \CO{down-to-earth attitude/personality}
            \end{ExplainCard}

            \begin{ExplainCard}{conservative}[adj][C1]
            \EN{not willing to accept changes or new ideas.}
            \VI{bảo thủ.}
            \SY{traditional; conventional}
            \EX{He has conservative views on education.}
            \EX{A conservative partner may resist innovation.}
            \CO{conservative attitude/approach}
            \end{ExplainCard}

            \begin{ExplainCard}{surrogate}[n][C2]
            \EN{a substitute, especially a person acting for someone else.}
            \VI{người/vật thay thế.}
            \SY{substitute; proxy}
            \EX{She saw the dog as a surrogate for her child.}
            \EX{Robots may serve as surrogates for workers.}
            \CO{surrogate role/mother}
            \end{ExplainCard}

            \begin{ExplainCard}{couch potato}[idiom][C1]
            \EN{a person who spends a lot of time sitting and watching TV.}
            \VI{người lười biếng, suốt ngày ngồi xem TV.}
            \SY{idler; loafer}
            \EX{He’s turned into a real couch potato since retiring.}
            \EX{A work-free society may create couch potatoes.}
            \CO{become a couch potato}
            \end{ExplainCard}

            \begin{ExplainCard}{implausible}[adj][C1]
            \EN{difficult to believe; not reasonable.}
            \VI{không hợp lý, khó tin.}
            \SY{unlikely; unreasonable}
            \EX{His excuse was implausible.}
            \EX{The scenario of a world without work is implausible.}
            \CO{implausible explanation/story}
            \end{ExplainCard}

            \begin{ExplainCard}{theoretically}[adv][C1]
            \EN{in theory but perhaps not in reality.}
            \VI{về lý thuyết.}
            \SY{in principle; hypothetically}
            \EX{Theoretically, the plan should work.}
            \EX{Theoretically, all jobs are equal, but in reality, they differ.}
            \CO{theoretically possible/valid}
            \end{ExplainCard}

            \begin{ExplainCard}{tough-going}[adj][C1]
            \EN{difficult, requiring effort.}
            \VI{khó khăn, vất vả.}
            \SY{arduous; demanding}
            \EX{The negotiations were tough-going.}
            \EX{Some jobs are tough-going but rewarding.}
            \CO{tough-going task/work}
            \end{ExplainCard}

            \begin{ExplainCard}{exclusively}[adv][C1]
            \EN{only; and not shared with others.}
            \VI{dành riêng, độc quyền.}
            \SY{solely; uniquely}
            \EX{This offer is exclusively for new members.}
            \EX{Workaholics devote their time exclusively to work.}
            \CO{exclusively focus on}
            \end{ExplainCard}

            \begin{ExplainCard}{remuneration}[n][C2]
            \EN{payment or reward for work or service.}
            \VI{tiền thù lao, tiền công.}
            \SY{payment; compensation}
            \EX{She received high remuneration for her work.}
            \EX{Remuneration packages include bonuses and insurance.}
            \CO{remuneration package/level}
            \end{ExplainCard}

            \begin{ExplainCard}{holiday entitlement}[n][C1]
            \EN{the number of days that a worker is officially allowed as holiday.}
            \VI{chế độ nghỉ phép.}
            \SY{leave; holiday allowance}
            \EX{The job offers 25 days’ holiday entitlement a year.}
            \EX{Refusing holiday entitlement can harm health.}
            \CO{statutory holiday entitlement}
            \end{ExplainCard}

            \begin{ExplainCard}{within moments}[phrase][C1]
            \EN{in a very short time.}
            \VI{trong chốc lát.}
            \SY{instantly; immediately}
            \EX{The building collapsed within moments.}
            \EX{Workaholics may destroy their health within moments.}
            \CO{happen within moments}
            \end{ExplainCard}
        \end{VocabExplain}

    \begin{VocabHighlights}
        \VH{to be confined to}{(p2) to be kept inside the limits of a particular activity}{(phần từ 2) bị giữ lại, giới hạn}
        \VH{immobile}{(adj) unable to move}{(tính từ) tĩnh}
        \VH{dexterous}{(adj) skilful with your hands; skilfully done}{(tính từ) khéo tay}
        \VH{speciality}{(n) a pursuit, area of study, or skill to which someone has devoted time and effort and in which they are expert}{(danh từ) lĩnh vực chuyên sâu, thế mạnh}
        \VH{must-see}{(adj) highly recommended as worth seeing}{(tính từ) cần chắc chắn phải xem}
        \VH{every nook and cranny}{(idiom) every part or aspect of something}{(thành ngữ) mọi ngóc ngách}
        \VH{to scratch the surface of}{(idiom) to visit a place briefly}{(thành ngữ) lướt qua nơi nào đó, cưỡi ngựa xem hoa}
        \VH{to slip my mind}{(idiom) forget}{(thành ngữ) quên bẵng mất}
        \VH{to deal with}{(phr.v) to cope with}{(cụm động từ) đối phó với}
        \VH{to be promoted to}{(p2) get higher position}{(phần từ 2) thăng chức}
        \VH{to push the boat out}{(idiom) to spend a lot of money on celebrating something}{(thành ngữ) dành nhiều tiền tổ chức ăn mừng gì}
        \VH{to pay a visit to}{(phrase) spend a journey time to go somewhere}{(cụm từ) dành một chuyến đi đến}
        \VH{to be overwhelmed with}{(p2) feeling too many emotions right now}{(phần từ 2) choáng ngợp trong cảm xúc}
        \VH{to show around}{(phr.v) to take someone to all parts, or the main parts, of a place that they have not visited before, so that they can see what it is like or learn about it}{(cụm động từ) dẫn tham quan xung quanh}
        \VH{to be dynamic}{(adj) positive in attitude and full of energy and new idea}{(tính từ) năng động}
        \VH{fresh from (=fresh out of)}{(idiom) having just finished education or training, not having a lot of experience}{(thành ngữ) vừa mới học xong, chưa dạn dĩ chân ráo}
        \VH{congenial}{(adj) pleasant to spend time with because their interests and character are similar to your own}{(tính từ) thoải mái}
        \VH{job satisfaction}{(n) a measure of workers’ contentedness with their job}{(danh từ) sự hài lòng trong công việc}
        \VH{an armchair critic}{(n) a person who knows or pretends to know a lot about something in theory rather than practice}{(danh từ) người nói lý thuyết suông}
        \VH{a big mouth}{(n) an indiscreet or boastful person}{(danh từ) người không đáng tin}
        \VH{hamper}{(v) to prevent somebody from easily doing or achieving something}{(động từ) cản trở}
        \VH{nourish}{(v) to keep a person, an animal or a plant alive and healthy with food, etc}{(động từ) nuôi dưỡng}
        \VH{laid-back}{(adj) relaxed in manner and character}{(tính từ) thong thả}
        \VH{stuffy}{(adj) warm in an unpleasant way and without enough fresh air}{(tính từ) ngột ngạt}
        \VH{compact}{(adj) smaller than is usual for things of the same kind}{(tính từ) nhỏ gọn}
        \VH{down-to-earth}{(adj) practical, reasonable and friendly}{(tính từ) thân thiện, hòa đồng}
        \VH{conservative}{(adj) opposed to great or sudden social change; showing that you prefer traditional styles and values}{(tính từ) bảo thủ}
        \VH{surrogate}{(n) something that replaces or is used instead of something else}{(danh từ) sự thay thế}
        \VH{a couch potato}{(n) a person who spends little or no time exercising \& a great deal of time watching TV}{(danh từ) người lười biếng}
        \VH{implausible}{(adj) not seeming reasonable or likely to be true}{(tính từ) vô lý, khó tin}
        \VH{to make headway in/with}{(idiom) to move forward or make progress}{(thành ngữ) tiến bộ, tiến bước lên}
        \VH{theoretically}{(adv) in a way that is concerned with the ideas and principles on which a particular subject is based, rather than with practice and experiment}{(trạng từ) về mặt lý thuyết}
        \VH{ultimately}{(adv) in the end; finally}{(trạng từ) cuối cùng}
        \VH{tough going}{(adj) progress that is difficult}{(tính từ) gian truân}
        \VH{exclusively}{(adv) for only one particular person, group or use}{(trạng từ) duy nhất}
        \VH{remuneration}{(n) payment for work or services}{(danh từ) mức lương, đãi ngộ}
        \VH{holiday entitlements}{(phrase) the number of days of paid holiday a year that a worker is entitled to take}{(cụm từ) chế độ ngày nghỉ được hưởng}
        \VH{within moments}{(idiom) within a very short amount of time}{(thành ngữ) rất nhanh chóng}
    \end{VocabHighlights}
    \end{test}
\end{glossarymc}