\begin{glossarymc}[Cambridge 10]
    \begin{test}{TEST 1}
    \noindent
    \textbf{Part 1. Weekends}
    \begin{qa}{How do you usually spend your weekends? [Why?]}
    Weekends are the quality time for my family so I \textbf{indulged myself with} a long sleep at night and woke up later than usual on weekdays, I frequently drive my family away to entertainment \textbf{hubs} or places of interests. That pretty sums up a typical weekend.
    \end{qa}

    \begin{qa}{Which is your favourite part of the weekend? [Why?]}
    The best part of the weekend is absolutely that I can \textbf{let my hair down} by spending quality time with family and hanging out without the need to care for my work. \textbf{Putting my nose to the grindstone} 5 days a week is enough and everyone needs time to \textbf{revitalize} their lost energy, right?
    \end{qa}

    \begin{qa}{Do you think your weekends are long enough? [Why?/Why not?]}
    Believe it or not, I only have Sundays off and Saturdays are still the days on which I \textbf{work like a dog} at not only my workplace but also my cram classes. So, I wish my weekends could be longer to \textbf{let off steam}.
    \end{qa}

    \begin{qa}{How important do you think it is to have free time at the weekends? [Why?]}
    It is \textbf{of paramount importance} to \textbf{idle away the hours} at weekends. Being given a chance to \textbf{blow off steam} is crucial to one's increased productivity the following week. \textbf{Pent-up} energy accumulated during weekdays should be \textbf{vented} on free time at weekends or else one would be under huge pressure and encounter mental problems \textbf{in the long run}.
    \end{qa}

        \begin{VocabExplain}[Part 1]
            \begin{ExplainCard}{indulge (oneself) with/in}[v][B2]
            \EN{to allow yourself to have or do something you enjoy, often more than is good for you.}
            \SY{pamper; treat yourself; luxuriate}
            \VI{nuông chiều bản thân; tự thưởng.}
            \EX{On Sundays I indulge myself with a late breakfast.}
            \EX{Short recovery breaks let employees indulge in brief leisure without hurting output.}
            \CO{\textit{indulge (oneself) in/with} sweets/a nap; \textit{self-indulgence}}
            \end{ExplainCard}

            \begin{ExplainCard}{hub}[n][B1]
            \EN{the center of activity or influence; a place where many things happen.}
            \SY{center; hotspot; nucleus}
            \VI{trung tâm; điểm tụ họp.}
            \EX{The park is a weekend entertainment hub.}
            \EX{The city aims to become a regional hub for creative industries.}
            \CO{\textit{transport/entertainment/tech} hub}
            \end{ExplainCard}

            \begin{ExplainCard}{let one's hair down}[idiom][B2]
            \EN{to relax and enjoy yourself without worrying about rules or work.}
            \SY{unwind; chill out; loosen up}
            \VI{xả hơi; thư giãn hết mình.}
            \EX{After exams we went dancing to let our hair down.}
            \EX{Retreats help teams let their hair down and bond.}
            \CO{\textit{let your hair down} at the weekend/party}
            \end{ExplainCard}

            \begin{ExplainCard}{put one's nose to the grindstone}[idiom][B2]
            \EN{to work very hard for a long period of time.}
            \SY{graft; knuckle down; toil}
            \VI{cắm đầu vào làm việc chăm chỉ.}
            \EX{He put his nose to the grindstone before deadlines.}
            \EX{Researchers must put their noses to the grindstone during field seasons.}
            \CO{\textit{keep/put} your nose to the grindstone}
            \end{ExplainCard}

            \begin{ExplainCard}{revitalize}[v][B2]
            \EN{to give new energy or life to something.}
            \SY{rejuvenate; reinvigorate; refresh}
            \VI{hồi phục sức sống; tiếp thêm năng lượng.}
            \EX{A short hike revitalized me.}
            \EX{Green spaces can revitalize urban well-being, studies show.}
            \CO{\textit{revitalize} energy/the body/the area}
            \end{ExplainCard}

            \begin{ExplainCard}{work like a dog}[idiom][B2]
            \EN{to work extremely hard.}
            \SY{work one's socks off; slave away}
            \VI{làm việc đầu tắt mặt tối.}
            \EX{She worked like a dog all week.}
            \EX{New founders often work like a dog to launch products.}
            \CO{\textit{work like a dog} at/on sth}
            \end{ExplainCard}

            \begin{ExplainCard}{let off steam}[idiom][B2]
            \EN{to release strong feelings or energy by doing something active.}
            \SY{unwind; decompress; blow off steam}
            \VI{xả stress; giải toả năng lượng.}
            \EX{He jogs to let off steam after work.}
            \EX{Breakout games let students let off steam between lectures.}
            \CO{\textit{let/blow off} steam}
            \end{ExplainCard}

            \begin{ExplainCard}{of paramount importance}[phrase][C1]
            \EN{more important than anything else; of the highest priority.}
            \SY{crucial; vital; imperative}
            \VI{cực kỳ quan trọng; tối quan trọng.}
            \EX{Sleep is of paramount importance to recovery.}
            \EX{Data integrity is of paramount importance in clinical trials.}
            \CO{\textit{be of} paramount importance}
            \end{ExplainCard}

            \begin{ExplainCard}{idle away the hours}[phrase][B2]
            \EN{to spend time doing very little, often in a pleasantly lazy way.}
            \SY{while away time; laze around}
            \VI{giết thời gian thong thả; thư nhàn.}
            \EX{We idled away the hours by the lake.}
            \EX{Tourists often idle away the hours in cafés between tours.}
            \CO{\textit{idle/while away} the hours/time}
            \end{ExplainCard}

            \begin{ExplainCard}{blow off steam}[idiom][B2]
            \EN{to do something that helps you get rid of strong feelings of stress or anger.}
            \SY{decompress; vent; let off steam}
            \VI{xả bực bội/căng thẳng.}
            \EX{He plays drums to blow off steam.}
            \EX{Post-exam socials allow students to blow off steam safely.}
            \CO{\textit{blow/let off} steam}
            \end{ExplainCard}

            \begin{ExplainCard}{pent-up}[adj][B2]
            \EN{(of emotions/energy) not expressed or released; kept inside.}
            \SY{repressed; bottled-up}
            \VI{dồn nén; bị kìm nén.}
            \EX{Pent-up frustration can spill over at home.}
            \EX{Lockdowns led to pent-up travel demand, economists note.}
            \CO{\textit{pent-up} energy/demand/emotions}
            \end{ExplainCard}

            \begin{ExplainCard}{vent}[v][C1]
            \EN{to express strong feelings so they are released.}
            \SY{release; unburden; discharge}
            \VI{xả (cảm xúc); trút ra.}
            \EX{Talk to a friend to vent your worries.}
            \EX{Writing journals helps participants vent negative affect constructively.}
            \CO{\textit{vent} anger/frustration/stress}
            \end{ExplainCard}

            \begin{ExplainCard}{in the long run}[idiom][B2]
            \EN{over a long period of time; eventually.}
            \SY{ultimately; over time}
            \VI{về lâu dài; rốt cuộc.}
            \EX{Regular rest pays off in the long run.}
            \EX{Preventive care reduces costs in the long run, research suggests.}
            \CO{\textit{in the} long run}
            \end{ExplainCard}
        \end{VocabExplain}

    \noindent
    \textbf{Part 2.}
    \begin{qa}{Describe someone you know who does something well. You should say:}
    \begin{itemize}
    \item Who this person is
    \item How you know this person
    \item What they do well
    \item and explain why you think this person is so good at doing this.
    \end{itemize}
    
    I have met so many people in my life and each person has made a long-lasting impression of who they are and what they \textbf{have a knack for}. Today, I would like to give you an account of one person who \textbf{has a natural bent for} English. It is my friend, Mr Tuan. The first thing I would like to mention is that he is 26 years old, one year older than me. Thanks to his regular workout at a gym center, he is athletic with strong muscles, so he is always the \textbf{center of attention} whenever he \textbf{turns up}. I knew him 3 months ago. At that time, I was working for an event-planning company, and there was a foreign presenter who was about to deliver a speech in an economic forum. He was \textbf{in charge of} interpreting the speech of the presenter from English into Vietnamese and vice versa. To be honest, I used to work as a translator in a small company 2 years ago, so at first, I was not impressed by him. The thing was that the presenter did not \textbf{have a good command of} English, and his accent was weird so it was \textbf{all Greek to me}. It was \textbf{like pulling teeth} to make sense of what he said. But Tuan \textbf{slayed me} with the clear messages delivered to the audience. The reason why I think he \textbf{did a good job} is that he has a \textbf{deep understanding} of the field he works on. To be able to work as a successful interpreter, good \textbf{expertise} in many fields and use of words in the right contexts are pivotal. Wrong interpretation could lead to misunderstanding among listeners and make interpreters become \textbf{a laughing stock}. In that sense, impressed by his nuance in conveying the message smoothly while dealing with the speaker's strange accent, I have been respectful of him since that time.
    \end{qa}

        \begin{VocabExplain}[Part 2]
            \begin{ExplainCard}{have a knack for}[phrase][B2]
            \EN{to have a natural skill or ability for something.}
            \SY{be good at; have a flair for}
            \VI{có khiếu; giỏi tự nhiên về điều gì.}
            \EX{She has a knack for explaining tough ideas simply.}
            \EX{Recruiters value candidates who have a knack for problem-solving.}
            \CO{\textit{have a knack for} languages/teaching/fixing things}
            \end{ExplainCard}

            \begin{ExplainCard}{(have) a natural bent for}[noun phrase][B2]
            \EN{a strong natural inclination or talent for something.}
            \SY{aptitude; predisposition; flair}
            \VI{thiên hướng/bẩm chất tự nhiên về.}
            \EX{He has a natural bent for public speaking.}
            \EX{Students with a natural bent for math progress rapidly.}
            \CO{\textit{a natural bent for} music/science}
            \end{ExplainCard}

            \begin{ExplainCard}{the center of attention}[noun][C1]
            \EN{the person everyone is watching or interested in.}
            \SY{focus; focal point}
            \VI{tâm điểm chú ý.}
            \EX{In group projects, she often becomes the center of attention.}
            \EX{The keynote speaker was the center of attention throughout the forum.}
            \CO{\textit{be the center of attention}}
            \end{ExplainCard}

            \begin{ExplainCard}{turn up}[phr.v][B2]
            \EN{to arrive or appear, especially unexpectedly or after being absent.}
            \SY{show up; appear}
            \VI{xuất hiện; có mặt.}
            \EX{He suddenly turned up at the meeting.}
            \EX{More participants turned up than the venue could host.}
            \CO{\textit{turn up} late/early/unexpectedly}
            \end{ExplainCard}

            \begin{ExplainCard}{in charge of}[phrase][B2]
            \EN{responsible for and having control over someone or something.}
            \SY{responsible for; head of}
            \VI{phụ trách; chịu trách nhiệm.}
            \EX{She is in charge of the translation team.}
            \EX{A moderator in charge of Q\&A ensured smooth discussion.}
            \CO{\textit{be/put} in charge of a project/team}
            \end{ExplainCard}

            \begin{ExplainCard}{have a good command of}[phrase][B2]
            \EN{to be able to use a language or skill very well.}
            \SY{be proficient in; have mastery of}
            \VI{thành thạo; sử dụng thuần thục.}
            \EX{Applicants must have a good command of English.}
            \EX{Researchers need a good command of statistics for data analysis.}
            \CO{\textit{good/strong} command of English/skills}
            \end{ExplainCard}

            \begin{ExplainCard}{(be) all Greek to me}[idiom][B2]
            \EN{completely impossible to understand.}
            \SY{incomprehensible; baffling}
            \VI{khó hiểu như “tiếng Hy Lạp”; hoàn toàn không hiểu.}
            \EX{Without subtitles the dialect was all Greek to me.}
            \EX{For non-coders, raw logs can be all Greek at first.}
            \CO{\textit{it's} all Greek \textit{to me/him/her}}
            \end{ExplainCard}

            \begin{ExplainCard}{like pulling teeth}[idiom][B2]
            \EN{very difficult and unpleasant to do.}
            \SY{arduous; a slog}
            \VI{khó nhọc, mệt mỏi (như nhổ răng).}
            \EX{Getting clear instructions was like pulling teeth.}
            \EX{Securing approvals can be like pulling teeth in big organizations.}
            \CO{\textit{be} like pulling teeth \textit{to}}
            \end{ExplainCard}

            \begin{ExplainCard}{slay (someone)}[v][B2]
            \EN{(1) to kill (literal/old use). (2) informal: to impress or amuse greatly; perform excellently.}
            \SY{(2) wow; dazzle; knock out}
            \VI{(2) gây ấn tượng/bùng nổ; “đỉnh của chóp”.}
            \EX{Her presentation slayed the audience.}
            \EX{The interpreter slayed with concise, accurate renderings.}
            \CO{\textit{slay} the audience/performance; absolutely \textit{slay}}
            \end{ExplainCard}

            \begin{ExplainCard}{do a good job}[phrase][B2]
            \EN{to perform a task well and effectively.}
            \SY{perform well; do well}
            \VI{làm tốt; hoàn thành công việc hiệu quả.}
            \EX{You did a good job handling questions.}
            \EX{Teams that plan carefully do a good job under pressure.}
            \CO{\textit{do} a good/excellent job (of) V-ing}
            \end{ExplainCard}

            \begin{ExplainCard}{deep understanding}[noun phrase][B2]
            \EN{thorough and detailed knowledge of a subject.}
            \SY{profound grasp; thorough comprehension}
            \VI{hiểu biết sâu sắc.}
            \EX{Her deep understanding of context prevents mistranslation.}
            \EX{Deep understanding of users drives better product design.}
            \CO{\textit{have/possess} a deep understanding (of)}
            \end{ExplainCard}

            \begin{ExplainCard}{expertise}[n][B2]
            \EN{high-level knowledge or skill in a particular area.}
            \SY{proficiency; mastery; specialism}
            \VI{chuyên môn; tay nghề cao.}
            \EX{Linguistic expertise is essential for interpreters.}
            \EX{The panel pooled expertise from law, tech, and ethics.}
            \CO{\textit{technical/professional} expertise; \textit{area of} expertise}
            \end{ExplainCard}

            \begin{ExplainCard}{a laughing stock}[noun][B2]
            \EN{a person or thing that is ridiculed by many people.}
            \SY{object of ridicule; joke}
            \VI{trò cười cho thiên hạ.}
            \EX{Inaccurate translations can make a speaker a laughing stock.}
            \EX{Poor quality control turned the product into a laughing stock online.}
            \CO{\textit{become/make sb} a laughing stock}
            \end{ExplainCard}
        \end{VocabExplain}

    \noindent
    \textbf{Part 3.}
    \begin{qa}{What skills and abilities do people most want to have today? Why?}
    To survive in the modern world today, people need to become \textbf{knowledgable} in their \textbf{domains}. Personally, language proficiency and communication \textbf{skills} are the most desirable skills which set people apart from their peers. If people are \textbf{adept at} one or two foreign languages, this means they would stand a better chance of integrating into the outer world. Additionally, good communication skills will allow people to express themselves in a positive and clear manner, resulting in good impression and increased likelihood for \textbf{promotion}.
    \end{qa}

    \begin{qa}{Which skills should children learn at school? Are there any skills which they should learn at home? What are they?}
    Well, it is a \textbf{fallacy} if people regulate the skills students should learn at school or at home. From my own perspective, school or family are ideal environment for students to \textbf{hone} their skills. For example, chatting with either parents or friends at school is a good way to enrich student's communication ability. \textbf{Self-belief} or \textbf{agility} can be nurtured from physical activities students take part with their family or school children.
    \end{qa}

    \begin{qa}{Which skills do you think will be important in the future? Why?}
    Basically, most skills will \textbf{pave the way} for people to reach their goals. On top of that, leadership skills might be one of destinations that many \textbf{yearn} to approach because this type of skill displays in every careers. Those who are \textbf{furnished} with leadership skills can see the \textbf{big picture} and could direct the staff to undergo \textbf{upheavals}. They are , also good at \textbf{negotiation} and persuasion to accommodate different preferences of their staff.
    \end{qa}

    \begin{qa}{Which kinds of jobs have the highest salaries in your country? Why is this?}
    I suppose the \textbf{salary range} is determined according to the duty and expertise of employees, so picking out the job which has the highest salary is impossible. Nonetheless, business and financial occupations are among \textbf{lucrative} careers, because these jobs \textbf{exact} formal qualifications from employees and \textbf{quick-witted} responses to unexpected circumstances. However, when there is an economic \textbf{recession}, most salary would be adversely affected, I believe.
    \end{qa}

    \begin{qa}{Are there any other jobs that you think should have high salaries? Why do you think that?}
    In my opinion, doctors or lawyers deserve deep respect and a generous income as they are usually under constant stress. Without doctors, diseases could \textbf{claim the life of} \textbf{a plethora of} patients. This is also the case for lawyers, who fight for social \textbf{justice} and help to \textbf{eliminate criminality}. They run the risk of being revenged, which might even \textbf{claim their lives}. Therefore, I am of the opinion that their salary should be \textbf{handsome} to pay off what they have done.
    \end{qa}

    \begin{qa}{Some people say it'd be better for society if everyone got the same salary. What do you think about that? Why?}
    I would refute that statement. On the one hand, I understand that if there is no \textbf{disparity} in salary pay, there will be no \textbf{envy} as a result. But the same salary would impede the development of employees as they know how matter conscientious they are, there is no change in their salary. Beside, we are living in an \textbf{egalitarian} society which prioritizes equality, hence we should not pay the same amount for everyone.
    \end{qa}

        \begin{VocabExplain}[Part 3]
            \begin{ExplainCard}{knowledgeable}[adj][C1]
            \EN{having a lot of information or understanding about a subject.}
            \SY{well-informed; versed}
            \VI{am hiểu; có nhiều kiến thức.}
            \EX{You need to be knowledgeable about tech trends.}
            \EX{Knowledgeable reviewers improve the quality of submissions.}
            \CO{be \textit{knowledgeable about} sth; a \textit{knowledgeable} expert}
            \end{ExplainCard}

            \begin{ExplainCard}{adept (at)}[adj][B2]
            \EN{very skilled or proficient in doing something.}
            \SY{skilled; proficient}
            \VI{thành thạo; điêu luyện (về).}
            \EX{She's adept at small talk.}
            \EX{Bilingual staff adept at negotiation are in demand.}
            \CO{\textit{adept at} V-ing/NP}
            \end{ExplainCard}

            \begin{ExplainCard}{promotion}[n][C1]
            \EN{advancement to a higher position or rank at work.}
            \SY{advancement; elevation}
            \VI{thăng chức; thăng tiến.}
            \EX{Strong soft skills help with promotion.}
            \EX{Promotion criteria include leadership and impact.}
            \CO{\textit{win/earn} a promotion; promotion \textit{prospects}}
            \end{ExplainCard}

            \begin{ExplainCard}{fallacy}[n][C1]
            \EN{a false idea or belief based on faulty reasoning.}
            \SY{misconception; mistaken belief}
            \VI{ngụy biện; quan niệm sai lầm.}
            \EX{It's a fallacy that multitasking boosts focus.}
            \EX{The paper exposes statistical fallacies in prior work.}
            \CO{\textit{common/widespread} fallacy; logical \textit{fallacy}}
            \end{ExplainCard}

            \begin{ExplainCard}{hone}[v][B2]
            \EN{to improve a skill by practicing it.}
            \SY{sharpen; refine}
            \VI{mài giũa; rèn luyện.}
            \EX{Debate club honed her argument skills.}
            \EX{Internships hone students' professional competencies.}
            \CO{\textit{hone} skills/talent/technique}
            \end{ExplainCard}

            \begin{ExplainCard}{self-belief}[n][B2]
            \EN{confidence in one's own abilities.}
            \SY{self-confidence; self-assurance}
            \VI{niềm tin vào bản thân.}
            \EX{Athletes need self-belief to perform.}
            \EX{Mentoring strengthens learners' self-belief over time.}
            \CO{\textit{build/boost} self-belief}
            \end{ExplainCard}

            \begin{ExplainCard}{agility}[n][C1]
            \EN{ability to move or think quickly and easily.}
            \SY{nimbleness; flexibility}
            \VI{sự nhanh nhẹn; linh hoạt.}
            \EX{Dance improves physical agility.}
            \EX{Organizational agility helps firms adapt to shocks.}
            \CO{physical/mental \textit{agility}}
            \end{ExplainCard}

            \begin{ExplainCard}{pave the way (for)}[idiom][B2]
            \EN{to make it possible for something to happen later.}
            \SY{prepare the ground; enable}
            \VI{mở đường; tạo tiền đề (cho).}
            \EX{Internships pave the way for jobs.}
            \EX{Policy pilots paved the way for nationwide adoption.}
            \CO{\textit{pave the way for} reform/success}
            \end{ExplainCard}

            \begin{ExplainCard}{yearn (for/to)}[v][B2]
            \EN{to long strongly for something.}
            \SY{long for; crave}
            \VI{khao khát; mong mỏi.}
            \EX{Graduates yearn for meaningful work.}
            \EX{Many yearn to study abroad for wider exposure.}
            \CO{\textit{yearn for} NP; \textit{yearn to} V}
            \end{ExplainCard}

            \begin{ExplainCard}{(be) furnished with}[adj][B2]
            \EN{equipped or provided with something.}
            \SY{equipped with; supplied with}
            \VI{được trang bị/cung cấp (với).}
            \EX{Leaders furnished with data make better calls.}
            \EX{New hires are furnished with onboarding guides.}
            \CO{\textit{be furnished with} skills/evidence}
            \end{ExplainCard}

            \begin{ExplainCard}{the big picture}[n][B2]
            \EN{the overall perspective or objective, not the details.}
            \SY{overall view; macro view}
            \VI{bức tranh tổng thể; cái nhìn toàn cục.}
            \EX{Step back and see the big picture.}
            \EX{Strategic roles require big-picture thinking.}
            \CO{\textit{see/focus on} the big picture}
            \end{ExplainCard}

            \begin{ExplainCard}{upheaval}[n][C1]
            \EN{a big change that causes a lot of trouble or confusion.}
            \SY{turmoil; disruption}
            \VI{biến động lớn; đảo lộn.}
            \EX{Job market upheavals worry graduates.}
            \EX{Digital upheaval reshaped media industries.}
            \CO{economic/political \textit{upheaval}}
            \end{ExplainCard}

            \begin{ExplainCard}{negotiation}[n][C1]
            \EN{discussion to reach an agreement.}
            \SY{bargaining; talks}
            \VI{đàm phán; thương lượng.}
            \EX{She led the salary negotiation calmly.}
            \EX{Cross-border negotiation skills are vital in trade.}
            \CO{\textit{enter into} negotiation; contract \textit{negotiations}}
            \end{ExplainCard}

            \begin{ExplainCard}{salary range}[n][B2]
            \EN{the span between the minimum and maximum pay for a role.}
            \SY{pay band; compensation range}
            \VI{khung lương; dải lương.}
            \EX{Ask HR about the salary range.}
            \EX{Transparent salary ranges reduce inequity.}
            \CO{\textit{within the} salary range; define/set a \textit{range}}
            \end{ExplainCard}

            \begin{ExplainCard}{lucrative}[adj][C1]
            \EN{producing a lot of money; profitable.}
            \SY{profitable; well-paid}
            \VI{béo bở; sinh lợi.}
            \EX{Consulting can be lucrative.}
            \EX{Data shows cybersecurity remains a lucrative field.}
            \CO{\textit{lucrative} career/contract/market}
            \end{ExplainCard}

            \begin{ExplainCard}{exact (from)}[v][B2]
            \EN{to demand and obtain something, often with difficulty.}
            \SY{demand; require}
            \VI{đòi hỏi; yêu cầu (gắt gao).}
            \EX{The role exacts high standards.}
            \EX{Crisis work exacts a heavy emotional toll.}
            \CO{\textit{exact} standards/penalties; \textit{exacts} a toll}
            \end{ExplainCard}

            \begin{ExplainCard}{quick-witted}[adj][B2]
            \EN{able to think and respond quickly and cleverly.}
            \SY{sharp; nimble-minded}
            \VI{nhanh trí; lanh lợi.}
            \EX{A quick-witted host saved the show.}
            \EX{Negotiators must be quick-witted under pressure.}
            \CO{\textit{be} quick-witted; a \textit{quick-witted} reply}
            \end{ExplainCard}

            \begin{ExplainCard}{recession}[n][C1]
            \EN{a period of temporary economic decline.}
            \SY{downturn; contraction}
            \VI{suy thoái kinh tế.}
            \EX{Hiring freezes are common in a recession.}
            \EX{Recession risk alters graduates' job choices.}
            \CO{\textit{enter/fall into} recession; recession \textit{risk}}
            \end{ExplainCard}

            \begin{ExplainCard}{a plethora of}[phrase][B2]
            \EN{a very large amount or number of something.}
            \SY{an abundance of; a wealth of}
            \VI{rất nhiều; vô số.}
            \EX{Doctors treat a plethora of cases daily.}
            \EX{The survey collected a plethora of responses.}
            \CO{\textit{a plethora of} options/patients/data}
            \end{ExplainCard}

            \begin{ExplainCard}{justice}[n][B2]
            \EN{fair treatment and due reward according to law or ethics.}
            \SY{fairness; equity}
            \VI{công lý; công bằng.}
            \EX{Lawyers fight for justice.}
            \EX{Access to justice improves social trust.}
            \CO{\textit{social/criminal} justice; seek \textit{justice}}
            \end{ExplainCard}

            \begin{ExplainCard}{eliminate criminality}[phrase][C1]
            \EN{to remove or reduce crime in society.}
            \SY{combat crime; curb offending}
            \VI{loại trừ/tấn giảm tội phạm.}
            \EX{Stronger policing can't alone eliminate criminality.}
            \EX{Education programs help eliminate criminality long-term.}
            \CO{\textit{eliminate/reduce} criminality}
            \end{ExplainCard}

            \begin{ExplainCard}{claim (one's) life}[phrase][B2]
            \EN{(of an illness/accident) to cause someone's death.}
            \SY{take a life; be fatal to}
            \VI{cướp đi mạng sống.}
            \EX{The outbreak claimed many lives.}
            \EX{Road crashes claim lives every year worldwide.}
            \CO{\textit{claim/take} lives; \textit{claim the life of} sb}
            \end{ExplainCard}

            \begin{ExplainCard}{handsome (salary)}[adj][B2]
            \EN{large and attractive (of amounts of money).}
            \SY{generous; substantial}
            \VI{hậu hĩnh (về lương/thù lao).}
            \EX{They offered a handsome salary.}
            \EX{Handsome stipends attract top candidates.}
            \CO{\textit{handsome} pay/package/bonus}
            \end{ExplainCard}

            \begin{ExplainCard}{disparity}[n][C1]
            \EN{a significant difference, especially unfair one.}
            \SY{inequality; gap}
            \VI{chênh lệch; bất bình đẳng.}
            \EX{Salary disparity hurts morale.}
            \EX{Reports track regional income disparity.}
            \CO{\textit{income/gender} disparity; reduce \textit{disparities}}
            \end{ExplainCard}

            \begin{ExplainCard}{envy}[n][B1]
            \EN{a feeling of wanting what someone else has.}
            \SY{jealousy; covetousness}
            \VI{sự ghen tị; lòng đố kỵ.}
            \EX{Equal pay reduces envy among staff.}
            \EX{Envy can distort workplace cooperation, studies show.}
            \CO{\textit{feel/stir} envy; object of \textit{envy}}
            \end{ExplainCard}

            \begin{ExplainCard}{egalitarian}[adj][B2]
            \EN{believing that all people are equal and deserve equal rights.}
            \SY{equal; non-hierarchical}
            \VI{bình đẳng chủ nghĩa.}
            \EX{An egalitarian culture values transparency.}
            \EX{Egalitarian policies aim to widen opportunity.}
            \CO{\textit{egalitarian} society/policy/workplace}
            \end{ExplainCard}
        \end{VocabExplain}

    \begin{VocabHighlights}
        \VH{to indulge oneself with}{(v) to allow oneself to have/do something he/she likes}{(động từ) tự thưởng cho bản thân}
        \VH{hub}{(n) the central and most important part of a particular place or activity}{(danh từ) khu vực trung tâm}
        \VH{to let one's hair down}{(idiom) to allow yourself to behave much more freely than usual and enjoy yourself}{(thành ngữ) cho phép bản thân xả hơi}
        \VH{to put/keep one's nose to the grindstone}{(idiom) work hard and continuously}{(thành ngữ) làm việc chăm chỉ, liên tục, không ngừng nghỉ}
        \VH{to revitalize}{(v) to make something stronger, more active or more healthy}{(động từ) làm hồi sinh, tái tạo lại}
        \VH{to work like a dog}{(idiom) to work very hard}{(thành ngữ) làm việc chăm chỉ}
        \VH{to let/blow off steam}{(idiom) get rid of pent-up energy or strong emotion}{(thành ngữ) xả hơi}
        \VH{of paramount importance}{(phrase) more important than anything else}{(cụm từ) quan trọng hơn tất cả}
        \VH{to idle something away}{(phr.v) to spend a period of time relaxing and doing very little}{(cụm động từ) dành thời gian thư giãn}
        \VH{pent-up}{(adj) closely confined or held back}{(tính từ) bị kìm nén, kiềm tỏa}
        \VH{to vent}{(v) to express feelings, especially anger, strongly}{(động từ) trút (giận, năng lượng...)}
        \VH{in the long run}{(phrase) over or after a long period of time; eventually}{(cụm từ) về lâu về dài}
        \VH{to have a knack for}{(idiom) do something well}{(thành ngữ) giỏi làm gì}
        \VH{to have a natural bent for}{(idiom) have talent for}{(thành ngữ) có năng khiếu}
        \VH{to deliver a speech}{(phrase) send a speech}{(cụm từ) có bài phát biểu}
        \VH{to have a good command of}{(idiom) have a good level of something}{(thành ngữ) có một trình độ tốt}
        \VH{to be/sound all Greek to somebody}{(idiom) sound strange to somebody's ears}{(thành ngữ) nghe lạ hoặc khó hiểu với ai đó}
        \VH{to be like pulling teeth}{(idiom) be extremely difficult}{(thành ngữ) rất khó}
        \VH{to slay}{(v) to impress someone very much}{(động từ) gây ấn tượng rất nhiều}
        \VH{to do a good job}{(idiom) to perform a task well}{(thành ngữ) làm tốt công việc}
        \VH{to have a deep understanding of}{(phrase) know something very well}{(cụm từ) hiểu sâu rộng và sâu sắc}
        \VH{expertise}{(n) a high level of knowledge or skill}{(danh từ) chuyên môn}
        \VH{to become a laughing stock}{(phrase) supposed to be important or serious but have been made to seem ridiculous}{(cụm từ) trở thành trò cười}
        \VH{to convey messages}{(phrase) send messages}{(cụm từ) truyền tải thông điệp}
        \VH{for all}{(phrase) despite}{(cụm từ) mặc dù}
        \VH{knowledgable}{(adj) knowing a lot}{(tính từ) hiểu biết nhiều}
        \VH{domain}{(n) an area of knowledge or activity; especially one that somebody is responsible for}{(danh từ) lĩnh vực}
        \VH{to be adept at}{(adj) having a natural ability to do something that needs skill}{(tính từ) có kĩ năng giỏi về cái gì}
        \VH{promotion}{(n) a move to a more important job or rank in a company or an organization}{(danh từ) thăng chức}
        \VH{fallacy}{(n) a false idea that many people believe is true}{(danh từ) hiểu sai}
        \VH{self-belief}{(n) confidence in your own abilities or judgment}{(danh từ) sự tự tin}
        \VH{agility}{(n) the ability to move quickly and easily}{(danh từ) nhanh nhẹn}
        \VH{to pave the way for}{(phrase) create the circumstances to enable (something) to happen or be done}{(cụm từ) tạo điều kiện cho cái gì phát triển}
        \VH{to yearn to V}{(v) to wish very strongly, especially for something that is very difficult to have}{(động từ) mong mỏi làm gì}
        \VH{to be furnished with}{(phr.v) to be equipped with}{(cụm từ) được trang bị}
        \VH{the big picture}{(idiom) the most important facts about a situation and the effects of that situation on other things}{(thành ngữ) bức tranh toàn cảnh}
        \VH{upheaval}{(n) a big change that causes a lot of confusion, worry and problems}{(danh từ) biến động}
        \VH{negotiation}{(n) formal discussion between people who are trying to reach an agreement}{(danh từ) đàm phán}
        \VH{salary range}{(phrase) the range of pay established by employers to pay to employees performing a particular job or function}{(cụm từ) mức lương}
        \VH{lucrative}{(adj) producing a lot of money}{(tính từ) tạo ra nhiều tiền}
        \VH{to exact}{(v) to demand and get something from somebody}{(động từ) đòi hỏi}
        \VH{quick-witted}{(adj) showing or characterized by an ability to think or respond quickly and effectively}{(tính từ) nhanh nhẹn}
        \VH{recession}{(n) a difficult time for the economy of a country, when there is less trade and industrial activity than usual and more people are unemployed}{(danh từ) suy thoái kinh tế}
        \VH{generous}{(adj) giving or willing to give freely; given freely}{(tính từ) hào phóng}
        \VH{to claim the life of}{(phrase) if a violent event, fighting, or a disease claims someone's life, it kills that person}{(cụm từ) tước đoạt mạng sống}
        \VH{a plethora of}{(phrase) a large or excessive amount of (something)}{(cụm từ) rất nhiều}
        \VH{justice}{(n) the fair treatment of people}{(danh từ) công lý}
        \VH{criminality}{(n) the fact of people being involved in crime; criminal acts}{(danh từ) sự phạm tội}
        \VH{to claim one's life}{(idiom) to kill somebody}{(thành ngữ) giết ai}
        \VH{handsome}{(adj) substantial, very large}{(tính từ) hậu hĩnh, nhiều}
        \VH{disparity}{(n) a difference, especially one connected with unfair treatment}{(danh từ) sự khác biệt}
        \VH{envy}{(n) the feeling of wanting to be in the same situation as somebody else; the feeling of wanting something that somebody else has}{(danh từ) đố kỵ}
        \VH{egalitarian}{(adj) relating to or believing in the principle that all people are equal and deserve equal rights and opportunities}{(tính từ) bình đẳng}
    \end{VocabHighlights}
    \end{test}

    \begin{test}{TEST 2}
    \noindent
    \textbf{Part 1. Music}
    \begin{qa}{What types of music do you like to listen to? [Why?]}
    \textbf{Believe it or not}, heavy metal is my favorite. This genre was \textbf{trendy} in the 1970s and 1980s. However, in the 21st century, its popularity is \textbf{on the wane} due to the rise of other genres such as pop, hip-hop, R\&B, etc. To tell the truth, listening to heavy metal may \textbf{send me into ecstasies} and prove to be effective in relieving my daily stress.
    \end{qa}

    \begin{qa}{At what times of day do you like to listen to music? [Why?]}
    Well, I usually \textbf{have too many irons in the fire} during daytime so I can only listen to music \textbf{on the move}. Enjoying music when I'm on the way to work sounds great to me. Besides, listening to music and singing with my favorite songs while taking a shower is interesting enough for me.
    \end{qa}

    \begin{qa}{Did you learn to play a musical instrument when you were a child? [Why?/Why not?]}
    I did not. When I was young, playing sports was \textbf{given priority over} \textbf{having a go at} playing an instrument. It was not until I was 25 that I first learned how to play one. However, at that time, my hand became \textbf{calloused} and I felt I lacked the \textbf{dexterity} to play the guitar well. I was also \textbf{tone-deaf} so playing the guitar was quite tough.
    \end{qa}

    \begin{qa}{Do you think all children should learn to play a musical instrument? [Why?/Why not?]}
    Yes, I do. Children shouldn't \textbf{model themselves on} me but started to learn how to use an instrument as soon as possible. It is a \textbf{soft skill} which will certainly be beneficial for them \textbf{in the fullness of time} as they can not only socialize with friends but also perform in front of the audience later on. I have never \textbf{brought the house down} by playing music and I do wish I could experience that feeling at least once in my life.
    \end{qa}

        \begin{VocabExplain}[Part 1]
            \begin{ExplainCard}{Believe it or not}[idiom][B2]
            \EN{used to introduce something surprising or hard to believe.}
            \SY{surprisingly; as strange as it may seem}
            \VI{tin hay không thì tuỳ; nghe khó tin nhưng sự thật là.}
            \EX{Believe it or not, I enjoy heavy metal.}
            \EX{Believe it or not, survey data show teens read more print than expected.}
            \CO{\textit{Believe it or not}, + clause}
            \end{ExplainCard}

            \begin{ExplainCard}{trendy}[adj][B2]
            \EN{very fashionable or popular at the moment.}
            \SY{fashionable; in vogue; hip}
            \VI{thịnh hành; “hot”.}
            \EX{Vinyl records are trendy again.}
            \EX{Trendy formats like short videos reshape music discovery.}
            \CO{\textit{trendy} clothes/genre/spot; become \textit{trendy}}
            \end{ExplainCard}

            \begin{ExplainCard}{on the wane}[idiom][B2]
            \EN{becoming weaker, smaller, or less popular.}
            \SY{in decline; dwindling}
            \VI{đang suy giảm; hết thời.}
            \EX{CD sales are on the wane.}
            \EX{Interest in long ads is on the wane among younger audiences.}
            \CO{\textit{be} on the wane; popularity \textit{on the wane}}
            \end{ExplainCard}

            \begin{ExplainCard}{send (sb) into ecstasies}[phrase][B2]
            \EN{to make someone feel intense joy or excitement.}
            \SY{thrill; exhilarate}
            \VI{làm ai đó phấn khích/tê mê.}
            \EX{The chorus sent the crowd into ecstasies.}
            \EX{A surprise scholarship sent her into ecstasies, according to interviews.}
            \CO{\textit{send/fill} sb \textit{with} ecstasy; be \textit{in} ecstasies}
            \end{ExplainCard}

            \begin{ExplainCard}{have too many irons in the fire}[idiom][B2]
            \EN{to be involved in too many activities at once.}
            \SY{be overcommitted; spread yourself too thin}
            \VI{ôm đồm quá nhiều việc cùng lúc.}
            \EX{I have too many irons in the fire on weekdays.}
            \EX{Leaders with too many irons in the fire risk burnout.}
            \CO{\textit{have/keep} many irons in the fire}
            \end{ExplainCard}

            \begin{ExplainCard}{on the move}[phrase][B1]
            \EN{(1) while traveling; (2) busy and active.}
            \SY{(1) in transit \quad (2) on the go}
            \VI{(1) đang di chuyển; (2) bận rộn.}
            \EX{I listen to podcasts on the move.}
            \EX{Mobile workers are constantly on the move between sites.}
            \CO{\textit{listen/work} on the move; always \textit{on the move}}
            \end{ExplainCard}

            \begin{ExplainCard}{give priority over / given priority over}[phrase][B2]
            \EN{to treat something as more important than something else.}
            \SY{take precedence over; prioritize}
            \VI{ưu tiên hơn; đặt lên hàng đầu.}
            \EX{Sports were given priority over music when I was young.}
            \EX{Safety should take priority over speed in operations.}
            \CO{\textit{give} priority \textit{to} A \textit{over} B}
            \end{ExplainCard}

            \begin{ExplainCard}{have a go at}[phrase][B2]
            \EN{to try doing something.}
            \SY{give it a try; attempt}
            \VI{thử làm; thử sức.}
            \EX{I finally had a go at guitar at 25.}
            \EX{Students had a go at composing a short jingle.}
            \CO{\textit{have a go at} V-ing/sth}
            \end{ExplainCard}

            \begin{ExplainCard}{calloused}[adj][B2]
            \EN{(of skin) hardened and thick from repeated use or friction.}
            \SY{hardened; roughened}
            \VI{chai sạn (da tay).}
            \EX{Months of practice left my fingertips calloused.}
            \EX{Manual workers often have calloused palms, ergonomics notes.}
            \CO{\textit{calloused} hands/fingers/skin}
            \end{ExplainCard}

            \begin{ExplainCard}{dexterity}[n][C1]
            \EN{(1) skill in using the hands; (2) skillful mental agility.}
            \SY{(1) deftness \quad (2) adroitness}
            \VI{(1) sự khéo tay; (2) sự linh hoạt trí óc.}
            \EX{(1) Finger dexterity is vital for guitarists.}
            \EX{(2) Negotiation requires mental dexterity under pressure.}
            \CO{\textit{manual/mental} dexterity; show \textit{dexterity in} sth}
            \end{ExplainCard}

            \begin{ExplainCard}{tone-deaf}[adj][B2]
            \EN{(1) unable to perceive differences in musical pitch; (2) figuratively, insensitive to a situation's mood.}
            \SY{(1) unmusical \quad (2) insensitive}
            \VI{(1) không cảm âm; (2) vô cảm (nghĩa bóng).}
            \EX{(1) I'm tone-deaf, so guitar was tough.}
            \EX{(2) The brand's tone-deaf ad drew criticism.}
            \CO{\textit{be} tone-deaf; a \textit{tone-deaf} response}
            \end{ExplainCard}

            \begin{ExplainCard}{model oneself on (sb)}[phrase][B2]
            \EN{to copy or imitate someone as a role model.}
            \SY{emulate; take as a model}
            \VI{noi gương; học theo ai.}
            \EX{Don't model yourself on my late start with music.}
            \EX{Young conductors often model themselves on celebrated maestros.}
            \CO{\textit{model oneself on} a mentor/hero}
            \end{ExplainCard}

            \begin{ExplainCard}{soft skill}[n][B2]
            \EN{a personal, non-technical ability that helps effective interaction (e.g., communication, teamwork).}
            \SY{people skill; interpersonal skill}
            \VI{kỹ năng mềm.}
            \EX{Performance builds students' soft skills.}
            \EX{Employers rate soft skills as critical to leadership roles.}
            \CO{\textit{develop/build} soft skills; soft-skill training}
            \end{ExplainCard}

            \begin{ExplainCard}{in the fullness of time}[idiom][B2]
            \EN{after a long time has passed; eventually.}
            \SY{in due course; eventually}
            \VI{rồi sớm muộn; đến lúc thích hợp.}
            \EX{In the fullness of time, practice pays off.}
            \EX{Impacts emerge in the fullness of time as cohorts mature.}
            \CO{\textit{in the fullness of time}, + clause}
            \end{ExplainCard}

            \begin{ExplainCard}{bring the house down}[idiom][B2]
            \EN{to get a very enthusiastic reaction from an audience.}
            \SY{bring down the house; wow the crowd}
            \VI{làm khán giả vỗ tay nồng nhiệt; “quẩy tung sân khấu”.}
            \EX{Her solo brought the house down.}
            \EX{The finale brought the house down at the conservatory recital.}
            \CO{\textit{bring/brought} the house down; a \textit{bring-the-house-down} performance}
            \end{ExplainCard}
        \end{VocabExplain}

    \noindent
    \textbf{Part 2.}
    \begin{qa}{Describe a shop near where you live that you sometimes use. You should say:}
    \begin{itemize}
    \item What sorts of product or service it sells
    \item What the shop looks like
    \item Where it is located
    \item and explain why you use this shop.
    \end{itemize}

    To be honest, I am a \textbf{shopaholic}, so in my free time, I often \textbf{do window-shopping}. I have visited many shops, but I am a regular customer of this shop because it is \textbf{a stone's throw} from my house. It is BICKY, one of the big chain stores in my area. Speaking of its location, it is situated on Nguyen Chi Thanh street, which is a \textbf{vibrant} area. It is very large with 5 floors, so it is easily seen \textbf{from afar}. Indeed, it is considered a fashion center, because it offers many clothing items. On the second floor, there were clothes from big \textbf{high street names}, while shoes and bags are \textbf{on display} on the third and fourth floor, respectively. I am a huge fan of hats, so I usually pay a visit to the showroom on the fifth floor which is \textbf{devoted to} accessories. To the best of my knowledge, the owner of this shop is a multi-talented singer in Vietnamese showbiz, Ngo Thanh Van. She is a \textbf{slave to} fashion, so she opened this shop with a view to helping people \textbf{dress smartly}. The reason why I \textbf{frequent} this shop is because clothes \textbf{are often bought for a song}. Quality and styles are \textbf{prioritized}, and it offers \textbf{loyalty cards}, so I usually get a discount. More importantly, I feel confident and comfortable when I \textbf{put on} items from this shop.
    \end{qa}

        \begin{VocabExplain}[Part 2]
            \begin{ExplainCard}{shopaholic}[n][B2]
            \EN{a person who is excessively fond of shopping and finds it hard to resist buying things.}
            \SY{compulsive shopper; shopping addict}
            \VI{nghiện mua sắm.}
            \EX{I'm a bit of a shopaholic during sales.}
            \EX{Studies link shopaholic tendencies to impulse-control issues.}
            \CO{be a \textit{shopaholic}; recover from \textit{shopping addiction}}
            \end{ExplainCard}

            \begin{ExplainCard}{window-shopping}[n/phrase][B2]
            \EN{looking at goods in shop windows without intending to buy.}
            \SY{browse; look around}
            \VI{ngắm hàng qua tủ kính; đi dạo xem đồ.}
            \EX{We love window-shopping downtown after dinner.}
            \EX{Window-shopping informs consumer preferences without purchase.}
            \CO{\textit{do/go} window-shopping}
            \end{ExplainCard}

            \begin{ExplainCard}{a stone's throw (away/from)}[idiom][B2]
            \EN{a very short distance.}
            \SY{a hop, skip, and a jump; nearby}
            \VI{cách một quãng rất ngắn; ngay gần.}
            \EX{The mall is a stone's throw from my flat.}
            \EX{Field sites located a stone's throw from campus reduce costs.}
            \CO{\textit{a stone's throw from} + place}
            \end{ExplainCard}

            \begin{ExplainCard}{vibrant}[adj][C1]
            \EN{full of energy and activity; lively.}
            \SY{lively; bustling; dynamic}
            \VI{sôi động; nhộn nhịp.}
            \EX{It's a vibrant shopping district.}
            \EX{Vibrant retail hubs stimulate local economies.}
            \CO{\textit{vibrant} area/culture/scene}
            \end{ExplainCard}

            \begin{ExplainCard}{from afar}[adv][B2]
            \EN{from a long distance away.}
            \SY{at a distance; from far away}
            \VI{từ xa.}
            \EX{You can spot the sign from afar.}
            \EX{The landmark is visible from afar across the plain.}
            \CO{see/visible \textit{from afar}}
            \end{ExplainCard}

            \begin{ExplainCard}{high street names}[n][B2]
            \EN{well-known mainstream retail fashion brands.}
            \SY{big-name brands; mainstream labels}
            \VI{thương hiệu thời trang đại chúng nổi tiếng.}
            \EX{This floor stocks high street names.}
            \EX{High-street names dominate mid-price segments.}
            \CO{\textit{big} high-street name; high-street \textit{brand/chain}}
            \end{ExplainCard}

            \begin{ExplainCard}{on display}[phrase][B2]
            \EN{arranged so that people can see it; exhibited.}
            \SY{exhibited; showcased}
            \VI{trưng bày; bày ra.}
            \EX{New arrivals are on display near the entrance.}
            \EX{Merchandise on display increases impulse purchases.}
            \CO{\textit{put/place} sth on display; seasonal \textit{display}}
            \end{ExplainCard}

            \begin{ExplainCard}{devoted to}[adj phrase][B2]
            \EN{used or set aside for a special purpose.}
            \SY{dedicated to; reserved for}
            \VI{dành riêng cho.}
            \EX{The top floor is devoted to accessories.}
            \EX{A wing devoted to archives improves access for researchers.}
            \CO{\textit{be} devoted to + noun/V-ing}
            \end{ExplainCard}

            \begin{ExplainCard}{a slave to (fashion)}[idiom][B2]
            \EN{strongly influenced by something (e.g., fashion) to the point of dependence.}
            \SY{fashion victim; captive of trends}
            \VI{nô lệ (thời trang); chạy theo mốt.}
            \EX{He's a slave to fashion and buys every drop.}
            \EX{Being a slave to trends can distort long-term branding.}
            \CO{\textit{become} a slave to fashion/tech}
            \end{ExplainCard}

            \begin{ExplainCard}{dress smartly}[v phrase][B2]
            \EN{to wear neat, stylish, and appropriate clothes.}
            \SY{dress sharply; dress well}
            \VI{ăn mặc lịch sự, chỉnh tề.}
            \EX{Staff are asked to dress smartly at weekends.}
            \EX{Candidates who dress smartly enjoy better first-impression ratings.}
            \CO{\textit{dress smartly/appropriately}; smart \textit{dress code}}
            \end{ExplainCard}

            \begin{ExplainCard}{frequent}[v][C1]
            \EN{to visit a place regularly.}
            \SY{patronize; haunt}
            \VI{lui tới thường xuyên.}
            \EX{Locals frequent the store after work.}
            \EX{Students frequently frequent cafés near campus (usage data).}
            \CO{\textit{frequent} a café/venue/market}
            \end{ExplainCard}

            \begin{ExplainCard}{(buy) for a song}[idiom][B2]
            \EN{to purchase very cheaply.}
            \SY{at a steal; for next to nothing}
            \VI{mua với giá rẻ bèo.}
            \EX{I picked up this hat for a song.}
            \EX{Off-season stock is often sold for a song.}
            \CO{\textit{get/buy} sth \textit{for a song}}
            \end{ExplainCard}

            \begin{ExplainCard}{prioritize}[v][B2]
            \EN{to treat something as more important than other things.}
            \SY{rank; give priority to; foreground}
            \VI{ưu tiên; đặt lên hàng đầu.}
            \EX{They prioritize fit and quality over hype.}
            \EX{Retailers prioritize customer retention in tight markets.}
            \CO{\textit{prioritize} quality/safety/requests}
            \end{ExplainCard}

            \begin{ExplainCard}{loyalty card}[n][B2]
            \EN{a card that records purchases and offers rewards or discounts to regular customers.}
            \SY{membership card; rewards card}
            \VI{thẻ khách hàng thân thiết.}
            \EX{Flash your loyalty card for 10\% off.}
            \EX{Loyalty-card data reveals repeat-purchase patterns.}
            \CO{\textit{issue/use} a loyalty card; loyalty \textit{program}}
            \end{ExplainCard}

            \begin{ExplainCard}{put on}[phr.v][B2]
            \EN{(1) to wear a piece of clothing; (2) to organize or present (a show/event).}
            \SY{(1) don \quad (2) stage; mount}
            \VI{(1) mặc/đội; (2) tổ chức, dàn dựng.}
            \EX{(1) I put on a new cap from this shop.}
            \EX{(2) The gallery put on a capsule exhibition last month.}
            \CO{\textit{put on} a coat/hat; \textit{put on} a show/event}
            \end{ExplainCard}
        \end{VocabExplain}

    \noindent
    \textbf{Part 3.}
    \begin{qa}{What types of local business are there in your neighbourhood? Are there any restaurants, shops or dentists for example?}
    The residential area that my family are living in is relatively convenient as it has an \textbf{entertainment complex} including a \textbf{waterfront park} and numerous coffee bars where customers can buy takeaway drinks very easily. There are also a chain of \textbf{dining establishments} ranging from Thai to Hong Kong cuisine. And, I should not forget to mention a dentist office has just opened few months ago in the street corner.
    \end{qa}

    \begin{qa}{Do you think local businesses are important for a neighbourhood? In what way?}
    Obviously, most local businesses are the \textbf{backbone} of local economy. Compared to giant \textbf{conglomerates}, local sellers can provide more \textbf{personal touch} and know their customers well enough to consult products. On a social perspective, local companies not only promote local products but also \textbf{foster} a sense of relationship among neighbours who pop by shops on a daily basis.
    \end{qa}

    \begin{qa}{How do large shopping malls and commercial centres affect small local businesses? Why do you think that is?}
    Well, shopping \textbf{corporations} and supermarkets will gain \textbf{competitive edge} compared to small businesses. For one, to customers' delight, big shopping malls are equipped with \textbf{top-notch} facilities such as \textbf{ventilation} system or spacious parking space to cater for every individual need. Besides, \textbf{electronic signage} in shopping malls can act as salesmen to provide sufficient information for buyers among various choices. Personally, the modernization of shopping malls may put the existence of small ones \textbf{in jeopardy}.
    \end{qa}

    \begin{qa}{Why do some people want to start their own business?}
    I guess the reasons why some \textbf{get their own business off the ground} may lie in the fact that some employees are \textbf{fed up with} the \textbf{monotonous routine} that they have to follow every day in their workplace. Gradually, they have a change of mind and want to take on a new career as an \textbf{entrepreneur}. The second reason can be attributed to financial benefits. A lot of people get their own business underway once they have \textbf{accumulated} enough experience, and being a boss means they can earn more money than before.
    \end{qa}

    \begin{qa}{Are there any disadvantages to running a business? Which is the most serious?}
    Of course, every business has its own \textbf{potential risks}. The stress that comes with having full responsibility for their business is one of the disadvantages of business \textbf{ownership}. From initial business planning to day-to-day operations, their \textbf{full commitment} is required to keep things going. Also, they may find themselves \textbf{wearing many hats}, especially when they are \textbf{starting from the scratch} and might not yet have any staff, which is so stressful.
    \end{qa}

    \begin{qa}{What are the most important qualities that a good business person needs? Why is that?}
    Being a leader has never been \textbf{a doddle}, I must say, so a mixture of qualities is necessary. \textbf{Courage} and ambition are the two features that I want to take first. If leaders possess these qualities, they will not hesitate to take risks and \textbf{initiatives} to advance, the business and boost revenue. Then, credibility is something favoured by lots of staff because they know they can rely on their boss and ask for constructive advice.
    \end{qa}

        \begin{VocabExplain}[Part 3]
            \begin{ExplainCard}{entertainment complex}[n][B2]
            \EN{a large site with multiple leisure venues (cinemas, cafés, arcades, etc.).}
            \SY{leisure center; recreation complex}
            \VI{khu phức hợp giải trí.}
            \EX{The new entertainment complex draws families at weekends.}
            \EX{Urban plans place the entertainment complex near transit hubs.}
            \CO{open/build an \textit{entertainment complex}; mixed-use \textit{complex}}
            \end{ExplainCard}

            \begin{ExplainCard}{waterfront park}[n][B2]
            \EN{a public park located along a river, lake, or seaside.}
            \SY{riverside park; lakeside promenade}
            \VI{công viên ven mặt nước.}
            \EX{Joggers love the waterfront park at dawn.}
            \EX{Waterfront parks can mitigate heat-island effects in dense cities.}
            \CO{\textit{revitalize/design} a waterfront park}
            \end{ExplainCard}

            \begin{ExplainCard}{dining establishment}[n][C1]
            \EN{a place that serves food to customers.}
            \SY{restaurant; eatery}
            \VI{cơ sở phục vụ ăn uống; nhà hàng/quán ăn.}
            \EX{New dining establishments keep the area lively.}
            \EX{Licensing rules apply to all dining establishments downtown.}
            \CO{\textit{fine/causal} dining establishment}
            \end{ExplainCard}

            \begin{ExplainCard}{backbone}[n][B2]
            \EN{the most important support or foundation of something.}
            \SY{mainstay; bedrock; pillar}
            \VI{xương sống; trụ cột.}
            \EX{Small firms are the backbone of the local economy.}
            \EX{SMEs form the backbone of national employment statistics.}
            \CO{the \textit{backbone of} industry/agriculture}
            \end{ExplainCard}

            \begin{ExplainCard}{conglomerate}[n][B2]
            \EN{a very large company that owns several different businesses.}
            \SY{corporate group; holding company}
            \VI{tập đoàn đa ngành.}
            \EX{Local shops struggle against global conglomerates.}
            \EX{Conglomerates diversify to spread market risk.}
            \CO{multinational \textit{conglomerate}; media/tech \textit{conglomerate}}
            \end{ExplainCard}

            \begin{ExplainCard}{personal touch}[n][B2]
            \EN{an individual, friendly way of dealing with customers.}
            \SY{human touch; personalized service}
            \VI{sự chăm sóc mang tính cá nhân.}
            \EX{Her bakery's personal touch wins loyal clients.}
            \EX{Adding a personal touch improves service satisfaction scores.}
            \CO{add/bring a \textit{personal touch} to service}
            \end{ExplainCard}

            \begin{ExplainCard}{competitive edge}[n][B2]
            \EN{an advantage that makes someone/something more successful.}
            \SY{competitive advantage; edge}
            \VI{lợi thế cạnh tranh.}
            \EX{Top-notch service gives boutiques a competitive edge.}
            \EX{Analytics can create a durable competitive edge.}
            \CO{\textit{gain/maintain} a competitive edge}
            \end{ExplainCard}

            \begin{ExplainCard}{top-notch}[adj][B2]
            \EN{of the highest quality.}
            \SY{first-rate; premium; excellent}
            \VI{hàng đầu; chất lượng tuyệt hảo.}
            \EX{The mall boasts top-notch facilities.}
            \EX{Top-notch logistics cut delivery times by half.}
            \CO{\textit{top-notch} service/facilities/talent}
            \end{ExplainCard}

            \begin{ExplainCard}{ventilation}[n][C1]
            \EN{the movement of fresh air into and out of an enclosed space.}
            \SY{air circulation; airflow}
            \VI{thông gió; lưu thông không khí.}
            \EX{Good ventilation keeps the store comfortable.}
            \EX{Standards require adequate ventilation in public venues.}
            \CO{\textit{adequate/proper} ventilation; ventilation \textit{system}}
            \end{ExplainCard}

            \begin{ExplainCard}{electronic signage}[n][B2]
            \EN{digital displays used to show information or adverts.}
            \SY{digital sign; LED display}
            \VI{biển hiệu điện tử; màn hình quảng cáo.}
            \EX{Electronic signage guides shoppers to promotions.}
            \EX{Airports rely on electronic signage for real-time updates.}
            \CO{install/update \textit{electronic signage}}
            \end{ExplainCard}

            \begin{ExplainCard}{in jeopardy}[phrase][B2]
            \EN{in danger of harm, loss, or failure.}
            \SY{at risk; in peril}
            \VI{gặp nguy cơ; bị đe doạ.}
            \EX{Rising rents put small stores in jeopardy.}
            \EX{Policy cuts placed heritage programs in jeopardy.}
            \CO{\textit{put/place} sth \textit{in jeopardy}}
            \end{ExplainCard}

            \begin{ExplainCard}{get (a business) off the ground}[idiom][B2]
            \EN{to start something successfully and make it work.}
            \SY{launch; kick off; set up}
            \VI{khoi động/đưa (doanh nghiệp) vào hoạt động.}
            \EX{They got the café off the ground with family help.}
            \EX{Seed funding helped the platform get off the ground.}
            \CO{\textit{get/set} sth \textit{off the ground}}
            \end{ExplainCard}

            \begin{ExplainCard}{fed up with}[adj][B2]
            \EN{annoyed or bored with something you have experienced for too long.}
            \SY{weary of; sick of}
            \VI{chán ngán; chịu hết nổi.}
            \EX{She's fed up with the commute.}
            \EX{Workers were fed up with monotonous processes, surveys show.}
            \CO{\textit{be/get} fed up with sth}
            \end{ExplainCard}

            \begin{ExplainCard}{monotonous routine}[n][B2]
            \EN{a set of tasks that are dull and repetitive.}
            \SY{tedious schedule; humdrum routine}
            \VI{nhịp điệu đơn điệu, nhàm chán.}
            \EX{He left to escape a monotonous routine.}
            \EX{Automation reduces the most monotonous routine work.}
            \CO{\textit{break/escape} a monotonous routine}
            \end{ExplainCard}

            \begin{ExplainCard}{entrepreneur}[n][B2]
            \EN{a person who starts and runs a business, taking on financial risk.}
            \SY{founder; business owner}
            \VI{doanh nhân khởi nghiệp; chủ doanh nghiệp.}
            \EX{Young entrepreneurs are reshaping retail.}
            \EX{Programs mentor entrepreneurs through the first year.}
            \CO{\textit{aspiring/serial} entrepreneur; tech \textit{entrepreneur}}
            \end{ExplainCard}

            \begin{ExplainCard}{wear many hats}[idiom][B2]
            \EN{to have many different roles or responsibilities.}
            \SY{multitask; juggle roles}
            \VI{đảm nhiệm nhiều vai trò.}
            \EX{Small-shop owners wear many hats daily.}
            \EX{In startups, leaders wear many hats from sales to ops.}
            \CO{\textit{have to} wear many hats}
            \end{ExplainCard}

            \begin{ExplainCard}{start from scratch}[idiom][B2]
            \EN{to begin with no advantage or preparation.}
            \SY{begin anew; build from the ground up}
            \VI{bắt đầu từ con số không.}
            \EX{They started from scratch with a pop-up stall.}
            \EX{The team rebuilt the dataset from scratch for accuracy.}
            \CO{\textit{start/rebuild} from scratch}
            \end{ExplainCard}

            \begin{ExplainCard}{(be) a doddle}[idiom][B2]
            \EN{British informal: very easy to do.}
            \SY{a breeze; a cakewalk}
            \VI{dễ ợt; quá đơn giản.}
            \EX{Managing a kiosk isn't a doddle at all.}
            \EX{What seems a doddle in theory can be hard in practice.}
            \CO{\textit{be} a doddle; \textit{no} doddle}
            \end{ExplainCard}

            \begin{ExplainCard}{initiative}[n][C1]
            \EN{(1) a new plan or action; (2) the ability to take the first step independently.}
            \SY{(1) program \quad (2) enterprise; drive}
            \VI{(1) sáng kiến/chương trình; (2) tính chủ động.}
            \EX{(1) The city launched a green-retail initiative.}
            \EX{(2) Good managers show initiative under pressure.}
            \CO{take/show \textit{initiative}; policy \textit{initiative}}
            \end{ExplainCard}
        \end{VocabExplain}

    \begin{VocabHighlights}
        \VH{believe it or not}{(phrase) this is surprising but true}{(cụm từ) tin hay không thì tùy}
        \VH{trendy}{(adj) influenced by or expressing the most recent fashions or ideas; modern in style}{(tính từ) hợp mốt, thịnh hành}
        \VH{on the wane}{(phrase) it is becoming weaker or less common}{(cụm từ) hết thời, yếu dần}
        \VH{to send somebody into ecstasies}{(phrase) to make somebody experience a feeling or state of very great happiness}{(cụm từ) làm ai sướng phát cuồng lên}
        \VH{to have too many irons in the fire}{(idiom) to be engaged in too many activities}{(thành ngữ) tham gia quá nhiều hoạt động khác nhau, bận rộn}
        \VH{on the move}{(phrase) in the process of moving from one place or job to another}{(cụm từ) đang di chuyển}
        \VH{to be given priority over}{(phrase) to be prioritized over}{(cụm từ) được ưu tiên hơn so với}
        \VH{to have a go at}{(idiom) to try}{(thành ngữ) thử làm gì}
        \VH{calloused}{(adj) made rough and hard, usually by hard work}{(tính từ) chai sạn}
        \VH{dexterity}{(n) skill in using your hands or your mind}{(danh từ) sự khéo tay}
        \VH{tone-deaf}{(adj) (of a person) unable to perceive differences of musical pitch accurately}{(tính từ) không phân biệt được các nốt nhạc}
        \VH{to model oneself on}{(phr.v) to copy the behaviour, style, etc. of somebody you like and respect in order to be like them}{(cụm động từ) học tập, bắt chước ai, lấy ai làm gương}
        \VH{in the fullness of time}{(idiom) when the time is appropriate, usually after a long period}{(thành ngữ) khi chín muồi}
        \VH{to bring the house down}{(idiom) make an audience respond with great enthusiasm, typically as shown by their laughter or applause}{(thành ngữ) làm khán giả vỗ tay hoan hô nhiệt liệt}
        \VH{a shopaholic}{(n) a person who enjoys shopping very much and does it a lot}{(danh từ) người nghiện mua sắm}
        \VH{complex}{(adj) made of many different things or parts that are connected; difficult to understand}{(tính từ) phức tạp}
        \VH{waterfront}{(n) a part of a town or an area that is next to water, for example in a harbor}{(danh từ) bờ sông}
        \VH{dining establishments}{(phrase) a public place where food and drinks are served}{(cụm từ) nhà hàng}
        \VH{backbone}{(n) the chief support of a system or organization; the mainstay}{(danh từ) trụ cột}
        \VH{conglomerate}{(n) a company that owns several smaller businesses whose products or services are usually very different}{(danh từ) tập đoàn}
        \VH{personal touch}{(n) an element or feature contributed by someone to make something less impersonal}{(danh từ) liên lạc cá nhân}
        \VH{to foster}{(v) to encourage something to develop}{(động từ) nuôi dưỡng}
        \VH{a corporation}{(n) a large business company}{(danh từ) tập đoàn; công ty lớn}
        \VH{competitive edge}{(phrase) the fact that a company has an advantage over its competitors}{(cụm từ) lợi thế cạnh tranh}
        \VH{top-notch}{(adj) excellent; of the highest quality}{(tính từ) hạng nhất, hàng đầu}
        \VH{ventilation}{(n) the fact of allowing fresh air to enter and move around a room, building, etc.}{(danh từ) thông gió, thông hơi}
        \VH{electronic signage}{(phrase) digital displays use technologies such as LCD, LED, projection}{(cụm từ) bảng hiệu kĩ thuật số}
        \VH{in jeopardy}{(phrase) in danger of being damaged or destroyed}{(cụm từ) ở trong hoàn cảnh nguy hiểm}
        \VH{to get something off the ground}{(idiom) to start something or to succeed in V-ing}{(thành ngữ) bắt đầu hoặc thành công}
        \VH{to be fed up with}{(p2) annoyed or upset at a situation or treatment}{(phân từ 2) cảm thấy chán ngán}
        \VH{monotonous}{(adj) never changing and therefore boring}{(tính từ) đơn điệu}
        \VH{entrepreneur}{(n) a person who makes money by starting or running businesses, especially when this involves taking financial risks}{(danh từ) doanh nhân}
        \VH{to accumulate}{(v) to gradually get more and more of something over a period of time}{(động từ) tích lũy}
        \VH{ownership}{(n) the fact of owning something}{(danh từ) quyền sở hữu}
        \VH{commitment}{(n) a promise to do something or to behave in a particular way; a promise to support somebody/something; the fact of committing yourself}{(danh từ) cam kết, dấn thân}
        \VH{to wear many hats}{(idiom) to have many jobs or roles}{(thành ngữ) đảm đương nhiều việc, nhiều chức vụ cùng lúc}
        \VH{to start from the scratch}{(phrase) begin at the beginning}{(cụm từ) bắt đầu lại từ đầu}
        \VH{doddle}{(n) a very easy task}{(danh từ) một nhiệm vụ dễ dàng}
        \VH{courage}{(n) the ability to do something dangerous, or to face pain or opposition, without showing fear}{(danh từ) lòng can đảm}
        \VH{an initiative}{(n) a new plan for dealing with a particular problem or for achieving a particular purpose}{(danh từ) sáng kiến, khởi xướng}
    \end{VocabHighlights}
    \end{test}

    \begin{test}{TEST 3}
    \noindent
    \textbf{Part 1. Travel}
    \begin{qa}{Do you enjoy travelling? [Why?/Why not?]}
    Travelling is \textbf{my cup of tea}. It enables me to \textbf{broaden my horizons} by \textbf{getting acquainted} with different cultures in lots of regions nationwide and worldwide. As a result, it also helps me \textbf{think outside the box}. For example, I never thought there would be such thing as a Red Light District in Amsterdam, where a sensitive matter like prostitution is legal and \textbf{taken for granted} although it is strictly forbidden in Vietnam.
    \end{qa}

    \begin{qa}{Have you done much travelling? [Why?/Why not?]}
    Well, many people call me an \textbf{avid traveller} as I've been to 15 countries across Asia and Europe, other continents exclusive. I had to go on some business trips in Asian countries like Korea, Brunei so I \textbf{couldn't pass up the chance} to explore their cultures as well. When I was in the U.K as a postgraduate student, I also managed to \textbf{seize} this opportunity to travel around European countries. It normally \textbf{costs an arm and a leg} to fly from Vietnam to Europe but the flight tickets were relatively affordable as I departed from the U.K. The longer the distance is, the more costly the fares are.
    \end{qa}

    \begin{qa}{Do you think it's better to travel alone or with other people? [Why?]}
    It depends on the purpose of the trip. If I choose leisure travel, then I \textbf{have a soft spot} for travelling in groups, say family relatives or friends. If adventure travel is my choice, I believe going alone suits me as there's no need for me to \textbf{drag on} in the same place to take numerous photos like in leisure travel. Instead, I can go \textbf{from place to place} as adventure travel grants me the freedom to explore the world.
    \end{qa}

    \begin{qa}{Where would you like to travel in the future? [Why?]}
    I haven't had a chance to visit the USA yet so perhaps I'd try to pay a visit to this country later on. It's not only a place where a \textbf{throng} of my friends are based but also an \textbf{entertainment hub} that may someone who prefers a hectic lifestyle like me. Reuniting with friends and having them \textbf{take me around} sound fantastic to me.
    \end{qa}

        \begin{VocabExplain}[Part 1]
            \begin{ExplainCard}{my cup of tea}[idiom][B2]
            \EN{something that one particularly likes or is good at.}
            \SY{be one's thing; suit someone}
            \VI{đúng “gu”/sở thích của ai.}
            \EX{City breaks are my cup of tea.}
            \EX{Qualitative research may not be every engineer's cup of tea.}
            \CO{not really my cup of tea; very much my cup of tea}
            \end{ExplainCard}

            \begin{ExplainCard}{broaden one's horizons}[phrase][B2]
            \EN{to increase the range of one's knowledge or experiences.}
            \SY{expand outlook; widen perspective}
            \VI{mở mang tầm mắt/kiến thức.}
            \EX{Backpacking broadened my horizons.}
            \EX{Study-abroad programs broaden students' horizons.}
            \CO{travel/study \textit{to} broaden one's horizons}
            \end{ExplainCard}

            \begin{ExplainCard}{get acquainted (with)}[phr.v][B2]
            \EN{to become familiar with a person, place, or thing.}
            \SY{familiarize oneself with; get to know}
            \VI{làm quen; tìm hiểu.}
            \EX{We got acquainted with local customs.}
            \EX{New hires get acquainted with tools during onboarding.}
            \CO{\textit{get/become} acquainted with sb/sth}
            \end{ExplainCard}

            \begin{ExplainCard}{think outside the box}[idiom][B2]
            \EN{to think creatively and unconventionally.}
            \SY{innovate; break the mold}
            \VI{suy nghĩ sáng tạo, vượt khuôn mẫu.}
            \EX{The brief asks us to think outside the box.}
            \EX{Design sprints help teams think outside the box.}
            \CO{\textit{encourage} people to think outside the box}
            \end{ExplainCard}

            \begin{ExplainCard}{take for granted}[v phrase][B2]
            \EN{(1) to accept something as normal or true without question; (2) to fail to appreciate someone/something.}
            \SY{(1) assume \quad (2) undervalue}
            \VI{(1) coi là hiển nhiên; (2) xem nhẹ/không trân trọng.}
            \EX{We often take clean water for granted.}
            \EX{Don't take your team's help for granted.}
            \CO{\textit{be} taken for granted; \textit{take} sth for granted}
            \end{ExplainCard}

            \begin{ExplainCard}{avid traveller}[n phrase][B2]
            \EN{a person who is extremely enthusiastic about travelling.}
            \SY{keen traveler; travel enthusiast}
            \VI{người mê du lịch.}
            \EX{As an avid traveller, she tracks every UNESCO site.}
            \EX{Avid travellers often write detailed trip reports.}
            \CO{\textit{be} an avid traveller; avid \textit{reader/collector}}
            \end{ExplainCard}

            \begin{ExplainCard}{pass up the chance}[phrase][C1]
            \EN{to not use or accept an opportunity.}
            \SY{miss/turn down an opportunity}
            \VI{bỏ lỡ/cự tuyệt cơ hội.}
            \EX{Don't pass up the chance to see Kyoto in spring.}
            \EX{He passed up the chance to intern abroad.}
            \CO{\textit{pass up} a chance/offer/opportunity}
            \end{ExplainCard}

            \begin{ExplainCard}{seize (an opportunity)}[v][B2]
            \EN{to take something quickly and eagerly.}
            \SY{grab; capitalize on}
            \VI{chớp lấy/nắm bắt (cơ hội).}
            \EX{She seized the opportunity to present.}
            \EX{Firms seized opportunities created by open borders.}
            \CO{\textit{seize} the chance/opportunity/moment}
            \end{ExplainCard}

            \begin{ExplainCard}{cost an arm and a leg}[idiom][B2]
            \EN{to be very expensive.}
            \SY{be a fortune; cost a bomb}
            \VI{đắt cắt cổ; tốn cả “một tay một chân”.}
            \EX{Flights in peak season cost an arm and a leg.}
            \EX{Downtown rents cost an arm and a leg for startups.}
            \CO{\textit{cost} an arm and a leg}
            \end{ExplainCard}

            \begin{ExplainCard}{have a soft spot (for)}[idiom][B2]
            \EN{to feel a particular fondness or affection for someone/something.}
            \SY{be fond of; have a weakness for}
            \VI{rất thích/thiên vị (ai/cái gì).}
            \EX{I have a soft spot for group trips.}
            \EX{Reviewers sometimes have a soft spot for vintage gear.}
            \CO{\textit{have} a soft spot for sth/sb}
            \end{ExplainCard}

            \begin{ExplainCard}{drag on}[phr.v][B2]
            \EN{to continue for longer than necessary in a dull or tedious way.}
            \SY{linger; overrun}
            \VI{kéo dài lê thê; dai dẳng.}
            \EX{The tour dragged on at the photo stop.}
            \EX{Meetings that drag on sap team energy.}
            \CO{\textit{drag on} for hours; discussion \textit{drags on}}
            \end{ExplainCard}

            \begin{ExplainCard}{from place to place}[phrase][B2]
            \EN{moving or traveling between many locations.}
            \SY{around; here and there}
            \VI{từ nơi này sang nơi khác.}
            \EX{Backpackers move from place to place freely.}
            \EX{The field team travelled from place to place collecting data.}
            \CO{\textit{go/travel} from place to place}
            \end{ExplainCard}

            \begin{ExplainCard}{throng}[n][B2]
            \EN{a large, densely packed crowd of people.}
            \SY{crowd; multitude; horde}
            \VI{đám đông đông đúc.}
            \EX{A throng gathered in Times Square.}
            \EX{Festivals attract throngs of visitors each summer.}
            \CO{a \textit{throng of} tourists/fans}
            \end{ExplainCard}

            \begin{ExplainCard}{entertainment hub}[n][B2]
            \EN{an area with many venues and activities for leisure.}
            \SY{entertainment center; hotspot}
            \VI{trung tâm giải trí.}
            \EX{Las Vegas is an entertainment hub.}
            \EX{New transit lines turned the waterfront into an entertainment hub.}
            \CO{\textit{become/develop into} an entertainment hub}
            \end{ExplainCard}

            \begin{ExplainCard}{take (sb) around}[phr.v][B2]
            \EN{to show someone the interesting places in a town or building.}
            \SY{show (sb) around; give a tour}
            \VI{dẫn ai đi tham quan/đi chơi vòng quanh.}
            \EX{Local friends will take me around the city.}
            \EX{Volunteers took visitors around the campus.}
            \CO{\textit{take/show} sb around + place}
            \end{ExplainCard}
        \end{VocabExplain}

    \noindent
    \textbf{Part 2.}
    \begin{qa}{Describe a child that you know. You should say:}
    \begin{itemize}
    \item Who this child is and how often you see him or her
    \item How old this child is
    \item What he or she is like
    \item and explain what you feel about this child.
    \end{itemize}

    To be honest, I am a teacher, so I have to work with numerous children on a daily basis. There is a child that has made a profound impression on me. It is Dang Nhat Anh, who is \textbf{my pet}. I would like to \textbf{highlight the fact} that he is just 6 years old. He is studying at Thanh Cong primary school, and he is the son of my next-door neighbor who I've gotten on well with for more than 10 years. That's why I \textbf{know him like the palm of my hand}. We often see each other daily, because his mother is my close neighbor. Each day, when she \textbf{picks him up} from school, she usually \textbf{stops by} my house to \textbf{chit-chat} or gives me stuff because I usually help her shop for vegetables and fruit. The reason I \textbf{have a soft spot for} Nhat Anh is that he is very \textbf{brainy}. In other words, he is \textbf{quick on the uptake}. He is \textbf{apt at} English, so he is more \textbf{advanced} in language development than his \textbf{peers}. What I like most about him is that he is very kind. He usually \textbf{gives his classmates a hand} if they have difficulty in learning English. For example, if a friend does not remember new words, Nhat Anh will help his buddy \textbf{jog his memory} by creating a quiz to study together. That's why he is \textbf{the apple of my eye}.
    \end{qa}

        \begin{VocabExplain}[Part 2]
            \begin{ExplainCard}{pet (teacher's pet)}[n][B2]
            \EN{a student who is especially favored by the teacher.}
            \SY{favorite; blue-eyed boy/girl}
            \VI{học trò cưng; người được thiên vị.}
            \EX{Everyone joked that Nhat Anh was my pet.}
            \EX{Teachers should avoid creating a “teacher's pet” to keep class dynamics fair.}
            \CO{be the teacher's pet; make sb the teacher's pet}
            \end{ExplainCard}

            \begin{ExplainCard}{highlight the fact (that)}[phrase][B2]
            \EN{to emphasize or draw attention to a particular truth.}
            \SY{underscore; stress}
            \VI{nhấn mạnh thực tế rằng.}
            \EX{I must highlight the fact that he's only six.}
            \EX{Reports highlight the fact that early literacy predicts later success.}
            \CO{\textit{highlight the fact that} + clause}
            \end{ExplainCard}

            \begin{ExplainCard}{know (sb/sth) like the palm of one's hand}[idiom][B2]
            \EN{to know someone or something extremely well.}
            \SY{know inside out; be intimately familiar with}
            \VI{biết rõ như lòng bàn tay.}
            \EX{I know him like the palm of my hand.}
            \EX{Local guides know the trails like the palm of their hand.}
            \CO{know a place/person \textit{like the palm of one's hand}}
            \end{ExplainCard}

            \begin{ExplainCard}{pick (sb) up}[phr.v][B2]
            \EN{to collect someone in a vehicle from a place.}
            \SY{collect; fetch}
            \VI{đón ai (bằng xe) từ đâu.}
            \EX{She picks him up after school.}
            \EX{Parents must pick students up at the north gate.}
            \CO{\textit{pick up} a child from school/airport}
            \end{ExplainCard}

            \begin{ExplainCard}{stop by}[phr.v][B2]
            \EN{to make a short visit to a place.}
            \SY{drop by; pop in}
            \VI{tạt qua; ghé thăm nhanh.}
            \EX{She stops by my house every afternoon.}
            \EX{Participants could stop by the help desk for badges.}
            \CO{\textit{stop/drop} by + place}
            \end{ExplainCard}

            \begin{ExplainCard}{chit-chat}[n/v][B2]
            \EN{informal friendly talk about unimportant things.}
            \SY{small talk; chat}
            \VI{tán gẫu; chuyện phiếm.}
            \EX{We had a quick chit-chat at the gate.}
            \EX{Short chit-chat can build rapport at the start of interviews.}
            \CO{have a \textit{chit-chat}; engage in \textit{chit-chat}}
            \end{ExplainCard}

            \begin{ExplainCard}{have a soft spot for}[idiom][B2]
            \EN{to feel a strong fondness or affection for.}
            \SY{be fond of; have a weakness for}
            \VI{rất yêu mến; có cảm tình đặc biệt với.}
            \EX{I have a soft spot for helpful kids.}
            \EX{Donors often have a soft spot for local projects.}
            \CO{\textit{have} a soft spot for sb/sth}
            \end{ExplainCard}

            \begin{ExplainCard}{brainy}[adj][B2]
            \EN{informal: very intelligent.}
            \SY{bright; clever; smart}
            \VI{thông minh; lanh lợi.}
            \EX{He's a brainy little boy.}
            \EX{Brainy pupils thrive in enriched classrooms.}
            \CO{a \textit{brainy} kid/student; \textit{brainy} solutions}
            \end{ExplainCard}

            \begin{ExplainCard}{quick on the uptake}[idiom][B2]
            \EN{able to understand things rapidly.}
            \SY{sharp; quick-witted}
            \VI{tiếp thu nhanh; hiểu nhanh.}
            \EX{She's quick on the uptake in class.}
            \EX{Clinicians quick on the uptake adopt new protocols early.}
            \CO{\textit{be} quick/slow on the uptake}
            \end{ExplainCard}

            \begin{ExplainCard}{apt at}[adj phrase][B2]
            \EN{having a natural ability or skill for something.}
            \SY{adept at; good at; proficient in}
            \VI{có khiếu; giỏi về.}
            \EX{He is apt at English.}
            \EX{Students apt at coding pick up algorithms faster.}
            \CO{\textit{apt/adept at} + V-ing/NP}
            \end{ExplainCard}

            \begin{ExplainCard}{advanced}[adj][B2]
            \EN{at a higher, more difficult level than average.}
            \SY{upper-level; higher-level}
            \VI{trình độ cao; phát triển hơn.}
            \EX{She's in the advanced reading group.}
            \EX{An advanced syllabus accelerates language development.}
            \CO{\textit{advanced} class/course/skills}
            \end{ExplainCard}

            \begin{ExplainCard}{peers}[n][B2]
            \EN{people who are of the same age, status, or ability as another.}
            \SY{equals; contemporaries}
            \VI{bạn đồng trang lứa; người cùng nhóm.}
            \EX{He outperformed his peers in vocabulary.}
            \EX{Peer effects influence motivation in classrooms.}
            \CO{\textit{among/with} peers; peer \textit{group}}
            \end{ExplainCard}

            \begin{ExplainCard}{give (sb) a hand}[idiom][B2]
            \EN{to help someone do something.}
            \SY{lend a hand; assist}
            \VI{giúp một tay.}
            \EX{He gives his classmates a hand with homework.}
            \EX{Volunteers gave teachers a hand during exams.}
            \CO{\textit{give/lend} sb a hand \textit{with} sth}
            \end{ExplainCard}

            \begin{ExplainCard}{jog (sb's) memory}[idiom][C1]
            \EN{to cause someone to remember something.}
            \SY{prompt; refresh someone's memory}
            \VI{gợi nhớ; khơi lại ký ức.}
            \EX{The quiz helped jog his memory of new words.}
            \EX{Visual cues can jog participants' memory in recall tasks.}
            \CO{\textit{jog} my/your memory; \textit{memory-jogger}}
            \end{ExplainCard}

            \begin{ExplainCard}{the apple of one's eye}[idiom][B2]
            \EN{a person who is greatly cherished and loved.}
            \SY{darling; prized one}
            \VI{báu vật của lòng; người rất được yêu quý.}
            \EX{His little sister is the apple of his eye.}
            \EX{For many teachers, a hardworking class can be the apple of their eye.}
            \CO{\textit{be} the apple of sb's eye}
            \end{ExplainCard}
        \end{VocabExplain}

    \noindent
    \textbf{Part 3.}
    \begin{qa}{How much time do children spend with their parents in your country? Do you think that is enough?}
    Normally, the amount of time varies based on the professions of the parents and the studies of their children. Vietnamese tradition \textbf{dictates} that fathers often \textbf{bring home the bacon} while mothers are mainly in charge of \textbf{childrearing}. Things have changed due to the rise of \textbf{double-income families} and these days, both parents are responsible.
    \end{qa}

    \begin{qa}{How important do you think spending time together is for the relationships between parents \& children? Why?}
    Well, \textbf{family bonding time} is of great value to relationships among all family members. For one, this practice is a \textbf{demonstration} of love between parents and their children. \textbf{Maternal} bond or \textbf{paternal} love can be strongly developed through moments they spend together. Additionally, instead of gifts sold \textbf{at a premium} or \textbf{elaborate} meals, spending time with your children exclusively is the best way to get to know well about them and support them whenever they seek for advice. In other words, it is \textbf{imperative} that parents devote their time to \textbf{inculcating} their children with the right moral values. Once brought up in an ethically \textbf{upright} manner in a sufficient amount of quality time, the children will turn the society into a gracious one in the future.
    \end{qa}

    \begin{qa}{Have relationships between parents and children changed in recent years? Why do you think that is?}
    I would describe most parents are supportive, \textbf{affectionate} and strict if their children make a mistake, no matter how trivial or \textbf{grave} it is. Twenty years ago, I suppose kids could not expect to win an argument with their parents, and parental words were not just guidance but \textbf{command}. Having said that, things are a bit different these days. Parents nowadays are willing to listen to their children, and recognize their mistakes to \textbf{rectify} if necessary. It is a fair deal, I believe.
    \end{qa}

    \begin{qa}{What are the most popular free-time activities with children today?}
    Frankly speaking, I am an \textbf{outdoorsy} person, so I am quite fond of \textbf{collective activities}. In my opinion, conducting some physical activities such as playing badminton or simply cracking \textbf{perplexing} puzzles can strengthen family relationships. Otherwise, going for a picnic or heading for the beach are common to family members on their holidays.
    \end{qa}

    \begin{qa}{Do you think the free-time activities children do today are good for their health? Why is that?}
    No, not really. On the one hand, engaging in physical activities really does wonders for children's health because this can help them to \textbf{keep in trim} and boost nervous system. However, that is not the case for playing online games. It is true that excessive playing video games can lead to a \textbf{sedentary lifestyle} and \textbf{impair} children health. More seriously, their eyes might be \textbf{glued to the computer screen} for hours on end, which can cause the \textbf{loss of vision}.
    \end{qa}

    \begin{qa}{How do you think children's activities will change in the future? Will this be a positive change?}
    Yes, children's favorite activities will be subject to the changes in technological world. Compared to traditional games, playing computer games is simple, and hence more \textbf{enticing} to children. To illustrate, with electronic \textbf{gadgets} such as an Internet-connected smart phone, children can access \textbf{a bunch of} video games in their free time. However, it is sometimes inconvenient and impractical to invite some of their friends to play football, for example.
    \end{qa}

        \begin{VocabExplain}[Part 3]
            \begin{ExplainCard}{dictate}[v][B2]
            \EN{to state or decide something with authority; to lay down a rule.}
            \SY{decree; ordain; prescribe}
            \VI{quy định; chi phối; ra lệnh.}
            \EX{Tradition dictates that elders are respected first.}
            \EX{Market forces often dictate pricing strategies.}
            \CO{dictate terms/policy; tradition/convention \textit{dictates}}
            \end{ExplainCard}

            \begin{ExplainCard}{bring home the bacon}[idiom][B2]
            \EN{to earn money to support a family.}
            \SY{be the breadwinner; earn a living}
            \VI{kiếm tiền nuôi gia đình.}
            \EX{For years, her mother brought home the bacon.}
            \EX{In many households, both partners bring home the bacon now.}
            \CO{\textit{be} the one to bring home the bacon}
            \end{ExplainCard}

            \begin{ExplainCard}{childrearing}[n][B2]
            \EN{the process of raising and caring for children.}
            \SY{upbringing; parenting}
            \VI{nuôi dạy con cái.}
            \EX{Childrearing practices differ across cultures.}
            \EX{Work–life policies support balanced childrearing.}
            \CO{childrearing \textit{practices}/duties/responsibilities}
            \end{ExplainCard}

            \begin{ExplainCard}{double-income family}[n][B2]
            \EN{a household in which both parents earn wages.}
            \SY{dual-earner family; two-income household}
            \VI{gia đình hai nguồn thu nhập.}
            \EX{Urban areas see more double-income families.}
            \EX{Childcare costs shape choices in double-income families.}
            \CO{double-income \textit{household/family}}
            \end{ExplainCard}

            \begin{ExplainCard}{family bonding time}[n][B2]
            \EN{time spent together to strengthen family relationships.}
            \SY{quality time; family time}
            \VI{thời gian gắn kết gia đình.}
            \EX{Weekend meals are our family bonding time.}
            \EX{Programs promoting family bonding time improve children's outcomes.}
            \CO{\textit{have/prioritize} family bonding time}
            \end{ExplainCard}

            \begin{ExplainCard}{at a premium}[idiom][C1]
            \EN{costing more than usual; scarce and therefore valuable.}
            \SY{scarce; costly}
            \VI{hiếm và đắt; khan hiếm.}
            \EX{Babysitters are at a premium during holidays.}
            \EX{Urban space is at a premium near schools.}
            \CO{\textit{be} at a premium}
            \end{ExplainCard}

            \begin{ExplainCard}{elaborate}[adj][B2]
            \EN{carefully planned and detailed; complicated.}
            \SY{lavish; intricate}
            \VI{cầu kỳ; công phu.}
            \EX{They skipped elaborate gifts and chose time together.}
            \EX{An elaborate protocol governed the experiment.}
            \CO{\textit{elaborate} meal/plan/ceremony}
            \end{ExplainCard}

            \begin{ExplainCard}{imperative}[adj][C1]
            \EN{extremely important and needing immediate attention.}
            \SY{essential; crucial; pressing}
            \VI{cấp thiết; bắt buộc.}
            \EX{It's imperative that parents listen to their kids.}
            \EX{Rapid response is imperative in emergency care.}
            \CO{\textit{it is} imperative \textit{that} + clause}
            \end{ExplainCard}

            \begin{ExplainCard}{inculcate}[v][B2]
            \EN{to teach an idea or habit firmly by repetition.}
            \SY{instill; implant; imbue}
            \VI{thấm nhuần; gieo vào.}
            \EX{Adults should inculcate sound values early.}
            \EX{Ethics courses aim to inculcate professional norms.}
            \CO{\textit{inculcate} values/discipline/habits \textit{in} sb}
            \end{ExplainCard}

            \begin{ExplainCard}{upright}[adj][B2]
            \EN{behaving in a morally correct way; honest.}
            \SY{honorable; principled}
            \VI{chính trực; ngay thẳng.}
            \EX{An upright approach earns children's trust.}
            \EX{Upright conduct is central to civic education.}
            \CO{\textit{ethically} upright; an upright citizen/life}
            \end{ExplainCard}

            \begin{ExplainCard}{affectionate}[adj][B2]
            \EN{showing feelings of liking or love.}
            \SY{loving; warm; tender}
            \VI{trìu mến; yêu thương.}
            \EX{He's affectionate with his kids.}
            \EX{Affectionate parenting correlates with secure attachment.}
            \CO{\textit{be} affectionate \textit{towards} sb; an \textit{affectionate} hug}
            \end{ExplainCard}

            \begin{ExplainCard}{grave}[adj][B2]
            \EN{serious and worrying; solemn.}
            \SY{serious; severe; weighty}
            \VI{nghiêm trọng; hệ trọng.}
            \EX{Parents react quickly to grave mistakes.}
            \EX{Reports warned of grave environmental risks.}
            \CO{\textit{grave} concern/error/consequence}
            \end{ExplainCard}

            \begin{ExplainCard}{rectify}[v][B2]
            \EN{to correct something that is wrong.}
            \SY{correct; remedy; redress}
            \VI{sửa chữa; khắc phục.}
            \EX{They rectified the misunderstanding with a call.}
            \EX{Policies were rectified after the audit.}
            \CO{\textit{rectify} a problem/error/situation}
            \end{ExplainCard}

            \begin{ExplainCard}{outdoorsy}[adj][B2]
            \EN{enjoying outdoor activities.}
            \SY{open-air loving; nature-oriented}
            \VI{ưa hoạt động ngoài trời.}
            \EX{Our family is pretty outdoorsy.}
            \EX{Outdoorsy programs improve children's fitness levels.}
            \CO{\textit{an} outdoorsy \textit{family/person}; outdoorsy activities}
            \end{ExplainCard}

            \begin{ExplainCard}{perplexing}[adj][B2]
            \EN{confusing and difficult to understand.}
            \SY{puzzling; baffling}
            \VI{rối rắm; gây bối rối.}
            \EX{He likes solving perplexing puzzles.}
            \EX{Perplexing data patterns prompted further study.}
            \CO{\textit{perplexing} problem/question/pattern}
            \end{ExplainCard}

            \begin{ExplainCard}{keep in trim}[idiom][B2]
            \EN{to stay healthy and fit.}
            \SY{stay in shape; keep fit}
            \VI{giữ dáng; giữ sức khoẻ tốt.}
            \EX{Cycling helps kids keep in trim.}
            \EX{Regular exercise keeps older adults in trim, research shows.}
            \CO{\textit{keep/stay} in trim}
            \end{ExplainCard}

            \begin{ExplainCard}{sedentary lifestyle}[n][B2]
            \EN{a way of life with little physical activity.}
            \SY{inactive lifestyle; low-activity routine}
            \VI{lối sống ít vận động.}
            \EX{Too much screen time leads to a sedentary lifestyle.}
            \EX{Sedentary lifestyles increase health risks in youth cohorts.}
            \CO{\textit{lead to/avoid} a sedentary lifestyle}
            \end{ExplainCard}

            \begin{ExplainCard}{glued to the computer screen}[phrase][B2]
            \EN{spending long periods staring at a screen.}
            \SY{fixated on screens; screen-bound}
            \VI{dán mắt vào màn hình máy tính.}
            \EX{Kids are often glued to the computer screen after school.}
            \EX{Being glued to screens correlates with eye strain.}
            \CO{\textit{be} glued to the screen/phone}
            \end{ExplainCard}

            \begin{ExplainCard}{enticing}[adj][B2]
            \EN{attractive and tempting.}
            \SY{appealing; alluring; tempting}
            \VI{hấp dẫn; lôi cuốn.}
            \EX{Mobile games are enticing to children.}
            \EX{Enticing interfaces increase engagement metrics.}
            \CO{\textit{enticing} offer/idea/option}
            \end{ExplainCard}

            \begin{ExplainCard}{gadget}[n][B2]
            \EN{a small device or machine with a practical use.}
            \SY{device; gizmo; appliance}
            \VI{đồ công nghệ/thiết bị nhỏ.}
            \EX{Electronic gadgets are everywhere at home.}
            \EX{Gadgets with parental controls help manage usage.}
            \CO{electronic/smart \textit{gadget}; tech \textit{gadgets}}
            \end{ExplainCard}

            \begin{ExplainCard}{a bunch of}[phrase][B2]
            \EN{a large number or amount of something (informal).}
            \SY{a lot of; loads of}
            \VI{nhiều; một mớ.}
            \EX{They downloaded a bunch of games.}
            \EX{A bunch of studies reached similar conclusions.}
            \CO{\textit{a bunch of} friends/games/reasons}
            \end{ExplainCard}
        \end{VocabExplain}

    \begin{VocabHighlights}
        \VH{fat chance}{(idiom) definitely not}{(thành ngữ) chắc chắn không}
        \VH{to wither}{(v) to become, or cause something to become, weak, dry, and smaller}{(động từ) lụi tàn dần}
        \VH{durable}{(adj) able to last a long time without becoming damaged}{(tính từ) bền}
        \VH{versatile}{(adj) able to change easily from one activity to another or able to be used for many different purposes}{(tính từ) đa năng}
        \VH{my kind of thing}{(idiom) the type of person, thing, place etc that someone usually likes}{(thành ngữ) thứ ưa thích}
        \VH{in lieu of}{(phrase) instead of}{(cụm từ) thay vì}
        \VH{ornamental}{(adj) beautiful rather than useful}{(tính từ) mang tính trang trí}
        \VH{the in-thing}{(n) to be very fashionable at the moment}{(danh từ) thứ thịnh hành hiện tại}
        \VH{in the comfort of}{(phrase) at}{(cụm từ) ở nơi nào thoải mái}
        \VH{aesthetic appreciation}{(phrase) admiration of beauty}{(cụm từ) sự thẩm mỹ học}
        \VH{to haggle with}{(v) to argue with somebody in order to reach an agreement, especially about the price of something}{(động từ) mặc cả, cò kè}
        \VH{to a great extent}{(idiom) mainly}{(thành ngữ) chủ yếu là}
        \VH{to gear towards}{(phrase) to design something with a focus on a particular audience or objective}{(cụm từ) nhắm đến}
        \VH{flag carrier}{(phrase) an airline owned by or strongly identified with a nation}{(cụm từ) hãng hàng không hàng đầu ở một nước}
        \VH{to rise above adversities}{(phrase) to overcome problems}{(cụm từ) vượt qua khó khăn}
        \VH{grime}{(n) dirt that forms a layer on the surface of something}{(danh từ) bụi bề mặt}
        \VH{to be imprinted in somebody's mind}{(phrase) to be put something firmly and deeply into something else, or to be put into something in this way}{(cụm từ) khắc ghi vào}
        \VH{to sharpen}{(v) make or grow sharp}{(động từ) mài giũa}
        \VH{to reach a compromise}{(phrase) gain/achieve/obtain a compromise}{(cụm từ) đạt được một thỏa hiệp}
        \VH{heated}{(adj) excited or angry}{(tính từ) gay gắt hoặc rất sôi nổi}
        \VH{to put the blame on}{(phrase) blame somebody}{(cụm từ) đổ lỗi cho ai đó}
        \VH{industrious}{(adj) busy and hard-working}{(tính từ) siêng năng}
        \VH{a golden chance}{(phrase) a rare chance that unusually happens}{(cụm từ) cơ hội vàng}
        \VH{to place great emphasis on}{(phrase) to emphasize}{(cụm từ) rất chú trọng vào}
        \VH{to attach great importance to}{(phrase) to think that something is important or true and that it should be considered seriously}{(cụm từ) rất coi trọng}
        \VH{to instill a sense of passion}{(phrase) give somebody a passion}{(cụm từ) truyền đam mê}
        \VH{to broaden somebody's horizons}{(idiom) to widen somebody's knowledge}{(thành ngữ) mở mang đầu óc, mở mang tri thức}
        \VH{tricks of the trade}{(idiom) a skill associated with a particular job that makes one more proficient, often acquired through experience}{(thành ngữ) tuyệt chiêu, bí quyết}
        \VH{a meeting-goer}{(phrase) a person who goes to the meeting}{(cụm từ) người hay đi các cuộc hội thảo, họp mặt}
        \VH{interminable}{(adj) lasting a very long time and therefore boring or annoying}{(tính từ) vô tận, liên miên}
        \VH{agenda}{(n) a list of items to be discussed at a meeting}{(danh từ) nội dung cuộc họp, hội thảo}
        \VH{attendees}{(n) a person who attends a meeting, etc}{(danh từ) người tham dự}
        \VH{presentable}{(adj) looking clean and attractive and suitable to be seen in public}{(tính từ) tươm tất}
        \VH{executives}{(n) a person who has an important job as a manager of a company or an organization}{(danh từ) người quản lý, điều hành}
        \VH{to articulate}{(v) to express or explain your thoughts or feelings clearly in words}{(động từ) thể hiện, trình bày bằng lời nói}
        \VH{stakeholders}{(n) a person or company that is involved in a particular organization, project, system, etc., especially because they have invested money in it}{(danh từ) nhà đầu tư, các bên tham gia}
        \VH{a query}{(n) a question, especially one asking for information or expressing a doubt about something}{(danh từ) câu hỏi, vấn đề}
        \VH{competent}{(adj) having enough skill or knowledge to do something well or to the necessary standard}{(tính từ) thành thạo}
        \VH{profusion}{(n) a very large quantity of something}{(danh từ) dồi dào, phong phú}
        \VH{to arise}{(v) to happen; to start to exist}{(động từ) xuất hiện, nảy sinh}
        \VH{pressing}{(adj) requiring quick or immediate action or attention}{(tính từ) bức bối}
        \VH{veto}{(n) the right to refuse to allow something to be done, especially the right to stop a law from being passed or a decision from being taken}{(danh từ) quyền phủ quyết}
        \VH{intriguing}{(adj) very interesting because of being unusual or not having an obvious answer}{(tính từ) hấp dẫn gây hứng thú}
        \VH{conflict of interests}{(phrase) a situation in which the concerns or aims of two different parties are incompatible}{(cụm từ) xung đột lợi ích}
        \VH{social unrest}{(phrase) disagreements or fighting between different groups of people}{(cụm từ) bất ổn xã hội}
        \VH{terrorism}{(n) violent action for political purposes}{(danh từ) khủng bố}
        \VH{to ward off}{(phr.v) to prevent}{(cụm động từ) phòng tránh}
        \VH{to dominate}{(v) to have control over a place or person}{(động từ) thống trị}
    \end{VocabHighlights}
    \end{test}

    \begin{test}{TEST 4}
    \noindent
    \textbf{Part 1. School}
    \begin{qa}{Did you go to secondary/high school near to where you lived? [Why?/Why not?]}
    Yes, I did. It is \textbf{within walking distance} from my house to my secondary and high school. In fact, my high school, Hanoi - Amsterdam consists of both high school and secondary sectors so it is one school only. My school is \textbf{on my doorstep}, which saves me not only time but travelling expenses as well. I felt I \textbf{stroke it lucky} by living near my school.
    \end{qa}

    \begin{qa}{What did you like about your secondary/high school ? [Why?]}
    Well, the thing that \textbf{sets Hanoi - Amsterdam apart} from almost every public school in Vietnam is the tuition fees and students' lifestyle. Firstly, 50,000 VND per month, which equals \$2.2, is each student's monthly tuition fees, which is way too cheap for its \textbf{exceptional} teaching quality available there. Secondly, \textbf{top-notch} students admitted here are not nerds but they adopt an active lifestyle by having various chances to take part in tons of activities throughout an academic year. At Hanoi - Amsterdam, students engage in such events as “Ngay Hoi Anh Tai” - “Festival of the Talented” to select the “Most Wanted Class” and “Ams's Got Talent” based on the worldwide famous format “Got Talent”, etc.
    \end{qa}

    \begin{qa}{Tell me about anything you didn't like at your school?}
    Well, the only thing that \textbf{springs to my mind} is the school ground. When I was a student there, the school ground was so small that there was a single football pitch for me to play football with my friends. To play football at that time, we \textbf{had no alternative} but played on sandy grounds which later resulted in our faces' being covered with grime and dust. Fortunately, after I came back to become a teacher, the ground was renovated and now there are 3 football grounds of artificial grass. Memories of a dusty ground have faded, I guess.
    \end{qa}

    \begin{qa}{How do you think your school could be improved? [Why?/Why not?]}
    My school has \textbf{changed radically} since I graduated. If there's one thing I can \textbf{come up with} to improve my school, that would be introducing another class, Japanese in particular to the overall structure of the school. Japanese is \textbf{coming into vogue} in Vietnam thanks to closer cooperation between Vietnam and Japan, so having a Japanese-specialized class in addition to English, Chinese, French and Russian-specialized ones should be taken into consideration.
    \end{qa}

        \begin{VocabExplain}[Part 1]
            \begin{ExplainCard}{within walking distance}[phrase][C1]
            \EN{close enough to reach comfortably on foot.}
            \SY{nearby; a short walk away}
            \VI{cách một quãng có thể đi bộ được.}
            \EX{The campus is within walking distance of the dorm.}
            \EX{Housing within walking distance of transit reduces car use.}
            \CO{\textit{be} within walking distance (of/from)}
            \end{ExplainCard}

            \begin{ExplainCard}{on my doorstep}[idiom][B2]
            \EN{very near where one lives; right nearby.}
            \SY{right next door; close at hand}
            \VI{ngay gần nhà; sát cạnh.}
            \EX{A new library is on my doorstep.}
            \EX{Amenities on residents' doorstep improve quality of life.}
            \CO{\textit{right} on your doorstep; facilities \textit{on} the doorstep}
            \end{ExplainCard}

            \begin{ExplainCard}{stroke it lucky (\textit{common:} strike it lucky)}{idiom}
            \EN{to be unexpectedly lucky; to have good fortune.}
            \SY{get lucky; hit the jackpot}
            \VI{gặp may; may mắn bất ngờ.}
            \EX{I really stroked it lucky living near school.}
            \EX{Some projects strike it lucky with perfect timing.}
            \CO{\textit{strike/stroke it} lucky; \textit{get} lucky}
            \end{ExplainCard}

            \begin{ExplainCard}{set (sb/sth) apart}[phr.v][B2]
            \EN{to make someone or something different and better than others.}
            \SY{distinguish; differentiate}
            \VI{khiến khác biệt; làm nổi bật.}
            \EX{Strong alumni support sets the school apart.}
            \EX{Curriculum depth sets programs apart in rankings.}
            \CO{\textit{set} A \textit{apart from} B}
            \end{ExplainCard}

            \begin{ExplainCard}{exceptional}[adj][C1]
            \EN{unusually good; much better than average.}
            \SY{outstanding; excellent; remarkable}
            \VI{xuất sắc; vượt trội.}
            \EX{The teachers provide exceptional support.}
            \EX{Exceptional outcomes followed sustained reform.}
            \CO{\textit{exceptional} quality/performance/ability}
            \end{ExplainCard}

            \begin{ExplainCard}{top-notch}[adj][B2]
            \EN{of the highest quality or standard.}
            \SY{first-rate; elite; premium}
            \VI{hạng nhất; hàng đầu.}
            \EX{Only top-notch students were admitted.}
            \EX{Top-notch facilities attract talented applicants.}
            \CO{\textit{top-notch} students/facilities/service}
            \end{ExplainCard}

            \begin{ExplainCard}{spring to mind}[idiom][B2]
            \EN{to come quickly into your thoughts.}
            \SY{immediately occur; come to mind}
            \VI{chợt nảy ra trong đầu.}
            \EX{One complaint springs to my mind.}
            \EX{When asked about risks, several issues sprang to mind.}
            \CO{\textit{spring} to mind; immediately \textit{springs} to mind}
            \end{ExplainCard}

            \begin{ExplainCard}{have no alternative}[phrase][C1]
            \EN{to have no other option or choice.}
            \SY{have no choice; be forced to}
            \VI{không còn lựa chọn nào khác.}
            \EX{We had no alternative but to use the sandy pitch.}
            \EX{Budget limits left administrators with no alternative.}
            \CO{\textit{have} no alternative \textit{but to} + V}
            \end{ExplainCard}

            \begin{ExplainCard}{change radically}[phrase][B2]
            \EN{to change completely or to a great extent.}
            \SY{transform; overhaul}
            \VI{thay đổi triệt để; thay đổi mạnh mẽ.}
            \EX{The campus has changed radically since 2010.}
            \EX{Policies changed radically after the review.}
            \CO{\textit{change} radically/dramatically}
            \end{ExplainCard}

            \begin{ExplainCard}{come up with}[phr.v][B2]
            \EN{to think of or produce (an idea/plan/solution).}
            \SY{devise; formulate; generate}
            \VI{nghĩ ra; đề xuất.}
            \EX{Students came up with a new club idea.}
            \EX{The committee must come up with feasible reforms.}
            \CO{\textit{come up with} ideas/solutions/plans}
            \end{ExplainCard}

            \begin{ExplainCard}{come into vogue}[phrase][B2]
            \EN{to become fashionable or popular.}
            \SY{gain popularity; be in vogue}
            \VI{trở nên thịnh hành.}
            \EX{Japanese has come into vogue among teens.}
            \EX{Project-based learning came into vogue in recent years.}
            \CO{\textit{come/fall} into vogue; be \textit{in} vogue}
            \end{ExplainCard}
        \end{VocabExplain}

    \noindent
    \textbf{Part 2.}
    \begin{qa}{Describe something you don't have now but would really like to own in the future. You should say:}
    \begin{itemize}
    \item What this thing is
    \item How long you have wanted to own it
    \item Where you first saw it
    \item and explain why you would like to own it.
    \end{itemize}

    I would like to tell you about one item that I am \textbf{dying of} possessing in the future. It is an iPhone 11 Pro Max, which is a \textbf{cutting-edge} smartphone \& produced by Apple, a giant in technology. It is labelled the best smartphone to date. I have \textbf{harbored a dream} of owning it since it was first unveiled a few months ago. Its release \textbf{took the world by storm} because it was a subject of discussion in the social media. I also had the opportunity to see it in a promotional campaign launched in a mobile store in Hanoi. There are a couple of reasons why I want to possess an iPhone11 Pro Max. Firstly, it has a \textbf{futuristic} design and comes with new features to unlock the phone and a large OLED screen which \textbf{boasts} fantastic display. Another remarkable feature of this smartphone is \textbf{facial recognition}. To unlock the phone all, what the user needs to do is looking at the phone and then \textbf{swiping up}. In addition, this is the first time that Apple has \textbf{integrated} dual sim connection onto their device. Therefore, an iPhone11 Pro Max allows its users to use two sim cards simultaneously, which is of great convenience as well. More importantly, this phone has an amazing camera. The rear cameras have optical image \textbf{stabilisation} \& fast lenses which allow for great photos even in the case of dim light.
    \end{qa}

        \begin{VocabExplain}[Part 2]
            \begin{ExplainCard}{be dying to (do sth) / be dying for (sth)}[idiom][B2]
            \EN{to be extremely eager to do or have something.}
            \SY{itching to; longing to; can't wait to}
            \VI{rất khao khát/hao hức muốn làm hay có điều gì.}
            \EX{I'm dying to try the new iPhone.}
            \EX{Consumers dying for upgrades often queue overnight at launches.}
            \CO{be \textit{dying to} see/meet/try; be \textit{dying for} a chance}
            \end{ExplainCard}

            \begin{ExplainCard}{cutting-edge}[adj][B2]
            \EN{at the newest and most advanced stage of development.}
            \SY{state-of-the-art; leading-edge; advanced}
            \VI{tối tân; hiện đại nhất.}
            \EX{It's a cutting-edge smartphone with an OLED display.}
            \EX{Cutting-edge sensors enable low-light photography.}
            \CO{\textit{cutting-edge} technology/design/features}
            \end{ExplainCard}

            \begin{ExplainCard}{harbor (a dream/hope)}[v][B2]
            \EN{(1) to keep a thought, feeling, or ambition in one's mind for a long time; (2) to give shelter (nautical/literal).}
            \SY{(1) cherish; nurse \quad (2) shelter}
            \VI{(1) ấp ủ ước mơ/hi vọng; (2) che chở (nghĩa đen).}
            \EX{She has long harbored a dream of studying abroad.}
            \EX{Urban plans aim not to harbor pollutants in street canyons.}
            \CO{\textit{harbor} a dream/ambition/hope}
            \end{ExplainCard}

            \begin{ExplainCard}{take the world by storm}[idiom][B2]
            \EN{to become extremely popular or successful very quickly.}
            \SY{become a sensation; go viral}
            \VI{làm mưa làm gió; gây sốt toàn cầu.}
            \EX{The model took the world by storm after launch.}
            \EX{Short-form video has taken the world by storm in media.}
            \CO{\textit{take} (the world/the market) \textit{by storm}}
            \end{ExplainCard}

            \begin{ExplainCard}{futuristic}[adj][B2]
            \EN{having a very modern design or imagining technology of the future.}
            \SY{ultramodern; forward-looking}
            \VI{mang dáng dấp tương lai; siêu hiện đại.}
            \EX{Its futuristic design really stands out.}
            \EX{Futuristic interfaces reduce friction in device use.}
            \CO{\textit{futuristic} design/architecture/vision}
            \end{ExplainCard}

            \begin{ExplainCard}{boast}[v][B2]
            \EN{(of a place or thing) to have something that is impressive or desirable.}
            \SY{feature; offer; possess}
            \VI{(vật/chỗ) có, sở hữu (đáng tự hào).}
            \EX{The phone boasts a huge OLED screen.}
            \EX{The framework boasts strong security primitives.}
            \CO{\textit{boast} features/specs/performance}
            \end{ExplainCard}

            \begin{ExplainCard}{facial recognition}[n][C1]
            \EN{technology that identifies or verifies a person by analyzing facial features.}
            \SY{face ID; biometric identification}
            \VI{nhận diện khuôn mặt (công nghệ sinh trắc).}
            \EX{Facial recognition lets users unlock the device hands-free.}
            \EX{Policies regulate the use of facial-recognition systems.}
            \CO{\textit{facial recognition} system/unlock/feature}
            \end{ExplainCard}

            \begin{ExplainCard}{swipe up}[phr.v][B2]
            \EN{to move a finger upward on a touch screen to perform an action.}
            \SY{flick up; slide up}
            \VI{vuốt lên trên màn hình cảm ứng.}
            \EX{Just swipe up to unlock the phone.}
            \EX{Users swipe up to access the control center.}
            \CO{\textit{swipe up/down/left/right}; \textit{swipe up to} unlock/open}
            \end{ExplainCard}

            \begin{ExplainCard}{integrate}[v][B2]
            \EN{to combine parts so they work together as a whole.}
            \SY{incorporate; merge; unify}
            \VI{tích hợp; kết hợp.}
            \EX{Apple integrated dual-SIM support into this model.}
            \EX{Apps integrate hardware and cloud services seamlessly.}
            \CO{\textit{integrate} A \textit{into/with} B; fully \textit{integrated} system}
            \end{ExplainCard}

            \begin{ExplainCard}{stabilisation (optical image stabilisation)}[n][B2]
            \EN{the process/feature that keeps something steady; in cameras, tech that reduces blur from movement.}
            \SY{steadying; stabilization}
            \VI{sự ổn định; (ảnh) chống rung quang học.}
            \EX{Optical image stabilisation sharpens low-light shots.}
            \EX{Hardware stabilisation complements software noise reduction.}
            \CO{\textit{image/price} stabilisation; optical \textit{image stabilisation (OIS)}}
            \end{ExplainCard}
        \end{VocabExplain}

    \noindent
    \textbf{Part 3.}
    \begin{qa}{What types of things do young people in your country most want to own today? Why is this?}
    Regardless of what time are we living in, I believe every single person will more or less \textbf{aspire} to a \textbf{permanent} dwelling. Compared to our prehistoric men who led a \textbf{nomadic} life, people these days wish to have a stable life by living in their own accommodation. Moreover, a secured employment is another \textbf{impulse} for people because they need to meet their daily necessities and pay for luxurious items sometimes.
    \end{qa}

    \begin{qa}{Why do some people feel they need to own things?}
    In my opinion, a sense of ownership can be \textbf{hailed} as a protection against \textbf{ups and downs} of life. Particularly, in a \textbf{dog-eat-dog} world today, a small change can lead to \textbf{ripple} effects. For example, an economic downturn can force many companies to \textbf{downsize} and \textbf{lay off} their employees, which makes it hard for people to \textbf{scrape by}. Therefore, people with more possessions will feel more secured than others.
    \end{qa}

    \begin{qa}{Do you think that owning lots of things makes people happy? Why?}
    Well, it is not necessarily the case. As I was saying, it is safe for people to have got certain properties such as a \textbf{detached house} and a \textbf{viable} income so as to \textbf{get by} every day. However, the possession is not everything people should be obsessed with because spiritual values sometimes can be placed over anything else. For example, helping some disadvantaged people can give the helper a \textbf{sense of fulfillment} as much as having the latest smart phone.
    \end{qa}

    \begin{qa}{Do you think television and films can make people want to get new possessions? Why do they have this effect?}
    Honestly, TV programs and movies have enriched the spiritual life in many ways, but they also have \textbf{discernible} impacts on the way people do shopping. Today, film or program budgeting mainly depend on the donation of giant corporations, so there is alway a \textbf{trade-off}. \textbf{Product placement}, for example, is an effective way for the corporation to market their products to audiences through the program they watch. As a result, the number of sold products \textbf{rest on} the frequency they reach out to potential customers.
    \end{qa}

    \begin{qa}{Are there benefits to society of people wanting to get new possessions? Why do you think this is?}
    To be fair, the \textbf{upsurge} in shopping capacity can contribute to economic growth. Mass manufacturing and \textbf{outsourcing} will \textbf{prevail} in the trade market, leading to job creations for producing and customer servies, for example. But the thing is, customers should be \textbf{savvy} enough to avoid shopping tricks and \textbf{throwing their money away}.
    \end{qa}

    \begin{qa}{Do you think people will consider that having lots of possessions is a sign of success in the future? Why?}
    I would partly agree with the opinion. Traditionally, possessions can represent \textbf{social hierarchy}, and having loads of luxurious properties means owners may fall into the \textbf{upper-class}. Things have steadily changed, though. Today, having intellectual talents can gain enormous respect and a powerful voice as much as possessing luxurious items. This means the assessment of human nature is not totally \textbf{predicated} on their properties but on intangible assets such as a \textbf{sharp mind}.
    \end{qa}

        \begin{VocabExplain}[Part 3]
            \begin{ExplainCard}{aspire}[v][B2]
            \EN{to have a strong desire to achieve or obtain something.}
            \SY{yearn for; strive for}
            \VI{khao khát, hướng tới.}
            \EX{Many graduates aspire to home ownership.}
            \EX{She aspires to a leadership role in the long term.}
            \CO{aspire to + N/V-ing; aspire to be + N}
            \end{ExplainCard}

            \begin{ExplainCard}{permanent}[adj][C1]
            \EN{lasting or intended to last indefinitely.}
            \SY{enduring; lasting}
            \VI{vĩnh viễn; lâu dài.}
            \EX{They want a permanent dwelling.}
            \EX{Permanent contracts offer greater security.}
            \CO{permanent residence/job/change}
            \end{ExplainCard}

            \begin{ExplainCard}{nomadic}[adj][B2]
            \EN{moving from place to place rather than settling permanently.}
            \SY{itinerant; roving}
            \VI{du mục; nay đây mai đó.}
            \EX{Prehistoric groups lived a nomadic life.}
            \EX{Remote work enables a modern nomadic lifestyle.}
            \CO{nomadic life/tribe/lifestyle}
            \end{ExplainCard}

            \begin{ExplainCard}{impulse}[n][B2]
            \EN{(1) a sudden urge to act; (2) a driving force behind an action.}
            \SY{urge; impetus}
            \VI{(1) sự bốc đồng; (2) lực thúc đẩy.}
            \EX{She bought the bag on impulse.}
            \EX{The need for stability is an impulse to save.}
            \CO{on impulse; provide an impulse for}
            \end{ExplainCard}

            \begin{ExplainCard}{hail (as)}[v][B2]
            \EN{to praise or acclaim something as.}
            \SY{acclaim; applaud}
            \VI{tán dương, ca ngợi là.}
            \EX{The policy was hailed as a breakthrough.}
            \EX{Ownership is often hailed as a safeguard.}
            \CO{hail sb/sth as + N}
            \end{ExplainCard}

            \begin{ExplainCard}{ups and downs}[n][B2]
            \EN{the good and bad times that happen in life.}
            \SY{highs and lows; vicissitudes}
            \VI{thăng trầm.}
            \EX{Savings help families through life's ups and downs.}
            \EX{Markets experience ups and downs each cycle.}
            \CO{life's ups and downs; the ups and downs of + N}
            \end{ExplainCard}

            \begin{ExplainCard}{dog-eat-dog}[adj][B2]
            \EN{extremely competitive and ruthless.}
            \SY{cut-throat; fiercely competitive}
            \VI{cạnh tranh khốc liệt.}
            \EX{It's a dog-eat-dog housing market.}
            \EX{Startups face a dog-eat-dog landscape.}
            \CO{dog-eat-dog world/industry}
            \end{ExplainCard}

            \begin{ExplainCard}{ripple effect}[n][B2]
            \EN{a situation where one event causes a series of other events.}
            \SY{knock-on effect; chain reaction}
            \VI{hiệu ứng dây chuyền.}
            \EX{Layoffs can create a ripple effect in local retail.}
            \EX{Policy shifts had ripple effects across sectors.}
            \CO{create/trigger ripple effects; ripple effects on}
            \end{ExplainCard}

            \begin{ExplainCard}{downsize}[v][B2]
            \EN{to reduce the number of employees to cut costs.}
            \SY{scale down; cut staff}
            \VI{cắt giảm nhân sự/quy mô.}
            \EX{The firm downsized during the downturn.}
            \EX{Many companies downsize to stay solvent.}
            \CO{downsize the workforce/operation}
            \end{ExplainCard}

            \begin{ExplainCard}{lay off}[phr.v][B2]
            \EN{to dismiss employees temporarily or permanently for economic reasons.}
            \SY{make redundant; dismiss}
            \VI{sa thải (vì kinh tế).}
            \EX{Factories laid off hundreds of workers.}
            \EX{Tech firms laid off staff amid weak demand.}
            \CO{lay off staff/workers; mass layoffs}
            \end{ExplainCard}

            \begin{ExplainCard}{scrape by}[phr.v][B2]
            \EN{to manage to live with very little money.}
            \SY{get by; make ends meet}
            \VI{sống chật vật.}
            \EX{Many families scraped by on one income.}
            \EX{Students scrape by with part-time jobs.}
            \CO{scrape by on + amount}
            \end{ExplainCard}

            \begin{ExplainCard}{detached house}[n][B2]
            \EN{a stand-alone house not joined to another building.}
            \SY{single-family house}
            \VI{nhà riêng biệt.}
            \EX{They dream of a detached house in the suburbs.}
            \EX{Detached houses usually require larger plots.}
            \CO{buy/build a detached house}
            \end{ExplainCard}

            \begin{ExplainCard}{viable}[adj][C1]
            \EN{able to work successfully; financially sustainable.}
            \SY{feasible; sustainable}
            \VI{khả thi; đủ sống.}
            \EX{He needs a viable income to support kids.}
            \EX{A viable plan is essential before investing.}
            \CO{viable plan/business/income stream}
            \end{ExplainCard}

            \begin{ExplainCard}{get by}[phr.v][B2]
            \EN{to manage to live or cope with what you have.}
            \SY{manage; cope}
            \VI{xoay xở; sống tạm đủ.}
            \EX{They get by on a modest salary.}
            \EX{You can get by with a budget phone.}
            \CO{get by on/with + N}
            \end{ExplainCard}

            \begin{ExplainCard}{sense of fulfillment}[n][C1]
            \EN{deep satisfaction gained from achieving something meaningful.}
            \SY{contentment; purpose}
            \VI{cảm giác mãn nguyện.}
            \EX{Volunteering brings a sense of fulfillment.}
            \EX{Creative work gives many people fulfillment.}
            \CO{find/gain a sense of fulfillment}
            \end{ExplainCard}

            \begin{ExplainCard}{discernible}[adj][C1]
            \EN{able to be noticed or distinguished.}
            \SY{noticeable; perceptible}
            \VI{có thể nhận thấy.}
            \EX{Ads had a discernible effect on sales.}
            \EX{There's no discernible difference in quality.}
            \CO{discernible impact/trend/difference}
            \end{ExplainCard}

            \begin{ExplainCard}{trade-off}[n][B2]
            \EN{a balance between two desirable but incompatible features.}
            \SY{compromise; exchange}
            \VI{sự đánh đổi; thoả hiệp.}
            \EX{There's a trade-off between price and durability.}
            \EX{Policy design involves equity–efficiency trade-offs.}
            \CO{make/accept a trade-off; trade-off between A and B}
            \end{ExplainCard}

            \begin{ExplainCard}{product placement}[n][C1]
            \EN{promotion by featuring branded products within films/TV shows.}
            \SY{brand integration; embedded marketing}
            \VI{quảng cáo gài trong phim/chương trình.}
            \EX{Product placement nudges viewers to buy gadgets.}
            \EX{Regulators monitor covert product placement.}
            \CO{use/subtle/overt product placement}
            \end{ExplainCard}

            \begin{ExplainCard}{rest on}[phr.v][B2]
            \EN{to depend on something.}
            \SY{depend on; hinge on}
            \VI{phụ thuộc vào; dựa trên.}
            \EX{Success rests on consistent quality.}
            \EX{Their forecast rests on stable demand.}
            \CO{rest on assumptions/evidence/frequency}
            \end{ExplainCard}

            \begin{ExplainCard}{upsurge}[n][B2]
            \EN{a sudden notable increase.}
            \SY{surge; spike}
            \VI{sự tăng vọt.}
            \EX{An upsurge in demand creates jobs.}
            \EX{Hospitals reported an upsurge in visits.}
            \CO{an upsurge in sales/interest}
            \end{ExplainCard}

            \begin{ExplainCard}{outsourcing}[n][B2]
            \EN{using outside companies to perform tasks previously done in-house.}
            \SY{subcontracting; offshoring}
            \VI{thuê ngoài.}
            \EX{They moved support to an outsourcing partner.}
            \EX{Outsourcing can cut costs but risks quality lapses.}
            \CO{IT/business-process outsourcing; outsource to + vendor}
            \end{ExplainCard}

            \begin{ExplainCard}{prevail}[v][B2]
            \EN{to be widespread or dominant; to win out.}
            \SY{dominate; triumph}
            \VI{thắng thế; chiếm ưu thế.}
            \EX{Digital payments prevail in big cities.}
            \EX{Common sense finally prevailed in talks.}
            \CO{prevail in/among; prevail over}
            \end{ExplainCard}

            \begin{ExplainCard}{savvy}[adj][B2]
            \EN{having practical knowledge and good judgment.}
            \SY{shrewd; astute}
            \VI{tinh tường; sành sỏi.}
            \EX{Savvy shoppers spot fake discounts.}
            \EX{Be financially savvy about big purchases.}
            \CO{tech-/media-/financial-savvy; a savvy move}
            \end{ExplainCard}

            \begin{ExplainCard}{throw money away}[phrase][B2]
            \EN{to waste money on things that are not worthwhile.}
            \SY{squander; fritter away}
            \VI{ném tiền qua cửa sổ; phung phí.}
            \EX{Don't throw your money away on fads.}
            \EX{Without research, firms can throw money away on ads.}
            \CO{avoid/stop throwing money away on + N}
            \end{ExplainCard}

            \begin{ExplainCard}{social hierarchy}[n][B2]
            \EN{a system in which members of society are ranked by status.}
            \SY{social strata; pecking order}
            \VI{trật tự thứ bậc xã hội.}
            \EX{Possessions often signal social hierarchy.}
            \EX{Education can reshape social hierarchies.}
            \CO{climb/reflect social hierarchy}
            \end{ExplainCard}

            \begin{ExplainCard}{upper-class}[n/adj][B2]
            \EN{(n) people of the highest social rank; (adj) belonging to that group.}
            \SY{elite; high-class}
            \VI{(tầng lớp) thượng lưu.}
            \EX{Owning estates marked the upper class.}
            \EX{Upper-class tastes influenced fashion trends.}
            \CO{the upper class; upper-class lifestyle/family}
            \end{ExplainCard}

            \begin{ExplainCard}{be predicated on}[phrase][B2]
            \EN{to be based on or founded upon something.}
            \SY{be based on; hinge on}
            \VI{dựa trên; đặt nền tảng ở.}
            \EX{Success shouldn't be predicated on wealth alone.}
            \EX{The model is predicated on several assumptions.}
            \CO{be predicated on + N}
            \end{ExplainCard}

            \begin{ExplainCard}{a sharp mind}[n][B2]
            \EN{quick and clear intelligence.}
            \SY{keen intellect; quick wit}
            \VI{trí óc sắc sảo.}
            \EX{A sharp mind can be more valuable than possessions.}
            \EX{He solved the puzzle with a sharp mind.}
            \CO{have/keep a sharp mind}
            \end{ExplainCard}
        \end{VocabExplain}

    \begin{VocabHighlights}
        \VH{within walking distance}{(idiom) not very far}{(thành ngữ) khá gần}
        \VH{on one's doorstep}{(idiom) very close to where you live}{(thành ngữ) rất gần nơi mình sống}
        \VH{to strike it lucky}{(idiom) to suddenly have a lot of luck}{(thành ngữ) bất chợt gặp may}
        \VH{to set something / somebody apart}{(phr.v) to make someone or something different and special}{(cụm động từ) khiến ai/cái gì khác biệt}
        \VH{exceptional}{(adj) unusually good}{(tính từ) cực tốt}
        \VH{top-notch}{(adj) excellent; of the highest quality}{(tính từ) ở đẳng cấp cao nhất}
        \VH{to have no alternative}{(phrase) to have no other choices}{(cụm từ) không có lựa chọn khác}
        \VH{radically}{(adv) in a way that concerns the most basic and important parts of something; in a thorough and complete way}{(trạng từ) toàn diện, hoàn toàn}
        \VH{to come up with}{(phr.v) to find or produce an answer, a sum of money}{(cụm động từ) nảy ra}
        \VH{to come into vogue}{(idiom) becomes very popular and fashionable}{(thành ngữ) trở nên thời thượng, phổ biến}
        \VH{to aspire}{(v) to have a strong desire to achieve or to become something}{(động từ) khao khát}
        \VH{permanent}{(adj) lasting or intended to last or remain unchanged indefinitely}{(tính từ) lâu dài, vĩnh viễn}
        \VH{nomadic}{(adj) belonging to a community that moves with its animals from place to place}{(tính từ) du mục}
        \VH{impulse}{(n) a sudden strong wish or need to do something, without stopping to think about the results}{(danh từ) động lực}
        \VH{to hail}{(v) to describe somebody/something as being very good or special, especially in newspapers, etc}{(động từ) ngợi ca}
        \VH{ups and downs}{(phrase) a succession of both good and bad experiences}{(cụm từ) thăng trầm}
        \VH{dog-eat-dog}{(phrase) used to refer to a situation of fierce competition in which people are willing to harm each other in order to succeed}{(cụm từ) cạnh tranh khốc liệt}
        \VH{ripple}{(adj) a small wave on the surface of a liquid, especially water in a lake, etc}{(tính từ) lan tỏa}
        \VH{downturn}{(n) a fall in the amount of business that is done; a time when the economy becomes weaker}{(danh từ) suy thoái}
        \VH{to downsize}{(v) to reduce the number of people who work in a company, business, etc. in order to reduce costs}{(động từ) cắt giảm}
        \VH{to lay off}{(phr.v) to stop employing somebody}{(cụm động từ) cho thôi việc}
        \VH{to scrape by}{(phr.v) to live with barely enough money}{(cụm động từ) sống tằn tiện}
        \VH{a detached house}{(n) a stand-alone house}{(danh từ) nhà riêng}
        \VH{a sense of fulfillment}{(phrase) a feeling of happiness and satisfaction}{(cụm từ) cảm giác hạnh phúc, thỏa mãn}
        \VH{discernible}{(adj) that can be recognized or understood}{(tính từ) rõ ràng}
        \VH{donation}{(n) something that is given to a person or an organization such as a charity, in order to help them; the act of giving something in this way}{(danh từ) quyên góp}
        \VH{product placement}{(phrase) a practice in which manufacturers of goods or providers of a service gain exposure for their products by paying for them to be featured in movies and television programs}{(cụm từ) định vị sản phẩm}
        \VH{to rest on}{(v) to depend or rely on (someone or something)}{(động từ) phụ thuộc vào}
        \VH{upsurge}{(n) a sudden large increase in something}{(danh từ) tăng đột ngột}
        \VH{outsourcing}{(n) the process of arranging for somebody outside a company to do work or provide goods for that company}{(danh từ) gia công}
        \VH{savvy}{(n) practical knowledge or understanding of something}{(danh từ) tinh tường, hiểu biết}
        \VH{to throw somebody's money away}{(idiom) to waste money without regard of the consequences}{(thành ngữ) ném tiền qua cửa sổ}
        \VH{social hierarchy}{(phrase) a fundamental aspect of social organization that is established by fighting or display behavior and results in a ranking of the animals in a group}{(cụm từ) tầng lớp xã hội}
        \VH{the upper-class}{(phrase) the social group that has the highest status in society, especially the aristocracy}{(cụm từ) tầng lớp thượng lưu}
        \VH{a sharp mind}{(phrase) having or showing an ability to think and react very quickly}{(danh từ) đầu óc nhanh nhạy}
    \end{VocabHighlights}
    \end{test}
\end{glossarymc}