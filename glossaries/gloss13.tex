\begin{glossarymc}[Cambridge 15]
    \begin{test}{TEST 1}
    \noindent
    \textbf{Part 1. E-mail}
    \begin{qa}{What kinds of emails do you receive about your work or studies?}
    Since I \textbf{joined the workforce} roughly a decade ago, I have received e-mails on a daily basis. My boss frequently delivers business mails about the \textbf{tentative} schedules to me and other colleagues.
    \end{qa}

    \begin{qa}{Do you prefer to email, phone or text your friends? [Why?]}
    Although these methods of communication are generally free thanks to the ubiquity of the Internet, email still \textbf{comes tertiary} to phone and texting. That is because if I make a phone call or send a text message, \textbf{the chances are} that the recipients will reply instantly whereas email users generally spend more time preparing their answers.
    \end{qa}

    \begin{qa}{Do you reply to emails and messages as soon as you receive them? [Why/Why not?]}
    Well, it is \textbf{contingent on} the current situation. If I am free, I will send back a reply \textbf{in a jiffy}. If I am occupied with working or looking after my children, I will contact the senders later \textbf{asap}.
    \end{qa}

    \begin{qa}{Are you happy to receive emails that are advertising things? [Why/Why not?]}
    It depends. If I receive emails that relate to the products or manufacturers I subscribed to earlier, of course I will be eager to view the contents. Junk mails offering services I have no interest in surely \textbf{wind me up}.
    \end{qa}

        \begin{VocabExplain}[Part 1]
            \begin{ExplainCard}{join the workforce}[phr][B2]
            \EN{To start working in a job or career, especially after finishing school or university.}
            \SY{enter employment; begin working life}
            \VI{tham gia lực lượng lao động.}
            \EX{She joined the workforce immediately after graduation.}
            \EX{Many young people delay joining the workforce to pursue further study.}
            \CO{join the workforce; rejoin the workforce}
            \end{ExplainCard}

            \begin{ExplainCard}{tentative}[adj][C1]
            \EN{Not certain or fixed; done as a trial or experiment.}
            \SY{provisional; temporary; uncertain}
            \VI{tạm thời, chưa chắc chắn.}
            \EX{We made tentative plans for the weekend.}
            \EX{The meeting is scheduled for a tentative date in March.}
            \CO{tentative plan; tentative schedule; tentative agreement}
            \end{ExplainCard}

            \begin{ExplainCard}{comes tertiary}[phr][C1]
            \EN{To be ranked third in importance or preference.}
            \SY{comes third; less important}
            \VI{đứng thứ ba về mức độ quan trọng.}
            \EX{For me, watching TV comes tertiary to reading and exercising.}
            \EX{In this company, salary often comes tertiary to job satisfaction and career growth.}
            \CO{comes tertiary to sth}
            \end{ExplainCard}

            \begin{ExplainCard}{the chances are}[phrase][B2]
            \EN{Used to say that something is likely to happen.}
            \SY{probably; likely; odds are}
            \VI{nhiều khả năng là.}
            \EX{The chances are that he will arrive late.}
            \EX{The chances are you will succeed if you keep trying.}
            \CO{the chances are (that)...}
            \end{ExplainCard}

            \begin{ExplainCard}{contingent on}[phr][C1]
            \EN{Depending on something else in order to happen.}
            \SY{dependent on; conditional on}
            \VI{phụ thuộc vào, tùy thuộc vào.}
            \EX{Our trip is contingent on the weather.}
            \EX{Payment is contingent on successful completion of the project.}
            \CO{contingent on circumstances; contingent on approval}
            \end{ExplainCard}

            \begin{ExplainCard}{in a jiffy}[idiom][C1]
            \EN{Very quickly; in a very short time.}
            \SY{in no time; very soon}
            \VI{rất nhanh, trong chốc lát.}
            \EX{I’ll be back in a jiffy.}
            \EX{He finished the task in a jiffy.}
            \CO{back in a jiffy; done in a jiffy}
            \end{ExplainCard}

            \begin{ExplainCard}{asap}[abbrev][B2]
            \EN{As soon as possible.}
            \SY{without delay; promptly}
            \VI{càng sớm càng tốt.}
            \EX{Please reply to my message asap.}
            \EX{The report must be submitted asap.}
            \CO{respond asap; deliver asap}
            \end{ExplainCard}

            \begin{ExplainCard}{wind sb up}[phr.v][C1]
            \EN{To annoy or irritate someone.}
            \SY{annoy; provoke; irritate}
            \VI{làm ai bực mình.}
            \EX{It really winds me up when people are late.}
            \EX{Junk emails wind me up every morning.}
            \CO{wind sb up completely; things that wind sb up}
            \end{ExplainCard}
        \end{VocabExplain}

    \noindent
    \textbf{Part 2.}
    \begin{qa}{Describe a hotel you know. You should say:}
    \begin{itemize}
        \item Where this hotel is
        \item What this hotel looks like
        \item What facilities this hotel has
        \item Explain whether you think this is a nice hotel to stay in
    \end{itemize}

    I have visited many tourist spots and stayed in a number of hotels, both of professional and personal needs but only one hotel has left a long-lasting impression on me because of its \textbf{premium} services. The name of the hotel is Western Hotel, which is located \textbf{in the heart of} Hoi An Old Quarter.  

    It is a 10-storey building with more than 30 \textbf{well-lit} rooms. It draws on today’s \textbf{growing appetite for} outdoor living, so a lot of windows and ventilators are \textbf{delicately designed}. The hotel honors the preservation of ancestral \textbf{traditional} construction techniques by using clay bricks that are made manually and dried in the sun. Although it is a \textbf{family-owned} hotel, it has a reputation for its services. It is one of the most luxurious hotels with all modern facilities including a swimming pool, a sauna, and a gym. Most of the modern amenities are offered at the hotel for its clients. The services are \textbf{up to the mark} while the accommodation systems appear to be one of the greatest. \textbf{Hands down}, it is always crowded with tourists.  

    The reasons why I believe that it is an ideal place is that I \textbf{get the royal treatment}. Before I reached the hotel, the reception had informed me of the status of the room in advance and asked about my expected arrival time to cater for. When I set foot into the hotel, two people welcomed me with a smile and offered me a hand and I bet \textbf{they were at my beck and call}. The rooms are connected to Internet \textbf{free of charge} and have a small balcony for reading. I should not forget to mention that local fruits are served \textbf{on the house}, so it is a paradise for me to stay in. What a lovely hotel!
    \end{qa}

        \begin{VocabExplain}[Part 2]
            \begin{ExplainCard}{premium}[adj][C1]
            \EN{Of very high quality; better than usual or average.}
            \SY{exclusive; superior; deluxe}
            \VI{cao cấp, hảo hạng.}
            \EX{This restaurant offers premium services for its customers.}
            \EX{Premium products are often associated with high prices.}
            \CO{premium quality; premium service; premium brand}
            \end{ExplainCard}

            \begin{ExplainCard}{in the heart of}[phrase][B2]
            \EN{Located in the central or most important part of a place.}
            \SY{in the center; in the middle of}
            \VI{ngay trung tâm, ở giữa.}
            \EX{The hotel is located in the heart of the city.}
            \EX{Shops in the heart of town are usually more expensive.}
            \CO{in the heart of the city/town/district}
            \end{ExplainCard}

            \begin{ExplainCard}{well-lit}[adj][B2]
            \EN{Brightly illuminated; having a lot of natural or artificial light.}
            \SY{bright; illuminated}
            \VI{được chiếu sáng đầy đủ.}
            \EX{The office is well-lit and pleasant to work in.}
            \EX{A well-lit room makes studying easier.}
            \CO{well-lit street; well-lit building}
            \end{ExplainCard}

            \begin{ExplainCard}{growing appetite for}[phrase][C1]
            \EN{An increasing desire or demand for something.}
            \SY{rising demand; increasing desire}
            \VI{nhu cầu ngày càng tăng về cái gì.}
            \EX{There is a growing appetite for organic products.}
            \EX{The growing appetite for technology has transformed education.}
            \CO{growing appetite for sth}
            \end{ExplainCard}

            \begin{ExplainCard}{up to the mark}[idiom][C1]
            \EN{Of a good enough standard; meeting expectations.}
            \SY{adequate; satisfactory}
            \VI{đạt tiêu chuẩn, vừa đủ.}
            \EX{The hotel services were really up to the mark.}
            \EX{His performance was not up to the mark.}
            \CO{up to the mark standard/service}
            \end{ExplainCard}

            \begin{ExplainCard}{hands down}[idiom][C2]
            \EN{Without any doubt; unquestionably.}
            \SY{undoubtedly; certainly}
            \VI{không còn nghi ngờ gì nữa.}
            \EX{This is hands down the best Italian restaurant in town.}
            \EX{She is hands down the most talented singer of her generation.}
            \CO{hands down winner; hands down best/choice}
            \end{ExplainCard}

            \begin{ExplainCard}{get the royal treatment}[idiom][C1]
            \EN{To be treated in a very special and luxurious way.}
            \SY{be pampered; be indulged}
            \VI{được tiếp đãi như thượng khách.}
            \EX{Guests at the resort always get the royal treatment.}
            \EX{The celebrity got the royal treatment at the hotel.}
            \CO{receive/get the royal treatment}
            \end{ExplainCard}

            \begin{ExplainCard}{at one's beck and call}[idiom][C2]
            \EN{Always ready to obey someone’s orders or requests.}
            \SY{ready to serve; obedient}
            \VI{sẵn sàng phục vụ theo yêu cầu.}
            \EX{The servants were at the king’s beck and call.}
            \EX{She has assistants at her beck and call 24/7.}
            \CO{be at sb’s beck and call}
            \end{ExplainCard}

            \begin{ExplainCard}{on the house}[idiom][B2]
            \EN{Given free by the owner of a business.}
            \SY{free; complimentary}
            \VI{miễn phí, chủ quán mời.}
            \EX{The first drink is on the house.}
            \EX{We got desserts on the house because it was our anniversary.}
            \CO{food/drink on the house}
            \end{ExplainCard}
        \end{VocabExplain}

    \noindent
    \textbf{Part 3.}
    \begin{qa}{What things are important when people are choosing a hotel?}
    There are a few \textbf{determinants} of customers’ choices regarding their hotel selection. Firstly, most of the time, location should be one of the most \textbf{decisive} factors when it comes to choosing a hotel. Staying in a distant hotel will lead to increased \textbf{commuting} times, which leaves travelers little time for discovering the \textbf{highlights} of the city. Secondly, hotel prices should not be overlooked since they \textbf{affect} the \textbf{amenities} such as the living room or bathrooms that tourists enjoy. Thirdly, \textbf{testimonials} can also be used as a relatively credible source of references. Surfing websites specialized in tourism such as Tripadvisor, etc. could help travelers to distinguish between the good and bad hotels.
    \end{qa}

    \begin{qa}{Why do some people not like staying in hotels?}
    First of all, strange ambience is the reason why several tourists \textbf{shy away from} hotels, I believe. In fact, not all hotels can satisfy the basic needs of their customers, which obviously renders the customers not to \textbf{feel at home}. Besides, as I said earlier, prices play an important role in choosing a hotel. Although centrally located hotels can provide tourists great services such as a gym center or supermarkets \textbf{within sight}, they usually appeal to those who are \textbf{swimming in money}. For those who are backpackers travelling with a tight budget, cheaper options like booking residents’ spare rooms on AirBnB or couchsurfing.com are generally preferred.
    \end{qa}

    \begin{qa}{Do you think staying in a luxury hotel is a waste of money?}
    Personally, I have a \textbf{mixed feeling} of the idea. On the one hand, I acknowledge that hiring a first-class hotel room can be deemed as a waste of resources. This is simply because travelling and exploring tourist attractions are the main purpose of the journey, and most tourists only spend few hours at the hotel for relaxing, which cannot justify the \textbf{criminally expensive} price they have to pay. That being said, a luxury hotel can be suitable for business people who have a tight schedule and just want to have a sound sleep or the \textbf{well-to-do} who wish to show off their lives of luxury.
    \end{qa}

    \begin{qa}{Do you think hotel work is a good career for life?}
    Undeniably, hotel work is an \textbf{exhausting job} since it requires movements and stamina all the time. In fact, those who work in the \textbf{hospitality industry} should have a sharp mind and a set of skills like communication and problem-solving skills to address daily problems. However, I believe pursuing a career in hotels is a good choice. Since there are a variety of job roles ranging from sales to cooking, they can suit people with different \textbf{knacks}. Moreover, with tourism booming, there is a huge demand for hotel staff, across all job roles. This means that a career in hotels can bring \textbf{job security} to workers.
    \end{qa}

    \begin{qa}{How does working in a big hotel compare with working in a small hotel?}
    Well, large hotels definitely require more staff to function, so they will generate more \textbf{employment opportunities} than smaller ones. Another plus of big hotels is that they can offer a generous \textbf{allowance} and salary package to staff. On the other hand, with a smaller workforce, people can form strong relationships with other staff members. More importantly, they are more likely to come into regular contact with senior and managerial roles, meaning problems can be solved quickly. Small \textbf{establishments} also often accept employees who are still \textbf{green}, which adds \textbf{credibility} to their CV when they move up.
    \end{qa}

    \begin{qa}{What skills are needed to be a successful hotel manager?}
    Without a doubt, leadership and communication skills are key \textbf{attributes} that general managers should possess to \textbf{call the shots}. Firstly, to handle the overwhelming workload \textbf{pertaining to} hospitality industry, the hotel managers should be able to \textbf{empower} and influence others. They are expected to \textbf{delegate} tasks to assistants and orient the big team towards shared goals. On top of that, the manager should also be a proficient communicator. In that way, they can enhance interpersonal communication and settle conflicts with more ease.
    \end{qa}

        \begin{VocabExplain}[Part 3]
            \begin{ExplainCard}{determinant}[n][C1]
            \EN{a factor that strongly influences or decides the outcome of something.}
            \VI{\textit{yếu tố quyết định}.}
            \SY{deciding factor; element; influence}
            \EX{Location is a key determinant in hotel choice.}
            \EX{Education is often considered a determinant of income.}
            \CO{determinant factor; key determinant of success}
            \end{ExplainCard}

            \begin{ExplainCard}{well-to-do}[adj][C1]
            \EN{wealthy; having a lot of money.}
            \VI{\textit{giàu có, khá giả}.}
            \SY{affluent; prosperous; rich}
            \EX{The well-to-do often stay in five-star hotels.}
            \EX{He was born into a well-to-do family of landowners.}
            \CO{well-to-do family; well-to-do neighborhood}
            \end{ExplainCard}

            \begin{ExplainCard}{hospitality industry}[n][B2]
            \EN{the business sector that provides services to people, such as hotels, restaurants, and travel.}
            \VI{\textit{ngành dịch vụ khách sạn – nhà hàng – du lịch}.}
            \SY{service industry; tourism sector}
            \EX{The hospitality industry has suffered during the pandemic.}
            \EX{Good communication skills are crucial in the hospitality industry.}
            \CO{hospitality industry growth; work in the hospitality industry}
            \end{ExplainCard}

            \begin{ExplainCard}{call the shots}[idiom][C1]
            \EN{to be the person who controls or makes important decisions in a situation.}
            \VI{\textit{ra lệnh, đưa ra quyết định}.}
            \SY{be in charge; control; command}
            \EX{It’s always the hotel manager who calls the shots.}
            \EX{The board of directors call the shots in the company.}
            \CO{be the one to call the shots; who calls the shots}
            \end{ExplainCard}

            \begin{ExplainCard}{delegate}[v][C1]
            \EN{to give part of your work, responsibilities, or authority to someone else.}
            \VI{\textit{giao phó, ủy quyền}.}
            \SY{assign; entrust; authorize}
            \EX{Managers must learn to delegate tasks effectively.}
            \EX{She delegated responsibility to her assistant.}
            \CO{delegate authority; delegate responsibility; delegate tasks}
            \end{ExplainCard}
        \end{VocabExplain}

    \begin{VocabHighlights}
        \VH{join the workforce}{(phrase) to start working}{(cụm từ) bắt đầu đi làm}
        \VH{tentative}{(adj) (of a plan or idea) not certain or agreed, or (of a suggestion or action) said or done in a careful but uncertain way because you do not know if you are right}{(tính từ) dự kiến}
        \VH{come tertiary to}{(phrase) to be ranked third}{(cụm từ) xếp thứ 3 sau}
        \VH{the chances are}{(idiom) it is likely}{(thành ngữ) rất có thể là}
        \VH{be contingent on}{(adj) depending on something else in the future in order to happen}{(tính từ) tùy thuộc vào}
        \VH{in a jiffy}{(idiom) in a very short time}{(thành ngữ) trong 1 khoảng thời gian rất ngắn}
        \VH{asap}{(abbreviation) as soon as possible}{(viết tắt) nhanh nhất có thể}
        \VH{wind somebody up}{(phr.v) to annoy or upset someone}{(cụm động từ) gây khó chịu cho ai}
        \VH{premium}{(adj) used to refer to something that is of higher than usual quality}{(tính từ) cao cấp}
        \VH{in the heart}{(idiom) in the center of}{(thành ngữ) trung tâm}
        \VH{well-lit}{(adj) bright}{(tính từ) nhiều ánh sáng}
        \VH{growing appetite}{(phrase) the feeling that you want to eat food}{(cụm từ) thèm ăn}
        \VH{ventilators}{(n) an opening or a device that allows fresh air to come into a closed space}{(danh từ) ô thông gió}
        \VH{delicately designed}{(adj) designed with sophistication}{(tính từ) thiết kế tinh vi}
        \VH{up to the mark}{(idiom) to be good enough}{(thành ngữ) đủ tốt}
        \VH{Hands down}{(idiom) definitely}{(thành ngữ) rõ ràng}
        \VH{get the royal treatment}{(idiom) to receive extravagant treatment or elaborate attention and care}{(thành ngữ) được tiếp đãi chu đáo, trịnh trọng}
        \VH{they were at my beck and call}{(idiom) ready to do something for someone any time you are asked}{(thành ngữ) sẵn sàng có mặt để giúp đỡ}
        \VH{free of charge}{(phrase) without having to pay}{(cụm từ) miễn phí}
        \VH{on the house}{(phrase) free (at the restaurant)}{(cụm từ) miễn phí (ở nhà hàng)}
        \VH{determinant}{(n) a factor that decides whether or how something happens}{(danh từ) nhân tố quyết định}
        \VH{decisive}{(adj) able to make decisions quickly and confidently, or showing this quality}{(tính từ) mang tính chất quyết định}
        \VH{highlight}{(n) the best or most exciting, entertaining, or interesting part of something}{(danh từ) những điều nổi bật, đáng chú ý nhất}
        \VH{amenity}{(n) something, such as a swimming pool or shopping centre, that is intended to make life more pleasant or comfortable for the people in a town, hotel, or other place}{(danh từ) tổ hợp khu tiện ích}
        \VH{testimonial}{(n) a statement about the character or qualities of someone or something}{(danh từ) lời nhận xét, đánh giá chất lượng}
        \VH{shy away from something}{(phrase) to avoid something that you dislike, fear, or do not feel confident about}{(cụm từ) tránh xa khỏi cái gì}
        \VH{feel at home}{(phrase) to feel comfortable and relaxed}{(cụm từ) cảm giác thoải mái như ở nhà}
        \VH{within sight}{(phrase) in any place that you can see from where you are}{(cụm từ) trong tầm mắt}
        \VH{to be swimming in money}{(phrase) to have too much of something}{(cụm từ) bơi trong tiền, có rất nhiều tiền}
        \VH{to have a mixed feeling}{(phrase) to like or approve of some aspects of a situation and not like or approve of other aspects}{(cụm từ) chỉ đồng tình một số mặt của vấn đề}
        \VH{criminally expensive}{(adj) too expensive}{(tính từ) quá đắt đỏ}
        \VH{well-to-do}{(n) rich people}{(danh từ) người giàu}
        \VH{hospitality industry}{(n) a broad category of fields within the service industry that includes lodging, food and drink service, event planning, theme parks, and transportation. It includes hotels, restaurants and bars}{(danh từ) ngành kinh doanh khách sạn, dịch vụ}
        \VH{knack}{(n) a skill or an ability to do something easily and well}{(danh từ) tài năng, điểm mạnh}
        \VH{job security}{(n) the state of having a job that is secure and from which one is unlikely to be dismissed}{(danh từ) sự an toàn trong công việc, không lo bị mất việc}
        \VH{allowance}{(n) money that you are given regularly, especially to pay for a particular thing}{(danh từ) trợ cấp, phụ cấp}
        \VH{establishment}{(n) a business or other organization, or the place where an organization operates}{(danh từ) tổ chức kinh doanh}
        \VH{green}{(adj) not experienced or trained}{(tính từ) non trẻ, không có kinh nghiệm}
        \VH{credibility}{(n) the fact that someone can be believed or trusted}{(danh từ) sự uy tín}
        \VH{attribute}{(n) a quality or characteristic that someone or something has}{(danh từ) phẩm chất, tính cách}
        \VH{call the shots}{(phrase) to be in the position of being able to make the decisions that will influence a situation}{(cụm từ) chỉ huy}
        \VH{empower}{(v) to give someone official authority or the freedom to do something}{(động từ) trao quyền cho cấp dưới tự quyết định}
        \VH{delegate}{(v) to give a particular job, duty, right, etc. to someone else so that they do it for you}{(động từ) phân công nhiệm vụ, công việc}
    \end{VocabHighlights}
    \end{test}

    \begin{test}{TEST 2}
    \noindent
    \textbf{Part 1. Language}
    \begin{qa}{How many languages can you speak? [Why/Why not?]}
    I can manage to speak 3 languages. I have a good command of English, my second language that I have been learning and teaching for several years, and Vietnamese, my mother tongue, of course. I \textbf{made a try at} learning Japanese \textbf{a long time ago} but now, my Japanese seems \textbf{rusty}. At the moment I only have a \textbf{smattering of} Japanese.
    \end{qa}

    \begin{qa}{How useful will English be in your future? [Why/Why not?]}
    I work as an English teacher so English will always be \textbf{indispensable in the foreseeable future}. Teaching English enables me to earn a living to support my family. Without English, I’d not be who I am today, I mean, having a loving family and a \textbf{rewarding} job, and I’d have had to choose a different career path in the past.
    \end{qa}

    \begin{qa}{What do you remember about learning languages at school? [Why/Why not?]}
    To the best of my recollection, the act of learning and teaching languages at school mainly revolved around the \textbf{grammar-translation method}. In particular, in the past, teachers were in charge of translating every sentence into Vietnamese and explaining grammar rules to students. However, this method has been done away with and the \textbf{communicative language teaching method}, which focuses on teaching students in a communicative way, is \textbf{gaining ground}. That is what I have been doing since I became a teacher a long time ago.
    \end{qa}

    \begin{qa}{What do you think would be the hardest language for you to learn? [Why/Why not?]}
    With the exception of Vietnamese, I have not \textbf{had a shot at} learning other languages besides English and Japanese. I guess, learning Russian will be a \textbf{tough row to hoe} as this language uses a different form of alphabets. Moreover, reference \textbf{materials to boost} learners’ \textbf{proficiency} in Russian are not easily accessible these days.
    \end{qa}

        \begin{VocabExplain}[Part 1]
            \begin{ExplainCard}{rusty}[adj][C1]
            \EN{(of knowledge/skill) not as good as it used to be because you have not practiced or used it for a long time.}
            \VI{\textit{mai một, kém đi} (do lâu không sử dụng).}
            \SY{out of practice; deteriorated}
            \EX{My French is a bit rusty, but I can still hold a conversation.}
            \EX{Her piano skills became rusty after years without practice.}
            \CO{rusty language; rusty skill}
            \end{ExplainCard}

            \begin{ExplainCard}{smattering}[n][C1]
            \EN{a very small amount of knowledge of a language or subject.}
            \VI{\textit{chút ít kiến thức, hiểu biết sơ sài}.}
            \SY{bit; fragment; rudiment}
            \EX{He only has a smattering of Italian.}
            \EX{The student displayed just a smattering of knowledge in physics.}
            \CO{a smattering of knowledge; smattering of words}
            \end{ExplainCard}

            \begin{ExplainCard}{indispensable}[adj][C1]
            \EN{so important or necessary that it is impossible to do without.}
            \VI{\textit{không thể thiếu, thiết yếu}.}
            \SY{essential; crucial; vital}
            \EX{Good communication skills are indispensable in teaching.}
            \EX{Water is indispensable for all forms of life.}
            \CO{indispensable tool; indispensable role; indispensable for success}
            \end{ExplainCard}

            \begin{ExplainCard}{rewarding}[adj][B2]
            \EN{giving satisfaction, benefit, or pleasure; worthwhile.}
            \VI{\textit{bổ ích, đáng làm, thỏa mãn}.}
            \SY{satisfying; fulfilling; beneficial}
            \EX{Teaching can be a very rewarding career.}
            \EX{Volunteering is a rewarding experience.}
            \CO{rewarding career; rewarding experience}
            \end{ExplainCard}

            \begin{ExplainCard}{gaining ground}[idiom][C1]
            \EN{becoming more successful, popular, or accepted.}
            \VI{\textit{đang ngày càng phổ biến, giành được chỗ đứng}.}
            \SY{advancing; spreading; thriving}
            \EX{Online learning is rapidly gaining ground.}
            \EX{The new teaching method is gaining ground among schools.}
            \CO{gaining ground in popularity; gaining ground fast}
            \end{ExplainCard}

            \begin{ExplainCard}{tough row to hoe}[idiom][C2]
            \EN{a very difficult task or situation to deal with.}
            \VI{\textit{một công việc/sự việc khó khăn để đối phó}.}
            \SY{hard task; uphill battle; challenge}
            \EX{Raising three kids on your own is a tough row to hoe.}
            \EX{Learning Russian could be a tough row to hoe for many students.}
            \CO{face a tough row to hoe; prove a tough row to hoe}
            \end{ExplainCard}

            \begin{ExplainCard}{proficiency}[n][C1]
            \EN{a high degree of skill, expertise, or ability in a subject or activity.}
            \VI{\textit{trình độ thành thạo, sự thông thạo}.}
            \SY{skill; expertise; competence}
            \EX{She achieved proficiency in both English and French.}
            \EX{Computer proficiency is a requirement for the job.}
            \CO{language proficiency; gain proficiency; test of proficiency}
            \end{ExplainCard}
        \end{VocabExplain}

    \noindent
    \textbf{Part 2.}
    \begin{qa}{Describe a website that you bought something from. You should say: 
    \begin{itemize}
        \item What the website is
        \item What you bought from this website
        \item How satisfied you were with what you bought
        \item Explain why you liked or disliked about using this website
    \end{itemize}}

    \textbf{In this day and age}, e-commerce has developed to a point that shopping in a \textbf{brick-and-mortar shop} has been \textbf{marginalized}. One website I frequently visit to shop for items is Shopee, which \textbf{dominates} Vietnam’s e-commerce market. It is a social-first, mobile-centric marketplace where users can browse, shop and sell on the go.  

    When it comes to my completed orders, I tend to shop for mobile accessories, stationery and cosmetics because they are \textbf{a steal}. For example, mobile phone cover cases for my mobile phone, Note20 Ultra, are available at \textbf{popular prices}, ranging from 50,000 VND, a tag price that almost no physical shops can beat.  

    All in all, I am fairly \textbf{content with} this website. It has almost every product an ordinary user like me can ask for at such an \textbf{unbeatable} price. The only drawback, I guess, is sometimes related to the sellers’ \textbf{professionalism}. In particular, some cancelled my orders \textbf{at short notice} due to their goods being out of stock.  

    There are a bunch of reasons why I love this website. Firstly, its comparative advantage lies in that it offers \textbf{genuine reviews} from verified customers, which I can use as reference before I decide to \textbf{make a purchase}. Secondly, the \textbf{interface} of this website is \textbf{user-friendly}, so it is accessible to people of all ages. Even if you are not \textbf{computer literate}, you can still \textbf{navigate} the website \textbf{with ease}.
    \end{qa}

        \begin{VocabExplain}[Part 2]
            \begin{ExplainCard}{in this day and age}[phrase][B2]
            \EN{used to emphasize the modern period of time; nowadays.}
            \VI{\textit{ngày nay, trong thời đại hiện nay}.}
            \SY{nowadays; these days}
            \EX{In this day and age, everyone should have Internet access.}
            \EX{It’s shocking that illiteracy still exists in this day and age.}
            \end{ExplainCard}

            \begin{ExplainCard}{brick-and-mortar}[adj][C1]
            \EN{referring to physical buildings or shops, especially in contrast with online businesses.}
            \VI{\textit{cửa hàng vật lý, truyền thống}.}
            \SY{physical; offline}
            \EX{Many brick-and-mortar bookstores have closed due to e-books.}
            \EX{Brick-and-mortar shops are struggling against online retailers.}
            \end{ExplainCard}

            \begin{ExplainCard}{marginalize}[v][C1]
            \EN{to treat something as unimportant or less significant.}
            \VI{\textit{gạt ra ngoài lề, làm cho kém quan trọng}.}
            \SY{sideline; diminish}
            \EX{Traditional shops are being marginalized by e-commerce.}
            \EX{Artists often feel marginalized in political debates.}
            \end{ExplainCard}

            \begin{ExplainCard}{dominate}[v][B2]
            \EN{to control or have a lot of influence over something.}
            \VI{\textit{chiếm ưu thế, thống trị}.}
            \SY{control; prevail}
            \EX{The company dominates the smartphone market.}
            \EX{Shopee dominates Vietnam’s e-commerce industry.}
            \end{ExplainCard}

            \begin{ExplainCard}{a steal}[idiom][C1]
            \EN{something that is very cheap or good value for money.}
            \VI{\textit{món hời, mua rẻ}.}
            \SY{bargain; good deal}
            \EX{That watch was a steal at only \$30.}
            \EX{The apartment is a steal considering the location.}
            \end{ExplainCard}

            \begin{ExplainCard}{popular prices}[phrase][B2]
            \EN{prices that are low enough for ordinary people to afford.}
            \VI{\textit{giá cả phải chăng, bình dân}.}
            \SY{reasonable prices; affordable cost}
            \EX{The restaurant offers meals at popular prices.}
            \EX{Tickets were sold at popular prices to attract more customers.}
            \end{ExplainCard}

            \begin{ExplainCard}{content with}[adj][B2]
            \EN{satisfied with what one has or experiences.}
            \VI{\textit{hài lòng với}.}
            \SY{satisfied; pleased}
            \EX{She seems content with her life.}
            \EX{I’m content with the quality of the product.}
            \end{ExplainCard}

            \begin{ExplainCard}{unbeatable}[adj][B2]
            \EN{impossible to surpass in quality, price, or performance.}
            \VI{\textit{không thể đánh bại, tốt nhất}.}
            \SY{unsurpassed; matchless}
            \EX{They offer unbeatable service for the price.}
            \EX{This is an unbeatable opportunity for students.}
            \end{ExplainCard}

            \begin{ExplainCard}{professionalism}[n][C1]
            \EN{the skill, competence, and high standards expected of a professional.}
            \VI{\textit{tính chuyên nghiệp}.}
            \SY{expertise; competence}
            \EX{The staff showed great professionalism.}
            \EX{His professionalism won the trust of his clients.}
            \end{ExplainCard}

            \begin{ExplainCard}{at short notice}[phrase][C1]
            \EN{with little warning or time to prepare.}
            \VI{\textit{trong thời gian ngắn, gấp gáp}.}
            \SY{on the spot; suddenly}
            \EX{The meeting was canceled at short notice.}
            \EX{He was asked to speak at short notice.}
            \end{ExplainCard}

            \begin{ExplainCard}{genuine reviews}[phrase][C1]
            \EN{authentic and honest evaluations, not fake or manipulated.}
            \VI{\textit{đánh giá thật, đánh giá đáng tin cậy}.}
            \SY{authentic feedback; real opinions}
            \EX{The website displays genuine reviews from customers.}
            \EX{Genuine reviews help new buyers make decisions.}
            \end{ExplainCard}

            \begin{ExplainCard}{make a purchase}[phrase][B2]
            \EN{to buy something.}
            \VI{\textit{thực hiện việc mua hàng}.}
            \SY{buy; acquire}
            \EX{She made a purchase of two jackets online.}
            \EX{Customers can make a purchase directly from the website.}
            \end{ExplainCard}

            \begin{ExplainCard}{interface}[n][B2]
            \EN{the way a website, software, or device presents information and allows interaction.}
            \VI{\textit{giao diện}.}
            \SY{layout; user environment}
            \EX{The phone has a very user-friendly interface.}
            \EX{The interface makes the app easy to use.}
            \end{ExplainCard}

            \begin{ExplainCard}{user-friendly}[adj][B2]
            \EN{easy for people to understand or use.}
            \VI{\textit{dễ sử dụng}.}
            \SY{intuitive; simple}
            \EX{The new software is user-friendly.}
            \EX{This app is more user-friendly than its competitors.}
            \end{ExplainCard}

            \begin{ExplainCard}{computer literate}[adj][C1]
            \EN{having enough knowledge and skill to use computers effectively.}
            \VI{\textit{thành thạo máy tính}.}
            \SY{tech-savvy; digitally skilled}
            \EX{All applicants must be computer literate.}
            \EX{Even people who are not computer literate can use this app.}
            \end{ExplainCard}

            \begin{ExplainCard}{navigate}[v][B2]
            \EN{to move around and find one’s way through a system or website.}
            \VI{\textit{dò đường, điều hướng}.}
            \SY{browse; move around}
            \EX{It’s easy to navigate the website.}
            \EX{Users can navigate menus with a few clicks.}
            \end{ExplainCard}

            \begin{ExplainCard}{with ease}[phrase][B2]
            \EN{without difficulty or effort.}
            \VI{\textit{một cách dễ dàng}.}
            \SY{effortlessly; smoothly}
            \EX{She solved the problem with ease.}
            \EX{He navigated the app with ease.}
            \end{ExplainCard}
        \end{VocabExplain}

    \noindent
    \textbf{Part 3.}
    \begin{qa}{What kinds of things do people in your country often buy from online shops?}
    Since the emergence of online shopping, I believe books were the first products that are sold online. Understandably, it is not necessary for customers to physically and \textbf{literally} touch a book before they make a purchase. \textbf{Bulky} products such as luggage or electric items like household appliances can be ordered online because all of them are technically qualified and they usually come along with \textbf{warranty}. Finally, buying movie tickets online is increasingly \textbf{universal} as it helps to minimize \textbf{human involvement} and contributes to cost reduction.
    \end{qa}

    \begin{qa}{Why do you think online shopping has become so popular nowadays?}
    Generally speaking, online purchasing \textbf{affords} customers opportunities to find exactly what they need. This means there would be no crowds or endless queues, and customers can cut additional expenses such as parking \textbf{incurred} during the shopping trip. Another merit of online shopping is price comparison. Obviously, it is impossible for people to compare prices of physical shops at once. However, price recommendations are \textbf{at customers’ disposal} thorough online shopping websites, which allows them to \textbf{shun} \textbf{inflated} prices.
    \end{qa}

    \begin{qa}{What are some possible disadvantages of buying things from online shops?}
    Strictly speaking, many people are still \textbf{disinclined} to shopping online because of its potential drawbacks. For one thing, delay in delivery is unavoidable. In fact, the lack of proper \textbf{inventory} management or security \textbf{clearance problems} results in delays in shipment. Additionally, the convenience of online shopping is \textbf{at the expense} of personal touch. To elaborate, customers cannot inspect the real products and sometimes may encounter \textbf{spurious} products.
    \end{qa}

    \begin{qa}{Why do many people keep buying things which they do not need?}
    First of all, the \textbf{allure} of advertising should be blame for compulsive shopping. Actually, many people buy things \textbf{on a whim} simply because they are attracted by the persuasive and exaggerated \textbf{storytelling} of advertisements. Besides, many see shopping as \textbf{retail therapy}, a common solution to stress relief, regardless of the products they buy. Another primary cause is the \textbf{prevalence} of discounts, which further \textbf{exacerbates} the problem.
    \end{qa}

    \begin{qa}{Do you believe the benefits of a consumer society outweigh the disadvantages?}
    It depends. To some extent, \textbf{consumerism} is of paramount importance to economic growth. In other words, greater demands for goods and services will \textbf{translate into} more employment and collaborations among businesses. Once the economy is \textbf{bolstered}, it would have far-reaching effects on other developments. By contrast, such a society can cause \textbf{distress} after all because it will result in a \textbf{throw-away society} where unnecessary products have relatively short lifespan and are quickly discarded. This is one of the main \textbf{culprits} of waste disposal management and environmental pollution.
    \end{qa}

    \begin{qa}{How possible is it to avoid the culture of consumerism?}
    Of course, there are certain ways to escape from consumerism. One of feasible solutions is to reduce \textbf{media consumption}. That means people would not be \textbf{swamped} with \textbf{misleading} advertisements, a powerful motivation for their constant shopping. Besides, there are several healthy sources of relaxation instead of shopping. For example, reading or \textbf{meditation} have been scientifically proven to benefit positive feelings. Last but not least, households should plan a check-list before going shopping to avoid \textbf{impulse purchase}.
    \end{qa}

        \begin{VocabExplain}[Part 3]
            \begin{ExplainCard}{literally}[adv][B2]
            \EN{in a strict, exact sense; not figuratively.}
            \SY{exactly; strictly; in reality}
            \VI{theo nghĩa đen; đúng từng chữ.}
            \EX{He was literally shaking with excitement.}
            \EX{The term is used literally here to denote physical contact.}
            \CO{taken literally; mean literally}
            \end{ExplainCard}

            \begin{ExplainCard}{bulky}[adj][C1]
            \EN{large and difficult to carry or store.}
            \SY{oversized; cumbersome; unwieldy}
            \VI{cồng kềnh; to và khó mang vác.}
            \EX{That jacket is a bit too bulky for travel.}
            \EX{Bulky goods increase last-mile delivery costs.}
            \CO{bulky items/goods/package}
            \end{ExplainCard}

            \begin{ExplainCard}{warranty}[n][B2]
            \EN{a written promise to repair or replace a product within a stated period.}
            \SY{guarantee; coverage}
            \VI{bảo hành; cam kết sửa chữa/đổi trả.}
            \EX{This laptop comes with a two-year warranty.}
            \EX{Warranty terms influence post-purchase satisfaction.}
            \CO{under warranty; warranty period/claim}
            \end{ExplainCard}

            \begin{ExplainCard}{universal}[adj][C1]
            \EN{common to or done by all people or things in the world.}
            \SY{widespread; general; ubiquitous}
            \VI{phổ quát; phổ biến rộng rãi.}
            \EX{The film has universal appeal.}
            \EX{Universal access remains a policy objective in e-commerce.}
            \CO{universal access/standard/appeal}
            \end{ExplainCard}

            \begin{ExplainCard}{human involvement}[n][C1]
            \EN{the degree to which human actions are required in a process.}
            \SY{human participation; manual input}
            \VI{sự tham gia của con người (trong quy trình).}
            \EX{Automation cuts down human involvement.}
            \EX{Reducing human involvement can lower operational risk.}
            \CO{reduce/minimize human involvement}
            \end{ExplainCard}

            \begin{ExplainCard}{afford (to give)}[v][C1]
            \EN{to provide or supply (an opportunity or advantage).}
            \SY{provide; offer; grant}
            \VI{mang lại; tạo cơ hội/điều kiện.}
            \EX{The park affords great views.}
            \EX{Digital platforms afford consumers richer information.}
            \CO{afford opportunities/benefits/access}
            \end{ExplainCard}

            \begin{ExplainCard}{incur}[v][C1]
            \EN{to become subject to something unwelcome or unpleasant as a result of one’s actions.}
            \SY{sustain; bring upon oneself; attract (costs)}
            \VI{gánh chịu; phát sinh (chi phí, thiệt hại).}
            \EX{We incurred extra fees for late return.}
            \EX{Shoppers may incur ancillary costs such as parking.}
            \CO{incur costs/penalties/liabilities}
            \end{ExplainCard}

            \begin{ExplainCard}{at customers’ disposal}[phrase][C1]
            \EN{available for customers to use whenever they want.}
            \SY{available to; accessible to}
            \VI{sẵn để khách hàng sử dụng.}
            \EX{A hotline is at customers’ disposal 24/7.}
            \EX{Rich datasets are at users’ disposal through dashboards.}
            \CO{at sb’s disposal; resources/services at one’s disposal}
            \end{ExplainCard}

            \begin{ExplainCard}{shun}[v][C1]
            \EN{to avoid something or someone deliberately.}
            \SY{avoid; eschew; steer clear of}
            \VI{tránh xa; né tránh có chủ ý.}
            \EX{He shuns the spotlight.}
            \EX{Consumers shun inflated prices during downturns.}
            \CO{shun attention/risk/controversy}
            \end{ExplainCard}

            \begin{ExplainCard}{inflated}[adj][C1]
            \EN{unreasonably high or increased beyond the true value.}
            \SY{overpriced; excessive; bloated}
            \VI{bị thổi phồng; quá cao.}
            \EX{Those shoes are sold at an inflated price.}
            \EX{Inflated valuations can distort market signals.}
            \CO{inflated prices/claims/figures}
            \end{ExplainCard}

            \begin{ExplainCard}{disinclined}[adj][C1]
            \EN{not willing or prepared to do something.}
            \SY{reluctant; unwilling; averse}
            \VI{không muốn; ngần ngại.}
            \EX{I’m disinclined to buy now.}
            \EX{Older adults may be disinclined to adopt new apps.}
            \CO{be disinclined to do sth}
            \end{ExplainCard}

            \begin{ExplainCard}{inventory}[n][C1]
            \EN{the goods or materials a business holds for sale or use.}
            \SY{stock; holdings; merchandise}
            \VI{hàng tồn kho; tồn trữ.}
            \EX{The store keeps low inventory after holidays.}
            \EX{Inventory management directly affects delivery times.}
            \CO{inventory management/levels/turnover}
            \end{ExplainCard}

            \begin{ExplainCard}{clearance problems}[n][C1]
            \EN{issues related to gaining official permission or passing security/customs checks.}
            \SY{authorization issues; customs/security delays}
            \VI{trục trặc thủ tục thông quan/kiểm duyệt.}
            \EX{The parcel faced clearance problems at customs.}
            \EX{Security clearance problems can delay shipments significantly.}
            \CO{customs/security clearance; clearance delay}
            \end{ExplainCard}

            \begin{ExplainCard}{at the expense of}[idiom][C1]
            \EN{causing harm to or neglecting one thing to achieve another.}
            \SY{to the detriment of; sacrificing}
            \VI{đánh đổi; gây bất lợi cho.}
            \EX{He worked nonstop at the expense of his health.}
            \EX{Speed was improved at the expense of accuracy.}
            \CO{do sth at the expense of sth}
            \end{ExplainCard}

            \begin{ExplainCard}{spurious}[adj][C2]
            \EN{not genuine or valid; false, especially in appearance or claims.}
            \SY{fake; bogus; counterfeit}
            \VI{giả mạo; sai lệch.}
            \EX{They returned the spurious brand-name bag.}
            \EX{Spurious products undermine consumer trust.}
            \CO{spurious claims/correlation/products}
            \end{ExplainCard}

            \begin{ExplainCard}{allure}[n][C1]
            \EN{the attractive power or quality of something.}
            \SY{appeal; charm; attraction}
            \VI{sức hấp dẫn; sự lôi cuốn.}
            \EX{The allure of flash sales is hard to resist.}
            \EX{Brand allure shapes consumer preferences.}
            \CO{the allure of sth; irresistible allure}
            \end{ExplainCard}

            \begin{ExplainCard}{on a whim}[idiom][C1]
            \EN{suddenly and without careful thought.}
            \SY{impulsively; on impulse; spur-of-the-moment}
            \VI{ngẫu hứng; bốc đồng.}
            \EX{He booked the trip on a whim.}
            \EX{Purchases made on a whim inflate household spending.}
            \CO{buy/do sth on a whim}
            \end{ExplainCard}

            \begin{ExplainCard}{storytelling}[n][B2]
            \EN{the activity or skill of telling stories; in ads, persuasive narrative.}
            \SY{narrative; narration}
            \VI{kể chuyện; lối dẫn chuyện (trong quảng cáo).}
            \EX{Good storytelling makes ads memorable.}
            \EX{Narrative framing is central to brand storytelling.}
            \CO{brand/storytelling; persuasive storytelling}
            \end{ExplainCard}

            \begin{ExplainCard}{retail therapy}[n][C1]
            \EN{shopping done to improve one’s mood.}
            \SY{comfort shopping; mood shopping}
            \VI{mua sắm để giải toả cảm xúc.}
            \EX{She did some retail therapy after work.}
            \EX{Retail therapy is associated with short-term mood gains.}
            \CO{indulge in retail therapy}
            \end{ExplainCard}

            \begin{ExplainCard}{prevalence}[n][C1]
            \EN{the fact of being very common in a particular time or place.}
            \SY{commonness; pervasiveness; ubiquity}
            \VI{mức độ phổ biến; tần suất xuất hiện.}
            \EX{The prevalence of coupons drives traffic.}
            \EX{Researchers measured the prevalence of impulsive buying.}
            \CO{prevalence of sth; high/low prevalence}
            \end{ExplainCard}

            \begin{ExplainCard}{exacerbate}[v][C2]
            \EN{to make a problem or situation worse.}
            \SY{worsen; aggravate; intensify}
            \VI{làm trầm trọng thêm.}
            \EX{Heavy ads exacerbate overspending.}
            \EX{Supply shocks exacerbate inflationary pressures.}
            \CO{exacerbate a problem/inequality/crisis}
            \end{ExplainCard}

            \begin{ExplainCard}{consumerism}[n][C1]
            \EN{the social and economic focus on acquiring goods and services in ever-increasing amounts.}
            \SY{mass consumption; materialism}
            \VI{chủ nghĩa tiêu dùng.}
            \EX{Holiday consumerism peaks in December.}
            \EX{Consumerism can stimulate growth but strain resources.}
            \CO{culture/age of consumerism; rampant consumerism}
            \end{ExplainCard}

            \begin{ExplainCard}{translate into}[phr.v][C1]
            \EN{to result in or lead to a different state or outcome.}
            \SY{lead to; result in; convert to}
            \VI{chuyển thành; dẫn tới.}
            \EX{More traffic should translate into sales.}
            \EX{Innovation may translate into productivity gains.}
            \CO{translate into growth/benefits/outcomes}
            \end{ExplainCard}

            \begin{ExplainCard}{bolster}[v][C1]
            \EN{to support or strengthen something.}
            \SY{strengthen; reinforce; shore up}
            \VI{củng cố; tăng cường.}
            \EX{Positive reviews bolstered my decision.}
            \EX{Fiscal policy can bolster aggregate demand.}
            \CO{bolster confidence/economy/capacity}
            \end{ExplainCard}

            \begin{ExplainCard}{distress}[n][C1]
            \EN{great worry, sadness, or pain; suffering.}
            \SY{anguish; trouble; hardship}
            \VI{sự đau khổ; căng thẳng cực độ.}
            \EX{Debt can cause real distress.}
            \EX{Financial distress predicts firm failure.}
            \CO{financial/emotional distress; cause/suffer distress}
            \end{ExplainCard}

            \begin{ExplainCard}{throw-away society}[n][C1]
            \EN{a society that discards goods quickly rather than repairing or reusing them.}
            \SY{disposable culture; wasteful society}
            \VI{xã hội vứt bỏ; chuộng đồ dùng một lần.}
            \EX{Fast fashion fuels a throw-away society.}
            \EX{Policies aim to curb the throw-away society through recycling.}
            \CO{create/criticize a throw-away society}
            \end{ExplainCard}

            \begin{ExplainCard}{culprit}[n][C1]
            \EN{the main cause of a problem or bad situation.}
            \SY{cause; offender; source}
            \VI{thủ phạm; nguyên nhân chính.}
            \EX{Overbuying is the real culprit here.}
            \EX{Plastic packaging is a key culprit in marine pollution.}
            \CO{main/prime culprit; culprit behind}
            \end{ExplainCard}

            \begin{ExplainCard}{media consumption}[n][C1]
            \EN{the amount and manner in which people use media content.}
            \SY{media use; content consumption}
            \VI{mức độ tiêu thụ/truyền thông mà người dùng tiếp nhận.}
            \EX{I’ve reduced my media consumption lately.}
            \EX{High media consumption correlates with impulse buying.}
            \CO{reduce/limit/increase media consumption}
            \end{ExplainCard}

            \begin{ExplainCard}{swamped}[adj][C1]
            \EN{overwhelmed with a large amount of something.}
            \SY{overloaded; inundated; snowed under}
            \VI{bị ngập/ngập đầu; quá tải.}
            \EX{My inbox is swamped with promos.}
            \EX{Consumers are swamped with information in online markets.}
            \CO{be swamped with/by work/ads/requests}
            \end{ExplainCard}

            \begin{ExplainCard}{misleading}[adj][C1]
            \EN{giving the wrong idea or impression.}
            \SY{deceptive; inaccurate; false}
            \VI{gây hiểu lầm; đánh lừa.}
            \EX{That headline is misleading.}
            \EX{Misleading claims distort consumer choices.}
            \CO{misleading ads/statements/figures}
            \end{ExplainCard}

            \begin{ExplainCard}{meditation}[n][B2]
            \EN{the practice of focusing the mind for relaxation or awareness.}
            \SY{mindfulness; contemplation}
            \VI{thiền; thực hành chánh niệm.}
            \EX{Ten minutes of meditation calms me down.}
            \EX{Meditation interventions improve well-being metrics.}
            \CO{practice/do meditation; guided meditation}
            \end{ExplainCard}

            \begin{ExplainCard}{impulse purchase}[n][C1]
            \EN{a spontaneous, unplanned buying decision.}
            \SY{impulsive buy; spur-of-the-moment purchase}
            \VI{mua sắm bộc phát; mua trong phút bốc đồng.}
            \EX{He grabbed a candy bar as an impulse purchase.}
            \EX{Prominent displays increase impulse purchases in stores.}
            \CO{make/avoid impulse purchases}
            \end{ExplainCard}
        \end{VocabExplain}

    \begin{VocabHighlights}
        \VH{make a try at}{(idiom) to seize a chance or opportunity to do or attempt something}{(thành ngữ) thử làm gì}
        \VH{rusty}{(adj) not as good as it was because you have not practised it}{(tính từ) kém dần đi do lâu không luyện tập}
        \VH{smattering}{(n) a slight knowledge of something}{(danh từ) 1 chút, 1 tẹo}
        \VH{indispensable}{(adj) so good or important that you could not manage without}{(tính từ) không thể thiếu được}
        \VH{in the foreseeable future}{(phrase) as far into the future as you can imagine or plan for}{(cụm từ) trong tương lai gần}
        \VH{rewarding}{(adj) satisfying or beneficial}{(tính từ) xứng đáng, thỏa mãn}
        \VH{grammar-translation method}{(phrase) a method in which students learn grammatical rules and then apply those rules by translating sentences between the target language and the native language}{(cụm từ) cách học nhồi ngữ pháp – từ vựng truyền thống, trong đó học sinh học thuộc quy tắc ngữ pháp và áp dụng bằng cách đặt câu}
        \VH{communicative language teaching method}{(phrase) an approach to language teaching that emphasizes interaction as both the means and the ultimate goal of study}{(cụm từ) cách giảng dạy ngôn ngữ theo định hướng giao tiếp}
        \VH{gain ground}{(phrase) become more popular or accepted}{(cụm từ) trở nên phổ biến, công nhận rộng rãi hơn}
        \VH{have a shot at}{(idiom) to try something}{(thành ngữ) thử cái gì}
        \VH{a tough row to hoe}{(idiom) a difficult task}{(thành ngữ) việc khó khăn}
        \VH{in this day and age}{(phrase) at the present time; in the modern era}{(cụm từ) trong thời đại mới này}
        \VH{brick and mortar shop}{(phrase) existing as a physical building, especially a shop}{(cụm từ) cửa hàng truyền thống}
        \VH{be marginalized}{(phrase) be replaced}{(cụm từ) bị thay thế}
        \VH{dominate}{(v) to be the largest, most important, or most noticeable part of something}{(động từ) thống trị}
        \VH{a steal}{(idiom) be cheap}{(thành ngữ) rất rẻ, rẻ như cho}
        \VH{popular prices}{(phrase) a low price that people are willing to pay}{(cụm từ) giá rẻ}
        \VH{unbeatable}{(adj) unable to be defeated or improved because of excellent quality}{(tính từ) quá tốt (giá), không thể tốt hơn được nữa}
        \VH{professionalism}{(n) the combination of all the qualities that are connected with trained and skilled people}{(danh từ) tính chuyên nghiệp}
        \VH{at short notice}{(idiom) with little warning or time for preparation}{(thành ngữ) không báo trước}
        \VH{make a purchase}{(phrase) buy something}{(cụm từ) mua}
        \VH{interface}{(n) the way a computer program presents information to a user or receives information from a user, in particular the layout of the screen and the menus}{(danh từ) giao diện}
        \VH{user-friendly}{(adj) if something, especially something related to a computer, is user-friendly, it is simple for people to use}{(tính từ) dễ sử dụng}
        \VH{navigate}{(v) to move around a website or computer screen, or between websites or screens}{(động từ) (thao tác) di chuyển}
        \VH{with ease}{(phrase) easily}{(cụm từ) dễ dàng}
        \VH{literally}{(adv) using the real or original meaning of a word or phrase}{(trạng từ) về nghĩa đen}
        \VH{bulky}{(adj) too big and taking up too much space}{(tính từ) cồng kềnh}
        \VH{warranty}{(n) a written promise from a company to repair or replace a product that develops a fault within a particular period of time, or to do a piece of work again if it is not satisfactory}{(danh từ) sự bảo hành}
        \VH{universal}{(adj) existing everywhere or involving everyone}{(tính từ) phổ biến rộng rãi, toàn cầu}
        \VH{human involvement}{(n) an environment in which people have an impact on decisions and actions that affect their jobs}{(danh từ) sự tham gia, can thiệp của con người}
        \VH{afford}{(v) to allow someone to have something pleasant or necessary}{(động từ) tạo điều kiện}
        \VH{incur}{(v) to experience something, usually something unpleasant, as a result of actions you have taken}{(động từ) gây ra điều gì xấu}
        \VH{at customers’ disposal}{(phrase) available for someone to use}{(cụm từ) có sẵn để dùng}
        \VH{shun}{(v) to avoid something}{(động từ) tránh}
        \VH{inflated}{(adj) something that is higher than it should be, or higher than people think is reasonable}{(tính từ) lạm phát, thổi phồng}
        \VH{disinclined}{(adj) to not want to do something}{(tính từ) không muốn làm gì}
        \VH{inventory}{(n) a detailed list of all the things in a place}{(danh từ) hàng tồn kho}
        \VH{clearance}{(n) official permission for something or the state of having satisfied the official conditions of something}{(danh từ) sự thông quan (ở cửa khẩu)}
        \VH{at the expense of something}{(phrase) if you do one thing at the expense of another, doing the first thing harms the second thing}{(cụm từ) phải đánh đổi cái gì}
        \VH{spurious}{(adj) based on false reasoning or information that is not true, and therefore not to be trusted}{(tính từ) giả mạo}
        \VH{allure}{(n) the quality of being attractive, interesting, or exciting}{(danh từ) sự hấp dẫn, cuốn hút}
        \VH{on a whim}{(phrase) a sudden wish or idea, especially one that cannot be reasonably explained}{(cụm từ) trong thoáng chốc, không có kế hoạch trước}
        \VH{storytelling}{(n) the activity of writing, telling, or reading stories}{(danh từ) lối kể chuyện}
        \VH{retail therapy}{(n) the act of buying special things for yourself in order to feel better when you are unhappy}{(danh từ) sự nghiện mua sắm vì cho rằng mua sắm có thể khiến tâm trạng thoải mái}
        \VH{consumerism}{(n) the state of an advanced industrial society in which a lot of goods are bought and sold}{(danh từ) chủ nghĩa tiêu dùng, tập trung vào mua sắm mà bỏ đi các sản phẩm dù chúng còn dùng tốt}
        \VH{to translate into something}{(phrase) to change something into a different form}{(cụm từ) sẽ chuyển thành, tạo thành cái gì}
        \VH{bolster}{(v) to support or improve something or make it stronger}{(động từ) nâng đỡ, hỗ trợ, cải thiện}
        \VH{distress}{(n) a feeling of extreme worry, sadness, or pain}{(danh từ) sự lo lắng}
        \VH{a throw-away society}{(n) a human society strongly influenced by consumerism}{(danh từ) một xã hội bị ảnh hưởng nặng bởi chủ nghĩa tiêu dùng}
        \VH{to be swamped with}{(phrase) if something swamps a person, system, or place, more of it arrives than can be easily dealt with}{(cụm từ) bị ngập trong, quá tải bởi cái gì}
        \VH{meditation}{(n) the act of giving your attention to only one thing, either as a religious activity or as a way of becoming calm and relaxed}{(danh từ) sự ngồi thiền}
    \end{VocabHighlights}
    \end{test}

    \begin{test}{TEST 3}
    \noindent
    \textbf{Part 1. Swimming}
    \begin{qa}{Did you learn to swim when you were a child? [Why? Why not?]}
    Yes, I did. I made a \textbf{sustained} effort to learn swimming when I was a little boy but so far it has not been \textbf{brought to fruition} yet. I almost got drowned \textbf{beforehand} so the very thought of jumping into the water has \textbf{scarred me for life}.
    \end{qa}

    \begin{qa}{How often do you go swimming now? [Why? Why not?]}
    As I mentioned above, I do not know how to swim so I am not in the habit of going for a swim, maybe once or twice per year when my family is on vacation on the coast perhaps.
    \end{qa}

    \begin{qa}{What places are there for swimming where you live? [Why?]}
    I think there are two places that swimmers can \textbf{pop in}. Firstly, the apartment complex in front of my block has an open-air swimming pool, which requires membership registration. Secondly, my school has a sporting complex equipped with an indoor pool that can \textbf{cater for} every registered member throughout the year.
    \end{qa}

    \begin{qa}{Do you think it would be more enjoyable to go swimming outdoors or at an indoor pool? [Why?]}
    Well, I hardly ever go swimming so I don’t mind indoor or outdoor pool. However, I prefer outdoor areas, for example, a \textbf{sun-drenched} beach, where almost everyone can \textbf{bask in the sun} and \textbf{swim around for free}. In contrast, an indoor pool requires swimmers to pay some fees.
    \end{qa}

        \begin{VocabExplain}[Part 1]
            \begin{ExplainCard}{sustained}[adj][C1]
            \EN{continuing for a long time; maintained at a steady level or rate.}
            \SY{continuous; prolonged; steady}
            \VI{dai dẳng, liên tục; được duy trì đều đặn.}
            \EX{She made a sustained effort to improve her stroke.}
            \EX{Sustained practice is a key determinant of skill acquisition.}
            \CO{sustained effort/growth/attention}
            \end{ExplainCard}

            \begin{ExplainCard}{bring to fruition}[phrase][C1]
            \EN{to make a plan or effort succeed; to achieve an intended result.}
            \SY{realize; achieve; carry through}
            \VI{đưa tới thành công; biến thành hiện thực.}
            \EX{Years of training were finally brought to fruition at the meet.}
            \EX{Policy pilots must be brought to fruition through sustained funding.}
            \CO{plans/efforts brought to fruition; bring sth to fruition}
            \end{ExplainCard}

            \begin{ExplainCard}{beforehand}[adv][B2]
            \EN{earlier; in advance of a particular event.}
            \SY{in advance; ahead of time; earlier}
            \VI{trước đó; trước khi việc gì diễn ra.}
            \EX{Let me know beforehand if you’re coming.}
            \EX{Participants received safety instructions beforehand.}
            \CO{well/long beforehand; know/arrange/plan beforehand}
            \end{ExplainCard}

            \begin{ExplainCard}{scar (someone) for life}[v phrase][C1]
            \EN{to cause lasting psychological damage or fear.}
            \SY{traumatize; leave lasting damage}
            \VI{gây ám ảnh cả đời; để lại vết thương tinh thần lâu dài.}
            \EX{The near-drowning incident scarred him for life.}
            \EX{Early adverse experiences can scar individuals for life.}
            \CO{be/get scarred for life; experience/event that scars sb}
            \end{ExplainCard}

            \begin{ExplainCard}{pop in}[phr.v][B2]
            \EN{to visit or go somewhere quickly or briefly.}
            \SY{drop in; stop by; swing by}
            \VI{ghé qua nhanh; tạt vào.}
            \EX{We can pop in the pool after class.}
            \EX{Residents often pop in community facilities between activities.}
            \CO{pop in to/at/for; just pop in}
            \end{ExplainCard}

            \begin{ExplainCard}{cater for}[phr.v][C1]
            \EN{to provide what is needed or wanted for a particular group or purpose.}
            \SY{serve; accommodate; provide for}
            \VI{phục vụ/đáp ứng nhu cầu cho.}
            \EX{This pool caters for families at weekends.}
            \EX{The facility caters for beginners through advanced swimmers.}
            \CO{cater for needs/demands/guests}
            \end{ExplainCard}

            \begin{ExplainCard}{sun-drenched}[adj][C1]
            \EN{receiving a lot of strong sunlight.}
            \SY{sunlit; sunny; bright}
            \VI{ngập tràn nắng; đầy ánh mặt trời.}
            \EX{We relaxed on a sun-drenched beach.}
            \EX{Sun-drenched coasts attract seasonal tourism.}
            \CO{sun-drenched beach/terrace/coast}
            \end{ExplainCard}

            \begin{ExplainCard}{bask in the sun}[phrase][B2]
            \EN{to lie or relax in warm sunlight.}
            \SY{sunbathe; soak up the sun}
            \VI{tắm nắng; nằm phơi nắng.}
            \EX{People bask in the sun after a swim.}
            \EX{Visitors often bask in the sun to recover after training sessions.}
            \CO{bask in the sun/warmth; bask on the beach}
            \end{ExplainCard}

            \begin{ExplainCard}{swim around}[v][B1]
            \EN{to move about in water in a relaxed or aimless way.}
            \SY{paddle; splash about; drift}
            \VI{bơi lội quanh quẩn; bơi thong thả.}
            \EX{Kids like to swim around for free at the beach.}
            \EX{Participants were allowed to swim around during the cool-down period.}
            \CO{swim around/in; just swim around}
            \end{ExplainCard}
        \end{VocabExplain}

    \noindent
    \textbf{Part 2.}
    \begin{qa}{Describe a famous business person that you know about. You should say:}
    \begin{itemize}
    \item Who this person is
    \item What kind of business this person is involved in
    \item What you know about this business person
    \item And explain what you think of this business person
    \end{itemize}

    I have met so many people and almost every person has left a vivid impression on me of who they are, or what they possess and do. There is one person who I do admire because of his personality, outstanding business and outlook on life. It is Mr. Dung, who was listed one of 30 most successful and inspiring young businessmen in my city in the year 2019. He works in the education sector; in particular, he is a co-founder of Ezi English Center, offering English courses for people of all ages.

    You may ask me why I \textbf{know him inside out}. Well, it is due to the fact that I am a close colleague, so he usually confides in me his inner thoughts. Having graduated from Foreign Trade university, he quitted a decent job as a public relations officer at a multinational company to set up an English center in his hometown because people there rarely have access to English. It was a \textbf{milestone} in his life because it was a \textbf{win-lose situation}. People in his hometown hadn’t \textbf{realized} the importance of English, so at first, he \textbf{got off to a rocky start} as it was quite a challenge to convince them to pay for his courses. I still vividly remember the time when he told me that he was \textbf{powerless} to change people’s mind, or when he could not come up with a \textbf{killer idea} to achieve \textbf{a balance between} cheaper tuition fees and lower running costs, then how he \textbf{saw the light at the end of the tunnel}. Particularly, he offered courses with tuition fees which cost \textbf{next to nothing} for early registered students. Then, high achievers of IELTS having studied under his instruction in such courses \textbf{introduced} more students to his center. With his \textbf{resilience} and determination, he has \textbf{planted in their mind the fact that} English is \textbf{child’s play} and can transform their lives by helping them land a better paid job. All his efforts have \textbf{paid off} and his business starts to \textbf{take off} when he can build a good reputation for it.

    Thanks to his stories, what I can tell is that he is truly a \textbf{conscientious} person, \textbf{with a heart of gold}. He puts people first and \textbf{put himself in their shoes} to raise them up.
    \end{qa}

        \begin{VocabExplain}[Part 2]
            \begin{ExplainCard}{know (someone) inside out}[idiom][C1]
            \EN{to know someone extremely well, including their character and habits.}
            \SY{know thoroughly; know inside and out}
            \VI{biết rất rõ về ai đó; hiểu tường tận.}
            \EX{After years of working together, I know her inside out.}
            \EX{Long-term collaborators often know each other inside out, which streamlines teamwork.}
            \CO{know sb/sth inside out; understand inside out}
            \end{ExplainCard}

            \begin{ExplainCard}{milestone}[n][C1]
            \EN{an important event or stage in development or progress.}
            \SY{landmark; turning point; watershed}
            \VI{cột mốc quan trọng.}
            \EX{Launching the app was a milestone for the team.}
            \EX{Securing seed funding marked a milestone in the firm’s growth trajectory.}
            \CO{major/key milestone; mark/reach a milestone}
            \end{ExplainCard}

            \begin{ExplainCard}{win-lose situation}[n][C1]
            \EN{a scenario in which one side’s gain is another side’s loss.}
            \SY{zero-sum game; adversarial setup}
            \VI{tình huống kẻ thắng người thua (bên này được thì bên kia mất).}
            \EX{Negotiations became a win-lose situation.}
            \EX{Policy design should avoid win-lose situations by aligning incentives.}
            \CO{create/avoid a win-lose situation}
            \end{ExplainCard}

            \begin{ExplainCard}{realize}[v][B2]
            \EN{to become aware of or understand a fact or situation.}
            \SY{recognize; grasp; perceive}
            \VI{nhận ra; hiểu ra.}
            \EX{I didn’t realize the class started earlier.}
            \EX{Consumers often realize the benefits only after adoption.}
            \CO{suddenly/gradually realize; realize that + clause}
            \end{ExplainCard}

            \begin{ExplainCard}{get off to a rocky start}[idiom][C1]
            \EN{to begin something with difficulties or problems.}
            \SY{start badly; have a bumpy beginning}
            \VI{khởi đầu trắc trở; gặp nhiều khó khăn ban đầu.}
            \EX{The project got off to a rocky start.}
            \EX{Many ventures get off to a rocky start before finding product–market fit.}
            \CO{get off to a good/rocky start}
            \end{ExplainCard}

            \begin{ExplainCard}{powerless}[adj][C1]
            \EN{without the ability or authority to act or change a situation.}
            \SY{helpless; impotent; unable}
            \VI{bất lực; không có quyền lực/khả năng.}
            \EX{I felt powerless to help.}
            \EX{Stakeholders can feel powerless when decision rights are centralized.}
            \CO{feel/be powerless (to do sth)}
            \end{ExplainCard}

            \begin{ExplainCard}{killer idea}[n][C1]
            \EN{an exceptionally strong or effective idea.}
            \SY{brilliant idea; game-changing concept}
            \VI{ý tưởng “đỉnh”; cực kỳ hiệu quả.}
            \EX{Her pitch had one killer idea.}
            \EX{A killer idea can differentiate a startup in crowded markets.}
            \CO{come up with/land a killer idea}
            \end{ExplainCard}

            \begin{ExplainCard}{a balance between}[phrase][B2]
            \EN{a proper compromise or middle point between two competing things.}
            \SY{equilibrium; trade-off; middle ground}
            \VI{sự cân bằng giữa hai yếu tố đối lập.}
            \EX{We need a balance between price and quality.}
            \EX{Effective policy strikes a balance between equity and efficiency.}
            \CO{strike/achieve/maintain a balance between A and B}
            \end{ExplainCard}

            \begin{ExplainCard}{see the light at the end of the tunnel}[idiom][C1]
            \EN{to begin to see signs of improvement after a difficult period.}
            \SY{see hope; glimpse a turnaround}
            \VI{thấy tia hy vọng sau giai đoạn khó khăn.}
            \EX{After months of rehab, she saw the light at the end of the tunnel.}
            \EX{Revenue growth suggests the company is seeing the light at the end of the tunnel.}
            \CO{finally/gradually see the light at the end of the tunnel}
            \end{ExplainCard}

            \begin{ExplainCard}{next to nothing}[idiom][C1]
            \EN{almost no amount; for a very small cost.}
            \SY{hardly anything; for peanuts}
            \VI{gần như chẳng đáng kể; với giá cực rẻ.}
            \EX{We bought the chairs for next to nothing.}
            \EX{Early users gained access for next to nothing to spur adoption.}
            \CO{cost/charge/pay next to nothing}
            \end{ExplainCard}

            \begin{ExplainCard}{introduce (someone) to (something)}[v][B2]
            \EN{to make someone experience or learn about something for the first time; to refer new people to a place.}
            \SY{refer; present; acquaint}
            \VI{giới thiệu ai đến/cho biết về điều gì; giới thiệu khách hàng/học viên.}
            \EX{A friend introduced me to that gym.}
            \EX{Alumni networks often introduce high-caliber candidates to programs.}
            \CO{introduce sb to sth; introduce new clients/students}
            \end{ExplainCard}

            \begin{ExplainCard}{resilience}[n][C1]
            \EN{the ability to recover quickly from difficulties; toughness.}
            \SY{grit; perseverance; tenacity}
            \VI{sự kiên cường; khả năng phục hồi.}
            \EX{Her resilience helped her bounce back.}
            \EX{Organizational resilience mitigates shocks and accelerates recovery.}
            \CO{build/show resilience; resilient mindset/system}
            \end{ExplainCard}

            \begin{ExplainCard}{plant (an idea) in sb’s mind}[v phrase][C1]
            \EN{to make someone start believing or considering an idea.}
            \SY{instill; imprint; embed}
            \VI{gieo vào đầu ai một ý nghĩ; làm ai tin điều gì.}
            \EX{The coach planted in us the belief we could win.}
            \EX{Education campaigns plant in citizens’ minds sustainable habits.}
            \CO{plant/instill the idea/notion/belief}
            \end{ExplainCard}

            \begin{ExplainCard}{child’s play}[idiom][C1]
            \EN{something that is very easy to do.}
            \SY{a breeze; effortless; easy as pie}
            \VI{dễ như chơi.}
            \EX{After a few lessons, driving felt like child’s play.}
            \EX{With the right tooling, deployment becomes child’s play.}
            \CO{be child’s play; make sth child’s play}
            \end{ExplainCard}

            \begin{ExplainCard}{pay off}[phr.v][B2]
            \EN{to result in success or benefit after effort.}
            \SY{bear fruit; succeed; yield results}
            \VI{đem lại kết quả; sinh trái ngọt.}
            \EX{Months of practice finally paid off.}
            \EX{Investment in training pays off through higher productivity.}
            \CO{efforts/investments pay off; eventually/finally pay off}
            \end{ExplainCard}

            \begin{ExplainCard}{take off}[phr.v][C1]
            \EN{(of a business/idea) to become successful or popular quickly.}
            \SY{boom; surge; gain traction}
            \VI{cất cánh; phát triển nhanh chóng.}
            \EX{The café really took off after the rebrand.}
            \EX{User growth took off once network effects kicked in.}
            \CO{business/market/app takes off; quickly/really take off}
            \end{ExplainCard}

            \begin{ExplainCard}{conscientious}[adj][C1]
            \EN{careful, thorough, and guided by a strong sense of duty.}
            \SY{diligent; scrupulous; responsible}
            \VI{tận tâm; có trách nhiệm và tỉ mỉ.}
            \EX{She’s a conscientious mentor.}
            \EX{Conscientious leadership fosters trust and long-term loyalty.}
            \CO{a conscientious worker/leader/approach}
            \end{ExplainCard}

            \begin{ExplainCard}{with a heart of gold}[idiom][C1]
            \EN{very kind and generous.}
            \SY{kind-hearted; big-hearted; benevolent}
            \VI{rất tốt bụng; giàu lòng nhân ái.}
            \EX{He looks tough but has a heart of gold.}
            \EX{Community leaders with a heart of gold often spearhead outreach programs.}
            \CO{have/with a heart of gold}
            \end{ExplainCard}

            \begin{ExplainCard}{put oneself in (someone’s) shoes}[idiom][C1]
            \EN{to imagine how someone else feels in their situation.}
            \SY{empathize; see from another’s perspective}
            \VI{đặt mình vào vị trí của người khác.}
            \EX{Try to put yourself in her shoes before judging.}
            \EX{Design thinking urges teams to put themselves in users’ shoes.}
            \CO{put yourself in sb’s shoes; empathic perspective-taking}
            \end{ExplainCard}
        \end{VocabExplain}

    \noindent
    \textbf{Part 3.}
    \begin{qa}{What kinds of people are most famous in your country today?}
    Well, it is hard for me to give you a definite answer. Typically, \textbf{eminent} people who have expertise in any field can gain respect and fame effortlessly. They are usually \textbf{in the limelight} and become public figures for several programs. \textbf{That being said}, there are people who are notorious for scandals or \textbf{wrongdoings}. Even though they could set bad examples to society, their stories still reach a wide range of audiences due to \textbf{the aid} of media such as You Tube.
    \end{qa}

    \begin{qa}{Why are there so many stories about famous people in the news?}
    Admittedly, celebrity news usually \textbf{hits the headline} because of its \textbf{sensationalism}. Regardless of good or bad news, \textbf{they all draw much public attention}. Many \textbf{put this down to} the curiosity of the audience, which urges them to \textbf{pry into} the life of celebrities. Therefore, stories about famous people is increasingly popular in the news as a way to increase the \textbf{viewership} and make significant profits for producing agencies.
    \end{qa}

    \begin{qa}{Do you agree or disagree that many young people today want to be famous?}
    If you ask me, I would say yes. In fact, there is a growing tendency for young people to be widely \textbf{acclaimed}. For one thing, becoming famous is a way to earn a living. I mean, the popularity often \textbf{goes in hand with} more job opportunities and decent income. More importantly, many enjoy a \textbf{sense of publicity} as they want to devote their talents to the community. Examples can be seen in artists or politicians. For them, becoming famous is an \textbf{instinctive drive}.
    \end{qa}

    \begin{qa}{Do you think it is easy for famous people to earn a lot of money?}
    Honestly, I would say yes and no. It depends on whether their fame is related to their expertise or not. On the one hand, there are people who have a \textbf{meteoric rise to fame} without any talent. They may enjoy being in the spotlight, earning some money in the short run, but in the long run, \textbf{the chances} of them maintaining a six-figure income \textbf{are quite slim}. On the other hand, there are countless \textbf{high-profile} figures whose hard work pays off. Therefore, I would say their high income is \textbf{commensurate with} their tremendous efforts and remarkable talent.
    \end{qa}

    \begin{qa}{Why might famous people enjoy having fans?}
    Of course, there are certain reasons to explain this. To my knowledge, the number of fans can be \textbf{indicative of} popularity level. In other words, having a \textbf{sizable following} \textbf{would be accompanied by} more job opportunities and networks of contact, which \textbf{solidifies} the career path of celebrities. Furthermore, one \textbf{rewarding} aspect of becoming famous is \textbf{embracing} the support from the followers, who advocate the way of life of their idols. The support, in turn, act as a driving factor for famous people to flourish in their career.
    \end{qa}

    \begin{qa}{In what ways could famous people use their influence to do good things in the world?}
    Obviously, there are several ways that public figures could exert positive influences on society. Celebrities are one of the best ways to promote companies’ products, set different trends, and \textbf{voice} opinions in many fields. For example, many Vietnamese artists have \textbf{endorsed} The Earth Hour, a global campaign for energy saving and environmental conservation. Not surprisingly, this has a far-reaching effect on the attitude of other people towards an intact environment. Another example worthy of mention lies in the case of Thuy Tien, a famous Vietnamese singer. Thanks to her fame, she managed to mobilize billions of Vietnam Dong to donate to the needy in areas \textbf{ravaged} by floods. No matter what good \textbf{deed} celebrities have done, they are \textbf{magnified} and could set a trend to other people.
    \end{qa}

        \begin{VocabExplain}[Part 3]
            \begin{ExplainCard}{eminent}[adj][C1]
            \EN{famous and respected within a particular field.}
            \SY{distinguished; prominent; renowned}
            \VI{lỗi lạc; có danh tiếng trong một lĩnh vực.}
            \EX{Several eminent doctors spoke at the event.}
            \EX{Eminent scholars often shape debates in their disciplines.}
            \CO{an eminent scholar/figure; eminent in + field}
            \end{ExplainCard}
            \begin{ExplainCard}{in the limelight}[idiom][C1]
            \EN{receiving a lot of public attention.}
            \SY{in the spotlight; center of attention}
            \VI{ở tâm điểm chú ý của công chúng.}
            \EX{After the viral clip, she was in the limelight.}
            \EX{High-visibility roles keep leaders in the limelight.}
            \CO{be/stay/step into the limelight}
            \end{ExplainCard}

            \begin{ExplainCard}{That being said}[phrase][C1]
            \EN{despite what was just mentioned; however.}
            \SY{nevertheless; nonetheless; even so}
            \VI{tuy vậy; nói là vậy nhưng.}
            \EX{I like fame. That being said, privacy matters.}
            \EX{The data look promising; that being said, the sample is small.}
            \CO{That being said, + clause}
            \end{ExplainCard}

            \begin{ExplainCard}{wrongdoings}[n][C1]
            \EN{illegal or immoral actions.}
            \SY{misconduct; offenses; transgressions}
            \VI{hành vi sai trái; vi phạm.}
            \EX{The firm denied any wrongdoings.}
            \EX{Media scrutiny uncovers corporate wrongdoings.}
            \CO{alleged/serious wrongdoings; admit/deny wrongdoing}
            \end{ExplainCard}

            \begin{ExplainCard}{(with) the aid (of)}[n][B2]
            \EN{help or support provided to achieve something.}
            \SY{assistance; help; support}
            \VI{sự trợ giúp; nhờ vào.}
            \EX{He finished the project with the aid of friends.}
            \EX{With the aid of social media, campaigns scale quickly.}
            \CO{with the aid of; financial/technical aid}
            \end{ExplainCard}

            \begin{ExplainCard}{hit the headlines}[idiom][C1]
            \EN{to be reported widely in the news.}
            \SY{make the news; make headlines}
            \VI{lên trang nhất; gây chú ý trên báo chí.}
            \EX{The divorce hit the headlines overnight.}
            \EX{Policy leaks often hit the headlines before release.}
            \CO{story/scandal hits the headlines}
            \end{ExplainCard}

            \begin{ExplainCard}{sensationalism}[n][C1]
            \EN{the use of exciting or shocking stories at the expense of accuracy.}
            \SY{hype; exaggeration; tabloidism}
            \VI{khuynh hướng giật gân.}
            \EX{Some outlets trade in sensationalism.}
            \EX{Sensationalism can distort risk perception among audiences.}
            \CO{media sensationalism; sensationalist coverage}
            \end{ExplainCard}

            \begin{ExplainCard}{put (sth) down to (sth)}[phr.v][C1]
            \EN{to think that something is caused by something else.}
            \SY{attribute to; ascribe to}
            \VI{quy cho; cho là do.}
            \EX{They put the win down to luck.}
            \EX{Researchers put the surge down to seasonal effects.}
            \CO{put success/failure down to + cause}
            \end{ExplainCard}

            \begin{ExplainCard}{pry into}[phr.v][C1]
            \EN{to try to find out private facts about a person.}
            \SY{snoop into; intrude into; poke around}
            \VI{soi mói; tò mò chuyện riêng.}
            \EX{He hates people prying into his life.}
            \EX{Audiences often pry into celebrities’ private affairs.}
            \CO{pry into sb’s life/affairs}
            \end{ExplainCard}

            \begin{ExplainCard}{viewership}[n][C1]
            \EN{the number of people who watch a program or content.}
            \SY{audience size; ratings}
            \VI{lượng người xem.}
            \EX{The finale drew huge viewership.}
            \EX{Live streams boost viewership during major events.}
            \CO{boost/increase viewership; high/low viewership}
            \end{ExplainCard}

            \begin{ExplainCard}{acclaimed}[adj][C1]
            \EN{publicly praised and admired.}
            \SY{celebrated; lauded; renowned}
            \VI{được ca ngợi; nổi danh.}
            \EX{An acclaimed actor joined the cast.}
            \EX{The acclaimed study reshaped policy debates.}
            \CO{widely/critically acclaimed; an acclaimed work}
            \end{ExplainCard}

            \begin{ExplainCard}{go (hand) in hand with}[idiom][C1]
            \EN{to be closely connected and happen together.}
            \SY{accompany; coincide with; go together with}
            \VI{đi đôi với; gắn liền với.}
            \EX{Responsibility goes hand in hand with fame.}
            \EX{Economic growth often goes hand in hand with urbanization.}
            \CO{A goes hand in hand with B}
            \end{ExplainCard}

            \begin{ExplainCard}{publicity (a sense of ~)}[n][B2]
            \EN{public attention or notice, especially via the media.}
            \SY{exposure; attention; coverage}
            \VI{sự chú ý của công chúng; truyền thông.}
            \EX{He enjoys the publicity around new releases.}
            \EX{Careful publicity management protects brand reputation.}
            \CO{seek/attract publicity; publicity stunt/campaign}
            \end{ExplainCard}

            \begin{ExplainCard}{instinctive drive}[n][C1]
            \EN{a natural, inborn urge or motivation.}
            \SY{innate urge; natural impulse}
            \VI{động lực bản năng; thôi thúc tự nhiên.}
            \EX{He has an instinctive drive to perform.}
            \EX{An instinctive drive for status appears across cultures.}
            \CO{an instinctive drive to + V}
            \end{ExplainCard}

            \begin{ExplainCard}{meteoric rise (to fame)}[n][C1]
            \EN{a very rapid and spectacular increase in success or popularity.}
            \SY{rapid ascent; swift rise}
            \VI{sự nổi lên nhanh như “tên lửa”; nổi tiếng rất nhanh.}
            \EX{Her meteoric rise shocked critics.}
            \EX{Startups sometimes experience a meteoric rise after product–market fit.}
            \CO{meteoric rise to fame/power}
            \end{ExplainCard}

            \begin{ExplainCard}{slim chance}[n][B2]
            \EN{a very small probability of something happening.}
            \SY{remote chance; long shot; low likelihood}
            \VI{khả năng mong manh; cơ hội rất thấp.}
            \EX{There’s a slim chance it will rain.}
            \EX{Evidence suggests a slim chance of long-term returns.}
            \CO{a slim chance of + V-ing}
            \end{ExplainCard}

            \begin{ExplainCard}{high-profile}[adj][C1]
            \EN{attracting a lot of public attention.}
            \SY{prominent; well-known; high-visibility}
            \VI{nổi bật; được chú ý nhiều.}
            \EX{It was a high-profile case.}
            \EX{High-profile figures can mobilize large audiences.}
            \CO{high-profile figure/campaign/event}
            \end{ExplainCard}

            \begin{ExplainCard}{commensurate with}[adj][C2]
            \EN{in proportion to; matching in size, degree, or extent.}
            \SY{proportionate to; corresponding to}
            \VI{tương xứng với.}
            \EX{Pay should be commensurate with effort.}
            \EX{The penalty is commensurate with the level of risk imposed.}
            \CO{salary/benefits commensurate with experience}
            \end{ExplainCard}

            \begin{ExplainCard}{indicative of}[adj][C1]
            \EN{showing or suggesting something.}
            \SY{suggestive of; reflective of; symptomatic of}
            \VI{biểu hiện/cho thấy (điều gì).}
            \EX{A long queue is indicative of popularity.}
            \EX{Rising churn is indicative of product–market misfit.}
            \CO{be indicative of + noun}
            \end{ExplainCard}

            \begin{ExplainCard}{sizable following}[n][C1]
            \EN{a large number of fans or supporters.}
            \SY{substantial fan base; large following}
            \VI{lượng người hâm mộ/ủng hộ đáng kể.}
            \EX{The band has a sizable following online.}
            \EX{Creators with a sizable following attract sponsors.}
            \CO{build/gain a sizable following}
            \end{ExplainCard}

            \begin{ExplainCard}{be accompanied by}[phrase][B2]
            \EN{to happen together with; to come alongside something else.}
            \SY{come with; be coupled with}
            \VI{đi kèm với; kéo theo.}
            \EX{Price hikes were accompanied by protests.}
            \EX{Adoption is often accompanied by support and training.}
            \CO{be accompanied by risks/benefits/changes}
            \end{ExplainCard}

            \begin{ExplainCard}{solidify}[v][C1]
            \EN{to make something more certain or strong.}
            \SY{strengthen; consolidate; cement}
            \VI{củng cố; làm vững chắc.}
            \EX{Winning the award solidified her status.}
            \EX{Repeated successes solidify a leader’s legitimacy.}
            \CO{solidify support/reputation/position}
            \end{ExplainCard}

            \begin{ExplainCard}{rewarding}[adj][B2]
            \EN{giving satisfaction or benefit.}
            \SY{fulfilling; gratifying; worthwhile}
            \VI{xứng đáng; mang lại cảm giác hài lòng/lợi ích.}
            \EX{Coaching young swimmers is rewarding.}
            \EX{Volunteering proves rewarding for community well-being.}
            \CO{find sth rewarding; a rewarding aspect/experience}
            \end{ExplainCard}

            \begin{ExplainCard}{embrace}[v][C1]
            \EN{to accept or support something willingly.}
            \SY{adopt; welcome; champion}
            \VI{đón nhận; ủng hộ.}
            \EX{Fans quickly embraced the new style.}
            \EX{Organizations embrace innovation to remain competitive.}
            \CO{embrace change/ideas/support}
            \end{ExplainCard}

            \begin{ExplainCard}{voice}[v][C1]
            \EN{to express (an opinion or feeling) in words, especially publicly.}
            \SY{express; air; articulate}
            \VI{lên tiếng; bày tỏ (quan điểm).}
            \EX{They voiced concerns about safety.}
            \EX{Stakeholders voiced opinions during the consultation.}
            \CO{voice concerns/opinions/support}
            \end{ExplainCard}

            \begin{ExplainCard}{endorse}[v][C1]
            \EN{to publicly approve or support someone or something.}
            \SY{back; support; champion}
            \VI{ủng hộ/công khai tán thành.}
            \EX{Several stars endorsed the charity run.}
            \EX{Researchers endorse open data to advance science.}
            \CO{endorse a campaign/product/candidate}
            \end{ExplainCard}

            \begin{ExplainCard}{ravaged}[adj][C1]
            \EN{severely damaged or destroyed.}
            \SY{devastated; ruined; wrecked}
            \VI{bị tàn phá nặng nề.}
            \EX{They sent aid to ravaged villages.}
            \EX{Ravaged ecosystems require long-term restoration.}
            \CO{areas/regions ravaged by war/floods}
            \end{ExplainCard}

            \begin{ExplainCard}{deed}[n][B2]
            \EN{an intentional act, especially one that is good or notable.}
            \SY{act; action; feat}
            \VI{việc làm (thường là việc tốt); hành động.}
            \EX{She did a kind deed for a stranger.}
            \EX{Publicizing good deeds can inspire pro-social behavior.}
            \CO{good/heroic deed; do a deed}
            \end{ExplainCard}

            \begin{ExplainCard}{magnify}[v][C1]
            \EN{to make something seem greater or more significant.}
            \SY{amplify; heighten; exaggerate}
            \VI{phóng đại; khuếch đại (tác động/ảnh hưởng).}
            \EX{Social media can magnify small mistakes.}
            \EX{Celebrity platforms magnify the reach of charitable work.}
            \CO{magnify impact/effect/influence}
            \end{ExplainCard}
        \end{VocabExplain}

    \begin{VocabHighlights}
        \VH{sustained}{(adj) continuing for a long time}{(tính từ) có từ lâu}
        \VH{to be brought to fruition}{(idiom) to be successful}{(thành ngữ) thành công}
        \VH{beforehand}{(adv) earlier (than a particular time)}{(trạng từ) sớm hơn}
        \VH{scar somebody for life}{(idiom) have a permanent emotional effect on someone}{(thành ngữ) khiến ai ám ảnh cả đời}
        \VH{pop in}{(phr.v) visit there briefly}{(cụm động từ) tạt qua}
        \VH{sun-drenched}{(adj) receiving a lot of heat and light from the sun}{(tính từ) nhiều nắng}
        \VH{bask in (the sun)}{(phr.v) to enjoy sitting or lying in the heat or light of something, especially the sun}{(cụm động từ) tắm nắng}
        \VH{swim around}{(phr.v) swim aimlessly from place to place}{(cụm động từ) bơi loanh quanh}
        \VH{know somebody inside out}{(idiom) know somebody very well}{(thành ngữ) biết rõ}
        \VH{a milestone}{(n) a turning point}{(danh từ) bước ngoặt}
        \VH{a win-lose situation}{(n) success or failure}{(cụm từ) được ăn cả, ngã về không}
        \VH{get off to a rocky start}{(phrase) to start in a situation or relationship which is unstable and full of difficulties}{(cụm từ) khởi đầu gian nan}
        \VH{powerless}{(adj) having no power}{(tính từ) bất lực}
        \VH{a killer idea}{(idiom) a creative idea}{(thành ngữ) ý kiến đột phá, sáng tạo}
        \VH{see the light at the end of the tunnel}{(idiom) hope of success, happiness, or help after a long period of difficulty}{(thành ngữ) nhìn thấy niềm tin, hy vọng sau một chặng đường dài mệt mỏi}
        \VH{next to nothing}{(idiom) very cheap}{(thành ngữ) rẻ gần như cho}
        \VH{plant in their mind the fact}{(phrase) instill something}{(cụm từ) truyền vào trong đầu}
        \VH{a child's play}{(idiom) be easy}{(thành ngữ) dễ ợt, trò trẻ con}
        \VH{pay off}{(phr.v) it is successful}{(cụm động từ) đền đáp, trở nên thành công}
        \VH{take off}{(phrase) to suddenly start to be successful or popular}{(cụm từ) bắt đầu phát triển}
        \VH{conscientious}{(adj) putting a lot of effort into your work}{(tính từ) kiên trì, quyết tâm}
        \VH{a heart of gold}{(idiom) a kind and generous character}{(thành ngữ) có trái tim vàng}
        \VH{put himself in their shoes}{(idiom) to imagine how someone else feels in a difficult situation}{(thành ngữ) đặt bản thân mình vào vị trí người khác}
        \VH{eminent}{(adj) famous, respected, or important}{(tính từ) nổi bật, ưu tú}
        \VH{in the lime light}{(phrase) in the public attention and interest}{(cụm từ) thu hút sự chú ý của công chúng}
        \VH{That being said}{(phrase) however}{(cụm từ) tuy nhiên}
        \VH{wrongdoing}{(n) a bad or an illegal action}{(danh từ) việc làm sai trái}
        \VH{aid}{(n) help or support}{(danh từ) sự hỗ trợ}
        \VH{to hit the headline}{(phrase) to appear in the news suddenly or receive a lot of attention in news reports}{(cụm từ) trở thành tiêu điểm (báo chí, truyền thông)}
        \VH{sensationalism}{(n) the act by newspapers, television, etc. of presenting information in a way that is shocking or exciting}{(danh từ) sự giật gân}
        \VH{put something down to something}{(phrase) to think that a problem or situation is caused by a particular thing}{(cụm từ) cho rằng cái gì bắt nguồn từ đâu}
        \VH{pry into something}{(phrase) to try to find out private facts about a person}{(cụm từ) tò mò, tòi mói điều gì}
        \VH{viewership}{(n) the type or number of people who watch a particular television programme or station}{(danh từ) số lượng người xem}
        \VH{acclaimed}{(adj) attracting public approval and praise}{(tính từ) được ca ngợi}
        \VH{to go in hand with something}{(phrase) If something goes hand in hand with something else, it is closely related to it and happens at the same time as it or as a result of it}{(cụm từ) đi cùng với}
        \VH{a sense of publicity}{(n) the feeling of getting public attention}{(danh từ) cảm giác được chú ý}
        \VH{instinctive}{(adj) not thought about, planned, or learned}{(tính từ) thuộc về bản năng}
        \VH{The chances are slim}{(phrase) There is little hope}{(cụm từ) không có nhiều hy vọng}
        \VH{high-profile}{(adj) attracting a lot of attention and interest from the public and newspapers, television, etc.}{(tính từ) nổi tiếng, được nhiều người quan tâm}
        \VH{be commensurate with something}{(phrase) in a correct and suitable amount compared to something else}{(cụm từ) tương thích, phù hợp với}
        \VH{indicative}{(adj) being or relating to a sign that something exists, is true, or is likely to happen}{(tính từ) dấu hiệu chỉ ra cái gì}
        \VH{sizable}{(adj) considerable, fairly large}{(tính từ) tương đối lớn, đáng kể}
        \VH{following}{(n) a group of supporters}{(danh từ) nhóm người ủng hộ}
        \VH{rewarding}{(adj) satisfying or beneficial}{(tính từ) có tính thỏa mãn, hài lòng}
        \VH{embrace}{(v) to include something, often as one of a number of things}{(động từ) áp dụng}
        \VH{voice}{(v) to say what you think about a particular subject, especially to express a doubt, complaint, etc. that you have about it}{(động từ) lên tiếng, bày tỏ quan điểm}
        \VH{endorse}{(v) to make a public statement of your approval or support for something or someone}{(động từ) ủng hộ, tán thành một cách công khai}
        \VH{the needy}{(n) poor people}{(danh từ) người nghèo}
        \VH{ravage}{(v) to cause great damage to something}{(động từ) gây ảnh hưởng xấu đến cái gì}
        \VH{deed}{(n) an intentional act, especially a very bad or very good one}{(danh từ) một hành động bất kỳ, có thể là tốt hoặc xấu}
        \VH{magnify}{(v) to make something look larger than it is}{(động từ) phóng đại, lan truyền}
    \end{VocabHighlights}
    \end{test}

    \begin{test}{TEST 4}
    \noindent
    \textbf{Part 1. Jewellery}
    \begin{qa}{How often do you wear jewellery? [Why? Why not?]}
    To be honest, I \textbf{have an aversion to} \textbf{flashy} things like bracelets, necklaces, etc. The two items of jewellery that I have are the engagement ring and my wrist watch which I wear every day.
    \end{qa}

    \begin{qa}{What type of jewellery do you like best? [Why? Why not?]}
    I am in favor of wrist watches. I \textbf{lean towards} something that can provide me with both aesthetic and functional values. Wearing ornamental items such as bracelets or necklaces helps the owner \textbf{flaunt} his or her wealth only whereas I’d like to \textbf{keep a low profile}. My simple wrist watch enables me to \textbf{stick to} my schedule to avoid being late and that’s what counts
    \end{qa}

    \begin{qa}{When do people like to give jewellery in your country? [Why?]}
    There are two types of occasions, particularly life events and annual ones. The former refers to the engagement or wedding ceremonies. The latter is associated with birthdays or Vietnamese \& International Women’s Days, etc.
    \end{qa}

    \begin{qa}{Have you ever given jewellery to someone as a gift? [Why?]}
    I have never given any jewellery-related presents to anyone, except my wife. Not only has she been offered wrist watches, quartz and mechanical ones, on her birthdays and our anniversary but she has also received my engagement and wedding rings.
    \end{qa}

        \begin{VocabExplain}[Part 1]
            \begin{ExplainCard}{have an aversion to}[phrase][C1]
            \EN{to strongly dislike something and try to avoid it.}
            \SY{dislike; be averse to; shun}
            \VI{\textit{dị ứng/ghét} mạnh; không ưa và muốn tránh.}
            \EX{I have an aversion to bulky necklaces.}
            \EX{Some candidates have an aversion to self-promotion during interviews.}
            \CO{have an aversion to + noun/gerund; deep/strong aversion}
            \end{ExplainCard}

            \begin{ExplainCard}{flashy}[adj][B2]
            \EN{bright or showy in a way that attracts attention.}
            \SY{showy; ostentatious; gaudy}
            \VI{\textit{loè loẹt}, phô trương.}
            \EX{He prefers simple over flashy accessories.}
            \EX{Flashy branding can undermine a premium image in some markets.}
            \CO{flashy clothes/jewellery/advertising}
            \end{ExplainCard}

            \begin{ExplainCard}{lean towards}[phr.v][C1]
            \EN{to be inclined to choose or prefer something.}
            \SY{favor; be inclined toward; gravitate toward}
            \VI{\textit{nghiêng về}, thiên về lựa chọn gì.}
            \EX{I lean towards a minimalist watch.}
            \EX{Consumers often lean towards functional designs over ornamentation.}
            \CO{lean towards + option/approach/style}
            \end{ExplainCard}

            \begin{ExplainCard}{flaunt}[v][C1]
            \EN{to display something in a showy way to attract attention or admiration.}
            \SY{show off; parade; display}
            \VI{\textit{khoe khoang}, phô bày.}
            \EX{He likes to flaunt his new chain.}
            \EX{Celebrities sometimes flaunt wealth, shaping aspirational consumption.}
            \CO{flaunt wealth/status/assets}
            \end{ExplainCard}

            \begin{ExplainCard}{keep a low profile}[idiom][C1]
            \EN{to avoid attracting attention; to be inconspicuous.}
            \SY{stay under the radar; be discreet}
            \VI{\textit{giữ kín tiếng}, tránh gây chú ý.}
            \EX{I keep a low profile at parties.}
            \EX{Leaders may keep a low profile during sensitive negotiations.}
            \CO{keep/maintain a low profile; stay low-profile}
            \end{ExplainCard}

            \begin{ExplainCard}{stick to}[phr.v][B2]
            \EN{(1) to continue doing or using something; not change it. (2) to follow or adhere to rules, plans, or limits.}
            \SY{adhere to; keep to; follow}
            \VI{(1) \textit{giữ nguyên}, tiếp tục dùng; (2) \textit{tuân theo}, bám sát.}
            \EX{I stick to one simple watch every day.}
            \EX{Please stick to the schedule to ensure on-time delivery.}
            \CO{stick to a plan/schedule/budget; stick to basics}
            \end{ExplainCard}
        \end{VocabExplain}

    \noindent
    \textbf{Part 2.}
    \begin{qa}{Describe an interesting TV programme you watched about a science topic. You should say:}
    \begin{itemize}
    \item What science topic this TV programme was about
    \item When you saw this TV programme
    \item What you learnt from this TV programme about a science topic
    \item And explain why you found this TV programme interesting
    \end{itemize}

    Since I am quite \textbf{occupied with} my work, I do not \textbf{squeeze in} much time to watch TV programme every day. However, sometimes I turn on my television to \textbf{put my feet up} if I am available on weekends.

    Two days ago, I \textbf{got a kick out of} a TV programme on the power of sleep. It is the first science programme that has kept me \textbf{stay tuned} from the beginning to the end. The programme is called “The mysteries of simple things”, which accounts for the nature and inter-relationships of common things. I bet this programme is broadcast to \textbf{nurture} a passion for science among young people. The episode I watched is an account of sleep. It last over 30 minutes and commenced with a question of a baby: “I don’t wanna sleep. I wanna play with dolls. Why do I have to sleep?” \textbf{It goes without saying} that we can’t go on without sleep. But to give a \textbf{compelling}, explicit answer for kids is not \textbf{a breeze}. This question sparked some thoughts about me, and I was \textbf{struck dumb} for some seconds because I could not know how to answer if my niece asked me about it. When this programme ended, I gradually appreciated the importance of sleeping soundly for at least 7 hours per night.

    There is a common belief that TV programmes on science are \textbf{dead tedious}, but contrary to it, this episode \textbf{intrigued} viewers by demonstrating the case of a twin. One person \textbf{hits the sack} on time at 9 pm and sleep 8 hours a day, while the others stay up late and only sleep 5 hours a day. Over the course of a \textbf{fortnight}, one person was invariably in a good mood while the other one suffering from \textbf{sleep deprivation} is overweight and easy to \textbf{lose his cool}. This comparison grabs attention of viewers and pinpoints what stereotypes they want to follow. What an interesting programme!
    \end{qa}

        \begin{VocabExplain}[Part 2]
            \begin{ExplainCard}{occupied with}[adj phrase][B2]
            \EN{busy or fully engaged in a task or activity.}
            \SY{busy; tied up; preoccupied}
            \VI{bận rộn với; mải mê làm việc gì.}
            \EX{I’m occupied with deadlines this week.}
            \EX{Participants were often occupied with coursework during the trial.}
            \CO{be/keep sb occupied with; fully/quite occupied}
            \end{ExplainCard}

            \begin{ExplainCard}{squeeze in}[phr.v][C1]
            \EN{to manage to find time or space for something despite a busy schedule.}
            \SY{fit in; cram in; shoehorn}
            \VI{chen, tranh thủ làm trong quỹ thời gian ít ỏi.}
            \EX{I squeezed in a quick workout at lunch.}
            \EX{We can squeeze in an extra interview before noon.}
            \CO{squeeze in time/an appointment/a session}
            \end{ExplainCard}

            \begin{ExplainCard}{put one’s feet up}[idiom][B2]
            \EN{to relax, especially by sitting or lying down and resting.}
            \SY{unwind; relax; kick back}
            \VI{nghỉ ngơi thư giãn.}
            \EX{After work I just put my feet up and read.}
            \EX{Short recovery breaks let employees put their feet up and recharge.}
            \CO{come home and put your feet up}
            \end{ExplainCard}

            \begin{ExplainCard}{get a kick out of}[idiom][C1]
            \EN{to get great enjoyment or amusement from something.}
            \SY{enjoy; relish; take delight in}
            \VI{rất thích thú, thấy khoái chí.}
            \EX{Kids get a kick out of science experiments.}
            \EX{Many learners get a kick out of problem-solving tasks.}
            \CO{get a kick out of + V-ing}
            \end{ExplainCard}

            \begin{ExplainCard}{stay tuned}[idiom][B2]
            \EN{to keep watching/listening; to continue paying attention for more.}
            \SY{keep watching; keep listening; keep posted}
            \VI{tiếp tục theo dõi.}
            \EX{Stay tuned after the break.}
            \EX{Users stayed tuned for updates during the launch.}
            \CO{stay tuned for/to}
            \end{ExplainCard}

            \begin{ExplainCard}{nurture}[v][C1]
            \EN{(1) to care for and encourage the growth of someone or something; (2) to help a feeling/idea/skill develop over time.}
            \SY{foster; cultivate; nourish}
            \VI{(1) nuôi dưỡng, bồi đắp; (2) ươm mầm, vun đắp (ý tưởng/cảm hứng/kỹ năng).}
            \EX{Teachers nurture curiosity in class.}
            \EX{Incubators nurture innovation ecosystems.}
            \CO{nurture talent/passion/relationships}
            \end{ExplainCard}

            \begin{ExplainCard}{It goes without saying}[idiom][C1]
            \EN{used to state that something is obvious or generally accepted.}
            \SY{needless to say; obviously}
            \VI{khỏi phải nói; hiển nhiên.}
            \EX{It goes without saying that sleep matters.}
            \EX{It goes without saying that ethics are non-negotiable in research.}
            \CO{It goes without saying that + clause}
            \end{ExplainCard}

            \begin{ExplainCard}{compelling}[adj][C1]
            \EN{very convincing or powerfully interesting so that it holds attention.}
            \SY{persuasive; forceful; gripping}
            \VI{thuyết phục, cuốn hút.}
            \EX{She made a compelling case for change.}
            \EX{Compelling visuals increased viewer engagement.}
            \CO{compelling evidence/argument/story}
            \end{ExplainCard}

            \begin{ExplainCard}{a breeze}[n idiom][B2]
            \EN{something that is very easy to do.}
            \SY{a cinch; cakewalk; walkover}
            \VI{việc dễ như chơi.}
            \EX{For him, the quiz was a breeze.}
            \EX{With automation, deployment became a breeze.}
            \CO{be/feel a breeze; not a breeze}
            \end{ExplainCard}

            \begin{ExplainCard}{struck dumb}[adj phrase][C1]
            \EN{temporarily unable to speak because of shock or surprise.}
            \SY{speechless; dumbstruck; stunned}
            \VI{cứng họng, lặng người vì bất ngờ.}
            \EX{I was struck dumb by the news.}
            \EX{Novices were struck dumb when the results appeared.}
            \CO{be struck dumb with + emotion}
            \end{ExplainCard}

            \begin{ExplainCard}{dead tedious}[adj][C1]
            \EN{extremely boring; with \emph{dead} as an informal intensifier.}
            \SY{dreadfully boring; mind-numbing; dull}
            \VI{chán ngắt; buồn tẻ kinh khủng.}
            \EX{Some lectures felt dead tedious.}
            \EX{Overlong slides can make a session dead tedious.}
            \CO{dead tedious/boring/easy (intensifier pattern)}
            \end{ExplainCard}

            \begin{ExplainCard}{intrigue}[v][C1]
            \EN{to arouse curiosity or interest; to fascinate.}
            \SY{fascinate; captivate; engross}
            \VI{gây tò mò, lôi cuốn.}
            \EX{The twist intrigued the audience.}
            \EX{Unusual findings intrigue researchers to dig deeper.}
            \CO{intrigue viewers/readers; be intrigued by}
            \end{ExplainCard}

            \begin{ExplainCard}{hit the sack}[idiom][B2]
            \EN{to go to bed in order to sleep.}
            \SY{turn in; hit the hay; go to sleep}
            \VI{đi ngủ.}
            \EX{I’ll hit the sack early tonight.}
            \EX{Athletes hit the sack on schedule to recover.}
            \CO{decide to/try to hit the sack}
            \end{ExplainCard}

            \begin{ExplainCard}{fortnight}[n][B2]
            \EN{a period of two weeks.}
            \SY{two weeks; fourteen days}
            \VI{hai tuần lễ.}
            \EX{They tracked habits over a fortnight.}
            \EX{Data were collected every fortnight during the study.}
            \CO{over the course of a fortnight; in a fortnight’s time}
            \end{ExplainCard}

            \begin{ExplainCard}{sleep deprivation}[n][C1]
            \EN{the condition of not getting enough sleep, often leading to impairment.}
            \SY{lack of sleep; sleep loss}
            \VI{thiếu ngủ; tình trạng mất ngủ kéo dài.}
            \EX{Sleep deprivation makes me irritable.}
            \EX{Chronic sleep deprivation reduces cognitive performance.}
            \CO{chronic/acute sleep deprivation; effects of ~}
            \end{ExplainCard}

            \begin{ExplainCard}{lose one’s cool}[idiom][B2]
            \EN{to lose one’s temper; to become angry or agitated.}
            \SY{lose one’s temper; snap; blow up}
            \VI{mất bình tĩnh; nổi nóng.}
            \EX{Try not to lose your cool in debates.}
            \EX{Under time pressure, participants tended to lose their cool.}
            \CO{be easy/quick to lose one’s cool; keep your cool}
            \end{ExplainCard}
        \end{VocabExplain}

    \noindent
    \textbf{Part 3.}
    \begin{qa}{How interested are most people in your country in science?}
    In my recollection, I am not certain that science was one of the most preferable subjects when I was a high school student because of its \textbf{complexity} and boredom. This is because scientific subjects, like physics, are packed with \textbf{scholarly} disciplines and \textbf{theoretical} arguments, which definitely fascinated those having an \textbf{inquisitive} mind or \textbf{a head for} science-based subjects, not me. Nowadays, however, the growth of scientific breakthroughs has caught much society’s attention. For example, scientific insights into DNA test can allow people to \textbf{stamp out genetic diseases} or the understanding of automation have helped to create smart robots for household chores.
    \end{qa}

    \begin{qa}{Why do you think children today might be better at science than their parents?}
    Honestly, the improvements in nutrition science might be responsible for this. I believe today’s children have more well-balanced diets than their parents, which could better nurture their \textbf{intellectual} and \textbf{cognitive} development when studying. Besides, as science has \textbf{permeated} many social aspects, the \textbf{exponential} growth of scientific education is justifiable. Young students are sent to extra or \textbf{remedial} classes to catch up with the latest scientific trends. That is why children today can be proficient at science while their parents seem to be out of touch with this realm.
    \end{qa}

    \begin{qa}{How do you suggest the public can learn more about scientific developments?}
    While I acknowledge the significance of scientific advancements to our life, I also notice that there is a gap between the understanding and the application of these advancements. This can be derived from the lack of scientific background which \textbf{restrains} individuals from fully embracing new technologies. To bridge the gap, I bet that we should \textbf{arm} ourselves with theoretical and \textbf{empirical} research associated with science. This can be done by reading books, attending tutorial classes or searching on the Internet. No matter which ways people adopt, we would step closer to the scientific domain.
    \end{qa}

    \begin{qa}{What do you think are the most important scientific discoveries in the last 100 years?}
    Well, \textbf{I do not know the first thing} about this. I wish I could go back in time to investigate which one was the most powerful discovery. But it is impossible, so I would choose to say about the emergence of the Internet, an area of my interest. Admittedly, the advent of the Internet has made the world become a \textbf{global village}, where individuals are connected immediately and constantly regardless of geographical barriers and time \textbf{constraints}. Thanks to the Internet, international trade and cooperation are facilitated and these open doors to greater opportunities. For instance, countries are \textbf{inextricably} linked to encounter several global issues such as climate change or terrorism by keeping contact through the Internet.
    \end{qa}

    \begin{qa}{Do you agree or disagree that there are no more major scientific discoveries left to make?}
    On the one hand, I admit that there are much less world-changing discoveries today and science appears to witness a period of \textbf{stagnation}. This is because human life has been transformed dramatically since ancient times, and contemporary innovations can hardly make any giant \textbf{strides} to challenge the old ones. However, there are still \textbf{realms} that need to be \textbf{conquered}. Let’s say about space exploration, a solution to find a backup for the planet once natural resources are totally depleted. I think it is just the matter of time before any \textbf{monumental} achievements appears.
    \end{qa}

    \begin{qa}{Who should pay for scientific research — governments or private companies?}
    Actually, each has its own perks. From a societal perspective, the responsibility of undertaking research should fall on the government to \textbf{compromise} private interests and the needs of the many. For example, military research programs must be controlled by the government only; otherwise, our society would be at risk if \textbf{weapons of mass destruction} were widely available. Nevertheless, private research can be an alternative to public ones. The \textbf{uncertainty} of new inventions and their practicality sometimes deny them of state funding. For example, self-driving cars may be unappealing to a \textbf{layman}. It is safe to say that governments should not be allowed to control everything.
    \end{qa}

        \begin{VocabExplain}[Part 3]
            \begin{ExplainCard}{complexity}[n][C1]
            \EN{the state of having many parts and being difficult to understand.}
            \SY{complication; intricacy}
            \VI{tính phức tạp; sự rắc rối.}
            \EX{The complexity of quantum ideas puts many off.}
            \EX{Model complexity can increase error if data are limited.}
            \CO{inherent complexity; reduce/handle complexity}
            \end{ExplainCard}

            \begin{ExplainCard}{scholarly}[adj][C1]
            \EN{relating to serious academic study and research.}
            \SY{academic; erudite}
            \VI{mang tính học thuật; uyên bác.}
            \EX{She writes in a scholarly style.}
            \EX{Scholarly debates refine theories over time.}
            \CO{scholarly article/work/discipline}
            \end{ExplainCard}

            \begin{ExplainCard}{theoretical}[adj][C1]
            \EN{based on theory rather than practical experience.}
            \SY{conceptual; abstract}
            \VI{mang tính lý thuyết.}
            \EX{The course is highly theoretical.}
            \EX{Theoretical models guide experimental design.}
            \CO{theoretical framework/argument/basis}
            \end{ExplainCard}

            \begin{ExplainCard}{inquisitive}[adj][C1]
            \EN{curious and eager to learn or find out information.}
            \SY{curious; probing; inquiring}
            \VI{hiếu kỳ, ham học hỏi.}
            \EX{An inquisitive child asks endless questions.}
            \EX{An inquisitive mindset drives scientific discovery.}
            \CO{an inquisitive mind/nature; be inquisitive about}
            \end{ExplainCard}

            \begin{ExplainCard}{a head for (sth)}[idiom][C1]
            \EN{a natural ability for something (e.g., numbers, science).}
            \SY{aptitude; flair; knack}
            \VI{năng khiếu, “cái đầu” cho lĩnh vực nào đó.}
            \EX{She has a head for figures.}
            \EX{Students with a head for logic thrive in CS.}
            \CO{have a head for numbers/science}
            \end{ExplainCard}

            \begin{ExplainCard}{stamp out}[phr.v][C1]
            \EN{to stop or get rid of something completely.}
            \SY{eradicate; eliminate; wipe out}
            \VI{dập tắt; xóa bỏ hoàn toàn.}
            \EX{We must stamp out plagiarism.}
            \EX{Vaccination helps stamp out infectious diseases.}
            \CO{stamp out corruption/disease/practice}
            \end{ExplainCard}

            \begin{ExplainCard}{genetic disease}[n][C1]
            \EN{a disorder caused by changes in a person’s genes or chromosomes.}
            \SY{hereditary disorder; inherited condition}
            \VI{bệnh di truyền.}
            \EX{Some genetic diseases appear early in life.}
            \EX{Genome editing may target specific genetic diseases.}
            \CO{inherited/genetic disease; test for ~}
            \end{ExplainCard}

            \begin{ExplainCard}{intellectual}[adj][C1]
            \EN{relating to the ability to think and understand ideas.}
            \SY{cerebral; mental; cognitive}
            \VI{thuộc trí tuệ.}
            \EX{Puzzles offer intellectual challenge.}
            \EX{Intellectual growth correlates with enriched input.}
            \CO{intellectual development/ability/stimulation}
            \end{ExplainCard}

            \begin{ExplainCard}{cognitive}[adj][C1]
            \EN{connected with thinking, learning, and understanding.}
            \SY{mental; intellectual}
            \VI{thuộc nhận thức.}
            \EX{Sleep aids cognitive performance.}
            \EX{Cognitive skills underpin problem solving.}
            \CO{cognitive development/decline/load}
            \end{ExplainCard}

            \begin{ExplainCard}{permeate}[v][C1]
            \EN{to spread through and be present in every part of something.}
            \SY{pervade; penetrate}
            \VI{thấm vào; lan tỏa khắp.}
            \EX{Tech has permeated daily life.}
            \EX{STEM concepts permeate the modern curriculum.}
            \CO{permeate society/culture/industry}
            \end{ExplainCard}

            \begin{ExplainCard}{exponential}[adj][C1]
            \EN{increasing or growing very quickly at an accelerating rate.}
            \SY{rapid; explosive; soaring}
            \VI{tăng theo cấp số mũ; bùng nổ.}
            \EX{The app saw exponential growth.}
            \EX{Exponential increases strain infrastructure capacity.}
            \CO{exponential growth/rise/increase}
            \end{ExplainCard}

            \begin{ExplainCard}{remedial}[adj][C1]
            \EN{intended to improve skills of students who are behind.}
            \SY{supportive; corrective}
            \VI{bổ trợ, phụ đạo (khắc phục lỗ hổng kiến thức).}
            \EX{He joined a remedial math class.}
            \EX{Remedial programs boost foundational literacy.}
            \CO{remedial class/course/teaching}
            \end{ExplainCard}

            \begin{ExplainCard}{restrain}[v][C1]
            \EN{to hold back from action; to limit or control.}
            \SY{limit; curb; constrain}
            \VI{kiềm chế; hạn chế.}
            \EX{Fear restrained him from speaking up.}
            \EX{Costs restrain adoption of new tech.}
            \CO{restrain growth/impulse/expansion}
            \end{ExplainCard}

            \begin{ExplainCard}{arm (oneself) with}[v phrase][B2]
            \EN{to equip yourself with something useful (knowledge, tools).}
            \SY{equip; furnish; prepare}
            \VI{trang bị (kiến thức/công cụ).}
            \EX{Arm yourself with facts before debating.}
            \EX{Students arm themselves with methods before fieldwork.}
            \CO{arm oneself with data/skills/knowledge}
            \end{ExplainCard}

            \begin{ExplainCard}{empirical}[adj][C1]
            \EN{based on observation or experiment rather than theory alone.}
            \SY{evidence-based; experimental}
            \VI{thực chứng; dựa trên bằng chứng.}
            \EX{We need empirical proof for that claim.}
            \EX{Empirical studies validate theoretical models.}
            \CO{empirical data/evidence/research}
            \end{ExplainCard}

            \begin{ExplainCard}{not know the first thing (about)}[idiom][C1]
            \EN{to know absolutely nothing about a subject.}
            \SY{be clueless about; have no idea about}
            \VI{không biết chút gì về.}
            \EX{I don’t know the first thing about cars.}
            \EX{Many novices don’t know the first thing about coding practices.}
            \CO{not know the first thing about + noun}
            \end{ExplainCard}

            \begin{ExplainCard}{global village}[n][C1]
            \EN{the idea that the world is connected closely by modern communications.}
            \SY{networked world; interconnected world}
            \VI{“ngôi làng toàn cầu” (thế giới kết nối chặt chẽ).}
            \EX{Social media makes us a global village.}
            \EX{In a global village, crises spread rapidly across borders.}
            \CO{become/act as a global village}
            \end{ExplainCard}

            \begin{ExplainCard}{constraint}[n][C1]
            \EN{a limitation or restriction.}
            \SY{limitation; restraint; curb}
            \VI{ràng buộc; hạn chế.}
            \EX{Budget constraints delayed the plan.}
            \EX{Time constraints affect study design choices.}
            \CO{time/budget/legal constraints; operate under ~}
            \end{ExplainCard}

            \begin{ExplainCard}{inextricably}[adv][C2]
            \EN{in a way that is impossible to separate.}
            \SY{inseparably; tightly}
            \VI{một cách gắn bó/không thể tách rời.}
            \EX{Culture and language are inextricably linked.}
            \EX{Security is inextricably tied to privacy concerns.}
            \CO{be inextricably linked/tied/bound}
            \end{ExplainCard}

            \begin{ExplainCard}{stagnation}[n][C1]
            \EN{the state of not developing, growing, or changing.}
            \SY{standstill; slowdown}
            \VI{sự trì trệ; đình trệ.}
            \EX{Wage stagnation worries workers.}
            \EX{Long stagnation can sap innovation capacity.}
            \CO{economic/price stagnation; period of ~}
            \end{ExplainCard}

            \begin{ExplainCard}{(make) strides}[n/phr][C1]
            \EN{to make significant progress or improvements.}
            \SY{advance; progress; leap forward}
            \VI{tiến bộ, tạo bước tiến.}
            \EX{She’s making great strides in English.}
            \EX{Healthcare has made major strides in treatment outcomes.}
            \CO{make/achieve strides; giant strides}
            \end{ExplainCard}

            \begin{ExplainCard}{realm}[n][C1]
            \EN{an area of activity, interest, or knowledge.}
            \SY{domain; field; sphere}
            \VI{lĩnh vực; phạm vi.}
            \EX{Not my realm of expertise.}
            \EX{AI opens new realms of discovery.}
            \CO{the realm of science/politics; within/outside the realm}
            \end{ExplainCard}

            \begin{ExplainCard}{conquer}[v][C1]
            \EN{to gain control of or succeed in dealing with something.}
            \SY{overcome; master; defeat}
            \VI{chinh phục; vượt qua.}
            \EX{She conquered her fear of water.}
            \EX{Engineers seek to conquer long-distance space travel.}
            \CO{conquer fear/markets/challenges}
            \end{ExplainCard}

            \begin{ExplainCard}{monumental}[adj][C1]
            \EN{very big, important, or impressive.}
            \SY{epoch-making; momentous; immense}
            \VI{to lớn; mang tính cột mốc.}
            \EX{It was a monumental task.}
            \EX{A monumental discovery reshaped modern physics.}
            \CO{monumental effort/achievement/change}
            \end{ExplainCard}

            \begin{ExplainCard}{compromise}[v][C1]
            \EN{to balance competing interests by making concessions; to weaken by accepting standards lower than desired (sense 2).}
            \SY{(1) reconcile; accommodate \ (2) undermine}
            \VI{(1) dung hoà, điều hoà lợi ích; (2) làm suy giảm.}
            \EX{We must compromise to reach agreement.}
            \EX{Oversight ensures research isn’t compromised by conflicts.}
            \CO{compromise between A and B; compromise on sth}
            \end{ExplainCard}

            \begin{ExplainCard}{weapons of mass destruction}[n][C1]
            \EN{nuclear, chemical, or biological weapons capable of causing large-scale harm.}
            \SY{WMDs; strategic weapons}
            \VI{vũ khí hủy diệt hàng loạt.}
            \EX{Treaties aim to curb weapons of mass destruction.}
            \EX{Unauthorized access to WMDs poses existential risks.}
            \CO{ban/curb/spread of weapons of mass destruction}
            \end{ExplainCard}

            \begin{ExplainCard}{uncertainty}[n][B2]
            \EN{the state of being not known or not definite.}
            \SY{unpredictability; doubt; risk}
            \VI{sự bất định; không chắc chắn.}
            \EX{Market uncertainty makes investors cautious.}
            \EX{Uncertainty around outcomes complicates funding decisions.}
            \CO{reduce/face/manage uncertainty}
            \end{ExplainCard}

            \begin{ExplainCard}{layman}[n][C1]
            \EN{a person without professional or specialized knowledge in a particular subject.}
            \SY{nonexpert; non-specialist}
            \VI{người không chuyên; dân thường.}
            \EX{Explain it for the layman, please.}
            \EX{Layman-friendly summaries widen access to research.}
            \CO{in layman’s terms; a layman’s view}
            \end{ExplainCard}
        \end{VocabExplain}

    \begin{VocabHighlights}
        \VH{aversion}{(n) a strong feeling of not liking somebody/something}{(danh từ) sự ác cảm, ghét bỏ}
        \VH{flashy}{(adj) attracting attention by being bright, expensive, large, etc}{(tính từ) lấp lánh}
        \VH{lean towards}{(phr.v) to tend to prefer something, especially a particular opinion or interest}{(cụm động từ) có xu hướng thích gì}
        \VH{flaunt}{(v) to show something you are proud of to other people, in order to impress them}{(động từ) khoe khoang gì}
        \VH{keep a low profile}{(idiom) to avoid attracting attention to yourself}{(thành ngữ) tránh thu hút sự chú ý}
        \VH{stick to}{(phr.v) adhere to a commitment, belief, or rule}{(cụm động từ) tuân thủ theo}
        \VH{be occupied with}{(phrase) be busy with}{(cụm từ) bận}
        \VH{squeeze in}{(phr.v) to manage to do something or see someone in a short period of time or when you are very busy}{(cụm động từ) tranh thủ}
        \VH{put my feet up}{(idiom) relax}{(thành ngữ) thư giãn}
        \VH{got a kick out of}{(idiom) get pleasure from}{(thành ngữ) có được niềm vui từ}
        \VH{stay tuned}{(phrase) follow something}{(cụm từ) theo dõi}
        \VH{nurture}{(v) to help someone or something develop by encouraging that person or thing}{(động từ) nuôi dưỡng}
        \VH{it goes without saying}{(idiom) it is obvious that}{(thành ngữ) rõ ràng}
        \VH{compelling}{(adj) persuasive}{(tính từ) thuyết phục}
        \VH{a breeze}{(idiom) very easy}{(thành ngữ) rất dễ}
        \VH{be struck dumb}{(idiom) be too surprised}{(thành ngữ) bất ngờ đến nỗi đứng hình, không nói nên lời}
        \VH{dead tedious}{(idiom) very bored}{(thành ngữ) chán chết}
        \VH{intrigue}{(v) to fascinate, arouse the curiosity of or amuse}{(động từ) làm say mê}
        \VH{hit the sack}{(idiom) go to bed to sleep}{(thành ngữ) đi ngủ}
        \VH{sleep deprivation}{(phrase) sleep loss}{(cụm từ) sự mất ngủ}
        \VH{lose one’s cool}{(idiom) not to keep one’s cool}{(thành ngữ) mất bình tĩnh}
        \VH{permeate}{(v) to spread through something and be present in every part of it}{(động từ) len lỏi, xâm nhập vào}
        \VH{remedial}{(adj) relating to teaching that is intended to help people who have difficulties in reading or writing}{(tính từ) có tính phụ đạo}
        \VH{out of touch with something}{(phrase) not informed or not having the same ideas as most people about something, so that you make mistakes}{(cụm từ) lạc hậu, không bắt kịp cái gì}
        \VH{restrain}{(v) to control the actions or behaviour of someone by force, especially in order to stop them from doing something, or to limit the growth or force of something}{(động từ) ngăn cản, giới hạn sự phát triển của cái gì}
        \VH{inextricably}{(adv) in a way that is unable to be separated, released, or escaped from}{(trạng từ) gắn bó chặt chẽ}
        \VH{arm}{(v) to equip somebody with something}{(động từ) trang bị (kiến thức, vũ khí...)}
        \VH{empirical}{(adj) based on what is experienced or seen rather than on theory}{(tính từ) thuộc về kinh nghiệm, có kiểm chứng}
        \VH{do not know the first thing}{(phrase) you are emphasizing that you know absolutely nothing about something}{(cụm từ) không biết tí gì về một vấn đề nào đó}
        \VH{a global village}{(n) the world considered as a single community linked by telecommunications}{(danh từ) một thế giới thu nhỏ nhờ sự kết nối của các phương tiện truyền thông từ xa}
        \VH{constraint}{(n) something that controls what you do by keeping you within particular limits}{(danh từ) sự hạn chế, kìm hãm}
        \VH{stagnation}{(n) a situation in which something stays the same and does not grow and develop}{(danh từ) sự trì trệ}
        \VH{stride}{(n) an important positive development}{(danh từ) bước tiến quan trọng}
        \VH{realm}{(n) an area of interest or activity}{(danh từ) lĩnh vực}
        \VH{conquer}{(v) to take control or possession of foreign land, or a group of people, by force}{(động từ) chinh phục}
        \VH{monumental}{(adj) very big}{(tính từ) hùng vĩ, quan trọng}
        \VH{compromise}{(v) to accept that you will reduce your demands or change your opinion in order to reach an agreement with someone}{(động từ) hòa hoãn, điều hòa giữa những khác biệt}
        \VH{weapons of mass destruction}{(n) weapons, like nuclear bombs, that cause a lot of damage and kill many people}{(danh từ) vũ khí hủy diệt hàng loạt}
        \VH{uncertainty}{(n) a situation in which something is not known, or something that is not known or certain}{(danh từ) sự không chắc chắn}
        \VH{layman}{(n) someone who is not trained in or does not have a detailed knowledge of a particular subject}{(danh từ) một người không có kiến thức chuyên sâu về lĩnh vực nào đó}
    \end{VocabHighlights}
    \end{test}
\end{glossarymc}