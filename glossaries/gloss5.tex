\begin{glossarymc}[Cambridge 7]
    \begin{test}{TEST 1}
    \noindent
    \textbf{Part 1. Keeping in contact with people}
    \begin{qa}{How do you usually contact your friends? (Why?)}
    It depends on whom I contact. \textbf{In terms of} childhood, school or college friends, I contacted them \textbf{once in a blue moon} \textbf{via} online apps. The last time we called each other was to hold a reunion party at a friend’s wedding almost half a year ago. Since then, I have been \textbf{snowed under} with work as usual. \textbf{In regard to} my colleagues, I may contact them \textbf{once in a while} as I am a \textbf{family-oriented} person. To be frank, my family \textbf{is given precedence over} any colleagueship or friendship.
    \end{qa}

    \begin{qa}{Do you prefer to contact different people in different ways? (Why?)}
    Though I occasionally communicate with my friends, I’m interested in doing so in different ways. If it is something in emergency, I won’t hesitate to call them \textbf{at the drop of a hat}. In case it is just a trivial matter, I won’t bother my friends with calling directly but switch to using online apps such as Facebook Messenger, Viber, Zalo or Whatsapp instead. It is also \textbf{cost-efficient} to use these free apps.
    \end{qa}

    \begin{qa}{Do you find it easy to keep in contact with friends and family? (Why/Why not?)}
    It is \textbf{a piece of cake} to contact friends or family relatives via phone calls or OTT apps mentioned above. However, the content of the discussion is what matters most. In that sense, I find it pretty hard to \textbf{strike up a conversation} with friends and family. As the \textbf{breadwinner} of my family, I am constantly \textbf{slammed with} work and this results in my lack of time devoted to anything else.
    \end{qa}

    \begin{qa}{In your country, did people in the past keep in contact in the same ways as they do today? (Why/Why not?)}
    Of course not. In the past, being \textbf{in possession of} a mobile phone was \textbf{out of the question} due to limitations in technology at that time. However, with \textbf{the advent of} 4.0 era, increased access to technological devices has facilitated communication in various aspects. This results in more widespread communication in the forms of distant phone calls, chatting or even video-conferencing via free OTT apps. However, less face-to-face social interaction and more online communication are \textbf{two sides of the same coin}.
    \end{qa}
            \begin{VocabExplain}[Part 1]
                \begin{ExplainCard}{in terms of}[phrase][B2]
                \EN{with regard to; concerning a particular aspect.}
                \SY{regarding; concerning; with respect to}
                \VI{\textit{xét về/nhìn từ khía cạnh}.}
                \EX{\textbf{In terms of} speed, texting is fastest.}
                \EX{\textbf{In terms of} methodology, the study relies on longitudinal data.}
                \CO{in terms of + aspect; discuss/compare in terms of}
            \end{ExplainCard}

            \begin{ExplainCard}{once in a blue moon}[idiom][B2]
                \EN{very rarely; almost never.}
                \SY{rarely; scarcely; hardly ever}
                \VI{\textit{hiếm khi}, rất ít khi.}
                \EX{We meet \textbf{once in a blue moon}.}
                \EX{Such anomalies occur \textbf{once in a blue moon} in controlled trials.}
                \CO{happen/meet/occur once in a blue moon}
            \end{ExplainCard}

            \begin{ExplainCard}{via}[prep][B2]
                \EN{by way of; through the use of.}
                \SY{through; by means of}
                \VI{\textit{qua}, \textit{thông qua}.}
                \EX{Send the file \textbf{via} Messenger.}
                \EX{Data were collected \textbf{via} online questionnaires.}
                \CO{book/contact/pay via; send via + platform}
            \end{ExplainCard}

            \begin{ExplainCard}{snowed under}[idiom][B2]
                \EN{overwhelmed with too much work or too many tasks.}
                \SY{swamped; overloaded; inundated}
                \VI{\textit{ngập đầu} vì công việc.}
                \EX{I’m \textbf{snowed under} this week.}
                \EX{Clinics were \textbf{snowed under} during the flu season.}
                \CO{be/get snowed under (with work/emails)}
            \end{ExplainCard}

            \begin{ExplainCard}{in regard to}[phrase][B2]
                \EN{concerning; about.}
                \SY{regarding; with respect to; concerning}
                \VI{\textit{liên quan đến}/\textit{về}.}
                \EX{\textbf{In regard to} schedules, text me.}
                \EX{\textbf{In regard to} policy impacts, results were mixed.}
                \CO{in/with regard to + noun}
            \end{ExplainCard}

            \begin{ExplainCard}{once in a while}[phrase][B2]
                \EN{occasionally; from time to time.}
                \SY{now and then; occasionally; at times}
                \VI{\textit{thỉnh thoảng}.}
                \EX{We catch up \textbf{once in a while}.}
                \EX{Check backups \textbf{once in a while} to ensure integrity.}
                \CO{happen/meet/visit once in a while}
            \end{ExplainCard}

            \begin{ExplainCard}{family-oriented}[adj][B2]
                \EN{prioritizing family relationships and values.}
                \SY{family-centred; family-focused}
                \VI{\textit{đặt gia đình lên hàng đầu}.}
                \EX{She’s very \textbf{family-oriented}.}
                \EX{\textbf{Family-oriented} policies improve work–life balance indicators.}
                \CO{family-oriented culture/policies/person}
            \end{ExplainCard}

            \begin{ExplainCard}{give precedence over}[phrase][B2]
                \EN{to treat something as more important than something else.}
                \SY{prioritize; take priority over; outweigh}
                \VI{\textit{ưu tiên hơn}, đặt lên trước.}
                \EX{I \textbf{give precedence over} family time.}
                \EX{Patient safety must \textbf{take precedence over} throughput.}
                \CO{give/take precedence over; precedence of A over B}
            \end{ExplainCard}

            \begin{ExplainCard}{at the drop of a hat}[idiom][B2]
                \EN{immediately; without hesitation.}
                \SY{instantly; readily; on the spot}
                \VI{\textit{ngay lập tức}, không chần chừ.}
                \EX{Call me \textbf{at the drop of a hat}.}
                \EX{Volunteers responded \textbf{at the drop of a hat} during the drill.}
                \CO{do/act/respond at the drop of a hat}
            \end{ExplainCard}

            \begin{ExplainCard}{cost-efficient}[adj][B2]
                \EN{producing good results without costing a lot of money.}
                \SY{economical; cost-effective; efficient}
                \VI{\textit{tiết kiệm chi phí}, hiệu quả.}
                \EX{Using OTT apps is \textbf{cost-efficient}.}
                \EX{Open-source tools offer a \textbf{cost-efficient} alternative for startups.}
                \CO{cost-efficient solution/method/approach}
            \end{ExplainCard}

            \begin{ExplainCard}{a piece of cake}[idiom][B2]
                \EN{very easy to do.}
                \SY{a breeze; no-brainer; effortless}
                \VI{\textit{dễ ợt}.}
                \EX{Video calls are \textbf{a piece of cake}.}
                \EX{With automation, deployment becomes \textbf{a piece of cake}.}
                \CO{be/look a piece of cake}
            \end{ExplainCard}

            \begin{ExplainCard}{strike up a conversation}[phrase][C1]
                \EN{to begin a conversation with someone, often a stranger.}
                \SY{start; initiate; open a conversation}
                \VI{\textit{bắt chuyện}.}
                \EX{He \textbf{struck up a conversation} on the bus.}
                \EX{Facilitators \textbf{strike up conversations} to increase networking density at events.}
                \CO{strike up a conversation/friendship; easily/quickly strike up}
            \end{ExplainCard}

            \begin{ExplainCard}{breadwinner}[n][B1]
                \EN{the person who earns the main income for a family.}
                \SY{primary earner; main provider}
                \VI{\textit{trụ cột kinh tế} của gia đình.}
                \EX{She’s the \textbf{breadwinner} at home.}
                \EX{\textbf{Breadwinner} status correlates with time-pressure indicators.}
                \CO{family/household breadwinner; sole/primary breadwinner}
            \end{ExplainCard}

            \begin{ExplainCard}{slammed with (work)}[idiom][B2]
                \EN{extremely busy; burdened with many tasks.}
                \SY{swamped; snowed under; overloaded}
                \VI{\textit{bận ngập đầu} (việc).}
                \EX{I’m \textbf{slammed with} deadlines this month.}
                \EX{Support teams were \textbf{slammed with} tickets after launch.}
                \CO{be/get slammed with work/requests/emails}
            \end{ExplainCard}

            \begin{ExplainCard}{in possession of}[phrase][B2]
                \EN{having/owning something.}
                \SY{holding; owning; having}
                \VI{\textit{sở hữu}, có trong tay.}
                \EX{Few families were \textbf{in possession of} mobile phones then.}
                \EX{Applicants must be \textbf{in possession of} valid permits.}
                \CO{be/come into possession of; remain in possession of}
            \end{ExplainCard}

            \begin{ExplainCard}{out of the question}[idiom][C1]
                \EN{impossible or unacceptable.}
                \SY{impossible; not an option; ruled out}
                \VI{\textit{không thể}, loại trừ.}
                \EX{International calls were \textbf{out of the question}.}
                \EX{Without consent, data sharing is \textbf{out of the question}.}
                \CO{be out of the question; entirely/completely out of the question}
            \end{ExplainCard}

            \begin{ExplainCard}{the advent of}[phrase][B2]
                \EN{the arrival or beginning of a notable thing or event.}
                \SY{arrival; emergence; onset}
                \VI{\textit{sự ra đời/xuất hiện của}.}
                \EX{Communication changed with \textbf{the advent of} smartphones.}
                \EX{\textbf{The advent of} 5G enables ultra-low-latency applications.}
                \CO{the advent of technology/era/innovation}
            \end{ExplainCard}

            \begin{ExplainCard}{two sides of the same coin}[idiom][B2]
                \EN{two different aspects of the same situation.}
                \SY{flip side; closely linked aspects}
                \VI{\textit{hai mặt của cùng một vấn đề}.}
                \EX{Convenience and distraction are \textbf{two sides of the same coin}.}
                \EX{Efficiency and equity can be \textbf{two sides of the same coin} in policy trade-offs.}
                \CO{are/represent two sides of the same coin}
            \end{ExplainCard}
            \end{VocabExplain}

    \noindent
    \textbf{Part 2.}
    \begin{qa}{Describe a party that you enjoyed. You should say:}
    \begin{itemize}

    \item Whose party it was and what it was celebrating
    \item Where the party was held and who went to it
    \item What people did during the party
    \item and explain what you enjoyed about this party.
    \end{itemize}

    In my life, I have been to a couple of parties, but there is one \textbf{bash} that \textbf{comes flooding back} right now. It is the 20th birthday party of my company, Manulife. This title of the party is “Forever love with Manulife”. The party was thrown by the board of executives to express the gratitude to staff for their dedication to this company. It was organized at a four-star hotel, Lotus one year ago. The party was \textbf{packed to the rafters} with the number of attendees adding up to 1200. Most of them were \textbf{elite staff}, and more importantly, some senior leaders also showed up there. The party was very wonderful because of many exciting activities. The music was very \textbf{upbeat} with some massive hits of Latin American, so people couldn’t stop dancing. To be honest, some \textbf{couldn't get enough of} dancing, so I \textbf{danced the night away}. Another remarkable feature is that dishes were superb because they were cooked by a world-class chef. I could never \textbf{resist my temptation} from hot food, so I \textbf{ate like a horse}. But \textbf{the life and the soul of the party} was the MC. He was so hilarious, so we \textbf{had a whale of a time}!
    \end{qa}

        \begin{VocabExplain}[Part 2]
            \begin{ExplainCard}{bash}[n][B1]
                \EN{an informal, lively party or celebration.}
                \SY{party; shindig; blowout}
                \VI{\textit{bữa tiệc linh đình} (thân mật).}
                \EX{We threw a small birthday \textbf{bash} at home.}
                \EX{The department hosted an annual \textbf{bash} to celebrate research awards.}
                \CO{throw/host a bash; birthday/anniversary bash}
            \end{ExplainCard}

            \begin{ExplainCard}{come flooding back (to sb)}[idiom][B2]
                \EN{(of memories/feelings) to return suddenly and strongly.}
                \SY{come rushing back; resurface; well up}
                \VI{\textit{(kí ức/cảm xúc) ùa về}.}
                \EX{Hearing that song, the summer memories \textbf{came flooding back}.}
                \EX{For many participants, traumatic recollections \textbf{came flooding back} during interviews.}
                \CO{memories/emotions come flooding back; details come flooding back}
            \end{ExplainCard}

            \begin{ExplainCard}{packed to the rafters}[idiom][B2]
                \EN{extremely full; crowded with people.}
                \SY{jam-packed; crammed; heaving}
                \VI{\textit{đông nghịt}, chật kín người.}
                \EX{The hall was \textbf{packed to the rafters} by 8 p.m.}
                \EX{Stadiums were \textbf{packed to the rafters} for the final series.}
                \CO{venue/hall/stadium packed to the rafters}
            \end{ExplainCard}

            \begin{ExplainCard}{elite}[adj][B2]
                \EN{belonging to the best, most skilled, or most powerful group.}
                \SY{top-tier; high-calibre; select}
                \VI{\textit{tinh hoa}, ưu tú.}
                \EX{Only \textbf{elite} staff were invited to the gala.}
                \EX{\textbf{Elite} institutions often shape national policy networks.}
                \CO{elite staff/athletes/universities; the elite}
            \end{ExplainCard}

            \begin{ExplainCard}{upbeat}[adj][B2]
                \EN{(1) cheerful and optimistic; (2) (of music) lively and fast.}
                \SY{cheerful; buoyant; lively}
                \VI{(1) \textit{lạc quan}; (2) \textit{sôi động}.}
                \EX{(1) Everyone was in an \textbf{upbeat} mood at the party.}
                \EX{(2) The playlist stayed \textbf{upbeat}, increasing dance-floor activity in the study.}
                \CO{upbeat mood/tone; upbeat track/playlist}
            \end{ExplainCard}

            \begin{ExplainCard}{can't get enough of}[phrase][B2]
                \EN{to like something so much that you always want more of it.}
                \SY{be crazy about; be hooked on; crave}
                \VI{\textit{mê tít}/\textit{không thấy chán}.}
                \EX{Guests \textbf{can’t get enough of} the desserts.}
                \EX{Users \textbf{can’t get enough of} short-form videos, according to engagement metrics.}
                \CO{can't get enough of + noun/V-ing}
            \end{ExplainCard}

            \begin{ExplainCard}{dance the night away}[idiom][B2]
                \EN{to spend most of the night dancing.}
                \SY{keep dancing; stay on the floor all night}
                \VI{\textit{nhảy suốt đêm}.}
                \EX{We \textbf{danced the night away} at the wedding.}
                \EX{Festival attendees \textbf{danced the night away}, boosting local night-time economy.}
                \CO{dance the night away at/until dawn}
            \end{ExplainCard}

            \begin{ExplainCard}{resist (the) temptation (to do sth)}[phrase][B2]
                \EN{to stop yourself from doing something you want to do.}
                \SY{refrain from; fight off; withstand}
                \VI{\textit{cưỡng lại cám dỗ} (làm điều gì).}
                \EX{I couldn’t \textbf{resist the temptation} to try every dish.}
                \EX{Participants were asked to \textbf{resist temptation} during the impulse-control task.}
                \CO{resist the temptation to + V; resist temptation of}
            \end{ExplainCard}

            \begin{ExplainCard}{eat like a horse}[idiom][B2]
                \EN{to eat a lot.}
                \SY{have a huge appetite; wolf down food}
                \VI{\textit{ăn rất khoẻ}.}
                \EX{After the hike, I \textbf{ate like a horse}.}
                \EX{Growth spurts can make adolescents \textbf{eat like a horse}.}
                \CO{he/she eats like a horse; be eating like a horse}
            \end{ExplainCard}

            \begin{ExplainCard}{the life and soul of the party}[idiom][B2]
                \EN{a very lively, amusing person who keeps a party entertaining.}
                \SY{spark; live wire; crowd-pleaser}
                \VI{\textit{linh hồn của bữa tiệc}.}
                \EX{Our MC was \textbf{the life and soul of the party}.}
                \EX{Charismatic leaders often become \textbf{the life and soul of} networking events.}
                \CO{be/act as the life and soul of the party}
            \end{ExplainCard}

            \begin{ExplainCard}{have a whale of a time}[idiom][B2]
                \EN{to enjoy yourself very much.}
                \SY{have a blast; have great fun; have a ball}
                \VI{\textit{vui hết nấc}.}
                \EX{We \textbf{had a whale of a time} at the gala.}
                \EX{Attendees reported they \textbf{had a whale of a time}, reflected in high satisfaction scores.}
                \CO{have/are having a whale of a time}
            \end{ExplainCard}
        \end{VocabExplain}

    \noindent
    \textbf{Part 3.}
    \begin{qa}{What are the main reasons why people organise family parties in your country?}
    In my country, people have different reasons for \textbf{throwing a party} at home. But, regardless of causes, these special occasions often mark important milestones in people's life. People host a party \textbf{simply} to celebrate their birthday or \textbf{commemorate} their marriage anniversary. By contrast, university freshmen or seniors seem to hold family parties to celebrate course admission and completion, respectively.
    \end{qa}

    \begin{qa}{In some places people spend a lot of money on parties that celebrate special family events. Is this ever true in your country? Do you think this is a good trend or a bad trend?}
    In my country, some will go for \textbf{lavish} parties for \textbf{precious} moments while others will organize \textbf{cozy} family gatherings instead of going to \textbf{posh} restaurants. In general, this choice is justifiable if it is based on the importance of event and not go beyond the family's budget. As long as the funds dedicated to holding a party is not beyond the \textbf{family's means} and does not affect anyone, it is acceptable to do such a thing of \textbf{their own volition}. In my opinion, the attendance of family members is more important than any financial factors.
    \end{qa}

    \begin{qa}{Are there many differences between family parties and parties given by friends? Why do you think this is?}
    Basically, the parties held by friends are usually more \textbf{gratifying} and \textbf{comfy} as we can \textbf{dine and wine} with those we \textbf{get on well with}. By contrast, people can sense \textbf{solemn ambience} in events organized by their parents I guess. Because we often feel relaxed amid our buddies, we need to remember our \textbf{pecking order} in family parties. Exchanging vulgar jokes is common among friends but saying such things to the elderly in the family is considered a disgrace.
    \end{qa}

    \begin{qa}{What kinds of national celebration do you have in your country?}
    To begin with, there are \textbf{a plethora of} national festivities, most of which are landmarks in my country. \textbf{Liberation day}, April 30th, \textbf{signalled} the end of the American War and the flying start of north-south \textbf{reunification}. National Day, September 2nd, was the moment when the country's \textbf{declaration} of independence was issued, and of course the most important national holiday.
    \end{qa}

    \begin{qa}{Who tends to enjoy national celebrations more young people or old people? Why?}
    It is pretty hard to measure the \textbf{eager anticipation} of young and old people before any national events. In my opinion, the elderly would have a sense of \textbf{nostalgia} as each celebration reminds them of the old days. As a young person, I was born in a peace era so luckily, I have never experienced the horrors of war. However, I invariably take pride in any national events because we could not enjoy peacefulness today without the sacrifices of forefathers whom I am always \textbf{indebted to}.
    \end{qa}

    \begin{qa}{Why do you think some people think that national celebrations are a waste of government money? Would you agree or disagree with this view? Why?}
    I would \textbf{refute} this argument. The first reason is that the main purpose of national celebrations is to \textbf{reminisce} about the past, and to educate young generation on the enduring \textbf{legacy} such as patriotism. Moreover, if we abandoned national \textbf{rituals}, national identity would fade away and inhabitants would have no clue about their \textbf{ethnic origins} and the nation's proud history.
    \end{qa}

        \begin{VocabExplain}[Part 3]
            \begin{ExplainCard}{throw a party}[phrase][B2]
                \EN{to host or organize a social celebration or gathering.}
                \SY{host; hold; put on}
                \VI{\textit{tổ chức một bữa tiệc}.}
                \EX{We’ll \textbf{throw a party} for grandma’s birthday.}
                \EX{Companies often \textbf{throw parties} to strengthen workplace cohesion.}
                \CO{throw/host/hold a party; anniversary/birthday party}
            \end{ExplainCard}

            \begin{ExplainCard}{simply}[adv][B2]
                \EN{only; merely; used to emphasize an uncomplicated reason or action.}
                \SY{merely; just; purely}
                \VI{\textit{đơn giản là}; chỉ.}
                \EX{They \textbf{simply} wanted to celebrate together.}
                \EX{Some outcomes arise \textbf{simply} from sampling variation.}
                \CO{simply because/put; simply want/ask}
            \end{ExplainCard}

            \begin{ExplainCard}{commemorate}[v][B2]
                \EN{to do something to remember and honor a person or event.}
                \SY{honor; mark; observe}
                \VI{\textit{tưởng niệm}/kỉ niệm.}
                \EX{We \textbf{commemorate} our grandparents each spring.}
                \EX{Monuments \textbf{commemorate} pivotal moments in national history.}
                \CO{commemorate an anniversary/event/victory; commemorative ceremony}
            \end{ExplainCard}

            \begin{ExplainCard}{lavish}[adj][B2]
                \EN{rich, elaborate, or luxurious; involving a lot of expense.}
                \SY{opulent; extravagant; sumptuous}
                \VI{\textit{xa hoa}, hoành tráng.}
                \EX{They threw a \textbf{lavish} wedding at a resort.}
                \EX{\textbf{Lavish} spending on ceremonies can strain household finances.}
                \CO{lavish party/banquet/gift; lavish spending}
            \end{ExplainCard}

            \begin{ExplainCard}{precious (moment)}[adj][B2]
                \EN{of great value and to be treasured; emotionally important.}
                \SY{cherished; treasured; valued}
                \VI{\textit{quý giá}, đáng trân trọng.}
                \EX{Photos help us keep \textbf{precious moments}.}
                \EX{Rituals preserve \textbf{precious} cultural memory across generations.}
                \CO{precious memory/time/gift; preserve/treasure precious moments}
            \end{ExplainCard}

            \begin{ExplainCard}{cozy}[adj][B2]
                \EN{comfortable and warm; giving a feeling of intimacy.}
                \SY{snug; homely; intimate}
                \VI{\textit{ấm cúng}.}
                \EX{We prefer a \textbf{cozy} family dinner at home.}
                \EX{\textbf{Cozy} settings can foster more candid discussion.}
                \CO{cozy gathering/room/atmosphere}
            \end{ExplainCard}

            \begin{ExplainCard}{posh}[adj][B2]
                \EN{elegant and expensive in a way associated with the upper class.}
                \SY{luxurious; upscale; classy}
                \VI{\textit{sang trọng}; thượng lưu.}
                \EX{They booked a \textbf{posh} restaurant downtown.}
                \EX{Brand activations at \textbf{posh} venues target premium audiences.}
                \CO{posh hotel/venue/area; sound posh}
            \end{ExplainCard}

            \begin{ExplainCard}{(one's) means}[n][B2]
                \EN{the money and financial resources available to a person or family.}
                \SY{resources; finances; budget}
                \VI{\textit{khả năng tài chính}.}
                \EX{Keep celebrations \textbf{within the family’s means}.}
                \EX{Policies must be affordable \textbf{within households’ means}.}
                \CO{within/beyond one’s means; limited means}
            \end{ExplainCard}

            \begin{ExplainCard}{of one’s own volition}[phrase][C1]
                \EN{voluntarily and by one’s free choice, not by force or obligation.}
                \SY{willingly; freely; voluntarily}
                \VI{\textit{tự nguyện}, theo ý mình.}
                \EX{She donated \textbf{of her own volition}.}
                \EX{Participants withdrew \textbf{of their own volition} without penalty.}
                \CO{act/leave/participate of one’s own volition}
            \end{ExplainCard}

            \begin{ExplainCard}{gratifying}[adj][C1]
                \EN{giving pleasure or satisfaction.}
                \SY{rewarding; satisfying; pleasing}
                \VI{\textit{đáng thỏa mãn}, làm vui lòng.}
                \EX{Cooking for friends is \textbf{gratifying}.}
                \EX{Feedback indicated \textbf{gratifying} gains in learner autonomy.}
                \CO{find it gratifying; a gratifying result/experience}
            \end{ExplainCard}

            \begin{ExplainCard}{comfy}[adj][B2]
                \EN{informal for comfortable.}
                \SY{comfortable; snug; cozy}
                \VI{\textit{thoải mái}.}
                \EX{We chose a \textbf{comfy} café to chat.}
                \EX{\textbf{Comfy} seating improved audience dwell time.}
                \CO{comfy chair/sofa/spot; feel comfy}
            \end{ExplainCard}

            \begin{ExplainCard}{dine and wine}[phrase][B2]
                \EN{to eat and drink well, usually in a social or celebratory setting.}
                \SY{feast; eat and drink; wine and dine}
                \VI{\textit{ăn uống linh đình}.}
                \EX{We \textbf{dined and wined} with old friends.}
                \EX{Firms \textbf{dine and wine} clients during product launches.}
                \CO{dine and wine with guests/clients; a dining-and-wining culture}
            \end{ExplainCard}

            \begin{ExplainCard}{get on well with}[phr.v][B2]
                \EN{to have a friendly and harmonious relationship with someone.}
                \SY{get along with; hit it off with}
                \VI{\textit{hoà hợp}, hợp tính.}
                \EX{I \textbf{get on well with} my cousins.}
                \EX{Teams that \textbf{get on well} report higher collaboration scores.}
                \CO{get on well with sb; get along famously}
            \end{ExplainCard}

            \begin{ExplainCard}{solemn ambience}[collocation][C1]
                \EN{a serious, dignified atmosphere appropriate to formal occasions.}
                \SY{grave atmosphere; dignified mood}
                \VI{\textit{không khí trang nghiêm}.}
                \EX{The ceremony had a \textbf{solemn ambience}.}
                \EX{Lighting and music design can create a \textbf{solemn ambience} for commemorations.}
                \CO{solemn ambience/atmosphere/ceremony}
            \end{ExplainCard}

            \begin{ExplainCard}{pecking order}[n][B2]
                \EN{the hierarchy of status or authority within a group.}
                \SY{hierarchy; ranking; order of precedence}
                \VI{\textit{thứ bậc}, tôn ti trật tự.}
                \EX{At family meals we respect the \textbf{pecking order}.}
                \EX{Informal \textbf{pecking orders} influence decision rights in teams.}
                \CO{respect/establish the pecking order; high in the pecking order}
            \end{ExplainCard}

            \begin{ExplainCard}{a plethora of}[phrase][B2]
                \EN{a very large number or amount of something.}
                \SY{a multitude of; an abundance of; myriad}
                \VI{\textit{vô số}, rất nhiều.}
                \EX{The city hosts \textbf{a plethora of} festivals.}
                \EX{\textbf{A plethora of} studies examine holiday spending patterns.}
                \CO{a plethora of options/events/studies}
            \end{ExplainCard}

            \begin{ExplainCard}{Liberation Day}[n][C1]
                \EN{a national holiday marking a country’s liberation (in Việt Nam, April 30th).}
                \SY{Victory Day; Independence Day (contextual)}
                \VI{\textit{Ngày Giải phóng} (30/4).}
                \EX{Parades celebrate \textbf{Liberation Day}.}
                \EX{Tourism peaks around \textbf{Liberation Day} due to extended breaks.}
                \CO{celebrate/mark Liberation Day; Liberation Day parade}
            \end{ExplainCard}

            \begin{ExplainCard}{signal}[v][B2]
                \EN{to indicate or mark the occurrence or beginning of something.}
                \SY{mark; herald; denote}
                \VI{\textit{báo hiệu}, đánh dấu.}
                \EX{Fireworks \textbf{signalled} the start of the show.}
                \EX{The treaty \textbf{signalled} a shift in regional alignment.}
                \CO{signal a shift/start/end; signal intent}
            \end{ExplainCard}

            \begin{ExplainCard}{reunification}[n][C1]
                \EN{the process of uniting parts that were divided into a single whole.}
                \SY{unification; reunion; integration}
                \VI{\textit{tái thống nhất}.}
                \EX{Stories about national \textbf{reunification} fill museums.}
                \EX{\textbf{Reunification} narratives shape collective memory in post-conflict studies.}
                \CO{national/political reunification; path to reunification}
            \end{ExplainCard}

            \begin{ExplainCard}{declaration (of independence)}[n][B2]
                \EN{a formal public statement, especially announcing a nation’s status or intent.}
                \SY{proclamation; announcement}
                \VI{\textit{tuyên ngôn}/\textit{tuyên bố} (độc lập).}
                \EX{The \textbf{declaration of independence} is taught in schools.}
                \EX{\textbf{Declarations} function as foundational texts in state-building.}
                \CO{issue/sign/read a declaration; a formal declaration}
            \end{ExplainCard}

            \begin{ExplainCard}{eager anticipation}[collocation][C1]
                \EN{keen excitement while waiting for something that is expected.}
                \SY{keen expectancy; excitement; suspense}
                \VI{\textit{sự háo hức mong chờ}.}
                \EX{Children watched with \textbf{eager anticipation}.}
                \EX{\textbf{Anticipation} peaks before national events, according to sentiment analyses.}
                \CO{with eager anticipation; build/heighten anticipation}
            \end{ExplainCard}

            \begin{ExplainCard}{nostalgia}[n][C1]
                \EN{a sentimental longing for the past.}
                \SY{longing; wistfulness; homesickness}
                \VI{\textit{nỗi hoài niệm}.}
                \EX{Old songs filled her with \textbf{nostalgia}.}
                \EX{\textbf{Nostalgia} can strengthen identity in diaspora communities.}
                \CO{feel/evoke nostalgia; nostalgic for}
            \end{ExplainCard}

            \begin{ExplainCard}{indebted to (sb)}[adj/phrase][B2]
                \EN{grateful because someone has helped you; owing gratitude.}
                \SY{grateful to; obliged to; beholden to}
                \VI{\textit{mang ơn}, chịu ơn.}
                \EX{I’m deeply \textbf{indebted to} my teachers.}
                \EX{Communities remain \textbf{indebted to} veterans for their service.}
                \CO{be/feel indebted to sb for sth; remain indebted}
            \end{ExplainCard}

            \begin{ExplainCard}{refute}[v][B2]
                \EN{to prove that a statement or opinion is wrong or false.}
                \SY{disprove; rebut; counter}
                \VI{\textit{bác bỏ}, phản bác.}
                \EX{She \textbf{refuted} the rumor with evidence.}
                \EX{The study \textbf{refutes} claims of a causal link.}
                \CO{refute an argument/claim/allegation}
            \end{ExplainCard}

            \begin{ExplainCard}{reminisce (about)}[v][B2]
                \EN{to recall and talk about past experiences with pleasure.}
                \SY{look back; recall; recollect}
                \VI{\textit{hồi tưởng}, kể lại.}
                \EX{They \textbf{reminisced about} school days.}
                \EX{Participants \textbf{reminisced} to co-construct community histories.}
                \CO{reminisce about old times/memories}
            \end{ExplainCard}

            \begin{ExplainCard}{legacy}[n][C1]
                \EN{something handed down from the past, such as traditions or achievements.}
                \SY{heritage; inheritance; bequest}
                \VI{\textit{di sản}.}
                \EX{Patriotism is part of our national \textbf{legacy}.}
                \EX{\textbf{Legacy} effects of policy persist long after enactment.}
                \CO{cultural/historical legacy; leave a legacy of}
            \end{ExplainCard}

            \begin{ExplainCard}{ritual}[n][B2]
                \EN{a series of actions done in a fixed way, especially as part of a ceremony.}
                \SY{ceremony; rite; observance}
                \VI{\textit{nghi lễ}, nghi thức.}
                \EX{Lighting incense is a family \textbf{ritual}.}
                \EX{National \textbf{rituals} reinforce collective identity.}
                \CO{daily/family/national rituals; perform/observe a ritual}
            \end{ExplainCard}

            \begin{ExplainCard}{ethnic origins}[n][B2]
                \EN{a person’s ancestral ethnic background.}
                \SY{ancestry; heritage; roots}
                \VI{\textit{nguồn gốc dân tộc}.}
                \EX{He researched his \textbf{ethnic origins}.}
                \EX{Census questions on \textbf{ethnic origin} inform diversity policies.}
                \CO{trace/declare ethnic origin(s); mixed ethnic origins}
            \end{ExplainCard}
        \end{VocabExplain}

    \begin{VocabHighlights}
        \VH{in terms of}{(phrase) with regard to the particular aspect or subject specified}{(cụm từ) khi xét về}
        \VH{once in a blue moon}{(idiom) very rarely}{(thành ngữ) cực kì hiếm khi}
        \VH{to be snowed under with}{(idiom) to have more things, especially work, than you feel able to deal with}{(thành ngữ) bận rộn, vùi đầu vào}
        \VH{in regard to}{(phrase) in relation to someone or something}{(cụm từ) khi xét đến}
        \VH{once in a while}{(idiom) occasionally; sometimes}{(thành ngữ) hiếm khi}
        \VH{family-oriented}{(adj) a principle that puts family at the center and focuses on their values, strengths and relationships}{(tính từ) luôn hướng về gia đình}
        \VH{to be given precedence over}{(phrase) to be prioritized}{(cụm từ) được ưu tiên so với}
        \VH{at the drop of a hat}{(idiom) instantly}{(thành ngữ) ngay lập tức}
        \VH{cost-efficient}{(adj) giving the best possible profit or benefits in comparison with the money that is spent}{(tính từ) tiết kiệm}
        \VH{a piece of cake}{(idiom) a very easy task or accomplishment}{(thành ngữ) việc dễ dàng}
        \VH{in that sense}{(phrase) that means}{(cụm từ) điều này có nghĩa là}
        \VH{to strike up a conversation with}{(phrase) to start talking to}{(cụm từ) bắt đầu nói chuyện với}
        \VH{breadwinner}{(n) a person who supports their family with the money they earn}{(danh từ) người lao động chính trong nhà}
        \VH{to be slammed with}{(p2) be overwhelmed with}{(phần từ 2) bận rộn làm gì}
        \VH{to be in possession of}{(phrase) to own something}{(cụm từ) sở hữu cái gì}
        \VH{out of the question}{(idiom) not possible; having no chance}{(thành ngữ) không thể}
        \VH{with the advent of}{(phrase) with the coming of an important event, person, invention, etc}{(cụm từ) với sự hiện diện, xuất hiện của}
        \VH{two sides of the same coin}{(idiom) very closely related although they seem different}{(thành ngữ) 2 mặt của 1 vấn đề}
        \VH{bash}{(n) a party}{(danh từ) bữa tiệc}
        \VH{to come flooding back}{(idiom) if memories or feelings flood back, you suddenly remember them very clearly}{(thành ngữ) quay lại, ùa về}
        \VH{to be packed to the rafters}{(idiom) be full of}{(thành ngữ) đông}
        \VH{upbeat}{(adj) happy and positive because you are confident that you will get what you want}{(tính từ) bốc, mạnh mẽ}
        \VH{couldn’t get enough of}{(idiom) to like something very much and want a lot of it}{(thành ngữ) thích}
        \VH{to dance the night away}{(idiom) dance all night long}{(thành ngữ) nhảy xuyên đêm}
        \VH{can’t resist my temptation}{(phrase) can’t adjust yourself because of your desire of something}{(cụm từ) không cưỡng nổi}
        \VH{to eat like a horse}{(idiom) to always eat a lot of food}{(thành ngữ) ăn rất nhiều}
        \VH{the life and the soul of the party}{(idiom) someone who is energetic and funny and at the centre of activity during social occasions}{(thành ngữ) linh hồn của bữa tiệc}
        \VH{to have a whale of a time}{(idiom) have a lot of fun}{(thành ngữ) có nhiều niềm vui}
        \VH{to throw a party}{(phrase) to have a party}{(cụm từ) tổ chức tiệc}
        \VH{to commemorate}{(v) to remind people of an important person or event from the past with a special action or object; to exist to remind people of a person or an event from the past}{(động từ) tưởng nhớ, kỷ niệm}
        \VH{lavish}{(adj) large in amount, or impressive, and usually costing a lot of money}{(tính từ) xa xỉ, tốn kém}
        \VH{precious}{(adj) valuable or important and not to be wasted}{(tính từ) quý báu}
        \VH{posh}{(adv) in a way that is typical of or used by people who belong to a high social class}{(tính từ) sang trọng}
        \VH{of one’s own volition}{(phrase) at one’s disposal}{(cụm từ) tùy ý ai}
        \VH{gratifying}{(adj) giving pleasure or satisfaction}{(tính từ) vui vẻ, thoải mái}
        \VH{comfy}{(adj) comfortable}{(tính từ) dễ chịu}
        \VH{solemn ambience}{(phrase) very serious atmosphere}{(không khí) nghiêm trang}
        \VH{to dine and wine}{(idiom) to go to restaurants, etc. and enjoy good food and drink; to entertain somebody by buying them good food and drink}{(thành ngữ) chiêu đãi, thiết đãi}
        \VH{to get on (well) with somebody}{(phrase) feel comfortable with somebody}{(cụm từ) hòa hợp với ai đó}
        \VH{solemn}{(adj) formal and dignified}{(tính từ) nghiêm túc, nghiêm trang}
        \VH{ambience}{(n) the character and atmosphere of a place}{(danh từ) bầu không khí (của một sự kiện, một hoạt động...)}
        \VH{pecking order}{(phrase) a hierarchy of status seen among members of a group of people or animals}{(cụm từ) tôn ti trật tự}
        \VH{a plethora of}{(phrase) a lot of}{(cụm từ) nhiều}
        \VH{liberation day}{(noun) a day, often a public holiday, that marks the liberation of a place, similar to an independence day}{(danh từ) ngày giải phóng, thống nhất đất nước}
        \VH{to signal}{(v) to be a sign that something exists or is likely to happen}{(động từ) báo hiệu trước}
        \VH{reunification}{(n) the act of joining together two or more regions or parts of a country so that they form a single political unit again}{(danh từ) sự thống nhất}
        \VH{declaration}{(n) an official or formal statement, especially about the plans of a government or an organization; the act of making such a statement}{(danh từ) bản tuyên bố; tuyên ngôn}
        \VH{eager anticipation}{(n) a feeling of excitement about something that is going to happen}{(danh từ) sự háo hức}
        \VH{nostalgia}{(n) a feeling of pleasure and also slight sadness when you think about things that happened in the past}{(danh từ) sự hoài niệm}
        \VH{indebted}{(adj) owing gratitude for a service or favor}{(tính từ) cảm thấy mắc nợ ai, cái gì}
        \VH{to refute}{(v) to say that something is not true or fair}{(động từ) không đồng ý, bác bỏ}
        \VH{to reminisce}{(v) to think, talk or write about a happy time in your past}{(động từ) hồi tưởng}
        \VH{enduring legacy}{(n) something that is a part of your history or that remains from an earlier time}{(danh từ) di sản lâu đời}
        \VH{rituals}{(n) a series of actions that are always performed in the same way, especially as part of a religious ceremony}{(danh từ) nghi lễ, nghi thức}
        \VH{ethnic origins}{(phrase) connected with or belonging to a nation, race or people that shares a cultural tradition}{(cụm từ) nguồn gốc dân tộc}
    \end{VocabHighlights}
    \end{test}

    \begin{test}{TEST 2}
    \noindent
    \textbf{Part 1. Laughing}
    \begin{qa}{What kinds of thing make you laugh?}
    Regarding laughter, I am not the kind of person who can \textbf{utter} a laugh easily. I can only laugh in two cases. Firstly, it is when something \textbf{ridiculous} or awkward happens. For example, in an episode of "Just for laughs" series, I saw a football player trying to \textbf{show off} his techniques but then, he was \textbf{nutmegged} and fell off the ground unexpectedly. Secondly, I can \textbf{die laughing} thanks to \textbf{witty} remarks. For example, "Gap nhau cuoi nam" or "Year-end meeting", one of the most expected TV programmes annually, can bring \textbf{hysterical} laughs to me owing to the actors' using humorous lines in a creative way.
    \end{qa}

    \begin{qa}{Do you like making other people laugh? (Why/Why not?)}
    Yes I do. \textbf{Having others in stitches} is often associated with inspiring feelings of happiness. However, it depends on each context I am involved in. In the case of \textbf{dismal} atmosphere like taking part in the funeral, it is awkward and disrespectable to \textbf{burst out laughing}. \textbf{On the flip side}, in a friendly, cozy and comfortable atmosphere, initiating a conversation with a few jokes is acceptable.
    \end{qa}

    \begin{qa}{Do you think it's important for people to laugh? (Why/Why not?)}
    Yes, definitely. Laughter is \textbf{clinically proven} to exert profound impacts on our \textbf{well-being}, health and emotions. In particularly, it releases serotonin, a \textbf{determinant} of feelings of satisfaction and happiness within each body. As I said earlier, it is much more important to laugh in the right context. Laughing in an intense or unpleasant situation can often lead to embarrassment, though.
    \end{qa}

    \begin{qa}{Is laughing the same as feeling happy, do you think? (Why/Why not?)}
    No, it is not. Even though the majority of laughter are connected with happiness, others may still \textbf{laugh off} their problems they encounter in daily lives. In their mind, hiding their sorrow via laughter can make their friends and relatives worry less about them.
    \end{qa}
        \begin{VocabExplain}[Part 1]
            \begin{ExplainCard}{utter}[v][B2]
                \EN{to say or make a sound with the voice; to produce (a laugh, cry, word).}
                \SY{voice; articulate; emit}
                \VI{\textit{thốt ra}, phát ra (tiếng cười/lời nói).}
                \EX{He could barely \textbf{utter} a laugh after the prank.}
                \EX{Witnesses \textbf{uttered} brief exclamations before the alarm was raised.}
                \CO{utter a laugh/cry/word; utter a sound}
            \end{ExplainCard}

            \begin{ExplainCard}{ridiculous}[adj][B2]
                \EN{very silly or unreasonable; deserving to be laughed at.}
                \SY{absurd; preposterous; laughable}
                \VI{\textit{lố bịch}, nực cười.}
                \EX{That hat looks \textbf{ridiculous} on me.}
                \EX{The claim was dismissed as \textbf{ridiculous} by the review panel.}
                \CO{utterly/completely ridiculous; a ridiculous idea/price}
            \end{ExplainCard}

            \begin{ExplainCard}{show off}[phr.v][B2]
                \EN{to behave in a way intended to attract attention or admiration.}
                \SY{flaunt; parade; boast}
                \VI{\textit{khoe khoang}, ra vẻ.}
                \EX{He kept \textbf{showing off} his new skills.}
                \EX{Users tend to \textbf{show off} achievements on social platforms.}
                \CO{show off skills/wealth/physique; blatant showing-off}
            \end{ExplainCard}

            \begin{ExplainCard}{nutmeg}[v][B2]
                \EN{in football, to kick the ball between an opponent’s legs.}
                \SY{meg (informal)}
                \VI{\textit{xỏ háng} (trong bóng đá).}
                \EX{She \textbf{nutmegged} the defender and scored.}
                \EX{The striker was repeatedly \textbf{nutmegged} during drills, drawing laughter.}
                \CO{nutmeg a defender/opponent; cheeky nutmeg}
            \end{ExplainCard}

            \begin{ExplainCard}{die laughing}[idiom][B2]
                \EN{to laugh a great deal; to find something extremely funny.}
                \SY{crack up; be in stitches; howl with laughter}
                \VI{\textit{cười muốn chết đi được}.}
                \EX{We \textbf{died laughing} at the sketch.}
                \EX{The audience \textbf{died laughing}, as reflected in prolonged applause.}
                \CO{make sb die laughing; almost die laughing}
            \end{ExplainCard}

            \begin{ExplainCard}{witty}[adj][B2]
                \EN{clever and funny in the way you say things.}
                \SY{quick-witted; droll; humorous}
                \VI{\textit{hóm hỉnh}, dí dỏm.}
                \EX{Her \textbf{witty} remarks broke the ice.}
                \EX{\textbf{Witty} one-liners improved the speaker’s engagement metrics.}
                \CO{witty remark/retort/banter; dry/quick wit}
            \end{ExplainCard}

            \begin{ExplainCard}{hysterical}[adj][B2]
                \EN{(1) extremely funny; causing uncontrollable laughter. (2) very upset and unable to control feelings.}
                \SY{(1) hilarious \quad (2) frantic; overwrought}
                \VI{(1) \textit{cực kỳ buồn cười}; (2) \textit{quá khích, kích động}.}
                \EX{(1) The parody was \textbf{hysterical}.}
                \EX{(2) The crowd grew \textbf{hysterical} after the shock announcement.}
                \CO{hysterical laughter/response; become/be hysterical}
            \end{ExplainCard}

            \begin{ExplainCard}{have/put others in stitches}[idiom][B2]
                \EN{to make people laugh a lot.}
                \SY{crack up; split sides; slay (informal)}
                \VI{\textit{làm ai cười nghiêng ngả}.}
                \EX{His impressions \textbf{had us in stitches}.}
                \EX{The presenter \textbf{put the audience in stitches}, boosting session energy.}
                \CO{leave sb in stitches; have the room in stitches}
            \end{ExplainCard}

            \begin{ExplainCard}{dismal}[adj][B2]
                \EN{very sad, gloomy, or poor in quality.}
                \SY{gloomy; bleak; depressing}
                \VI{\textit{ảm đạm}, tệ hại.}
                \EX{The weather was \textbf{dismal} all week.}
                \EX{Quarterly results were \textbf{dismal}, missing every target.}
                \CO{dismal mood/performance/prospects}
            \end{ExplainCard}

            \begin{ExplainCard}{burst out laughing}[phr.v][B2]
                \EN{to suddenly start laughing loudly.}
                \SY{crack up; explode with laughter}
                \VI{\textit{phì cười}, bật cười.}
                \EX{She \textbf{burst out laughing} at the meme.}
                \EX{Participants often \textbf{burst out laughing} when the punchline landed.}
                \CO{suddenly/instantly burst out laughing}
            \end{ExplainCard}

            \begin{ExplainCard}{on the flip side}[idiom][B2]
                \EN{expressing an alternative or opposite point.}
                \SY{conversely; by contrast; on the other hand}
                \VI{\textit{ngược lại}/mặt khác.}
                \EX{I like puns; \textbf{on the flip side}, some find them cringy.}
                \EX{\textbf{On the flip side}, tighter rules may stifle creativity.}
                \CO{on the flip side of the debate/issue}
            \end{ExplainCard}

            \begin{ExplainCard}{clinically proven}[collocation][B2]
                \EN{shown to be true by medical or scientific trials.}
                \SY{evidence-based; validated; demonstrated}
                \VI{\textit{được chứng minh lâm sàng}.}
                \EX{The method is \textbf{clinically proven} to reduce stress.}
                \EX{Serotonin’s mood effects are \textbf{clinically proven} across studies.}
                \CO{clinically proven benefits/effects/treatments}
            \end{ExplainCard}

            \begin{ExplainCard}{well-being}[n][B2]
                \EN{the state of being healthy, happy, and comfortable.}
                \SY{welfare; wellness; health}
                \VI{\textit{phúc lợi}/sức khoẻ tinh thần và thể chất.}
                \EX{Laughter boosts \textbf{well-being}.}
                \EX{Workplace \textbf{well-being} programs improved retention metrics.}
                \CO{mental/physical well-being; promote/undermine well-being}
            \end{ExplainCard}

            \begin{ExplainCard}{determinant}[n][C1]
                \EN{a factor that strongly influences an outcome.}
                \SY{driver; factor; predictor}
                \VI{\textit{yếu tố quyết định}.}
                \EX{Sleep is a key \textbf{determinant} of mood.}
                \EX{Income is a major \textbf{determinant} of health disparities.}
                \CO{determinant of X; key/primary determinant}
            \end{ExplainCard}

            \begin{ExplainCard}{laugh off}[phr.v][B2]
                \EN{to try to make people think that something is not serious by laughing about it.}
                \SY{brush off; shrug off; play down}
                \VI{\textit{cười cho qua}, xem nhẹ.}
                \EX{He \textbf{laughed off} the mistake.}
                \EX{Officials attempted to \textbf{laugh off} the criticism as mere banter.}
                \CO{laugh off criticism/mistakes/problems}
            \end{ExplainCard}
        \end{VocabExplain}

    \noindent
    \textbf{Part 2.}
    
    \begin{qa}{Describe an idea you had for improving something at work or college. You should say:}
    \begin{itemize}
    \item When and where you had your idea\
    \item What your idea was\
    \item Who you told about your idea\
    \item and explain why you thought your idea would make an improvement.
    \end{itemize}

    In all sincerity, I am working as an IELTS trainer at the Institution of American Education. It is built by overseas students and its mission is to open the doors for \textbf{academically gifted students} to gain access to quality education through scholarships. Although some of my students were hard-working, their speaking skills were still poor. My personal view is that they should be given more opportunities to \textbf{hone} their English skills by having more extracurricular activities. The idea \textbf{crossed my mind} when one of my students \textbf{lamented} that the results of speaking skills were much lower than those of the remaining skills. It came as a surprise to me because most of speaking lessons were led by qualified foreigner teachers, so I did not think that their poor outcomes \textbf{stemmed from} instructors. Luckily, I had an opportunity to observe a class, and the root cause was shyness among students. I also discussed the issue with a \textbf{handful} of colleagues and they advised me to report the case to the academic manager. Luckily, he agreed to my suggestion right away. The reason why my idea would work is that extracurricular activities could create favorable conditions for students to \textbf{bridge their gap}. They would talk and play with each other, which helped to \textbf{break their invisible barriers}. I should not forget to mention that the nature of speaking is sharing opinions. If we \textbf{foster a dynamic} environment, they will feel open to speak out their thoughts. And the truth of matter is that students enhanced their speaking skills thanks to these activities.
    \end{qa}

        \begin{VocabExplain}[Part 2]
            \begin{ExplainCard}{academically gifted students}[n phrase][B2]
                \EN{students with exceptionally high academic ability or potential.}
                \SY{high-achieving students; talented pupils; advanced learners}
                \VI{\textit{học sinh giỏi/ có năng khiếu học thuật}.}
                \EX{Our school offers scholarships for \textbf{academically gifted students}.}
                \EX{Early identification of \textbf{academically gifted students} predicts long-term attainment.}
                \CO{identify/support academically gifted students; programs for the academically gifted}
            \end{ExplainCard}

            \begin{ExplainCard}{hone}[v][B2]
                \EN{to improve a skill by practicing it; to make something sharper or more effective.}
                \SY{sharpen; refine; polish}
                \VI{\textit{mài giũa}, rèn luyện.}
                \EX{She joined a debate club to \textbf{hone} her speaking.}
                \EX{Internships \textbf{hone} graduates’ workplace competencies.}
                \CO{hone skills/technique/instincts; hone in on (AmE alt. meaning)}
            \end{ExplainCard}

            \begin{ExplainCard}{cross (one’s) mind}[idiom][B2]
                \EN{to occur to someone; to be thought of briefly.}
                \SY{occur to; strike; come to mind}
                \VI{\textit{thoáng nảy ra trong đầu}.}
                \EX{It suddenly \textbf{crossed my mind} to survey the class.}
                \EX{The possibility of bias never \textbf{crossed their minds} during design.}
                \CO{it crosses/never crosses sb’s mind that…}
            \end{ExplainCard}

            \begin{ExplainCard}{lament}[v][C1]
                \EN{to express sadness, disappointment, or regret about something.}
                \SY{mourn; bemoan; rue}
                \VI{\textit{than phiền}, than thở; tiếc nuối.}
                \EX{Students \textbf{lamented} the lack of speaking practice.}
                \EX{Researchers \textbf{lament} persistent inequities in access to enrichment.}
                \CO{lament the fact/loss/decline; widely lamented}
            \end{ExplainCard}

            \begin{ExplainCard}{stem from}[v][B2]
                \EN{to originate or result from a particular cause.}
                \SY{arise from; derive from; spring from}
                \VI{\textit{bắt nguồn từ}.}
                \EX{Their shyness \textbf{stems from} fear of making mistakes.}
                \EX{Achievement gaps often \textbf{stem from} unequal opportunities.}
                \CO{problems/benefits stem from; largely/primarily stem from}
            \end{ExplainCard}

            \begin{ExplainCard}{handful}[n][B2]
                \EN{(1) a small number or amount. (2) a person who is difficult to control.}
                \SY{(1) few; sprinkling \quad (2) troublemaker; live wire}
                \VI{(1) \textit{một vài}/một nhúm. (2) \textit{người khó quản}.}
                \EX{(1) I consulted a \textbf{handful} of colleagues.}
                \EX{(2) The toddler is a real \textbf{handful} at meal times.}
                \CO{a handful of + N; be quite a handful}
            \end{ExplainCard}

            \begin{ExplainCard}{bridge the gap}[phrase][B2]
                \EN{to reduce or remove a difference between groups, levels, or abilities.}
                \SY{close the gap; narrow disparities; connect}
                \VI{\textit{thu hẹp/khắc phục khoảng cách}.}
                \EX{Pair work helped \textbf{bridge the gap} between shy and confident speakers.}
                \EX{Targeted tutoring \textbf{bridges learning gaps} after disruptions.}
                \CO{bridge the achievement/skills/cultural gap}
            \end{ExplainCard}

            \begin{ExplainCard}{break (down) invisible barriers}[phrase][B2]
                \EN{to overcome unseen psychological or social obstacles that block progress.}
                \SY{overcome; dismantle; tear down barriers}
                \VI{\textit{phá vỡ những rào cản vô hình}.}
                \EX{Ice-breaker games \textbf{broke their invisible barriers}.}
                \EX{Peer mentoring \textbf{breaks down social barriers} to participation.}
                \CO{break down social/psychological barriers; remove barriers to + N}
            \end{ExplainCard}

            \begin{ExplainCard}{foster a dynamic (environment)}[collocation][B2]
                \EN{to encourage and nurture an energetic, responsive atmosphere.}
                \SY{cultivate; nurture; stimulate}
                \VI{\textit{nuôi dưỡng} một môi trường \textit{năng động}.}
                \EX{Clubs and projects \textbf{foster a dynamic} classroom culture.}
                \EX{Leadership practices that \textbf{foster a dynamic environment} correlate with innovation output.}
                \CO{foster a dynamic/creative/collaborative environment; foster growth/engagement}
            \end{ExplainCard}
        \end{VocabExplain}

    \noindent
    \textbf{Part 3.}
    \begin{qa}{Some people think that education should be about memorising the important ideas of the past. Do you agree or disagree? Why?}
    I would like to \textbf{kick off} by saying the past is \textbf{of the essence} to every single nation. Without history, \textbf{descendants} could never \textbf{perceive} what happened and learn from past mistakes. This will raise national spirit and help young people to \textbf{steer clear of} the same mistake. However, life is ever-changing, and people need to better prepare \textbf{themselves} for the future as well. In other words, students need to know how to take each day as it comes. \textbf{By and large}, it is unfair if the education only aims to cover the past while \textbf{turning a blind eye} to the \textbf{forthcoming} future.
    \end{qa}

    \begin{qa}{Should education encourage students to have their own new ideas? Why?}
    Definitely, children should be given \textbf{autonomy} to come up with their own ideas during the schooling. I guess some \textbf{convincing} reasons can explain this. Firstly, \textbf{freedom of speech} has been recognized in many parts of the world, and this will benefit \textbf{democracy} to some extent. Secondly, if students are allowed to raise their voice, their independent thinking skills would develop and help them take \textbf{a firm stand}, which is also necessary to deal with the \textbf{erratic} nature of the world nowadays.
    \end{qa}

    \begin{qa}{How do you think teachers could help students to develop and share their own ideas?}
    As far as I am concerned, some measures could be \textbf{implemented} to \textbf{incentivise} students to \textbf{weigh in} on opinions. \textbf{Open-ended} questions should be provided more to ask about students’ experience instead of finding a \textbf{definite} answer. Besides, group discussions are another solution to let students develop their \textbf{viewpoints}. Some students are \textbf{reluctant} to \textbf{voice} out their ideas for fear of making mistakes and being made fun of by their classmates. In that sense, teachers should be \textbf{open-minded} to accept every answer possible and \textbf{utilizing motivational strategies} to maintain positive atmosphere in class.
    \end{qa}

    \begin{qa}{Should employers encourage their workers to have new ideas about improving the company? Why?}
    Clearly, employers are part of the company, and they are \textbf{contributors} to the success of companies. \textbf{Hence}, staff should be encouraged to \textbf{mull over} and propose their ideas towards the shared goal of their company. There is an old saying, together we can change the world; therefore, individuals’ ideas should be valued as supervisors’ ideas.
    \end{qa}

    \begin{qa}{Do you think people sometimes dislike ideas just because they are new? Why?}
    Yes, I admit that people \textbf{shun} new ideas in some specific circumstances and fear of the unknown is quite common in terms of psychology. Actually, those who avoid innovations are \textbf{conservative} and afraid to change. They hardly make any \textbf{bold} decisions simply because new ideas will probably accompany negative \textbf{implications} and they are not in no sense courageous enough to shoulder responsibilities.
    \end{qa}

    \begin{qa}{What is more difficult: having new ideas or putting them into practice? Which is more important for a successful company?}
    In my opinion, inventing ideas and applying them into practice are both tough battles. Given the fact that the global market is likely to be saturated, it has never been a \textbf{child’s play} for individuals to \textbf{think outside the box}. Then, once new ideas have been accepted, the hypotheses and even samples will be tested in either a \textbf{laboratory} or on a specific population before officially being \textbf{put into effect}. And, every stages should deserve equal attention.
    \end{qa}

        \begin{VocabExplain}[Part 3]
            \begin{ExplainCard}{kick off}[phr.v][B2]
                \EN{to begin something in an energetic or official way.}
                \SY{start; commence; launch}
                \VI{\textit{mở đầu}, bắt đầu.}
                \EX{Let me \textbf{kick off} with a quick story.}
                \EX{The symposium \textbf{kicked off} with a keynote on curriculum reform.}
                \CO{kick off a meeting/event/debate}
            \end{ExplainCard}

            \begin{ExplainCard}{of the essence}[idiom][C1]
                \EN{extremely important and requiring swift attention.}
                \SY{crucial; paramount; vital}
                \VI{\textit{vô cùng quan trọng}.}
                \EX{In emergencies, speed is \textbf{of the essence}.}
                \EX{Timely feedback is \textbf{of the essence} for effective learning cycles.}
                \CO{be of the essence; time is of the essence}
                \end{ExplainCard}

                \begin{ExplainCard}{descendant}[n][B2]
                \EN{a person related to someone from a previous generation.}
                \SY{offspring; heir; progeny}
                \VI{\textit{hậu duệ}, con cháu.}
                \EX{She is a \textbf{descendant} of immigrants.}
                \EX{\textbf{Descendants} preserve oral histories that complement archives.}
                \CO{direct/remote descendant of; trace one’s descendants}
                \end{ExplainCard}

                \begin{ExplainCard}{perceive}[v][C1]
                \EN{to become aware of or understand something.}
                \SY{discern; recognize; apprehend}
                \VI{\textit{nhận thức}, nhìn nhận.}
                \EX{Students often \textbf{perceive} history as dull.}
                \EX{Learners’ self-efficacy is \textbf{perceived} to improve after mentoring.}
                \CO{perceive A as B; widely/perceptibly perceived}
                \end{ExplainCard}

                \begin{ExplainCard}{steer clear of}[phr.v][B2]
                \EN{to avoid something or someone that may cause problems.}
                \SY{avoid; shun; sidestep}
                \VI{\textit{tránh xa}.}
                \EX{We should \textbf{steer clear of} plagiarism.}
                \EX{Policies advise schools to \textbf{steer clear of} unvetted apps.}
                \CO{steer clear of trouble/controversy/pitfalls}
                \end{ExplainCard}

                \begin{ExplainCard}{by and large}[phrase][B2]
                \EN{generally; for the most part.}
                \SY{overall; in general; on the whole}
                \VI{\textit{nói chung}.}
                \EX{\textbf{By and large}, students liked the new format.}
                \EX{\textbf{By and large}, randomized trials produce consistent estimates.}
                \CO{by and large it seems/appears that…}
                \end{ExplainCard}

                \begin{ExplainCard}{turn a blind eye (to)}[idiom][B2]
                \EN{to deliberately ignore something wrong or problematic.}
                \SY{ignore; overlook; wink at}
                \VI{\textit{làm ngơ}, nhắm mắt bỏ qua.}
                \EX{We can’t \textbf{turn a blind eye to} cheating.}
                \EX{Regulators sometimes \textbf{turn a blind eye} to minor infractions.}
                \CO{turn a blind eye to misconduct/abuse}
                \end{ExplainCard}

                \begin{ExplainCard}{forthcoming}[adj][B2]
                \EN{(1) about to happen; upcoming. (2) willing to share information.}
                \SY{upcoming; imminent; approaching}
                \VI{(1) \textit{sắp tới}; (2) \textit{cởi mở}.}
                \EX{The \textbf{forthcoming} exam worries many students.}
                \EX{Authors were not \textbf{forthcoming} about limitations in the paper.}
                \CO{forthcoming event/report; be forthcoming about}
                \end{ExplainCard}

                \begin{ExplainCard}{autonomy}[n][C1]
                \EN{the ability to make decisions independently.}
                \SY{independence; self-direction; agency}
                \VI{\textit{tự chủ}, tự quyết.}
                \EX{Project \textbf{autonomy} motivates learners.}
                \EX{Educational \textbf{autonomy} correlates with creativity outputs.}
                \CO{grant/encourage student autonomy; learner autonomy}
                \end{ExplainCard}

                \begin{ExplainCard}{convincing}[adj][B2]
                \EN{persuasive because it is clear and credible.}
                \SY{persuasive; cogent; compelling}
                \VI{\textit{thuyết phục}.}
                \EX{She gave a \textbf{convincing} argument.}
                \EX{\textbf{Convincing} evidence supports active-learning methods.}
                \CO{convincing case/evidence/explanation}
                \end{ExplainCard}

                \begin{ExplainCard}{freedom of speech}[n][B2]
                \EN{the right to express opinions without censorship.}
                \SY{free expression; free speech}
                \VI{\textit{tự do ngôn luận}.}
                \EX{Universities protect \textbf{freedom of speech}.}
                \EX{\textbf{Freedom of speech} underpins democratic deliberation.}
                \CO{protect/limit freedom of speech; exercise freedom of speech}
                \end{ExplainCard}

                \begin{ExplainCard}{democracy}[n][C1]
                \EN{a system in which citizens exercise power by voting.}
                \SY{popular rule; self-government}
                \VI{\textit{dân chủ}.}
                \EX{Active citizenship sustains \textbf{democracy}.}
                \EX{Civic education is foundational to robust \textbf{democracies}.}
                \CO{liberal/representative democracy; strengthen democracy}
                \end{ExplainCard}

                \begin{ExplainCard}{a firm stand}[n phrase][B2]
                \EN{a strong, clear position on an issue.}
                \SY{strong stance; resolute position}
                \VI{\textit{lập trường vững vàng}.}
                \EX{Teachers should take \textbf{a firm stand} against bullying.}
                \EX{The board adopted \textbf{a firm stand} on data protection.}
                \CO{take/maintain a firm stand on/against}
                \end{ExplainCard}

                \begin{ExplainCard}{erratic}[adj][B2]
                \EN{unpredictable and inconsistent.}
                \SY{irregular; capricious; volatile}
                \VI{\textit{thất thường}, khó đoán.}
                \EX{Her schedule is quite \textbf{erratic}.}
                \EX{Funding cycles can be \textbf{erratic}, affecting program continuity.}
                \CO{erratic behavior/patterns/performance}
                \end{ExplainCard}

                \begin{ExplainCard}{implement}[v][C1]
                \EN{to put a plan or policy into action.}
                \SY{enact; execute; carry out}
                \VI{\textit{triển khai}, thực thi.}
                \EX{Schools \textbf{implemented} new rubrics.}
                \EX{Well-\textbf{implemented} interventions improve attainment.}
                \CO{implement a policy/plan/strategy}
                \end{ExplainCard}

                \begin{ExplainCard}{incentivise}[v][C1]
                \EN{to motivate by offering incentives.}
                \SY{motivate; encourage; spur}
                \VI{\textit{tạo động lực}/khuyến khích (bằng thưởng).}
                \EX{Badges \textbf{incentivise} participation.}
                \EX{Grants \textbf{incentivise} research in priority areas.}
                \CO{incentivise students/staff/behavior}
                \end{ExplainCard}

                \begin{ExplainCard}{weigh in (on)}[phr.v][B2]
                \EN{to join a discussion with an opinion.}
                \SY{chime in; contribute; opine}
                \VI{\textit{lên tiếng}/đưa ý kiến.}
                \EX{Please \textbf{weigh in on} the proposal.}
                \EX{Experts \textbf{weighed in} during the open consultation.}
                \CO{weigh in on a debate/issue}
                \end{ExplainCard}

                \begin{ExplainCard}{open-ended}[adj][B2]
                \EN{allowing for many possible answers; not limited to one outcome.}
                \SY{unrestricted; exploratory; non-directive}
                \VI{\textit{mở}, không giới hạn.}
                \EX{Use \textbf{open-ended} questions in class.}
                \EX{\textbf{Open-ended} tasks foster divergent thinking.}
                \CO{open-ended question/task/inquiry}
                \end{ExplainCard}

                \begin{ExplainCard}{definite}[adj][B2]
                \EN{clearly fixed or certain.}
                \SY{explicit; clear-cut; unequivocal}
                \VI{\textit{rõ ràng}, dứt khoát.}
                \EX{There’s no \textbf{definite} answer here.}
                \EX{The trial showed a \textbf{definite} improvement in outcomes.}
                \CO{a definite plan/answer/improvement}
                \end{ExplainCard}

                \begin{ExplainCard}{viewpoint}[n][B2]
                \EN{a particular attitude or way of considering something.}
                \SY{perspective; stance; opinion}
                \VI{\textit{quan điểm}.}
                \EX{Share your \textbf{viewpoint} respectfully.}
                \EX{Multiple \textbf{viewpoints} enrich policy deliberation.}
                \CO{from a/the viewpoint of; differing viewpoints}
                \end{ExplainCard}

                \begin{ExplainCard}{reluctant}[adj][B2]
                \EN{unwilling and hesitant.}
                \SY{hesitant; loath; disinclined}
                \VI{\textit{miễn cưỡng}, ngại.}
                \EX{Some students are \textbf{reluctant} to speak.}
                \EX{Parents were \textbf{reluctant} to consent to data sharing.}
                \CO{reluctant to do sth; a reluctant participant}
                \end{ExplainCard}

                \begin{ExplainCard}{voice (one’s ideas)}[v][B2]
                \EN{to express ideas or opinions out loud.}
                \SY{articulate; express; air}
                \VI{\textit{bày tỏ}/nói lên (ý kiến).}
                \EX{Invite learners to \textbf{voice} their concerns.}
                \EX{Forums let stakeholders \textbf{voice} perspectives transparently.}
                \CO{voice concerns/opinions/ideas}
                \end{ExplainCard}

                \begin{ExplainCard}{open-minded}[adj][B2]
                \EN{willing to consider new ideas.}
                \SY{receptive; broad-minded; tolerant}
                \VI{\textit{cởi mở}.}
                \EX{Be \textbf{open-minded} about feedback.}
                \EX{\textbf{Open-minded} leadership fosters innovation.}
                \CO{remain/stay open-minded; be open-minded about}
                \end{ExplainCard}

                \begin{ExplainCard}{motivational strategies}[collocation][B2]
                \EN{planned methods to increase learners’ drive and engagement.}
                \SY{engagement tactics; incentive mechanisms}
                \VI{\textit{chiến lược tạo động lực}.}
                \EX{Gamification is one of our \textbf{motivational strategies}.}
                \EX{\textbf{Motivational strategies} improved persistence in large courses.}
                \CO{design/apply/evaluate motivational strategies}
                \end{ExplainCard}

                \begin{ExplainCard}{contributor}[n][B2]
                \EN{a person or factor that helps bring about a result.}
                \SY{participant; actor; factor}
                \VI{\textit{người/nhân tố đóng góp}.}
                \EX{Every employee is a \textbf{contributor} to success.}
                \EX{Socioeconomic status is a major \textbf{contributor} to outcomes.}
                \CO{key/significant contributor to}
                \end{ExplainCard}

                \begin{ExplainCard}{hence}[adv][C1]
                \EN{for this reason; as a result.}
                \SY{therefore; thus; accordingly}
                \VI{\textit{vì thế}, do đó.}
                \EX{Deadlines were tight; \textbf{hence} the overtime.}
                \EX{Demand surged, \textbf{hence} the policy adjustment.}
                \CO{and hence; henceforth (different meaning)}
                \end{ExplainCard}

                \begin{ExplainCard}{mull over}[phr.v][B2]
                \EN{to think about something carefully for a period of time.}
                \SY{ponder; contemplate; deliberate}
                \VI{\textit{suy nghĩ kỹ}, cân nhắc.}
                \EX{Let’s \textbf{mull over} the options tonight.}
                \EX{Boards \textbf{mull over} mergers before voting.}
                \CO{mull over a proposal/idea/offer}
                \end{ExplainCard}

                \begin{ExplainCard}{shun}[v][B2]
                \EN{to avoid someone or something deliberately.}
                \SY{avoid; eschew; steer clear of}
                \VI{\textit{lảng tránh}, xa lánh.}
                \EX{Some \textbf{shun} new tech at first.}
                \EX{Firms that \textbf{shun} risk often innovate more slowly.}
                \CO{shun publicity/controversy/change}
                \end{ExplainCard}

                \begin{ExplainCard}{conservative}[adj][C1]
                \EN{resistant to change; favoring traditional views.}
                \SY{cautious; traditional; risk-averse}
                \VI{\textit{bảo thủ}, thận trọng.}
                \EX{He is \textbf{conservative} about redesigns.}
                \EX{A \textbf{conservative} approach limited variance but slowed progress.}
                \CO{conservative attitude/estimate/strategy}
                \end{ExplainCard}

                \begin{ExplainCard}{bold}[adj][B2]
                \EN{showing willingness to take risks; confident and courageous.}
                \SY{daring; audacious; decisive}
                \VI{\textit{táo bạo}, liều lĩnh tích cực.}
                \EX{We need a \textbf{bold} redesign.}
                \EX{\textbf{Bold} policy moves can catalyze systemic change.}
                \CO{a bold decision/move/vision}
                \end{ExplainCard}

                \begin{ExplainCard}{implication}[n][C1]
                \EN{a possible effect or result of an action or decision.}
                \SY{consequence; ramification; repercussion}
                \VI{\textit{hệ quả}, hàm ý.}
                \EX{Consider the \textbf{implications} of tracking data.}
                \EX{The reform has fiscal and social \textbf{implications}.}
                \CO{policy/ethical implications; far-reaching implications}
                \end{ExplainCard}

                \begin{ExplainCard}{child’s play}[idiom][B2]
                \EN{something very easy to do.}
                \SY{a breeze; no big deal; simple}
                \VI{\textit{dễ như chơi}.}
                \EX{For her, calculus is \textbf{child’s play}.}
                \EX{Compared to deployment, prototyping is \textbf{child’s play}.}
                \CO{be/looks like child’s play}
                \end{ExplainCard}

                \begin{ExplainCard}{think outside the box}[idiom][B2]
                \EN{to think creatively beyond conventional ideas.}
                \SY{innovate; think laterally; break the mold}
                \VI{\textit{suy nghĩ khác lối mòn}.}
                \EX{We must \textbf{think outside the box} for funding.}
                \EX{Hackathons push students to \textbf{think outside the box}.}
                \CO{encourage/learn to think outside the box}
                \end{ExplainCard}

                \begin{ExplainCard}{laboratory}[n][B2]
                \EN{a room with scientific equipment for experiments.}
                \SY{lab; research facility}
                \VI{\textit{phòng thí nghiệm}.}
                \EX{Samples were tested in the \textbf{laboratory}.}
                \EX{\textbf{Laboratory} findings inform field trials.}
                \CO{laboratory testing/setting/conditions}
                \end{ExplainCard}

                \begin{ExplainCard}{put into effect}[phrase][B2]
                \EN{to make a plan or rule start to operate.}
                \SY{enforce; implement; bring into force}
                \VI{\textit{đưa vào hiệu lực}/thực thi.}
                \EX{The policy was \textbf{put into effect} in May.}
                \EX{Pilot results must be reviewed before \textbf{putting} reforms \textbf{into effect}.}
                \CO{put a law/policy/measure into effect}
                \end{ExplainCard}
        \end{VocabExplain}

        \begin{VocabHighlights}
            \VH{to utter}{(v) to make a sound with your voice; to say something}{(động từ) bật ra (tiếng)}
            \VH{to show off}{(phr. v) to try to impress others by talking about your abilities, possessions, etc}{(cụm động từ) khoe khoang}
            \VH{to nutmeg}{(v) in football, to kick the ball through an opponent's legs}{(động từ) xỏ háng}
            \VH{to die laughing}{(idiom) laugh heartily or uncontrollably}{(thành ngữ) cười không nhặt được mồm}
            \VH{hysterical}{(adj) in a state of extreme excitement, and crying, laughing, etc. in an uncontrolled way}{(tính từ) điên dại}
            \VH{to have somebody in the stitches}{(idiom) to make somebody laugh uncontrollably}{(thành ngữ) khiến ai cười lăn lộn}
            \VH{dismal}{(adj) gloomy, miserable}{(tính từ) ảm đạm}
            \VH{to burst out v-ing}{(phr. v) to suddenly say something loudly}{(cụm động từ) phá lên, hét toáng lên}
            \VH{on the flip side}{(phrase) looking at a different or opposite aspect}{(cụm từ) nhìn ở khía cạnh khác}
            \VH{clinically proven}{(phrase) tested by doctors}{(cụm từ) đã được các bác sĩ kiểm tra}
            \VH{well-being}{(n) general health and happiness}{(danh từ) sự khỏe mạnh, thịnh vượng}
            \VH{determinant}{(n) a thing that decides whether or how something happens}{(danh từ) thứ quyết định}
            \VH{to laugh off}{(phr. v) to try to make people think that something is not serious or important, especially by making a joke about it}{(cụm động từ) cười trừ, cho qua}
            \VH{academically-gifted students}{(phrase) students are good at academic subjects}{(cụm từ) học sinh giỏi}
            \VH{to hone}{(v) to make an object sharp}{(động từ) mài dũa}
            \VH{to cross my mind}{(idiom) an idea happens in your mind}{(thành ngữ) nảy ra trong đầu}
            \VH{to lament}{(v) express disappointment about something}{(động từ) than vãn}
            \VH{to stem from}{(v) to start or develop as the result of something}{(động từ) xuất phát từ}
            \VH{a handful of}{(phrase) some}{(cụm từ) một vài}
            \VH{to bridge their gap}{(phrase) make the differences less marked}{(cụm từ) thu hẹp khoảng cách}
            \VH{to break their invisible barriers}{(phrase) remove hidden obstacles}{(cụm từ) phá vỡ rào cản vô hình}
            \VH{to foster}{(v) encourage the development of}{(động từ) nuôi dưỡng, bồi dưỡng}
            \VH{dynamic}{(adj) positive in attitude and full of energy and new idea}{(tính từ) năng động}
            \VH{kick off}{(v) to start a discussion, a meeting, an event, etc}{(động từ) bắt đầu}
            \VH{to be of the essence}{(idiom) necessary and very important}{(thành ngữ) rất quan trọng}
            \VH{descendant}{(n) a person's descendants are their children, their children's children, and all the people who live after them who are related to them}{(danh từ) hậu duệ}
            \VH{to perceive}{(v) to notice or become aware of something}{(động từ) nhận thức}
            \VH{to steer clear of}{(phrase) take care to avoid or keep away from}{(cụm từ) tránh xa khỏi}
            \VH{autonomy}{(n) the ability to act and make decisions without being controlled by anyone else}{(danh từ) sự tự chủ}
            \VH{schooling}{(n) the education you receive at school}{(danh từ) quá trình đi học}
            \VH{convincing}{(adj) that makes somebody believe that something is true}{(tính từ) có sức thuyết phục}
            \VH{freedom of speech}{(phrase) the right of individuals and organizations to exchange information without fear of repercussion or censorship}{(cụm từ) tự do ngôn luận}
            \VH{democracy}{(n) a system of government in which all the people of a country can vote to elect their representatives}{(danh từ) dân chủ}
            \VH{to incentivise}{(v) provide someone with a reward for doing something}{(động từ) sự khuyến khích}
            \VH{definite}{(adj) sure or certain; unlikely to change}{(tính từ) rõ ràng, cụ thể}
            \VH{a contributor}{(n) a person or thing that provides money to help pay for something, or support something}{(danh từ) người đóng góp}
            \VH{to mull over}{(v) to think carefully about something for a long time}{(động từ) suy nghĩ kỹ}
            \VH{to shun}{(v) avoid something}{(động từ) tránh, xa lánh}
            \VH{conservative}{(adj) opposed to great or sudden social change; showing that you prefer traditional styles and values}{(tính từ) bảo thủ}
            \VH{bold}{(adj) brave and confident; not afraid to say what you feel or to take risks}{(tính từ) táo bạo}
            \VH{implication}{(n) something that is suggested or indirectly stated (=something that is implied)}{(danh từ) hệ lụy, hậu quả}
            \VH{to think outside the box}{(idiom) to think imaginatively using new ideas instead of traditional or expected ideas}{(thành ngữ) suy nghĩ mới lạ, độc đáo}
            \VH{a laboratory}{(n) a room or building used for scientific research, experiments, testing, etc}{(danh từ) phòng thí nghiệm}
            \VH{to put into effect}{(idiom) to implement; to execute; to carry out}{(thành ngữ) áp dụng, tiến hành}
        \end{VocabHighlights}
    \end{test}

    \begin{test}{TEST 3}
    \noindent
    \textbf{Part 1. Cold weather}
    \begin{qa}{Have you ever been in very cold weather? (When?)}
    Yes. This \textbf{traced back} to my time in the U.K to pursue my Master degree. I was admitted to Newcastle University located in the Northeast of the U.K. This city is famous for its \textbf{harsh} weather especially in winter. One day, temperatures even dropped to minus 10 Celsius degree. Consequently, I happened to see snowfall with my own eyes for the very first time because I came from a tropical country, Vietnam, where snow is quite \textbf{elusive}.
    \end{qa}

    \begin{qa}{How often is the weather cold where you come from?}
    In a tropical country where the weather is \textbf{muggy} almost every month, I do not often \textbf{grab the opportunity} to \textbf{bundle up}. \textbf{To the best of my belief}, I only endure \textbf{a cold snap} \textbf{for a fortnight} or so every winter.
    \end{qa}

    \begin{qa}{Are some parts of your country colder than others? (Why?)}
    Yes they are. In the North, there exist 4 seasons: spring, summer, autumn and winter whereas in the South, only 2 seasons are in existence, namely dry and wet ones. Accordingly, the Northern climate is more severe when compared to the Southern one. The temperature in the North might \textbf{plummet} to 6 or 7 Celsius degrees while the Southern temperature hardly ever falls below 25 Celsius degrees.
    \end{qa}

    \begin{qa}{Would you prefer to live in a hot place or a cold place? (Why?)}
    To be honest, I am of the opinion of living in areas of \textbf{mild} weather instead. The \textbf{sultry} weather of the Northern region of Vietnam often \textbf{creates favorable conditions for} infectious diseases to \textbf{go rampant}. However, the lack of warmth due to cold weather in Newcastle sometimes results in an increased \textbf{susceptibility} to illnesses such as cold. That’s why I would like to live in countries of mild weather in Mediterranean regions, Portugal, Spain and Italy inclusive.
    \end{qa}
    
        \begin{VocabExplain}[Part 1]
            \begin{ExplainCard}{trace back (to)}[phr.v][B2]
                \EN{to originate from or be found to have begun at a particular time or source.}
                \SY{originate; derive; date back}
                \VI{\textit{bắt nguồn từ}, truy nguyên về.}
                \EX{My love of tea \textbf{traces back to} my grandmother.}
                \EX{The outbreak was \textbf{traced back to} contaminated water in the dataset.}
                \CO{trace back to origins/roots; be traced back to}
            \end{ExplainCard}

            \begin{ExplainCard}{harsh (weather)}[adj][B2]
                \EN{severe or unpleasantly rough; difficult to live in.}
                \SY{severe; bitter; inclement}
                \VI{\textit{khắc nghiệt}.}
                \EX{We had a \textbf{harsh} winter last year.}
                \EX{\textbf{Harsh} climatic conditions reduce agricultural yields.}
                \CO{harsh winter/conditions/climate}
            \end{ExplainCard}

            \begin{ExplainCard}{elusive}[adj][C1]
                \EN{difficult to find, catch, or achieve; hard to define.}
                \SY{evasive; hard-to-catch; intangible}
                \VI{\textit{khó nắm bắt}, hiếm gặp.}
                \EX{A good night’s sleep feels \textbf{elusive}.}
                \EX{An \textbf{elusive} concept complicates operational definitions in research.}
                \CO{prove/remain elusive; elusive answer/species}
            \end{ExplainCard}

            \begin{ExplainCard}{muggy}[adj][B2]
                \EN{uncomfortably warm and damp.}
                \SY{humid; sultry; sticky}
                \VI{\textit{oi bức}, ẩm ướt.}
                \EX{It’s \textbf{muggy} tonight—open a window.}
                \EX{\textbf{Muggy} conditions increase heat-stress risk in cities.}
                \CO{muggy air/night/weather}
            \end{ExplainCard}

            \begin{ExplainCard}{grab the opportunity (to)}[phrase][B2]
                \EN{to seize a chance quickly and use it.}
                \SY{seize; capitalize on; take advantage of}
                \VI{\textit{nắm lấy cơ hội}.}
                \EX{You should \textbf{grab the opportunity} to travel.}
                \EX{Firms \textbf{grab the opportunity} created by green subsidies.}
                \CO{grab/seize/take the opportunity to + V}
            \end{ExplainCard}

            \begin{ExplainCard}{bundle up}[phr.v][B2]
                \EN{to dress warmly in many layers.}
                \SY{wrap up; layer up}
                \VI{\textit{mặc ấm}, quấn áo ấm.}
                \EX{It’s freezing—\textbf{bundle up}!}
                \EX{Visitors are advised to \textbf{bundle up} during sub-zero expeditions.}
                \CO{bundle up against the cold; bundle sb up}
            \end{ExplainCard}

            \begin{ExplainCard}{to the best of my belief}[phrase][B2]
                \EN{as far as I know or believe to be true.}
                \SY{to my knowledge; as far as I can tell}
                \VI{\textit{theo như tôi tin/biết}.}
                \EX{\textbf{To the best of my belief}, he’s honest.}
                \EX{\textbf{To the best of our belief}, no records were lost in transit.}
                \CO{to the best of my/our knowledge/belief}
            \end{ExplainCard}

            \begin{ExplainCard}{cold snap}[n][B2]
                \EN{a short period of unusually cold weather.}
                \SY{cold spell; cold wave}
                \VI{\textit{đợt rét ngắn}.}
                \EX{A \textbf{cold snap} hit the city last week.}
                \EX{The \textbf{cold snap} drove up electricity demand by 20\%.}
                \CO{a cold snap hits/arrives; during a cold snap}
            \end{ExplainCard}

            \begin{ExplainCard}{for a fortnight}[adv phrase][B2]
                \EN{for a period of two weeks.}
                \SY{for two weeks; biweekly period}
                \VI{\textit{trong hai tuần}.}
                \EX{I stayed in London \textbf{for a fortnight}.}
                \EX{Participants logged their diet \textbf{for a fortnight}.}
                \CO{for a fortnight; every fortnight}
            \end{ExplainCard}

            \begin{ExplainCard}{plummet}[v][B2]
                \EN{to fall or drop quickly and steeply.}
                \SY{plunge; nosedive; tumble}
                \VI{\textit{giảm/tuột nhanh}.}
                \EX{Temperatures \textbf{plummeted} overnight.}
                \EX{Stock values \textbf{plummeted} after the announcement.}
                \CO{plummet to/from; prices/temperatures/rates plummet}
            \end{ExplainCard}

            \begin{ExplainCard}{mild (weather)}[adj][B2]
                \EN{not extreme or severe; moderately warm.}
                \SY{temperate; moderate; balmy}
                \VI{\textit{ôn hòa}.}
                \EX{Winters are fairly \textbf{mild} here.}
                \EX{\textbf{Mild} climates support year-round agriculture.}
                \CO{mild winter/climate/temperatures}
            \end{ExplainCard}

            \begin{ExplainCard}{sultry}[adj][B2]
                \EN{hot and very humid; oppressively warm.}
                \SY{sweltering; stifling; muggy}
                \VI{\textit{oi bức, ngột ngạt}.}
                \EX{We walked home on a \textbf{sultry} evening.}
                \EX{\textbf{Sultry} conditions exacerbate heat-related illness rates.}
                \CO{sultry weather/night/air}
            \end{ExplainCard}

            \begin{ExplainCard}{create favorable conditions for}[phrase][B2]
                \EN{to make an environment that helps something happen.}
                \SY{facilitate; foster; pave the way for}
                \VI{\textit{tạo điều kiện thuận lợi cho}.}
                \EX{Good mentoring \textbf{creates favorable conditions for} learning.}
                \EX{Tax reforms \textbf{create favorable conditions for} investment inflows.}
                \CO{create favorable conditions for growth/development}
            \end{ExplainCard}

            \begin{ExplainCard}{run/go rampant}[phrase][B2]
                \EN{to spread or grow quickly without being controlled.}
                \SY{spread unchecked; proliferate; rage}
                \VI{\textit{hoành hành}, lan tràn.}
                \EX{Rumors \textbf{ran rampant} online.}
                \EX{Without containment, vectors can \textbf{run rampant} across regions.}
                \CO{run rampant in/among; allow sth to run rampant}
            \end{ExplainCard}

            \begin{ExplainCard}{susceptibility (to)}[n][B2]
                \EN{the state of being likely to be harmed or affected by something.}
                \SY{vulnerability; proneness; sensitivity}
                \VI{\textit{sự dễ bị ảnh hưởng}, dễ mắc.}
                \EX{Lack of sleep increases \textbf{susceptibility to} colds.}
                \EX{Genetic factors shape \textbf{susceptibility to} infectious diseases.}
                \CO{susceptibility to disease/infection; increase/reduce susceptibility}
            \end{ExplainCard}
        \end{VocabExplain}

    \noindent
    \textbf{Part 2.}
    \begin{qa}{A competition (TV, college work or sports competition) that you took part in. You should say:}
    \begin{itemize}
    \item What kind of competition it was and how you found out about it
    \item What you had to do
    \item What the prizes were
    \item and explain why you chose to take part in this competition.
    \end{itemize}
    
    I have attended \textbf{a couple of} competitions in my life, but today, I would like to talk about my first contest I \textbf{went in for}. It was the design contest titled “Global Design Competition” which I knew \textbf{by chance}. While I was browsing the Internet, the title of advertisement \textbf{caught my attention}. The competition was primarily for amateur designers who \textbf{harbored a dream of pursuing} a career as a fashion designer. The first prize is a two-month course in one of the \textbf{leading} fashion institutions in Vietnam, along with 30 million Vietnam Dong of prize money. I thought that it would be a golden chance for me so I had to \textbf{jump at this opportunity}. To take part in this competition, I had to \textbf{come up with} a new design for evening gowns. This outfit will be worn by Miss Vietnam 2014, so I needed to \textbf{think outside the box}. There were 2 stages. The first phase was that I had to send a copy of my drawings to the contest. The \textbf{jury} of the contest would evaluate the works of candidates and make a short list for the next round. The following stage was that I had to make the outfit and present my ideas in front of the \textbf{jury}. The reason why I chose to \textbf{engage in} this competition is that it aimed to nurture design talent and create access to international stage. I should not forget to mention that it was not merely a \textbf{platform} for \textbf{aspiring} designers to showcase their ideas and work, but it was also designed to \textbf{brush up on} skills in every aspect related to fashion design and branding. Thanks to this competition, I could \textbf{accumulate} more \textbf{hands-on} experience and expand my social network.
    \end{qa}

        \begin{VocabExplain}[Part 2]
            \begin{ExplainCard}{a couple of}[phrase][B2]
                \EN{two; a small number of.}
                \SY{a few; two or so}
                \VI{\textit{vài}; \textit{hai}.}
                \EX{I’ve entered \textbf{a couple of} contests this year.}
                \EX{\textbf{A couple of} variables were controlled in the experiment.}
                \CO{a couple of years/times/ideas}
            \end{ExplainCard}

            \begin{ExplainCard}{go in for}[phr.v][B2]
                \EN{to enter or take part in (a competition, exam, etc.).}
                \SY{enter; take part in; compete in}
                \VI{\textit{tham gia}/\textit{đăng ký dự}.}
                \EX{She \textbf{went in for} the city marathon.}
                \EX{Thousands \textbf{go in for} national qualifying rounds annually.}
                \CO{go in for an exam/competition}
                \end{ExplainCard}

                \begin{ExplainCard}{by chance}[phrase][B1]
                \EN{accidentally; without planning.}
                \SY{accidentally; coincidentally}
                \VI{\textit{tình cờ}.}
                \EX{I found the ad \textbf{by chance}.}
                \EX{The sample was selected \textbf{by chance}, ensuring randomness.}
                \CO{meet/see/come across by chance}
                \end{ExplainCard}

                \begin{ExplainCard}{catch (one’s) attention}[phrase][C1]
                \EN{to attract someone’s notice or interest.}
                \SY{grab/attract/draw attention}
                \VI{\textit{thu hút sự chú ý}.}
                \EX{The headline \textbf{caught my attention}.}
                \EX{Bright packaging \textbf{catches attention} at point of sale.}
                \CO{immediately/instantly catch attention; attention-grabbing}
                \end{ExplainCard}

                \begin{ExplainCard}{harbor a dream (of …)}[v phrase][B2]
                \EN{to keep and nourish an ambition or desire, often secretly.}
                \SY{nurture; cherish; entertain (a dream)}
                \VI{\textit{ấp ủ giấc mơ} (làm gì).}
                \EX{She \textbf{harbors a dream of} becoming a designer.}
                \EX{Many \textbf{harbor dreams of} entrepreneurship despite constraints.}
                \CO{harbor a dream/ambition/hope}
                \end{ExplainCard}

                \begin{ExplainCard}{leading}[adj][B2]
                \EN{most important or successful in a particular area.}
                \SY{top; foremost; premier}
                \VI{\textit{hàng đầu}.}
                \EX{A \textbf{leading} fashion school offered the prize.}
                \EX{\textbf{Leading} institutions often set industry standards.}
                \CO{leading company/expert/institution}
                \end{ExplainCard}

                \begin{ExplainCard}{jump at (an) opportunity}[idiom][C1]
                \EN{to accept eagerly as soon as it appears.}
                \SY{seize; grab; leap at}
                \VI{\textit{chớp lấy cơ hội}.}
                \EX{I \textbf{jumped at the opportunity} to enter.}
                \EX{Startups \textbf{jump at opportunities} created by policy shifts.}
                \CO{jump/leap at the chance/opportunity}
                \end{ExplainCard}

                \begin{ExplainCard}{come up with}[phr.v][B2]
                \EN{to think of or produce (an idea/solution).}
                \SY{devise; conceive; generate}
                \VI{\textit{nghĩ ra}, đề xuất.}
                \EX{We \textbf{came up with} a new sketch.}
                \EX{The team \textbf{came up with} a novel algorithm.}
                \CO{come up with an idea/plan/answer}
                \end{ExplainCard}

                \begin{ExplainCard}{think outside the box}[idiom][B2]
                \EN{to think creatively, beyond conventional limits.}
                \SY{innovate; think laterally; break the mold}
                \VI{\textit{suy nghĩ khác lối mòn}.}
                \EX{Designers must \textbf{think outside the box}.}
                \EX{Workshops train students to \textbf{think outside the box} in problem-solving.}
                \CO{encourage/learn to think outside the box}
                \end{ExplainCard}

                \begin{ExplainCard}{jury}[n][B1]
                \EN{a panel of judges who evaluate entries in a contest.}
                \SY{panel; board of judges; adjudicators}
                \VI{\textit{ban giám khảo}.}
                \EX{The \textbf{jury} announced the shortlist.}
                \EX{\textbf{Jury} feedback improved the prototypes iteratively.}
                \CO{jury panel; present before the jury}
                \end{ExplainCard}

                \begin{ExplainCard}{engage in}[v][B2]
                \EN{to take part in or become involved with.}
                \SY{participate in; involve oneself in}
                \VI{\textit{tham gia vào}.}
                \EX{She \textbf{engages in} community art projects.}
                \EX{Students \textbf{engaged in} collaborative design tasks during the study.}
                \CO{engage in discussion/research/activities}
                \end{ExplainCard}

                \begin{ExplainCard}{platform (for)}[n][B2]
                \EN{a venue or means that enables people to present or develop work.}
                \SY{stage; forum; springboard}
                \VI{\textit{nền tảng}/diễn đàn (để…)}.
                \EX{The show is a \textbf{platform for} young artists.}
                \EX{Incubators provide a \textbf{platform for} commercializing research.}
                \CO{platform for creators/innovation/talent}
                \end{ExplainCard}

                \begin{ExplainCard}{aspiring}[adj][B2]
                \EN{having ambitions to become a specified type of person.}
                \SY{budding; up-and-coming; would-be}
                \VI{\textit{đầy khát vọng}, mới vào nghề.}
                \EX{An \textbf{aspiring} designer needs exposure.}
                \EX{\textbf{Aspiring} entrepreneurs benefited from mentorship schemes.}
                \CO{aspiring artist/writer/designer}
                \end{ExplainCard}

                \begin{ExplainCard}{brush up on}[phr.v][B2]
                \EN{to improve your knowledge/skill quickly by revising.}
                \SY{refresh; polish; revise}
                \VI{\textit{ôn luyện}/trau dồi (kỹ năng).}
                \EX{I’m \textbf{brushing up on} pattern-making.}
                \EX{Participants \textbf{brushed up on} statistics before analysis.}
                \CO{brush up on skills/grammar/techniques}
                \end{ExplainCard}

                \begin{ExplainCard}{accumulate}[v][B2]
                \EN{to gather or build up gradually over time.}
                \SY{amass; acquire; build up}
                \VI{\textit{tích lũy}.}
                \EX{She \textbf{accumulated} experience through internships.}
                \EX{Firms \textbf{accumulate} capabilities via repeated projects.}
                \CO{accumulate experience/wealth/knowledge}
                \end{ExplainCard}

                \begin{ExplainCard}{hands-on}[adj][B2]
                \EN{involving active, practical participation rather than theory.}
                \SY{practical; experiential; applied}
                \VI{\textit{thực hành}, trực tiếp.}
                \EX{The course offers \textbf{hands-on} workshops.}
                \EX{\textbf{Hands-on} training improved performance metrics.}
                \CO{hands-on experience/practice/training}
                \end{ExplainCard}
        \end{VocabExplain}

    \noindent
    \textbf{Part 3.}
    \begin{qa}{Why do you think some school teachers use competitions as class activities?}
    Understandably, competitions are an effective way for children to \textbf{gather momentum} in classes. Basically, \textbf{peer pressure} is usually a \textbf{stimulus} for students to \textbf{pull out all the stops} and achieve better outcomes. School children will never want to \textbf{fall behind} their classmates or receive \textbf{unsatisfactory} comments. Therefore, the introduction of competitions should be welcomed in schools.
    \end{qa}

    \begin{qa}{Do you think it is a good thing to give prizes to children who do well at school? Why?}
    Well, it is just as well that students who \textbf{outperform} others at school should \textbf{reap} the reward for their \textbf{dedication}, I believe. This practice is reasonable to encourage and cheer students to make further \textbf{attempts}. However, students who are given a \textbf{compliment} too often may develop an \textbf{arrogant} attitude towards other students, so teachers should consider before giving any prizes.
    \end{qa}

    \begin{qa}{Would you say that schools for young children have become more or less competitive since you were that age? Why?}
    Of course, schools were not as competitive as those in today. In the past, teachers were far too \textbf{lenient} and students had more time to \textbf{put their feet up}. However, these days, more \textbf{principles} are established, forcing students to finish more homework and, the establishment of extra classes means they have less spare time to \textbf{engage in} other physical activities. When I was around my students’ age, I only took part in 2 or 3 cram classes at most per week but my students now often enroll in 4 extra classes weekly at least.
    \end{qa}

    \begin{qa}{What are the advantages and disadvantages of intensive training for young sportspeople?}
    Well, intensive training is almost \textbf{compulsory} for every young athlete. The \textbf{coaching} session will provide sportspeople with \textbf{sufficient} skills to yield the best performance to \textbf{carve out stellar careers} later on. However, young athletes who undergo intensive practice and high-level competition from an early age can suffer from injuries and chronic stress if they do not follow \textbf{safety ground procedures}. Some injuries are so \textbf{horrendous} that they may wreck one’s promising career.
    \end{qa}

    \begin{qa}{Some people think that competition leads to a better performance from sports stars. Others think it just makes players feel insecure. What is your opinion?}
    Clearly, participating in contests really boost athletes’ performance. First and foremost, competitions are true tests of skills. If players are \textbf{drilled} for hours to perfect their techniques, a fight can \textbf{gauge} how excellent they are in terms of \textbf{stamina} and \textbf{athletic ability}. Having said that, I do agree that there are \textbf{occupational injuries} that make them feel unsafe whenever taking part in a run for their money. Even though there are safety procedures to prevent accidents, I guess many competitions are \textbf{putting a strain} on participants.
    \end{qa}

    \begin{qa}{Do you think that it is possible to become too competitive in sport? In what way?}
    I would say it is \textbf{in the cards}, but it should not happen too frequently. Although the competition is a chance for sportspeople to \textbf{tweak} their performance, too much competitiveness can lead to \textbf{counterproductive} effects. Trauma and stress are among two \textbf{psychological} problems that athletes often suffer from long-term training. Besides, the conflicts among the \textbf{rival} sports team and fans are not something sports lover expect to watch.
    \end{qa}

        \begin{VocabExplain}[Part 3]
            \begin{ExplainCard}{gather momentum}[phrase][B2]
            \EN{to begin to progress or develop more quickly and strongly.}
            \SY{build up steam; pick up pace}
            \VI{\textit{tăng tốc}, lấy đà.}
            \EX{The project \textbf{gathered momentum} after the first win.}
            \EX{Policy reforms \textbf{gather momentum} when early outcomes are positive.}
            \CO{gather momentum/pace/steam}
            \end{ExplainCard}
            \begin{ExplainCard}{peer pressure}[n][B2]
            \EN{influence from people of the same age or status to behave in a certain way.}
            \SY{social pressure; group influence}
            \VI{\textit{sức ép từ bạn bè}.}
            \EX{\textbf{Peer pressure} can push teens to study harder.}
            \EX{\textbf{Peer-pressure} effects are significant in classroom performance models.}
            \CO{resist/succumb to peer pressure}
            \end{ExplainCard}
            \begin{ExplainCard}{stimulus}[n][B2]
            \EN{something that causes growth, activity, or reaction.}
            \SY{incentive; spur; catalyst}
            \VI{\textit{kích thích}, tác nhân thúc đẩy.}
            \EX{Praise was the \textbf{stimulus} he needed.}
            \EX{Financial \textbf{stimuli} can raise participation in training programs.}
            \CO{provide/act as a stimulus; stimulus for/to}
            \end{ExplainCard}
            \begin{ExplainCard}{pull out all the stops}[idiom][B2]
            \EN{to make every possible effort to achieve something.}
            \SY{go all out; spare no effort}
            \VI{\textit{dốc toàn lực}.}
            \EX{They \textbf{pulled out all the stops} for the final.}
            \EX{Universities \textbf{pull out all the stops} during accreditation cycles.}
            \CO{really/always pull out all the stops}
            \end{ExplainCard}
            \begin{ExplainCard}{fall behind}[phr.v][B2]
            \EN{to fail to keep up with others in progress or achievement.}
            \SY{lag; trail; be left behind}
            \VI{\textit{tụt lại phía sau}.}
            \EX{He \textbf{fell behind} in maths last term.}
            \EX{Regions \textbf{fall behind} when investment declines.}
            \CO{fall behind in/with; fall behind peers}
            \end{ExplainCard}
            \begin{ExplainCard}{unsatisfactory}[adj][B2]
            \EN{not good enough; not meeting expectations.}
            \SY{subpar; inadequate; poor}
            \VI{\textit{không đạt yêu cầu}.}
            \EX{Her report was \textbf{unsatisfactory}.}
            \EX{\textbf{Unsatisfactory} outcomes triggered a program review.}
            \CO{unsatisfactory result/performance/explanation}
            \end{ExplainCard}
            \begin{ExplainCard}{outperform}[v][B2]
            \EN{to do better than someone or something else.}
            \SY{excel; outstrip; surpass}
            \VI{\textit{vượt trội}, làm tốt hơn.}
            \EX{Our girls’ team \textbf{outperformed} rivals.}
            \EX{New models \textbf{outperform} baselines across metrics.}
            \CO{outperform competitors/benchmarks/peers}
            \end{ExplainCard}
            \begin{ExplainCard}{reap (the rewards)}[v][B2]
            \EN{to receive something good as a result of actions.}
            \SY{gain; harvest; obtain}
            \VI{\textit{gặt hái} (thành quả).}
            \EX{Study hard now to \textbf{reap the rewards}.}
            \EX{Firms \textbf{reap} productivity gains from training investments.}
            \CO{reap benefits/rewards/dividends}
            \end{ExplainCard}
            \begin{ExplainCard}{dedication}[n][C1]
            \EN{the quality of being committed to a task or purpose.}
            \SY{devotion; commitment; perseverance}
            \VI{\textit{sự cống hiến}, tận tâm.}
            \EX{Her \textbf{dedication} inspired the team.}
            \EX{\textbf{Dedication} is a predictor of long-term expertise acquisition.}
            \CO{show/recognize dedication; dedication to}
            \end{ExplainCard}
            \begin{ExplainCard}{lenient}[adj][C1]
            \EN{not strict; tolerant and permissive.}
            \SY{soft; forgiving; indulgent}
            \VI{\textit{dễ dãi}, khoan dung.}
            \EX{The coach is quite \textbf{lenient} on late arrivals.}
            \EX{\textbf{Lenient} grading can distort performance indicators.}
            \CO{be lenient with/towards; lenient policy}
            \end{ExplainCard}
            \begin{ExplainCard}{put one’s feet up}[idiom][B2]
            \EN{to relax, especially by sitting or lying down.}
            \SY{unwind; take it easy; rest}
            \VI{\textit{nghỉ ngơi, thư giãn}.}
            \EX{After the match, we \textbf{put our feet up}.}
            \EX{Short recovery windows help athletes \textbf{put their feet up} between sessions.}
            \CO{just/ finally put your feet up}
            \end{ExplainCard}
            \begin{ExplainCard}{engage in}[v][B2]
            \EN{to take part in or do an activity.}
            \SY{participate in; involve oneself in}
            \VI{\textit{tham gia vào}.}
            \EX{Kids should \textbf{engage in} sports daily.}
            \EX{Students \textbf{engage in} collaborative inquiry for deeper learning.}
            \CO{engage in activities/discussion/practice}
            \end{ExplainCard}
            \begin{ExplainCard}{compulsory}[adj][C1]
            \EN{required by rules or law; mandatory.}
            \SY{mandatory; obligatory; required}
            \VI{\textit{bắt buộc}.}
            \EX{PE is \textbf{compulsory} at our school.}
            \EX{\textbf{Compulsory} modules ensure minimum competency standards.}
            \CO{compulsory course/attendance/education}
            \end{ExplainCard}
            \begin{ExplainCard}{coaching}[n][B2]
            \EN{training or instruction to improve skills and performance.}
            \SY{tuition; mentoring; instruction}
            \VI{\textit{huấn luyện}.}
            \EX{He pays for private \textbf{coaching}.}
            \EX{\textbf{Coaching} interventions improve team efficiency in meta-analyses.}
            \CO{coaching session/staff/program}
            \end{ExplainCard}
            \begin{ExplainCard}{sufficient}[adj][C1]
            \EN{enough for a particular purpose.}
            \SY{adequate; ample; satisfactory}
            \VI{\textit{đủ}.}
            \EX{We didn’t have \textbf{sufficient} time to warm up.}
            \EX{\textbf{Sufficient} sample sizes increase statistical power.}
            \CO{sufficient time/resources/evidence}
            \end{ExplainCard}
            \begin{ExplainCard}{carve out (a) career}[phr.v][B2]
            \EN{to succeed in achieving a career through effort.}
            \SY{forge; build; shape}
            \VI{\textit{gầy dựng} sự nghiệp.}
            \EX{She \textbf{carved out a career} in design.}
            \EX{Athletes can \textbf{carve out stellar careers} via early specialization.}
            \CO{carve out a niche/career/path}
            \end{ExplainCard}
            \begin{ExplainCard}{(follow) safety procedures}[n][B2]
            \EN{official steps to keep people from harm.}
            \SY{protocols; guidelines; safeguards}
            \VI{\textit{quy trình an toàn}.}
            \EX{Always \textbf{follow safety procedures} in the gym.}
            \EX{Noncompliance with \textbf{safety procedures} increases injury incidence.}
            \CO{follow/violate safety procedures; strict procedures}
            \end{ExplainCard}
            \begin{ExplainCard}{horrendous}[adj][B2]
            \EN{extremely unpleasant or shocking.}
            \SY{appalling; terrible; dreadful}
            \VI{\textit{khủng khiếp}.}
            \EX{He suffered a \textbf{horrendous} fall.}
            \EX{\textbf{Horrendous} injuries can end elite careers prematurely.}
            \CO{horrendous injury/accident/conditions}
            \end{ExplainCard}
            \begin{ExplainCard}{drill}[v][B2]
            \EN{to train repeatedly to improve a skill.}
            \SY{rehearse; practice intensively; coach}
            \VI{\textit{rèn luyện}, luyện tập kỹ.}
            \EX{They \textbf{drilled} free kicks for an hour.}
            \EX{Teams were \textbf{drilled} in set plays to raise efficiency.}
            \CO{drill students/players; drill techniques}
            \end{ExplainCard}
            \begin{ExplainCard}{gauge}[v][B2]
            \EN{to measure or judge something, especially ability or reaction.}
            \SY{assess; evaluate; measure}
            \VI{\textit{đánh giá}, đo lường.}
            \EX{It’s hard to \textbf{gauge} the crowd’s mood.}
            \EX{We \textbf{gauged} performance using standardized tests.}
            \CO{gauge ability/impact/progress}
            \end{ExplainCard}
            \begin{ExplainCard}{stamina}[n][C1]
            \EN{the physical or mental strength to keep going.}
            \SY{endurance; staying power}
            \VI{\textit{sức bền}.}
            \EX{Marathons demand huge \textbf{stamina}.}
            \EX{Aerobic training significantly improves \textbf{stamina} indices.}
            \CO{build/improve stamina; stamina training}
            \end{ExplainCard}
            \begin{ExplainCard}{athletic ability}[n][C1]
            \EN{natural or developed physical skill in sports.}
            \SY{athleticism; physical aptitude}
            \VI{\textit{năng lực thể thao}.}
            \EX{Her \textbf{athletic ability} stood out at trials.}
            \EX{\textbf{Athletic ability} correlates with coordination and speed measures.}
            \CO{display/develop athletic ability}
            \end{ExplainCard}
            \begin{ExplainCard}{occupational injury}[n][B2]
            \EN{an injury occurring in the course of work or activity.}
            \SY{work-related injury; job injury}
            \VI{\textit{chấn thương nghề nghiệp}.}
            \EX{Back pain is a common \textbf{occupational injury}.}
            \EX{Sports coaching carries elevated \textbf{occupational injury} risks.}
            \CO{prevent/report occupational injuries}
            \end{ExplainCard}
            \begin{ExplainCard}{put a strain on}[phrase][B2]
            \EN{to place stress or pressure on someone or something.}
            \SY{overburden; tax; stress}
            \VI{\textit{gây áp lực} lên.}
            \EX{Tight schedules \textbf{put a strain on} players.}
            \EX{Frequent travel \textbf{puts a strain on} athlete recovery systems.}
            \CO{put a strain on resources/relationships/health}
            \end{ExplainCard}
            \begin{ExplainCard}{in the cards}[idiom][B2]
            \EN{likely or possible to happen.}
            \SY{on the cards (BrE); likely; in prospect}
            \VI{\textit{có khả năng xảy ra}.}
            \EX{A rematch is \textbf{in the cards}.}
            \EX{Further regulation seems \textbf{in the cards} given current trends.}
            \CO{be/looks in the cards}
            \end{ExplainCard}
            \begin{ExplainCard}{tweak}[v][B2]
            \EN{to make small adjustments to improve something.}
            \SY{fine-tune; adjust; refine}
            \VI{\textit{chỉnh nhẹ}, tinh chỉnh.}
            \EX{She \textbf{tweaked} her routine before finals.}
            \EX{We \textbf{tweaked} the model to reduce error rates.}
            \CO{tweak a plan/design/performance}
            \end{ExplainCard}
            \begin{ExplainCard}{counterproductive}[adj][C1]
            \EN{having the opposite effect to what was intended.}
            \SY{self-defeating; adverse; backfiring}
            \VI{\textit{phản tác dụng}.}
            \EX{Too much pressure is \textbf{counterproductive}.}
            \EX{Overtraining is \textbf{counterproductive} for long-term performance.}
            \CO{prove/become counterproductive; counterproductive strategy}
            \end{ExplainCard}
            \begin{ExplainCard}{psychological}[adj][C1]
            \EN{relating to the mind or mental processes.}
            \SY{mental; cognitive; emotional}
            \VI{\textit{thuộc tâm lý}.}
            \EX{They faced \textbf{psychological} hurdles before competing.}
            \EX{\textbf{Psychological} stress impairs recovery and sleep quality.}
            \CO{psychological stress/factors/effects}
            \end{ExplainCard}
            \begin{ExplainCard}{rival}[n/adj][C1]
            \EN{a person or team competing with another for the same objective.}
            \SY{competitor; opponent; adversary}
            \VI{\textit{đối thủ}; đối địch.}
            \EX{We play our local \textbf{rivals} next week.}
            \EX{\textbf{Rival} teams exhibit distinct tactical profiles in analysis.}
            \CO{arch/close rival; beat/face a rival}
            \end{ExplainCard}
        \end{VocabExplain}

    \begin{VocabHighlights}
            \VH{to trace back to}{(phr. v) to derive or originate from}{(cụm động từ) bắt nguồn từ}
            \VH{harsh}{(adj) cruel, severe and unkind}{(tính từ) khắc nghiệt}
            \VH{elusive}{(adj) difficult to find, define or achieve}{(tính từ) khó mà đạt được}
            \VH{muggy}{(adj) (weather) unpleasantly warm and damp}{(tính từ) (thời tiết) nóng ẩm gây khó chịu}
            \VH{to grab the opportunity}{(phrase) to seize the opportunity}{(cụm từ) nắm bắt lấy cơ hội}
            \VH{to bundle up}{(v) to put warm clothes or coverings on somebody}{(động từ) khoác áo ấm lên người}
            \VH{to the best of my belief}{(phrase) as far as i know}{(cụm từ) theo tôi biết}
            \VH{a cold snap}{(phrase) a sudden, brief spell of cold weather}{(cụm từ) 1 đợt lạnh bất chợt}
            \VH{a fortnight}{(n) a period of two weeks}{(danh từ) 2 tuần}
            \VH{to plummet}{(v) to fall suddenly and quickly from a high level or position}{(động từ) giảm đột ngột}
            \VH{mild}{(adj) not severe or strong}{(tính từ) dịu nhẹ}
            \VH{sultry}{(adj) very hot and uncomfortable}{(tính từ) nóng nực, khó chịu}
            \VH{to create favorable conditions for}{(phrase) to help something develop}{(cụm từ) tạo điều kiện thuận lợi cho}
            \VH{to go rampant}{(phrase) to exist or spread everywhere in a way that cannot be controlled}{(cụm từ) tràn lan mọi nơi}
            \VH{susceptibility}{(n) the state of being very likely to be influenced, harmed or affected by something}{(danh từ) sự dễ bị, dễ mắc}
            \VH{a couple of}{(phrase) a small number of things}{(cụm từ) 1 vài}
            \VH{to go in for}{(phr.v) take part in}{(cụm động từ) tham gia}
            \VH{by chance}{(idiom) unexpectedly}{(thành ngữ) tình cờ}
            \VH{to catch somebody’s attention}{(phrase) to cause one to become interested in something}{(cụm từ) thu hút sự chú ý của ai}
            \VH{to harbor a dream of}{(phrase) have a dream of}{(cụm từ) ấp ủ một giấc mơ}
            \VH{to pursue}{(v) to follow someone or something, usually to try to catch them}{(động từ) theo đuổi}
            \VH{leading}{(adj) very important or most important}{(tính từ) hàng đầu}
            \VH{to come up with}{(phr.v) to think of a plan, an idea, or a solution to a problem}{(cụm động từ) nảy ra một ý kiến}
            \VH{to think outside the box}{(idiom) to use new ideas instead of traditional ideas when you think about something}{(thành ngữ) sáng tạo, suy nghĩ vượt khuôn khổ}
            \VH{jury}{(n) a group of people who decides who is the winner of a competition}{(cụm từ) ban giám khảo}
            \VH{to engage in}{(v) to take part in something}{(động từ) tham gia}
            \VH{platform}{(n) a standard for the hardware of a computer system, which determines what kinds of software it can run}{(danh từ) nền tảng}
            \VH{aspiring}{(adj) directing one’s hopes or ambitions towards becoming a specified type of person}{(tính từ) khao khát}
            \VH{to brush up on (skills)}{(phr.v) to improve your knowledge of something already learned but partly forgotten}{(cụm động từ) rèn luyện kĩ năng}
            \VH{to accumulate}{(v) to collect a large number of things over a long period of time}{(động từ) tích lũy}
            \VH{hands-on}{(adj) relating to, being, or providing direct practical experience in the operation or functioning of something}{(tính từ) thực tế}
            \VH{gather momentum}{(phrase) the force that keeps an object moving or keeps an event developing after it has started}{(cụm từ) có thêm động lực}
            \VH{peer pressure}{(phrase) influence from members of one’s peer group}{(cụm từ) áp lực cạnh tranh với những người ngang hàng}
            \VH{a stimulus}{(n) something that causes growth or activity}{(danh từ) động lực}
            \VH{to pull out all the stops}{(phrase) to make a very great effort to achieve something}{(cụm từ) bất chấp thử thách vượt qua}
            \VH{to fall behind}{(phr.v) fail to keep up with one’s competitors}{(cụm động từ) tụt lại đằng sau}
            \VH{unsatisfactory}{(adj) unacceptable because poor or not good enough}{(tính từ) không hài lòng}
            \VH{to outperform}{(v) to achieve better results than somebody/something}{(động từ) làm tốt hơn, xuất sắc hơn}
            \VH{to reap}{(v) to obtain something, especially something good, as a direct result of something that you have done}{(động từ) thu được kết quả gì}
            \VH{dedication}{(n) the hard work and effort that somebody puts into an activity or a purpose because they think it is important}{(danh từ) sự tận tụy}
            \VH{attempt}{(n) an act of trying to do something, especially something difficult, often with no success}{(danh từ) cố gắng, nỗ lực}
            \VH{a compliment}{(n) a remark that expresses praise or admiration of somebody}{(danh từ) lời khen ngợi}
            \VH{arrogant}{(adj) having or revealing an exaggerated sense of one’s own importance or abilities}{(tính từ) tự kiêu, kiêu ngạo}
            \VH{lenient}{(adj) not as strict as expected when punishing somebody or when making sure that rules are obeyed}{(tính từ) nhân hậu, khoan dung}
            \VH{to put somebody’s feet up}{(idiom) to relax}{(thành ngữ) thư giãn, xả hơi}
            \VH{principle}{(n) a moral rule or a strong belief that influences your actions}{(danh từ) nguyên lý, quy tắc}
            \VH{to engage in}{(v) to become involved, or have contact, with someone or something}{(động từ) tham gia vào cái gì}
            \VH{compulsory}{(adj) that must be done because of a law or a rule}{(tính từ) bắt buộc}
            \VH{coaching}{(n) the process of training somebody to play a sport, to do a job better or to improve a skill}{(danh từ) sự huấn luyện}
            \VH{sufficient}{(adj) enough for a particular purpose; as much as you need}{(tính từ) đầy đủ}
            \VH{to carve out stellar careers}{(phrase) to make steer careers}{(cụm từ) tạo dựng sự nghiệp vững chắc}
            \VH{safety ground procedures}{(phrase) the principles that make sure the participants can be safe when they involve in}{(cụm từ) quy trình đảm bảo an toàn}
            \VH{horrendous}{(adj) extremely unpleasant, horrifying, or terrible}{(tính từ) khủng khiếp}
            \VH{to drill}{(v) to teach somebody to do something by making them repeat it a lot of times}{(động từ) luyện tập bền bỉ}
            \VH{to gauge}{(v) using to estimate or judge something}{(động từ) ước lượng; đánh giá}
            \VH{stamina}{(n) the physical or mental strength that enables you to do something difficult for long periods of time}{(danh từ) khả năng chịu đựng; rắn rỏi}
            \VH{athletic ability}{(phrase) physically active and strong; good at athletics or sports}{(cụm từ) khả năng thể lực}
            \VH{occupational injuries}{(phrase) any personal injury, disease or death resulting from an occupational accident}{(cụm từ) chấn thương, tai nạn nghề nghiệp}
            \VH{to put a strain on}{(phrase) to put pressure on somebody/something}{(cụm từ) gây căng thẳng, đặt áp lực lên}
            \VH{in the cards}{(phrase) very possible or likely}{(cụm từ) có khả năng}
            \VH{to tweak}{(v) to change something slightly, especially in order to make it more correct, effective, or suitable}{(động từ) cải thiện}
            \VH{counterproductive}{(adj) having the opposite effect to the one which was intended}{(tính từ) phản tác dụng}
            \VH{psychological}{(adj) connected with a person’s mind and the way in which it works}{(tính từ) thuộc về tâm lý học}
            \VH{rival}{(n) a person, company, product, etc. competing with others for the same thing or in the same area}{(danh từ) đối thủ}
    \end{VocabHighlights}
    \end{test}

    \begin{test}{TEST 4}
    \noindent
    \textbf{Part 1. Travel to work or college}
    \begin{qa}{How do you usually travel to work or college? (Why?)}
    I usually drive to work. You see, the \textbf{sultry} yet wild weather in summer and winter, respectively, not to mention \textbf{intermittent} rains or \textbf{scattered} showers, can have \textbf{adverse} effects on my overall health. That's why going to work in cars in \textbf{shield me} from any external factors and indirectly contribute to increased productivity at my workplace.
    \end{qa}

    \begin{qa}{Have you always travelled to work/college in the same way? Why/Why not?}
    No, I haven't. Several years ago, I could \textbf{resort to} motorbike when travelling to work. Nevertheless, once I \textbf{got hitched}, I realized the importance of purchasing a car to protect my family from the \textbf{inhospitable} climate in Hanoi. I \textbf{went to great lengths} to \textbf{accrue} a certain amount of money and 2 years ago, I managed to buy one.
    \end{qa}

    \begin{qa}{What do you like about travelling to work/college this way?}
    Travelling this way does \textbf{yield} certain benefits beyond my expectation. Besides protecting myself and my family from the extreme climate, music played in my car also \textbf{puts me at ease}. Moreover, after lunch at work, spared from the distractions and noises that other colleagues might cause in office rooms, I can lie down in my car and \textbf{catch forty winks} to restore energy before heading back to work again in the afternoon.
    \end{qa}

    \begin{qa}{What changes would improve the way you travel to work/ college? (Why?)}
    \textbf{I hold the belief} that changes in the traffic system will certainly improve how I travel to work. Firstly, a number of \textbf{flyovers} have been constructed in Hanoi, lessening the traffic burdens at several intersections and allowing me to drive faster as a result. Secondly, some streets have been expanded to accommodate more passengers. It might also \textbf{alleviate} traffic congestion to some extent and assist me in getting to the workplace faster than usual.
    \end{qa}

    \begin{VocabExplain}[Part 1]
        \begin{ExplainCard}{sultry}[adj][B2]
            \EN{uncomfortably hot and humid.}
            \SY{muggy; sweltering; stifling}
            \VI{\textit{oi bức}, ẩm nóng.}
            \EX{It was a \textbf{sultry} evening, so I stayed indoors.}
            \EX{\textbf{Sultry} conditions are associated with higher heat-stress incidences in cities.}
            \CO{sultry weather/night/air}
        \end{ExplainCard}

        \begin{ExplainCard}{intermittent}[adj][B2]
            \EN{stopping and starting at irregular intervals.}
            \SY{sporadic; occasional; periodic}
            \VI{\textit{gián đoạn}, lúc có lúc không.}
            \EX{We had \textbf{intermittent} rain all morning.}
            \EX{\textbf{Intermittent} exposure to noise negatively affects commute satisfaction.}
            \CO{intermittent rain/service/episodes}
        \end{ExplainCard}

        \begin{ExplainCard}{scattered (showers)}[adj][B2]
            \EN{occurring over a wide area but not everywhere; irregularly distributed.}
            \SY{patchy; spotty; sporadic}
            \VI{\textit{rải rác} (mưa rào).}
            \EX{Forecast says \textbf{scattered} showers this afternoon.}
            \EX{\textbf{Scattered} precipitation complicates short-term traffic planning.}
            \CO{scattered showers/storms/clouds}
        \end{ExplainCard}

        \begin{ExplainCard}{adverse}[adj][B2]
            \EN{harmful or unfavorable; preventing success or development.}
            \SY{detrimental; negative; unfavorable}
            \VI{\textit{bất lợi}, có hại.}
            \EX{Driving in \textbf{adverse} weather makes me nervous.}
            \EX{\textbf{Adverse} conditions significantly increase accident risk during peak hours.}
            \CO{adverse effects/impact/conditions}
        \end{ExplainCard}

        \begin{ExplainCard}{shield (sb) from}[v][B2]
            \EN{to protect someone from danger or unpleasant influence.}
            \SY{protect; guard; insulate}
            \VI{\textit{che chở}, bảo vệ (khỏi).}
            \EX{A good coat \textbf{shields me from} the wind.}
            \EX{Cabin filtration \textbf{shields occupants from} fine particulates on busy roads.}
            \CO{shield from weather/risk/noise}
        \end{ExplainCard}

        \begin{ExplainCard}{resort to}[phr.v][B2]
            \EN{to use a less desirable option when nothing else is possible.}
            \SY{turn to; fall back on; make use of}
            \VI{\textit{phải dùng đến}, trông cậy vào.}
            \EX{During the strike we \textbf{resorted to} taxis.}
            \EX{Commuters often \textbf{resort to} informal transit when networks are disrupted.}
            \CO{resort to measures/borrowing/shortcuts}
        \end{ExplainCard}

        \begin{ExplainCard}{get hitched}[idiom][B2]
            \EN{to get married (informal).}
            \SY{tie the knot; marry}
            \VI{\textit{kết hôn} (khẩu ngữ).}
            \EX{They \textbf{got hitched} last spring.}
            \EX{Household travel patterns usually change after couples \textbf{get hitched}.}
            \CO{newly/recently hitched}
        \end{ExplainCard}

        \begin{ExplainCard}{inhospitable (climate)}[adj][B2]
            \EN{harsh and difficult to live in.}
            \SY{hostile; severe; unwelcoming}
            \VI{\textit{khắc nghiệt}, khó sống (khí hậu).}
            \EX{The desert’s \textbf{inhospitable} climate deterred us.}
            \EX{\textbf{Inhospitable} climates drive higher private-vehicle dependence.}
            \CO{inhospitable climate/terrain/conditions}
        \end{ExplainCard}

        \begin{ExplainCard}{go to great lengths}[idiom][B2]
            \EN{to make a great effort to achieve something.}
            \SY{spare no effort; make every effort}
            \VI{\textit{dốc sức}, làm mọi cách.}
            \EX{He \textbf{went to great lengths} to find parking near work.}
            \EX{Governments \textbf{go to great lengths} to cut peak congestion.}
            \CO{go to great/considerable lengths to + V}
        \end{ExplainCard}

        \begin{ExplainCard}{accrue}[v][B2]
            \EN{to accumulate or be received over time.}
            \SY{amass; accumulate; build up}
            \VI{\textit{tích lũy}, dồn lại.}
            \EX{I \textbf{accrued} enough savings to buy a car.}
            \EX{Travel-time benefits \textbf{accrue} after corridor upgrades.}
            \CO{accrue interest/benefits/savings}
        \end{ExplainCard}

        \begin{ExplainCard}{yield}[v][B2]
            \EN{(1) to produce or provide (a result/benefit). (2) to give way (traffic).}
            \SY{(1) generate; deliver \quad (2) cede; give way}
            \VI{(1) \textit{mang lại}; (2) \textit{nhường đường}.}
            \EX{Carpooling \textbf{yields} real cost savings.}
            \EX{Vehicles must \textbf{yield} at the roundabout to improve flow.}
            \CO{yield results/benefits; yield to traffic}
        \end{ExplainCard}

        \begin{ExplainCard}{put (sb) at ease}[phrase][B2]
            \EN{to make someone feel relaxed and comfortable.}
            \SY{calm; reassure; soothe}
            \VI{\textit{làm ai yên tâm}, thoải mái.}
            \EX{Soft music \textbf{puts me at ease} while driving.}
            \EX{Clear wayfinding \textbf{puts passengers at ease} during diversions.}
            \CO{put clients/patients/riders at ease}
        \end{ExplainCard}

        \begin{ExplainCard}{catch forty winks}[idiom][B2]
            \EN{to take a short nap.}
            \SY{doze; power-nap; catnap}
            \VI{\textit{ngủ chợp mắt một lát}.}
            \EX{I \textbf{catch forty winks} in the car at lunch.}
            \EX{Brief naps help drivers \textbf{catch forty winks} and restore alertness.}
            \CO{quickly/just catch forty winks}
        \end{ExplainCard}

        \begin{ExplainCard}{I hold the belief (that)}[phrase][B2]
            \EN{I am firmly convinced that; I believe.}
            \SY{I maintain; I am convinced; I contend}
            \VI{\textit{tôi cho rằng}/\textit{tôi tin rằng}.}
            \EX{\textbf{I hold the belief} that car-sharing cuts emissions.}
            \EX{\textbf{We hold the belief} that pricing reforms improve throughput.}
            \CO{hold the belief/view/opinion that}
        \end{ExplainCard}

        \begin{ExplainCard}{flyover}[n][B1]
            \EN{a bridge that carries a road over another road; overpass.}
            \SY{overpass; viaduct}
            \VI{\textit{cầu vượt}.}
            \EX{The new \textbf{flyover} shortened my commute.}
            \EX{\textbf{Flyovers} redistribute traffic at congested intersections.}
            \CO{build/close/use a flyover}
        \end{ExplainCard}

        \begin{ExplainCard}{alleviate}[v][B2]
            \EN{to make a problem or suffering less severe.}
            \SY{ease; mitigate; reduce}
            \VI{\textit{xoa dịu}, giảm nhẹ.}
            \EX{Dedicated bus lanes \textbf{alleviate} rush-hour jams.}
            \EX{Demand-management policies \textbf{alleviate} peak-period congestion externalities.}
            \CO{alleviate congestion/pressure/pain}
        \end{ExplainCard}
    \end{VocabExplain}

    \noindent
    \textbf{Part 2.}
    \begin{qa}{Describe a piece of electronic equipment that you find useful. You should say:}
    \begin{itemize}
    \item What it is
    \item How you learned to use it
    \item How long you have had it
    \item and explain why you find this piece of electronic equipment useful.
    \end{itemize}
 
    If you ask me about a piece of electronic equipment which is of great use to me, the first thing that \textbf{pops up in my mind} is my Galaxy Book 12, which is a 2-in-1 PC laptop. In particular, it can function as a laptop when \textbf{hooked up to} its separate keyboard cover and also work as a tablet when I detach the keyboard from it as well. There was no need for me to \textbf{familiarize} myself with it as it runs on Windows 10, an operating system that has been in use for half a decade, let alone I consider myself a \textbf{computer literate} person. Truth be told, it was not a brand new one. Though it was a used product, it was still \textbf{in mint condition} so I decided to buy it \textbf{in the blink of an eye}. Last year, I \textbf{scored great deals on it} from Ebay as its asking prices were roughly half the original price tag. Throughout one year of constant use, this device has assisted me greatly in a lot of aspects. When it comes to work, it is just about 1 kilogram including the keyboard, which is lightweight enough to be carried to my workplace to draft some documents or present my lessons. It has \textbf{built-in} 4G LTE connection, enabling me to get connected even if I am \textbf{on the move} \textbf{in the absence of} Wifi. My job requires me to sync data continuously so an \textbf{integrated} 4G LTE component will certainly \textbf{do wonders} for me.
    \end{qa}

        \begin{VocabExplain}[Part 2]
            \begin{ExplainCard}{pop up in one’s mind}[phrase][B2]
                \EN{to appear suddenly as a thought or idea.}
                \SY{spring to mind; occur to; come to mind}
                \VI{\textit{bất chợt nảy ra trong đầu}.}
                \EX{When you say “tablet,” my Galaxy instantly \textbf{pops up in my mind}.}
                \EX{In brainstorming, novel associations can \textbf{pop up in the mind} without deliberate search.}
                \CO{immediately/instantly pop up in one’s mind; what pops into your head}
            \end{ExplainCard}

            \begin{ExplainCard}{hook (sth) up to}[phr.v][B2]
                \EN{to connect a device to power, a network, or another device.}
                \SY{connect to; plug into; attach to}
                \VI{\textit{kết nối} (thiết bị) \textit{với}.}
                \EX{I \textbf{hooked it up to} a monitor for a bigger screen.}
                \EX{Sensors were \textbf{hooked up to} a data logger for continuous recording.}
                \CO{hook up to a network/monitor/power}
            \end{ExplainCard}

            \begin{ExplainCard}{familiarize (oneself) with}[v][B2]
                \EN{to learn about something so that you know it well.}
                \SY{acquaint; accustom; get to know}
                \VI{\textit{làm quen với}, nắm rõ.}
                \EX{New hires spend a day \textbf{familiarizing themselves with} the software.}
                \EX{Students were \textbf{familiarized with} the interface before testing began.}
                \CO{familiarize oneself with a system/procedure}
            \end{ExplainCard}

            \begin{ExplainCard}{computer-literate}[adj][B2]
                \EN{able to use computers and common software effectively.}
                \SY{tech-savvy; digitally literate; IT-literate}
                \VI{\textit{thành thạo máy tính}.}
                \EX{Most office roles require \textbf{computer-literate} staff.}
                \EX{\textbf{Computer literacy} is a baseline competency in modern curricula.}
                \CO{become/be computer-literate; computer literacy}
            \end{ExplainCard}

            \begin{ExplainCard}{in mint condition}[idiom][C1]
                \EN{in perfect, like-new condition.}
                \SY{pristine; immaculate; like new}
                \VI{\textit{như mới}, hoàn hảo.}
                \EX{I bought a used laptop \textbf{in mint condition}.}
                \EX{Collectors pay premiums for items \textbf{in mint condition}.}
                \CO{remain/stay in mint condition; a mint-condition copy}
            \end{ExplainCard}

            \begin{ExplainCard}{in the blink of an eye}[idiom][B2]
                \EN{very quickly; almost instantly.}
                \SY{in no time; in a flash; instantly}
                \VI{\textit{trong nháy mắt}, rất nhanh.}
                \EX{The app loads \textbf{in the blink of an eye}.}
                \EX{With SSDs, file retrieval occurs \textbf{in the blink of an eye}.}
                \CO{happen/change/finish in the blink of an eye}
            \end{ExplainCard}

            \begin{ExplainCard}{score a great deal (on)}[phrase][B2]
                \EN{to obtain something at a very good price.}
                \SY{snag; land; nab a bargain}
                \VI{\textit{săn được giá hời}.}
                \EX{She \textbf{scored a great deal on} that used tablet.}
                \EX{During clearance sales, consumers often \textbf{score great deals on} last-gen devices.}
                \CO{score a deal/bargain/discount on}
            \end{ExplainCard}

            \begin{ExplainCard}{built-in}[adj][B1]
                \EN{included as an integral, permanent part of a device.}
                \SY{integrated; embedded; inbuilt}
                \VI{\textit{tích hợp sẵn}.}
                \EX{My laptop has a \textbf{built-in} webcam.}
                \EX{\textbf{Built-in} LTE modules improve connectivity for fieldwork.}
                \CO{built-in camera/battery/modem; built-in feature}
            \end{ExplainCard}

            \begin{ExplainCard}{on the move}[phrase][B1]
                \EN{while travelling or being active away from a fixed place.}
                \SY{on the go; mobile}
                \VI{\textit{đang di chuyển}, đang bận rộn.}
                \EX{I answer emails \textbf{on the move}.}
                \EX{Cloud sync lets researchers access notes \textbf{on the move}.}
                \CO{work/connect on the move; people on the move}
            \end{ExplainCard}

            \begin{ExplainCard}{in the absence of}[phrase][B2]
                \EN{when something is not present or available.}
                \SY{without; lacking; in default of}
                \VI{\textit{khi thiếu/vắng}, \textit{trong trường hợp không có}.}
                \EX{Use mobile data \textbf{in the absence of} Wi-Fi.}
                \EX{\textbf{In the absence of} evidence, the claim remains tentative.}
                \CO{in the absence of Wi-Fi/data/evidence}
            \end{ExplainCard}

            \begin{ExplainCard}{integrated}[adj][C1]
                \EN{combined into a whole so parts work together seamlessly.}
                \SY{unified; consolidated; embedded}
                \VI{\textit{tích hợp}, đồng bộ.}
                \EX{An \textbf{integrated} LTE modem keeps me online outdoors.}
                \EX{\textbf{Integrated} systems reduce latency across the stack.}
                \CO{integrated module/system/solution}
            \end{ExplainCard}

            \begin{ExplainCard}{do wonders (for)}[idiom][B2]
                \EN{to have a very beneficial effect.}
                \SY{work miracles; help a lot; make a big difference}
                \VI{\textit{có tác dụng kỳ diệu}/rất có lợi.}
                \EX{A bigger battery would \textbf{do wonders for} travel days.}
                \EX{Automation \textbf{does wonders for} data accuracy and throughput.}
                \CO{do wonders for health/productivity/performance}
            \end{ExplainCard}
        \end{VocabExplain}

    \noindent
    \textbf{Part 3.}
    \begin{qa}{What kinds of machine are used for housework in modern homes in your country?}
    Speaking of \textbf{domestic appliances}, I guess every household would be \textbf{in possession} of common ones are vacuum cleaner, dishwasher, refrigerator and oven, \textbf{you name it}. However, \textbf{middle-class} or \textbf{underprivileged} families may not have many electrical machines like these. A cooker and a fridge are two basic and most useful appliances in every kitchen, I suppose.
    \end{qa}

    \begin{qa}{How have these machines benefited people? Are there any negative effects of using them?}
    Obviously, most of home appliances are really true \textbf{life-savers}. They make household chores become \textbf{a breeze} and housewives can save an \textbf{inordinate} amount of time and money. For example, they can store \textbf{leftovers} in the fridge to eat later. But, these machines consume lots of energy and sometimes release unwanted \textbf{substances} like CO\textsubscript{2} through the environment, which in turn may inflict damage on people's life.
    \end{qa}

    \begin{qa}{Do you think all new homes will be equipped with household machines in the future? Why?}
    No, not really. In most \textbf{metropolises}, \textbf{furnished} houses are commonplace as city life is getting busier, and working adults have no choices but resort to household appliances to carry out domestic tasks. In spite of this, well-equipped houses might be \textbf{a rarity} in mountainous and remote areas as not all people \textbf{living from hand to mouth} can afford the machines. Therefore, almost household chores are done \textbf{manually}.
    \end{qa}

    \begin{qa}{What kinds of equipment do most workers need to use in offices today?}
    To be honest, I have been working a \textbf{desk job} since graduation, and I am pretty \textbf{content with} the facilities. Personally, \textbf{not to mention} \textbf{stationery items}, computers, fax and printers will be used on a regular basis. The equipment has \textbf{made headway} on the \textbf{pattern} of employment. In other words, some types of \textbf{grunt work} will drop off the face of the earth or be replaced by the machine.
    \end{qa}

    \begin{qa}{How have developments in technology affected employment in your country?}
    Without a doubt, technological \textbf{progress} has revolutionized many aspects in the workplace. The \textbf{advents} of the Internet and hi-tech equipment like projectors have made \textbf{virtual meetings} \textbf{become realistic}. Everything considered, technological \textbf{advancements} has increased our productivity and \textbf{eliminated} the distance between the business and customers as well as decreased \textbf{redundancy} in our work.
    \end{qa}

    \begin{qa}{Some people think that technology has brought more stress than benefits to employed people nowadays. Would you agree or disagree? Why?}
    Well, I think the answer should be \textbf{case by case basis}. While the benefit of technology is \textbf{unarguable}, there are \textbf{hidden} drawbacks that can be anticipated. In other words, many jobs have become \textbf{obsolete} due to technology, especially in service sectors which technology-based customer services such as online marketing tools are popular. But, in my opinion, technology itself is not the problem. It is the way working adults learn and \textbf{adapt} to it in their career that matters.
    \end{qa}

        \begin{VocabExplain}[Part 3]
            \begin{ExplainCard}{domestic appliance}[n][C1]
                \EN{a machine used in the home for tasks such as cleaning, cooking, or storing food.}
                \SY{household appliance; home device}
                \VI{\textit{thiết bị gia dụng}.}
                \EX{Modern \textbf{domestic appliances} save us hours each week.}
                \EX{Ownership of \textbf{domestic appliances} correlates with time-use changes.}
                \CO{small/large domestic appliances}
            \end{ExplainCard}

            \begin{ExplainCard}{in possession (of)}[phrase][B2]
                \EN{having or owning something.}
                \SY{holding; owning}
                \VI{\textit{sở hữu}, \textit{có}.}
                \EX{Most homes are \textbf{in possession of} a fridge.}
                \EX{Firms \textbf{in possession of} key patents gain advantage.}
                \CO{be in possession of; take possession}
            \end{ExplainCard}

            \begin{ExplainCard}{you name it}[idiom][B2]
                \EN{used after a list to mean “and anything else you can think of.”}
                \SY{and so on; etc.}
                \VI{\textit{vân vân}, cái gì cũng có.}
                \EX{Vacuum, oven, dishwasher—\textbf{you name it}.}
                \EX{The dataset includes age, income, education—\textbf{you name it}.}
                \CO{—you name it (sentence-final)}
            \end{ExplainCard}

            \begin{ExplainCard}{middle-class}[adj][B1]
                \EN{belonging to the social group between the upper and working classes.}
                \SY{bourgeois; suburban}
                \VI{\textit{trung lưu}.}
                \EX{A \textbf{middle-class} family may buy multiple gadgets.}
                \EX{\textbf{Middle-class} households drive appliance adoption.}
                \CO{middle-class family/neighbourhood}
            \end{ExplainCard}

            \begin{ExplainCard}{underprivileged}[adj][B1]
                \EN{lacking basic social or economic advantages.}
                \SY{disadvantaged; deprived}
                \VI{\textit{thiệt thòi}.}
                \EX{\textbf{Underprivileged} families often share devices.}
                \EX{\textbf{Underprivileged} groups face digital divides.}
                \CO{underprivileged children/areas}
            \end{ExplainCard}

            \begin{ExplainCard}{life-saver}[n][B1]
                \EN{something that provides crucial help in a difficult situation.}
                \SY{godsend; boon}
                \VI{\textit{cứu cánh}.}
                \EX{The dishwasher is a \textbf{life-saver} on busy days.}
                \EX{Remote access proved a \textbf{life-saver} during lockdowns.}
                \CO{real/absolute life-saver}
            \end{ExplainCard}

            \begin{ExplainCard}{a breeze}[idiom][B1]
                \EN{very easy to do.}
                \SY{a cinch; effortless}
                \VI{\textit{dễ như chơi}.}
                \EX{This app makes budgeting \textbf{a breeze}.}
                \EX{Automation renders routine reporting \textbf{a breeze}.}
                \CO{be/feel a breeze; make sth a breeze}
            \end{ExplainCard}

            \begin{ExplainCard}{inordinate}[adj][B2]
                \EN{much more than is usual or reasonable.}
                \SY{excessive; disproportionate}
                \VI{\textit{quá mức}, \textit{quá nhiều}.}
                \EX{They spent an \textbf{inordinate} amount on gadgets.}
                \EX{\textbf{Inordinate} delays undermine service quality.}
                \CO{inordinate amount/number/delay}
            \end{ExplainCard}

            \begin{ExplainCard}{leftovers}[n][B1]
                \EN{food remaining after a meal.}
                \SY{remnants; remains}
                \VI{\textit{đồ ăn thừa}.}
                \EX{We saved the \textbf{leftovers} for lunch.}
                \EX{\textbf{Leftovers} reduce waste when stored properly.}
                \CO{save/eat/reheat leftovers}
            \end{ExplainCard}

            \begin{ExplainCard}{substance}[n][C1]
                \EN{a particular kind of matter with uniform properties.}
                \SY{material; compound}
                \VI{\textit{chất}, \textit{chất liệu}.}
                \EX{Some sprays release harmful \textbf{substances}.}
                \EX{CO\textsubscript{2} is a greenhouse \textbf{substance} emitted by devices.}
                \CO{toxic/chemical/trace substances}
            \end{ExplainCard}

            \begin{ExplainCard}{metropolis}[n][C1]
                \EN{a very large and important city.}
                \SY{mega-city; urban center}
                \VI{\textit{đô thị lớn}.}
                \EX{Appliance ownership is higher in \textbf{metropolises}.}
                \EX{\textbf{Metropolis} size predicts infrastructure demand.}
                \CO{global/regional metropolis}
            \end{ExplainCard}

            \begin{ExplainCard}{furnished}[adj][B2]
                \EN{equipped with furniture and basic household items.}
                \SY{equipped; fitted out}
                \VI{\textit{có sẵn nội thất}.}
                \EX{We rented a \textbf{furnished} apartment.}
                \EX{\textbf{Furnished} units command higher rents in cities.}
                \CO{fully/partly furnished apartment}
            \end{ExplainCard}

            \begin{ExplainCard}{a rarity}[n][C1]
                \EN{something uncommon or unusual.}
                \SY{scarcity; uncommonness}
                \VI{\textit{sự hiếm hoi}.}
                \EX{Dishwashers are \textbf{a rarity} in rural homes.}
                \EX{High-speed fiber remains \textbf{a rarity} in remote regions.}
                \CO{become/remain a rarity}
            \end{ExplainCard}

            \begin{ExplainCard}{live from hand to mouth}[idiom][B2]
                \EN{to have just enough money to live on with no savings.}
                \SY{scrape by; subsist}
                \VI{\textit{sống giật gấu vá vai}.}
                \EX{They \textbf{live from hand to mouth} and skip luxuries.}
                \EX{Many informal workers \textbf{live from hand to mouth} between gigs.}
                \CO{families/households live from hand to mouth}
            \end{ExplainCard}

            \begin{ExplainCard}{manually}[adv][B2]
                \EN{by hand rather than by machine or automation.}
                \SY{by hand; hand-operated}
                \VI{\textit{thủ công}.}
                \EX{We wash clothes \textbf{manually} at the cabin.}
                \EX{Data entered \textbf{manually} is prone to error.}
                \CO{do sth manually; manual entry}
            \end{ExplainCard}

            \begin{ExplainCard}{desk job}[n][B2]
                \EN{work that is mainly done sitting at a desk.}
                \SY{office job; clerical job}
                \VI{\textit{công việc văn phòng}.}
                \EX{He moved from retail to a \textbf{desk job}.}
                \EX{\textbf{Desk jobs} increase sedentary time among adults.}
                \CO{have/do a desk job}
            \end{ExplainCard}

            \begin{ExplainCard}{content with}[adj][B2]
                \EN{satisfied or pleased with something.}
                \SY{satisfied; pleased}
                \VI{\textit{hài lòng với}.}
                \EX{I’m \textbf{content with} my setup.}
                \EX{Employees \textbf{content with} tools report higher productivity.}
                \CO{be/feel content with}
            \end{ExplainCard}

            \begin{ExplainCard}{not to mention}[phrase][C1]
                \EN{used to add extra information that emphasizes what you have said.}
                \SY{let alone; as well as}
                \VI{\textit{chưa kể đến}.}
                \EX{We have laptops, \textbf{not to mention} printers.}
                \EX{Automation cuts costs, \textbf{not to mention} errors.}
                \CO{not to mention + noun/-ing}
            \end{ExplainCard}

            \begin{ExplainCard}{stationery items}[n][B2]
                \EN{office supplies such as pens, paper, and envelopes.}
                \SY{office supplies; writing materials}
                \VI{\textit{đồ văn phòng phẩm}.}
                \EX{Please order \textbf{stationery items} this week.}
                \EX{Usage of \textbf{stationery items} declines with digitization.}
                \CO{basic/essential stationery items}
            \end{ExplainCard}

            \begin{ExplainCard}{make headway (on)}[v][B2]
                \EN{to make progress.}
                \SY{advance; gain ground}
                \VI{\textit{tiến bộ}, \textit{đạt tiến triển}.}
                \EX{We’re \textbf{making headway} on automation.}
                \EX{The sector \textbf{made headway} in remote-work adoption.}
                \CO{make headway on/with}
            \end{ExplainCard}

            \begin{ExplainCard}{pattern (of employment)}[n][B2]
                \EN{the typical or repeated way in which employment is arranged or occurs.}
                \SY{trend; structure; configuration}
                \VI{\textit{mô hình}/\textit{xu hướng} việc làm.}
                \EX{WFH changed the \textbf{pattern of employment}.}
                \EX{Digitalization reshapes \textbf{employment patterns} nationwide.}
                \CO{change/shape employment patterns}
            \end{ExplainCard}

            \begin{ExplainCard}{grunt work}[n][B1]
                \EN{hard, boring work that does not require special skill.}
                \SY{drudge work; menial work}
                \VI{\textit{việc lặt vặt, nhàm chán}.}
                \EX{Interns often handle the \textbf{grunt work}.}
                \EX{Scripts eliminate \textbf{grunt work} in data cleaning.}
                \CO{do/automate the grunt work}
            \end{ExplainCard}

            \begin{ExplainCard}{progress}[n][B2]
                \EN{forward or onward movement toward a goal; improvement.}
                \SY{advancement; development}
                \VI{\textit{tiến bộ}.}
                \EX{There’s been huge \textbf{progress} in AI tools.}
                \EX{\textbf{Progress} in ICT boosts productivity growth.}
                \CO{make/achieve progress; technological progress}
            \end{ExplainCard}

            \begin{ExplainCard}{advent}[n][B2]
                \EN{the arrival of a notable person, thing, or event.}
                \SY{arrival; emergence; onset}
                \VI{\textit{sự xuất hiện}.}
                \EX{The \textbf{advent} of 5G changed streaming.}
                \EX{\textbf{Advent} of broadband enabled virtual teamwork.}
                \CO{the advent of + technology/era}
            \end{ExplainCard}

            \begin{ExplainCard}{virtual meeting}[n][B2]
                \EN{a meeting held via the internet using video or audio tools.}
                \SY{online meeting; video conference}
                \VI{\textit{cuộc họp trực tuyến}.}
                \EX{We have a \textbf{virtual meeting} every Monday.}
                \EX{\textbf{Virtual meetings} reduce travel-related emissions.}
                \CO{host/join virtual meetings}
            \end{ExplainCard}

            \begin{ExplainCard}{become realistic}[v phrase][B2]
                \EN{to turn from idea to practical reality.}
                \SY{materialize; be feasible}
                \VI{\textit{trở nên khả thi/thực tế}.}
                \EX{Remote work \textbf{became realistic} after upgrades.}
                \EX{High-fidelity telepresence has \textbf{become realistic} with fiber.}
                \CO{finally/now become realistic}
            \end{ExplainCard}

            \begin{ExplainCard}{advancement}[n][C1]
                \EN{development or improvement in something.}
                \SY{progress; innovation}
                \VI{\textit{bước tiến}, \textit{sự phát triển}.}
                \EX{Medical \textbf{advancements} save lives.}
                \EX{Rapid \textbf{advancements} in automation reshape labor demand.}
                \CO{technological/scientific advancements}
            \end{ExplainCard}

            \begin{ExplainCard}{eliminate}[v][C1]
                \EN{to remove or get rid of something.}
                \SY{eradicate; remove; abolish}
                \VI{\textit{loại bỏ}.}
                \EX{Cloud backup \textbf{eliminates} USB hassles.}
                \EX{Process redesign \textbf{eliminated} redundant steps.}
                \CO{eliminate waste/errors/barriers}
            \end{ExplainCard}

            \begin{ExplainCard}{redundancy}[n][B2]
                \EN{(workplace) the state of being no longer needed; job loss due to this.}
                \SY{layoff; surplus}
                \VI{\textit{sự dư thừa}, mất việc do thừa.}
                \EX{Automation can cause \textbf{redundancy}.}
                \EX{Policies aim to cushion workers from technological \textbf{redundancy}.}
                \CO{avoid/face redundancy; redundancy risk}
            \end{ExplainCard}

            \begin{ExplainCard}{case-by-case basis}[n][B2]
                \EN{considering each situation separately.}
                \SY{individually; ad hoc}
                \VI{\textit{từng trường hợp một}.}
                \EX{Decisions are made on a \textbf{case-by-case basis}.}
                \EX{Exemptions were granted on a \textbf{case-by-case basis}.}
                \CO{assess/judge on a case-by-case basis}
            \end{ExplainCard}

            \begin{ExplainCard}{unarguable}[adj][B2]
                \EN{impossible to disagree with; undeniable.}
                \SY{indisputable; incontrovertible}
                \VI{\textit{không thể phủ nhận}.}
                \EX{The benefits are \textbf{unarguable}.}
                \EX{There is \textbf{unarguable} evidence of efficiency gains.}
                \CO{unarguable fact/benefit/evidence}
            \end{ExplainCard}

            \begin{ExplainCard}{hidden}[adj][B2]
                \EN{not easily noticed; concealed.}
                \SY{latent; covert}
                \VI{\textit{tiềm ẩn}, \textit{ẩn}.}
                \EX{Upgrades may have \textbf{hidden} costs.}
                \EX{\textbf{Hidden} biases can distort algorithmic decisions.}
                \CO{hidden cost/agenda/risk}
            \end{ExplainCard}

            \begin{ExplainCard}{obsolete}[adj][C1]
                \EN{no longer used because something newer exists.}
                \SY{outdated; outmoded}
                \VI{\textit{lỗi thời}.}
                \EX{DVD drives are largely \textbf{obsolete}.}
                \EX{Routine clerical tasks became \textbf{obsolete} post-automation.}
                \CO{become/render obsolete}
            \end{ExplainCard}

            \begin{ExplainCard}{adapt (to)}[v][B2]
                \EN{to change your behavior so that it is suitable for a new situation.}
                \SY{adjust; acclimate; modify}
                \VI{\textit{thích nghi (với)}}.
                \EX{Older workers \textbf{adapt to} new apps with training.}
                \EX{Firms \textbf{adapt to} technological shocks by reskilling.}
                \CO{adapt to change/technology; adapt quickly}
            \end{ExplainCard}
        \end{VocabExplain}

    \begin{VocabHighlights}
        \VH{intermittent}{(adj) stopping and starting often over a period of time, but not regularly}{(tính từ) không liên tục}
        \VH{scattered}{(adj) spread far apart over a wide area or over a long period of time}{(tính từ) rải rác}
        \VH{to shield}{(v) to protect somebody/something from danger, harm or something unpleasant}{(động từ) bảo vệ khỏi}
        \VH{to get hitched}{(idiom) to marry}{(thành ngữ) kết hôn}
        \VH{inhospitable}{(adj) difficult to stay or live in, especially because there is no shelter from the weather}{(tính từ) khiến khó ở}
        \VH{to go to great lengths to V-inf}{(phrase) to make effort to V-inf}{(cụm từ) cố gắng làm gì}
        \VH{to accrue}{(v) to accumulate}{(động từ) tích lũy}
        \VH{to yield}{(v) to produce or provide something, for example a profit, result or crop}{(động từ) thu được}
        \VH{to put somebody at ease}{(idiom) to make (someone) feel calm and relaxed}{(thành ngữ) làm ai thư giãn}
        \VH{to wind down}{(phr. v) to relax}{(cụm động từ) thư giãn}
        \VH{to catch forty winks}{(idiom) to take a quick snap}{(thành ngữ) chợp mắt}
        \VH{I hold the belief that}{(phrase) I think that}{(cụm từ) tôi cho là}
        \VH{flyover}{(n) a bridge that carries one road over another one}{(danh từ) cầu vượt}
        \VH{to pop up in somebody’s mind}{(phrase) spring to one’s mind}{(cụm từ) nảy ra trong đầu}
        \VH{to be hooked up to}{(phrase) to connect a machine to a power supply or to another machine}{(cụm từ) kết nối}
        \VH{computer literate}{(phrase) able to use computers well}{(cụm từ) người sử dụng máy tính thành thạo}
        \VH{(in) mint condition}{(phrase) in excellent condition, as if new}{(cụm từ) còn rất mới}
        \VH{in the blink of an eye}{(idiom) extremely quickly}{(thành ngữ) rất nhanh, xảy ra nhanh trong nháy mắt}
        \VH{to score great deals}{(phrase) have big deals}{(cụm từ) có ưu đãi lớn}
        \VH{built-in}{(adj) forming part of something, and not separate from it}{(tính từ) dính liền}
        \VH{on the move}{(idiom) travelling from place to place}{(thành ngữ) di chuyển}
        \VH{in the absence of}{(phrase) without}{(cụm từ) vắng}
        \VH{integrated}{(adj) with two or more things combined in order to become more effective}{(tính từ) tích hợp vào}
        \VH{to do wonders for}{(idiom) to have a very good effect}{(thành ngữ) mang nhiều lợi ích}
        \VH{domestic appliances}{(n) a large piece of electrical equipment used in the home, especially in the kitchen}{(danh từ) đồ dùng nhà bếp}
        \VH{you name it}{(phrase) so on, so forth}{(cụm từ) cụm từ, dùng để liệt kê}
        \VH{middle-class}{(adj) typical of people from the middle social class}{(tính từ) thuộc tầng lớp trung lưu}
        \VH{underprivileged}{(adj) having less money and fewer opportunities than most people in society}{(tính từ) chịu thiệt thòi, có hoàn cảnh khó khăn}
        \VH{life-saver}{(n) a thing that helps somebody in a difficult situation; something that saves somebody’s life}{(danh từ) vật, đồ dùng hữu ích}
        \VH{a breeze}{(n) a thing that is easy to do}{(danh từ) việc dễ dàng}
        \VH{inordinate}{(adj) far more than is usual or expected}{(tính từ) vô số}
        \VH{leftovers}{(n) food that has not been eaten at the end of a meal}{(danh từ) thức ăn còn lại; thức ăn thừa}
        \VH{substances}{(n) a type of solid, liquid or gas that has particular qualities}{(danh từ) các chất rắn, chất thải}
        \VH{a metropolis}{(n) a large important city (often the capital city of a country or region)}{(danh từ) thành phố lớn}
        \VH{to be furnished}{(adj) of a house, room, etc.) containing furniture}{(tính từ) được trang bị đồ đạc, nội thất}
        \VH{a rarity}{(n) a person or thing that is unusual and is therefore often valuable or interesting}{(danh từ) sự hiếm thấy}
        \VH{mountainous}{(adj) having many mountains}{(tính từ) thuộc vùng núi}
        \VH{living from hand to mouth}{(idiom) to have just enough money to live on and nothing extra}{(thành ngữ) những người có thu nhập chỉ đủ trang trải cuộc sống}
        \VH{manually}{(adv) by hand rather than automatically or using electricity}{(trạng từ) bằng tay}
        \VH{a desk job}{(n) a job based at a desk}{(danh từ) công việc bàn giấy}
        \VH{to be content with}{(adj) pleased with your situation and not hoping for change or improvement}{(tính từ) bằng lòng với cái gì}
        \VH{stationery items}{(phrase) commercially manufactured writing materials, including cut paper, envelopes, writing implements, continuous form paper, and other office supplies}{(cụm từ) vật dụng văn phòng phẩm}
        \VH{to make headway}{(phrase) to make a progress}{(cụm từ) đạt được tiến bộ}
        \VH{pattern}{(n) a particular way in which something is done, is organized, or happens}{(danh từ) khuôn mẫu}
        \VH{grunt work}{(phrase) thankless and menial work}{(cụm từ) việc tay chân, việc vặt}
        \VH{to drop off the face of the earth}{(idiom) to stop existing}{(thành ngữ) ngừng tồn tại}
        \VH{progress}{(n) the process of improving or developing, or of getting nearer to achieving or completing something}{(danh từ) sự tiến bộ}
        \VH{the advent of something}{(phrase) the appearance of something}{(cụm từ) sự xuất hiện, ra đời của cái gì}
        \VH{a virtual meeting}{(n) a virtual meeting is when people around the world, regardless of their location, use video, audio, and text to link up online}{(danh từ) cuộc họp trực tuyến}
        \VH{to eliminate}{(v) to remove or get rid of something/somebody}{(động từ) loại bỏ}
        \VH{case by case basis}{(phrase) according to the particular facts relating to each situation}{(cụm từ) tùy từng trường hợp}
        \VH{unarguable}{(adj) that nobody can disagree with}{(tính từ) không bàn cãi}
        \VH{hidden}{(v) to put or keep somebody/something in a place where they/it cannot be seen or found}{(động từ) bị ẩn đi; che khuất}
        \VH{obsolete}{(adj) no longer produced or used; out of date}{(tính từ) lỗi thời}
        \VH{to adapt to}{(v) to change something in order to make it suitable for a new use or situation}{(động từ) thích nghi}
    \end{VocabHighlights}
    \end{test}
\end{glossarymc}