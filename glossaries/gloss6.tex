\begin{glossarymc}[Cambridge 8]
    \begin{test}{TEST 1}
    \noindent
    \textbf{Part 1. Neighbours}
    \begin{qa}{How well do you know the people who live next door to you? (Why/Why not?)}
    \textbf{Speaking candidly}, I only \textbf{know my neighbors by sight} instead of their background. This is partly because I’m \textbf{tied up} at work. I have to \textbf{set off} for work \textbf{first thing in the morning}, say 7 o’clock, and return home at around 9–9.30 p.m due to extra work. On returning home at that time, I’m \textbf{not in the mood for} interacting with others around me anymore.
    \end{qa}

    \begin{qa}{How often do you see them? (Why/Why not?)}
    Like what I said earlier, I leave home \textbf{at the crack of dawn} and return home \textbf{till the cows come home} so I do not frequently see anyone around on weekdays. Maybe I can meet my neighbor twice or three times per week at weekends when I have a day off.
    \end{qa}

    \begin{qa}{What kind of problem do people sometimes have with their neighbours?}
    There are some problems \textbf{stemming from} my neighbors that I’ve encountered. Firstly, due to their \textbf{inquisitive} nature, they sometimes \textbf{stick their nose in my business} and \textbf{talk behind my back} although they are not familiar with me. For example, I once watched \textbf{a film} that consists of scenes of a quarrel between a wife and her husband. Then the next day, rumor had it that my family had an argument, which sounded ridiculous to me. Secondly, some are ignorant of others’ break time. I once had to \textbf{put up with} the loudness from the home theatre system of the \textbf{adjacent} apartment until 10.30 p.m.
    \end{qa}

    \begin{qa}{How do you think neighbours can help each other?}
    There are lots of things neighbors can do to contribute to mutual benefits. Firstly, based on their expertise, they can share some useful advice and lend a helping hand if necessary. For example, I guess I really \textbf{lucked out} to have a \textbf{pediatrician around the corner} when he gave me a prescription for my daughter in case of some minor illnesses before he moved out. Secondly, as long as they do not \textbf{make up} stories about my family, I actually consider this kind of a great help for me.
    \end{qa}

        \begin{VocabExplain}[Part 1]
            \begin{ExplainCard}{speaking candidly}[phrase][B2]
                \EN{frankly; speaking in an honest and direct way.}
                \VI{nói thẳng, thành thật.}
                \SY{frankly; honestly; openly}
                \EX{\textit{Simple:} \textbf{Speaking candidly}, I don’t know my neighbors well.}
                \EX{\textit{Academic:} \textbf{Speaking candidly}, the committee acknowledged methodological flaws.}
                \CO{speaking candidly about; candidly admit/state}
            \end{ExplainCard}

            \begin{ExplainCard}{know (someone) by sight}[phrase][B2]
                \EN{recognize a person’s face without knowing them personally.}
                \VI{biết mặt nhưng không quen.}
                \SY{recognize; be familiar with (only) by sight}
                \EX{\textit{Simple:} I only \textbf{know the guard by sight}.}
                \EX{\textit{Academic:} Respondents reported they \textbf{knew} several officials \textbf{by sight} rather than by name.}
                \CO{only/by sight; recognize by sight}
                \end{ExplainCard}

                \begin{ExplainCard}{tied up}[adj][B2]
                \EN{busy and unable to do other things.}
                \VI{bận bịu, bị cuốn vào việc.}
                \SY{busy; swamped; snowed under}
                \EX{\textit{Simple:} Sorry, I’m \textbf{tied up} this evening.}
                \EX{\textit{Academic:} Faculty were \textbf{tied up} with accreditation activities.}
                \CO{be/get tied up with/in; remain tied up}
                \end{ExplainCard}

                \begin{ExplainCard}{set off}[phrasal verb][B1]
                \EN{begin a journey; leave.}
                \VI{lên đường, xuất phát.}
                \SY{depart; head off; leave}
                \EX{\textit{Simple:} We \textbf{set off} at 7 a.m.}
                \EX{\textit{Academic:} Data collection \textbf{set off} after ethics approval.}
                \CO{set off early/late; set off for}
                \end{ExplainCard}

                \begin{ExplainCard}{first thing in the morning}[phrase][B1]
                \EN{very early in the morning, before doing anything else.}
                \VI{sáng sớm, việc đầu tiên buổi sáng.}
                \SY{at daybreak; at dawn; bright and early}
                \EX{\textit{Simple:} Call me \textbf{first thing in the morning}.}
                \EX{\textit{Academic:} Participants arrived \textbf{first thing in the morning} for testing.}
                \CO{do/leave/meet first thing in the morning}
                \end{ExplainCard}

                \begin{ExplainCard}{not in the mood for}[phrase][B2]
                \EN{not feeling like doing something.}
                \VI{không có hứng/không muốn làm gì.}
                \SY{unwilling; disinclined; reluctant}
                \EX{\textit{Simple:} I’m \textbf{not in the mood for} talking.}
                \EX{\textit{Academic:} Students were \textbf{not in the mood for} additional surveys during exams.}
                \CO{not in the mood for/to; be in no mood for}
                \end{ExplainCard}

                \begin{ExplainCard}{at the crack of dawn}[idiom][B2]
                \EN{very early in the morning.}
                \VI{tờ mờ sáng, rất sớm.}
                \SY{daybreak; first light; sunrise}
                \EX{\textit{Simple:} We left \textbf{at the crack of dawn}.}
                \EX{\textit{Academic:} Fieldwork commenced \textbf{at the crack of dawn} to avoid heat.}
                \CO{leave/start/work at the crack of dawn}
                \end{ExplainCard}

                \begin{ExplainCard}{till the cows come home}[idiom][C1]
                \EN{for a very long time; endlessly.}
                \VI{rất lâu, mãi mãi.}
                \SY{for ages; indefinitely}
                \EX{\textit{Simple:} He can talk \textbf{till the cows come home}.}
                \EX{\textit{Academic:} Debates can continue \textbf{till the cows come home} without consensus.}
                \CO{argue/wait/talk till the cows come home}
                \end{ExplainCard}

                \begin{ExplainCard}{stem from}[phrasal verb][B2]
                \EN{to be caused by or originate from something.}
                \VI{bắt nguồn từ, xuất phát từ.}
                \SY{arise from; result from; derive from}
                \EX{\textit{Simple:} The noise \textbf{stems from} construction.}
                \EX{\textit{Academic:} Disparities \textbf{stem from} structural factors.}
                \CO{issues/problems that stem from}
                \end{ExplainCard}

                \begin{ExplainCard}{inquisitive}[adj][C1]
                \EN{too interested in other people’s affairs; curious.}
                \VI{tọc mạch; tò mò.}
                \SY{prying; nosy; curious}
                \EX{\textit{Simple:} Our \textbf{inquisitive} neighbor keeps asking questions.}
                \EX{\textit{Academic:} \textbf{Inquisitive} minds drive exploratory research.}
                \CO{inquisitive nature; inquisitive child/mind}
                \end{ExplainCard}

                \begin{ExplainCard}{stick one’s nose in (someone’s) business}[idiom][C1]
                \EN{to interfere in matters that do not concern you.}
                \VI{xía mũi vào chuyện người khác.}
                \SY{meddle; pry; intrude}
                \EX{\textit{Simple:} Don’t \textbf{stick your nose in my business}.}
                \EX{\textit{Academic:} External bodies that \textbf{stick their nose in} local affairs face pushback.}
                \CO{constantly/always stick one’s nose in}
                \end{ExplainCard}

                \begin{ExplainCard}{talk behind (someone’s) back}[idiom][B2]
                \EN{to say bad things about someone without their knowledge.}
                \VI{nói xấu sau lưng.}
                \SY{gossip about; bad-mouth}
                \EX{\textit{Simple:} She \textbf{talks behind my back}.}
                \EX{\textit{Academic:} Perceived colleagues \textbf{talking behind one’s back} reduces trust.}
                \CO{talk/gossip behind sb’s back}
                \end{ExplainCard}

                \begin{ExplainCard}{a film}[n][A2]
                \EN{a movie; a series of moving images shown in a cinema or on TV.}
                \VI{bộ phim.}
                \SY{movie; motion picture}
                \EX{\textit{Simple:} We watched \textbf{a film} last night.}
                \EX{\textit{Academic:} The study analyzed \textbf{a film} for narrative structure.}
                \CO{watch/make/review a film}
                \end{ExplainCard}

                \begin{ExplainCard}{put up with}[phrasal verb][B2]
                \EN{tolerate something unpleasant.}
                \VI{chịu đựng.}
                \SY{tolerate; endure; bear}
                \EX{\textit{Simple:} I can’t \textbf{put up with} the noise.}
                \EX{\textit{Academic:} Participants would not \textbf{put up with} extended delays.}
                \CO{have to/able to put up with; put up with noise/behavior}
                \end{ExplainCard}

                \begin{ExplainCard}{adjacent}[adj][C1]
                \EN{next to or very near something.}
                \VI{kề bên, liền kề.}
                \SY{adjoining; neighboring; contiguous}
                \EX{\textit{Simple:} We live in \textbf{adjacent} apartments.}
                \EX{\textit{Academic:} \textbf{Adjacent} parcels were consolidated for the study.}
                \CO{adjacent building/room/property; adjacent to}
                \end{ExplainCard}

                \begin{ExplainCard}{luck out}[phrasal verb][C1]
                \EN{to be very lucky; have good fortune unexpectedly.}
                \VI{gặp may bất ngờ.}
                \SY{be fortunate; catch a break}
                \EX{\textit{Simple:} I \textbf{lucked out} and found a seat.}
                \EX{\textit{Academic:} Teams that \textbf{lucked out} with favorable timing reported better outcomes.}
                \CO{really/just luck out; luck out with}
                \end{ExplainCard}

                \begin{ExplainCard}{pediatrician around the corner}[phrase][B2]
                \EN{a children’s doctor located very nearby.}
                \VI{bác sĩ nhi ở ngay gần.}
                \SY{nearby pediatrician; local children’s doctor}
                \EX{\textit{Simple:} We’re lucky to have a \textbf{pediatrician around the corner}.}
                \EX{\textit{Academic:} Access to a \textbf{pediatrician around the corner} improves routine care adherence.}
                \CO{have/find/consult a pediatrician around the corner}
                \end{ExplainCard}

                \begin{ExplainCard}{make up (stories)}[phrasal verb][B1]
                \EN{to invent (often something untrue).}
                \VI{bịa đặt (chuyện).}
                \SY{fabricate; invent; concoct}
                \EX{\textit{Simple:} Don’t \textbf{make up} stories about us.}
                \EX{\textit{Academic:} Witnesses may unconsciously \textbf{make up} details to fill memory gaps.}
                \CO{make up a story/excuse; completely make up}
                \end{ExplainCard}
        \end{VocabExplain}

    \noindent
    \textbf{Part 2.}
    \begin{qa}{Describe a time when you were asked to give your opinion in a questionnaire or survey.}
    \begin{itemize}
    \item What the questionnaire/survey was about
    \item Why you were asked to give your opinions
    \item What opinions you gave
    \item and explain how you felt about giving your opinions in this questionnaire/survey.
    \end{itemize}

    I would like to tell you a time when I was asked to give my feedback on a survey, and I will surely never forget this time. If my memory serves me right, I was in the Vincom center, which is a giant shopping mall in Hanoi. I was waiting for my friends in the corridor when a man approached me. He introduced a little about himself and invited me to do a survey on customer behavior. The survey lasted for 10 minutes. Although there were \textbf{a handful of} questions, the interviewer wanted me to provide \textbf{in-depth} answers so that he could analyse the data for his dissertation. I still remember a question related to factors that \textbf{drive} my final decision of owning my smartphone. I replied that it was my budget. If I felt \textbf{flush}, I would \textbf{splurge on} a new iPhone without careful consideration. To be honest, I was glad because I could help him complete his tasks. More importantly, it was an opportunity for me to have a close reflection of my shopping habits and think about \textbf{making the best out of} money because in the past, I used to \textbf{pour money down the drain} by shopping for something to \textbf{show off my social standing} rather than for its functions.
    \end{qa}

        \begin{VocabExplain}[Part 2]
            \begin{ExplainCard}{a handful of}[n phrase][B2]
                \EN{a very small number or amount of something.}
                \SY{a few; a small number; a sprinkling}
                \VI{một lượng rất ít; chỉ vài ba.}
                \EX{There were only a handful of questions on the survey.}
                \EX{Only a handful of participants met the strict criteria.}
                \CO{a handful of people/cases/items}
            \end{ExplainCard}

            \begin{ExplainCard}{in-depth}[adj][C1]
                \EN{thorough and detailed; showing careful, complete coverage of a topic.}
                \SY{thorough; comprehensive; exhaustive}
                \VI{chuyên sâu, chi tiết.}
                \EX{He asked for in-depth answers, not one-liners.}
                \EX{The report offers an in-depth analysis of consumer behavior.}
                \CO{in-depth study/analysis/interview/coverage}
            \end{ExplainCard}

            \begin{ExplainCard}{drive (a decision)}[v][C1]
                \EN{(1) to strongly influence or determine an outcome; (2) to provide energy or motivation for action.}
                \SY{(1) shape; dictate; govern \;(2) motivate; propel; spur}
                \VI{(1) chi phối/quyết định; (2) thúc đẩy, tạo động lực.}
                \EX{Price and battery life drive my final choice of phone.}
                \EX{Environmental concerns drive firms to redesign packaging.}
                \CO{drive demand/change/innovation; be driven by factors}
            \end{ExplainCard}

            \begin{ExplainCard}{flush (with cash)}[adj][C1]
                \EN{having plenty of money available, especially temporarily.}
                \SY{well-off; loaded (inf.); affluent}
                \VI{dư dả tiền bạc (thường trong ngắn hạn).}
                \EX{If I’m feeling flush, I might upgrade my phone.}
                \EX{Post-holiday shoppers are rarely flush enough for big purchases.}
                \CO{feel/be flush with cash/funds}
            \end{ExplainCard}

            \begin{ExplainCard}{splurge on}[phr.v][B2]
                \EN{to spend a lot of money on something enjoyable but not essential.}
                \SY{treat yourself to; splash out on; indulge in}
                \VI{vung tiền/chi đậm cho (một món vui thích).}
                \EX{She splurged on noise-cancelling headphones.}
                \EX{Households tend to splurge on non-essentials during booms.}
                \CO{splurge on gadgets/travel/luxuries}
            \end{ExplainCard}

            \begin{ExplainCard}{make the best of / out of}[idiom][B2]
                \EN{to get as much benefit as possible from a situation or resource, even if it is limited.}
                \SY{capitalize on; make the most of; optimize}
                \VI{tận dụng tối đa (dù điều kiện không lý tưởng).}
                \EX{I’m trying to make the best of my budget.}
                \EX{Teams made the best of scarce data by augmenting with surveys.}
                \CO{make the best of/out of time/money/situation}
            \end{ExplainCard}

            \begin{ExplainCard}{pour money down the drain}[idiom][C1]
                \EN{to waste money on something useless or unnecessary.}
                \SY{waste; squander; fritter away}
                \VI{ném tiền qua cửa sổ; phung phí tiền bạc.}
                \EX{I used to pour money down the drain on trendy accessories.}
                \EX{Without evaluation, advertising spend can pour money down the drain.}
                \CO{stop/avoid pouring money down the drain}
            \end{ExplainCard}

            \begin{ExplainCard}{show off}[phr.v][B2]
                \EN{to display something proudly to impress others.}
                \SY{flaunt; parade; boast about}
                \VI{khoe khoang, phô trương.}
                \EX{He bought designer gear just to show off.}
                \EX{Some brands show off social proof to boost credibility.}
                \CO{show off wealth/status/new purchases}
            \end{ExplainCard}

            \begin{ExplainCard}{social standing}[n][C1]
                \EN{a person’s rank or status within society.}
                \SY{status; social status; prestige}
                \VI{địa vị/uy tín xã hội.}
                \EX{She used luxury goods to raise her social standing.}
                \EX{Education level is a key predictor of social standing.}
                \CO{high/low social standing; improve social standing}
            \end{ExplainCard}

            \begin{ExplainCard}{dissertation}[n][C1]
                \EN{a long piece of academic writing submitted for a degree.}
                \SY{thesis; research paper; treatise}
                \VI{luận văn/luận án học thuật.}
                \EX{He needed survey data for his dissertation.}
                \EX{A clear methodology strengthens any dissertation.}
                \CO{write/defend a dissertation; dissertation topic/supervisor}
            \end{ExplainCard}

            \begin{ExplainCard}{If my memory serves me right}[phrase][B2]
                \EN{used to say you believe your recollection is correct.}
                \SY{if I recall correctly; to the best of my memory}
                \VI{nếu tôi nhớ không nhầm.}
                \EX{If my memory serves me right, it lasted ten minutes.}
                \EX{If memory serves, the pilot test used 50 respondents.}
                \CO{If (my) memory serves (me) right/correctly}
            \end{ExplainCard}

            \begin{ExplainCard}{budget}[n][B2]
                \EN{the amount of money available or planned for spending.}
                \SY{spending plan; allocation; funds}
                \VI{ngân sách; khoản tiền dự chi.}
                \EX{My budget ultimately decides which phone I buy.}
                \EX{Projects must operate within a tight budget.}
                \CO{on a tight/limited budget; set/allocate a budget}
            \end{ExplainCard}
        \end{VocabExplain}

    \noindent
    \textbf{Part 3.}
    \begin{qa}{What kinds of organisation want to find out about people's opinions?}
    In the \textbf{market economy}, many organizations, sales companies or supermarkets, would like to survey the interests of their customers in order to \textbf{make alteration} to their products. On the grounds that commercial products are \textbf{rolled out} constantly, \textbf{customer psychology} is of great importance for companies to have sound strategies to promote their commodities to maximise the profit.
    \end{qa}

    \begin{qa}{Do you think that questionnaire or surveys are good ways of finding out people's opinions?}
    It seems to me that surveys and polls are practical ways to approach \textbf{target customers} because they can \textbf{encompass} many \textbf{pertinent} questions to reveal the preference. The thing is, these types of research can be time-consuming and include \textbf{erroneous} information because lots of people do not feel like sharing personal information with outsiders. Just imagine, a salesman could not reach hundreds of his customers every day to ask about their favorites, so in case of urgency, carrying out surveys seems \textbf{unfeasible}.
    \end{qa}

    \begin{qa}{What reasons might people have for not wanting to give opinions?}
    Understandably, people hesitate to \textbf{disclose} their personal information to strangers or companies for fear of having their privacy breached. Some feel reluctant to fill out the survey but still remain \textbf{anonymous} for safety reasons. Another point I'd like to mention is that \textbf{every now and then} customers feel \textbf{bothered} because they are usually \textbf{bombarded} with questionnaires in unwanted moments. Therefore, surveys should be conducted when it is suitable for the customers.
    \end{qa}

    \begin{qa}{Do you think it would be a good idea for schools to ask students their opinions about lessons?}
    As a matter of fact, students are \textbf{beneficiaries} of education, hence, it would be \textbf{contentious} if the establishment of education is not based on students' \textbf{proficiency}. In that sense, I believe asking students is a \textbf{sensible} answer to the problem. Through students' feedback, necessary \textbf{modifications} will be made to fit each study level and to classify students more efficiently.
    \end{qa}

    \begin{qa}{What would be advantages for schools be if they asked students their opinions?}
    Well, I can imagine school might \textbf{reap tremendous benefits} through analysing students' opinions. For one, their opinions actually reflect the quality of education that \textbf{school administrators} fail to know due to ineffective communication, for example. The feedback, therefore, is a \textbf{concrete base} for education makers to propose any changes. Secondly, \textbf{student-centered education} is increasingly popular in many countries, which means students are the top priority in any education system. Accordingly, their responses need to be taken into account.
    \end{qa}

    \begin{qa}{Would there be any disadvantages in asking students' opinions?}
    Unfortunately, I would say yes. While I \textbf{concede} that collecting students' opinion is \textbf{worthwhile} to some points, there are few \textbf{downsides} that should be taken into consideration. Some of students' answer may be inaccurate or \textbf{exaggerated} to some extent, which can cause misunderstanding. For example, a student is sometimes biased, rating a strict teacher based on his intuition instead of the learning outcome. Besides, it is impossible to \textbf{compromise} all changes into one education system at once, so the board of management should only consider a majority of good opinions.
    \end{qa}
    
        \begin{VocabExplain}[Part 3]
            \begin{ExplainCard}{market economy}[n][C1]
                \EN{an economic system in which prices and production are determined by supply and demand.}
                \SY{free-market system; capitalist economy}
                \VI{nền kinh tế thị trường.}
                \EX{In a market economy, firms track customer opinions closely.}
                \EX{Transition to a market economy reshaped consumer behavior.}
                \CO{open/competitive market economy; transition to a market economy}
            \end{ExplainCard}

            \begin{ExplainCard}{make alteration (to)}[n phrase][B2]
                \EN{to make small changes or adjustments to something.}
                \SY{make changes; adjust; modify}
                \VI{thực hiện điều chỉnh/đổi nhỏ cho.}
                \EX{They made alterations to the product after the survey.}
                \EX{User feedback helps companies make alterations to design.}
                \CO{make alteration(s) to a plan/product/policy}
            \end{ExplainCard}

            \begin{ExplainCard}{roll out}[phr.v][C1]
                \EN{to launch or introduce a new product or service to the public.}
                \SY{launch; introduce; debut}
                \VI{ra mắt/triển khai sản phẩm, dịch vụ.}
                \EX{The brand rolls out new flavors every quarter.}
                \EX{Governments rolled out online surveys during the pilot.}
                \CO{roll out a product/campaign/program}
            \end{ExplainCard}

            \begin{ExplainCard}{customer psychology}[n][C1]
                \EN{study of how consumers think, feel, and act when buying.}
                \SY{consumer behavior; buyer psychology}
                \VI{tâm lý khách hàng/Người tiêu dùng.}
                \EX{An ad that taps customer psychology boosts sales.}
                \EX{Understanding customer psychology informs pricing strategy.}
                \CO{understand/apply customer psychology; insights into customer psychology}
            \end{ExplainCard}

            \begin{ExplainCard}{target customers}[n][B2]
                \EN{the specific group a product or campaign is intended for.}
                \SY{target audience; intended customers}
                \VI{khách hàng mục tiêu.}
                \EX{Surveys help identify target customers.}
                \EX{Defining target customers is central to market research.}
                \CO{identify/reach target customers; target-customer profile}
            \end{ExplainCard}

            \begin{ExplainCard}{encompass}[v][C1]
                \EN{to include a wide range of ideas, subjects, or things.}
                \SY{include; cover; embrace}
                \VI{bao gồm, bao trùm.}
                \EX{The form encompasses questions on price and quality.}
                \EX{Our framework encompasses three dimensions of satisfaction.}
                \CO{encompass topics/aspects/areas}
            \end{ExplainCard}

            \begin{ExplainCard}{pertinent}[adj][C1]
                \EN{relevant and directly related to the matter at hand.}
                \SY{relevant; germane; apt}
                \VI{phù hợp, liên quan trực tiếp.}
                \EX{Please ask only pertinent questions.}
                \EX{Pertinent variables were retained in the final model.}
                \CO{pertinent question/point/data}
            \end{ExplainCard}

            \begin{ExplainCard}{erroneous}[adj][C2]
                \EN{containing errors; not correct.}
                \SY{incorrect; faulty; mistaken}
                \VI{sai sót, không chính xác.}
                \EX{Self-reports can be erroneous.}
                \EX{Erroneous entries were removed during cleaning.}
                \CO{erroneous assumption/conclusion/data}
            \end{ExplainCard}

            \begin{ExplainCard}{unfeasible}[adj][C1]
                \EN{not practical or possible to do.}
                \SY{impracticable; infeasible; unrealistic}
                \VI{không khả thi.}
                \EX{Daily in-person checks are unfeasible.}
                \EX{The committee deemed the proposal unfeasible at scale.}
                \CO{prove/become unfeasible; an unfeasible plan}
            \end{ExplainCard}

            \begin{ExplainCard}{disclose}[v][C1]
                \EN{to reveal information that was previously private or secret.}
                \SY{reveal; divulge; share}
                \VI{tiết lộ, cung cấp (thông tin).}
                \EX{Some refuse to disclose income on forms.}
                \EX{Firms must disclose survey methods in reports.}
                \CO{disclose details/information/data}
            \end{ExplainCard}

            \begin{ExplainCard}{anonymous}[adj][B2]
                \EN{without a name or other identifying information.}
                \SY{unnamed; unidentified}
                \VI{ẩn danh, không nêu tên.}
                \EX{Responses remained anonymous.}
                \EX{Anonymous surveys reduce social-desirability bias.}
                \CO{stay/remain anonymous; anonymous response/survey}
            \end{ExplainCard}

            \begin{ExplainCard}{every now and then}[idiom][B2]
                \EN{from time to time; occasionally.}
                \SY{occasionally; sometimes; now and again}
                \VI{thỉnh thoảng, đôi khi.}
                \EX{Every now and then I fill out a poll.}
                \EX{Errors still occur every now and then in large datasets.}
                \CO{happen/occur every now and then}
            \end{ExplainCard}

            \begin{ExplainCard}{bothered}[adj][B2]
                \EN{annoyed, worried, or upset by something.}
                \SY{annoyed; irritated; troubled}
                \VI{khó chịu, phiền toái.}
                \EX{Customers felt bothered by repeated emails.}
                \EX{Participants reported being bothered by survey length.}
                \CO{feel/get bothered; bothered by noise/calls}
            \end{ExplainCard}

            \begin{ExplainCard}{bombarded (with)}[v][C1]
                \EN{to be hit with something continuously, especially requests or messages.}
                \SY{inundate; flood; pester}
                \VI{bị dồn dập/“oanh tạc” (thông tin, câu hỏi).}
                \EX{I’m bombarded with promotional surveys.}
                \EX{Users were bombarded with pop-ups during testing.}
                \CO{be/get bombarded with ads/emails/questions}
            \end{ExplainCard}

            \begin{ExplainCard}{beneficiary}[n][C1]
                \EN{a person who gains benefits from something.}
                \SY{recipient; advantaged party}
                \VI{người thụ hưởng; người nhận lợi ích.}
                \EX{Students are the direct beneficiaries of reforms.}
                \EX{Beneficiaries of the program reported higher retention.}
                \CO{primary/ultimate beneficiary; beneficiary of a policy}
            \end{ExplainCard}

            \begin{ExplainCard}{contentious}[adj][C1]
                \EN{likely to cause disagreement; controversial.}
                \SY{controversial; debatable; disputed}
                \VI{gây tranh cãi.}
                \EX{Ranking teachers can be contentious.}
                \EX{Funding models remain a contentious issue.}
                \CO{contentious issue/debate/policy}
            \end{ExplainCard}

            \begin{ExplainCard}{proficiency}[n][C1]
                \EN{a high degree of skill or competence in a subject.}
                \SY{competence; skill; mastery}
                \VI{trình độ thành thạo; năng lực.}
                \EX{Placement tests gauge students’ proficiency.}
                \EX{Language proficiency predicts academic success.}
                \CO{language/math proficiency; levels of proficiency}
            \end{ExplainCard}

            \begin{ExplainCard}{sensible}[adj][B2]
                \EN{showing good judgment; practical and reasonable.}
                \SY{reasonable; prudent; sound}
                \VI{hợp lý, khôn ngoan, thực tế.}
                \EX{Surveying learners seems sensible.}
                \EX{A sensible policy balances cost and benefit.}
                \CO{sensible decision/approach/choice}
            \end{ExplainCard}

            \begin{ExplainCard}{modification}[n][B2]
                \EN{a small change to improve or adapt something.}
                \SY{adjustment; alteration; tweak}
                \VI{sự điều chỉnh; chỉnh sửa nhỏ.}
                \EX{Curricula need modifications each term.}
                \EX{Minor modifications increased response rates.}
                \CO{make/require modifications; design modifications}
            \end{ExplainCard}

            \begin{ExplainCard}{reap tremendous benefits}[v phrase][C1]
                \EN{to gain very large advantages as a result of an action.}
                \SY{gain; derive; obtain significant benefits}
                \VI{thu được lợi ích lớn.}
                \EX{Schools can reap tremendous benefits from feedback.}
                \EX{Firms reap tremendous benefits when surveys guide R\&D.}
                \CO{reap benefits/gains/rewards from sth}
            \end{ExplainCard}

            \begin{ExplainCard}{school administrator}[n][B2]
                \EN{a person responsible for managing a school’s operations.}
                \SY{manager; principal; education official}
                \VI{cán bộ/nhà quản lý trường học.}
                \EX{School administrators review survey results.}
                \EX{Administrators coordinate policy implementation.}
                \CO{experienced school administrator; district administrators}
            \end{ExplainCard}

            \begin{ExplainCard}{concrete base}[n phrase][C1]
                \EN{a solid, factual foundation for decisions.}
                \SY{firm basis; solid ground; sound foundation}
                \VI{cơ sở vững chắc, có dữ liệu.}
                \EX{Student feedback gives a concrete base for changes.}
                \EX{A concrete base of evidence supports the reform.}
                \CO{provide/form a concrete base for decisions}
            \end{ExplainCard}

            \begin{ExplainCard}{student-centered education}[n][C1]
                \EN{an approach that prioritizes learners’ needs, interests, and active role.}
                \SY{learner-centered approach; student-focused learning}
                \VI{giáo dục lấy người học làm trung tâm.}
                \EX{Student-centered education values feedback.}
                \EX{Policies promote student-centered education across levels.}
                \CO{adopt/promote student-centered education}
            \end{ExplainCard}

            \begin{ExplainCard}{concede}[v][C1]
                \EN{to admit, often unwillingly, that something is true.}
                \SY{admit; acknowledge; grant}
                \VI{thừa nhận (thường miễn cưỡng).}
                \EX{I concede that surveys are useful at times.}
                \EX{Authors concede limitations in the discussion.}
                \CO{concede that + clause; concede a point}
            \end{ExplainCard}

            \begin{ExplainCard}{worthwhile}[adj][B2]
                \EN{worth the time, effort, or money spent.}
                \SY{valuable; rewarding; beneficial}
                \VI{đáng công/đáng làm.}
                \EX{Collecting feedback is worthwhile.}
                \EX{A pilot study is worthwhile before scaling up.}
                \CO{prove/remain worthwhile; a worthwhile effort/investment}
            \end{ExplainCard}

            \begin{ExplainCard}{downside}[n][B2]
                \EN{the negative part or disadvantage of something.}
                \SY{disadvantage; drawback; pitfall}
                \VI{điểm bất lợi; hạn chế.}
                \EX{Spam is a downside of online surveys.}
                \EX{Researchers discuss downsides of self-selection.}
                \CO{the downside is that...; potential downsides}
            \end{ExplainCard}

            \begin{ExplainCard}{exaggerated}[adj][C1]
                \EN{described as larger, better, or worse than it really is.}
                \SY{overstated; inflated; overstressed}
                \VI{phóng đại; nói quá.}
                \EX{Some ratings seemed exaggerated.}
                \EX{Exaggerated claims undermine credibility of data.}
                \CO{an exaggerated claim/response/figure}
            \end{ExplainCard}

            \begin{ExplainCard}{compromise}[v][C1]
                \EN{to reach an agreement by each side giving up part of its demands; or to weaken something by making concessions.}
                \SY{settle; reconcile; accommodate}
                \VI{thoả hiệp; nhượng bộ làm yếu đi.}
                \EX{We compromised on the changes to the syllabus.}
                \EX{Rushed reforms may compromise quality controls.}
                \CO{compromise on/over; be willing to compromise}
            \end{ExplainCard}
        \end{VocabExplain}

    \begin{VocabHighlights}
        \VH{speaking candidly}{(phrase) in all honesty; being totally truthful}{(cụm từ) thực sự mà nói thì}
        \VH{to know somebody by sight}{(phrase) to recognize someone or something based solely on appearance (without knowing any other information, such as a name)}{(cụm từ) chỉ biết mặt}
        \VH{to be tied up}{(phrase) to be extremely busy}{(cụm từ) cực kỳ bận}
        \VH{to set off for}{(phr. v) to depart for or begin traveling (to some place)}{(cụm động từ) khởi hành đi}
        \VH{first thing in the morning}{(idiom) at the very beginning of the day}{(thành ngữ) đầu giờ sáng}
        \VH{to be in the mood for}{(phrase) feeling a desire for something or to do something}{(cụm từ) có hứng làm gì}
        \VH{to be at the crack of dawn}{(phrase) a time very early in the morning}{(cụm từ) rất sớm buổi sáng}
        \VH{inquisitive}{(adj) wanting to discover as much as you can about things, sometimes in a way that annoys people}{(tính từ) tò mò}
        \VH{to stick one’s nose in one’s business}{(idiom) interfere in something that does not concern the doer}{(thành ngữ) chõ mũi vào việc của người khác}
        \VH{to talk behind one’s back}{(idiom) to talk bad things about a person who is not present}{(thành ngữ) nói xấu sau lưng ai}
        \VH{to put up with}{(v) to tolerate; endure}{(động từ) chịu đựng}
        \VH{adjacent}{(adj) next to or adjoining something else}{(tính từ) bên cạnh}
        \VH{to luck out}{(phr.v) to be very lucky}{(cụm động từ) gặp may}
        \VH{pediatrician}{(n) a doctor with special training in medical care for children}{(danh từ) bác sĩ nhi khoa}
        \VH{around the corner}{(idiom) very close to the place that you are}{(thành ngữ) rất gần}
        \VH{to make up}{(phr. v) to invent, say something untrue}{(cụm động từ) bịa ra}
        \VH{a handful of}{(phrase) some}{(cụm từ) một vài}
        \VH{in-depth}{(adj) done carefully and in great detail}{(tính từ) chi tiết}
        \VH{to drive}{(v) to force someone or something to go somewhere or do something}{(động từ) dẫn tới}
        \VH{flush}{(adj) have much money}{(tính từ) có nhiều tiền}
        \VH{to splurge on}{(v) to spend a lot of money (on somebody or something) in an indulgent or self-gratifying manner}{(động từ) vung tiền vào}
        \VH{to make the best out of}{(phrase) to gain the greatest possible advantage from something}{(cụm từ) tận dụng được hết}
        \VH{to pour money down the drain}{(idiom) to throw money away, waste money}{(thành ngữ) vứt tiền qua cửa sổ}
        \VH{to show off one’s social standing}{(phrase) to project your wealth}{(cụm từ) khoe đẳng cấp}
        \VH{market economy}{(n) an economic system in which production and prices are determined by unrestricted competition between privately owned businesses}{(danh từ) nền kinh tế thị trường}
        \VH{to make alteration}{(phrase) the act of making a change to something}{(cụm từ) làm thay đổi}
        \VH{to roll out}{(phr.v) to make a new product, service, or system available for the first time}{(cụm động từ) ra mắt sản phẩm}
        \VH{customer psychology}{(phrase) the study of why people buy things}{(cụm từ) tâm lý khách hàng}
        \VH{target customers}{(n) the type of person that a company wants to sell its products or services to}{(danh từ) khách hàng mục tiêu}
        \VH{to encompass}{(v) to include a large number or range of things}{(động từ) bao gồm}
        \VH{pertinent}{(adj) appropriate to a particular situation}{(tính từ) thích hợp}
        \VH{erroneous}{(adj) not correct; based on wrong information}{(tính từ) sai lệch}
        \VH{unfeasible}{(adj) not possible to do or achieve}{(tính từ) khó khả thi, ít có khả năng xảy ra}
        \VH{to disclose}{(v) to give somebody information about something, especially something that was previously secret}{(động từ) tiết lộ, chia sẻ}
        \VH{anonymous}{(adj) with a name that is not known or that is not made public}{(tính từ) nặc danh}
        \VH{every now and then}{(idiom) sometimes, but not regularly or often}{(thành ngữ) thỉnh thoảng}
        \VH{bothered}{(adj) concerned about something}{(tính từ) bị làm phiền}
        \VH{to bombard somebody with something}{(v) to attack somebody with a lot of questions, criticisms, etc. or by giving them too much information}{(động từ) làm phiền ai đó với rất nhiều (thông tin, câu hỏi...)}
        \VH{a beneficiary}{(n) a person who gains as a result of something}{(danh từ) người thụ hưởng}
        \VH{contentious}{(adj) likely to cause disagreement between people}{(tính từ) gây tranh cãi}
        \VH{proficiency}{(n) the ability to do something well because of training and practice}{(danh từ) sự thông thạo}
        \VH{sensible}{(adj) able to make good judgements based on reason and experience rather than emotion; practical}{(tính từ) khôn ngoan, hợp lý}
        \VH{a modification}{(n) the act or process of changing something in order to improve it or make it more acceptable; a change that is made}{(danh từ) sự sửa đổi, thay đổi}
        \VH{to reap tremendous benefits}{(phrase) to get something good as a result of your own actions}{(cụm từ) có rất nhiều lợi ích từ ai/cái gì}
        \VH{a school administrator}{(phrase) a person whose job is to manage and organize the public or business affairs of a company or an institution, or a person who works in an office dealing with records, accounts, etc.}{(cụm từ) ban giám hiệu nhà trường}
        \VH{concrete}{(adj) existing in a material or physical form; not abstract}{(tính từ) cụ thể}
        \VH{student-centered education}{(phrase) methods of teaching that shift the focus of instruction from the teacher to the student}{(cụm từ) học tập lấy học sinh làm trung tâm}
        \VH{to concede}{(v) to admit that something is true, logical, etc.}{(động từ) thừa nhận}
        \VH{worthwhile}{(adj) important, enjoyable, interesting, etc.; worth spending time, money or effort on}{(tính từ) đáng làm}
        \VH{exaggerated}{(adj) regarded or represented as larger, better, or worse than in reality}{(adj) bị phóng đại}
        \VH{compromise}{(n) an agreement made between two people or groups in which each side gives up some of the things they want so that both sides are happy at the end}{(danh từ) sự thỏa hiệp}
    \end{VocabHighlights}
    \end{test}

    \begin{test}{TEST 2}
    \noindent
    \textbf{Part 1. Newspapers and Magazines}
    \begin{qa}{Which magazines and newspapers do you read? [Why?]}
    A long time ago, I used to read sports-related newspapers and magazines in print. Nowadays, as the Internet \textbf{coverage} is more widespread than ever, I have switched to online forms instead thanks to their convenience. It allows me to read sports news \textbf{voraciously at my disposal} instead of leaving home to purchase one at a newspaper stall like I did before.
    \end{qa}

    \begin{qa}{What kinds of article are you most interested in? [Why?]}
    I consider myself an \textbf{avid} football fan so football-related article is what never fails to \textbf{grab my attention}. Owing to reading these articles, I may \textbf{be au courant with} any teams’ performance analysis, transfer updates or life stories of footballers. That literally \textbf{sums up} a typical day of mine.
    \end{qa}

    \begin{qa}{Have you ever read a newspaper or magazine in a foreign language? [When/Why?]}
    I used to read \textbf{tons of} newspapers and magazines in English when I was in the U.K to complete my Master degree half a decade ago. I would collect \textbf{a load of} “Metro” newspapers as they were given away for free at any stations. There was no Vietnamese newspapers or magazines in print form around so I had to \textbf{resort to} English papers instead.
    \end{qa}

    \begin{qa}{Do you think reading a newspaper or magazine in foreign language is a good way to learn the language? [Why/Why not?]}
    Yes, definitely. The act of reading a newspaper or magazine in any foreign language does \textbf{facilitate} learners in mastering any language. Readers can \textbf{familiarize} themselves with native author’s uses of words and writing styles. As a result, it may improve their former’s writing skills \textbf{no end}. For example, I myself compiled a list of lexical items to use later in my assignment and thesis, which was beneficial \textbf{up to a point}.
    \end{qa}

        \begin{VocabExplain}[Part 1]
            \begin{ExplainCard}{coverage}[n][B2]
                \EN{(1) reporting of a subject by the media; (2) the extent or reach of service or influence (e.g., internet/network).}
                \SY{reporting; exposure; reach; extent}
                \VI{(1) sự đưa tin của truyền thông; (2) phạm vi/độ phủ (mạng, dịch vụ).}
                \EX{The match received extensive media coverage.}
                \EX{Rural areas still have patchy internet coverage.}
                \CO{media/press coverage; extensive/limited coverage; network/internet coverage}
            \end{ExplainCard}

            \begin{ExplainCard}{voraciously}[adv][C1]
                \EN{in an extremely eager way, especially when reading, learning, or consuming information.}
                \SY{avidly; ravenously; greedily}
                \VI{một cách ngấu nghiến, say mê (đặc biệt khi đọc/học).}
                \EX{He started reading about football tactics voraciously.}
                \EX{Graduate students often devour literature voraciously during the first term.}
                \CO{read voraciously; learn voraciously; consume content voraciously}
            \end{ExplainCard}

            \begin{ExplainCard}{at (one's) disposal}[idiom][C1]
                \EN{available for someone to use whenever they need.}
                \SY{at one's command; available; on hand}
                \VI{sẵn để sử dụng khi cần; trong tay để tuỳ ý dùng.}
                \EX{With a tablet at my disposal, I can follow every game.}
                \EX{Researchers had extensive datasets at their disposal.}
                \CO{have sth at your disposal; resources at sb's disposal}
            \end{ExplainCard}

            \begin{ExplainCard}{avid}[adj][C1]
                \EN{showing keen enthusiasm or interest in something.}
                \SY{keen; ardent; devoted}
                \VI{hết sức đam mê, say mê.}
                \EX{I'm an avid fan of the Premier League.}
                \EX{An avid readership sustains the journal’s influence.}
                \CO{avid fan/reader/collector; be avid for sth}
            \end{ExplainCard}

            \begin{ExplainCard}{grab (someone's) attention}[phrase][B2]
                \EN{to attract or capture someone’s interest immediately.}
                \SY{catch; capture; draw}
                \VI{thu hút, giành lấy sự chú ý.}
                \EX{Bold headlines always grab my attention.}
                \EX{A striking abstract can grab readers’ attention in seconds.}
                \CO{grab/capture/catch attention; immediately/instantly grab attention}
            \end{ExplainCard}

            \begin{ExplainCard}{be au courant with}[adj phrase][C2]
                \EN{to be up to date or well informed about something.}
                \SY{abreast of; up-to-date with; conversant with}
                \VI{cập nhật, nắm bắt kịp thời về điều gì.}
                \EX{She’s au courant with all the latest transfer rumors.}
                \EX{Scholars must remain au courant with developments in their field.}
                \CO{au courant with trends/news/developments}
            \end{ExplainCard}

            \begin{ExplainCard}{sum up}[phr.v][B2]
                \EN{(1) to describe or express the important facts or qualities of something concisely; (2) to conclude a discussion or speech.}
                \SY{(1) encapsulate; epitomize; (2) conclude; wrap up}
                \VI{(1) tóm gọn/khái quát; (2) kết luận phần trình bày.}
                \EX{That sentence sums up my daily routine.}
                \EX{To sum up, the data supports our hypothesis.}
                \CO{perfectly/neatly sum up; to sum up, ...}
            \end{ExplainCard}

            \begin{ExplainCard}{tons of}[idiom][B2]
                \EN{a very large amount or number of something (informal).}
                \SY{loads of; heaps of; a great deal of}
                \VI{rất nhiều, vô số (thân mật).}
                \EX{There were tons of articles to read.}
                \EX{The lab generates tons of data each week.}
                \CO{tons of work/data/problems}
            \end{ExplainCard}

            \begin{ExplainCard}{a load of}[idiom][B2]
                \EN{a large quantity of something; many/much (informal).}
                \SY{a lot of; loads of; plenty of}
                \VI{nhiều, một đống (thân mật).}
                \EX{I picked up a load of free papers at the station.}
                \EX{We gathered a load of responses for the survey.}
                \CO{a load of papers/tasks/ideas}
            \end{ExplainCard}

            \begin{ExplainCard}{resort to}[v][C1]
                \EN{to do or use something, especially something undesirable, because no other options are available.}
                \SY{turn to; fall back on; make use of}
                \VI{phải dùng/nhờ đến (giải pháp ít mong muốn) khi không còn lựa chọn khác.}
                \EX{With no Vietnamese papers around, I resorted to English ones.}
                \EX{Some sites resort to clickbait to maintain traffic.}
                \CO{resort to violence/measures/means; last resort}
            \end{ExplainCard}

            \begin{ExplainCard}{facilitate}[v][C1]
                \EN{to make a process or action easier or more likely to happen.}
                \SY{ease; enable; expedite; streamline}
                \VI{tạo điều kiện, làm cho dễ dàng hơn.}
                \EX{Reading authentic texts facilitates vocabulary growth.}
                \EX{The platform facilitates collaboration across departments.}
                \CO{facilitate learning/communication/collaboration}
            \end{ExplainCard}

            \begin{ExplainCard}{familiarize (yourself/someone) with}[v][C1]
                \EN{(1) to make someone know or understand something; (2) to learn about something yourself so you know it well.}
                \SY{acquaint; brief; accustom}
                \VI{(1) giúp ai làm quen/hiểu; (2) tự làm quen, nắm vững.}
                \EX{The course familiarizes students with academic writing.}
                \EX{I familiarized myself with the journal’s style guide.}
                \CO{familiarize yourself with; be familiarized with; training familiarizes}
            \end{ExplainCard}

            \begin{ExplainCard}{no end}[idiom][C1]
                \EN{to a very great degree; very much (informal).}
                \SY{immensely; greatly; enormously}
                \VI{rất nhiều, vô cùng.}
                \EX{Reading quality prose helped my writing no end.}
                \EX{The upgrade improved system stability no end.}
                \CO{help/improve/benefit sb no end}
            \end{ExplainCard}

            \begin{ExplainCard}{up to a point}[idiom][C1]
                \EN{partly but not completely; to some extent.}
                \SY{to a degree; partially; in part}
                \VI{ở một mức độ nào đó; phần nào.}
                \EX{Reading alone is helpful up to a point.}
                \EX{The model explains the variance up to a point, but outliers remain.}
                \CO{agree up to a point; useful/helpful up to a point}
            \end{ExplainCard}
        \end{VocabExplain}

    \noindent
    \textbf{Part 2.}
    \begin{qa}{Describe a restaurant that you enjoyed going to.}
    \begin{itemize}
    \item Where the restaurant was
    \item Why you choose this restaurant
    \item What type of food you ate in this restaurant
    \item and explain why you thought the restaurant was good.
    \end{itemize}

    I have had the opportunity to \textbf{come by} dozens of restaurants, but I am a regular customer of only one restaurant. It is Konglao, which is named the best Thai restaurant in Hanoi according to a survey of 1,000 customers in 2018. As for its location, it is situated on the third floor in Vincom center, which is \textbf{a stone away} from my house. It only takes me only 3 minutes to reach it. To the best of my knowledge, this restaurant is owned by a woman whose husband is a Thai chef so authentic Thai flavors \textbf{of the first water} can be assured here. I knew this restaurant by chance. When it was newly opened, the restaurant offered a discount for customers, so my friends and I went there to \textbf{sample} new food on the weekend. Believe it or not, I am addicted to spicy food, and \textbf{it is common knowledge} that Thai food is generally heavily spiced. I \textbf{cannot resist my temptation from} hot food like hot pot so whenever I am \textbf{on the premises}, I may \textbf{polish off} everything I am served there. If you ask me the reasons why the restaurant came highly recommended, I believe the dishes there were cooked by a world-class chef. I also sampled Thai food in a world-class restaurant awarded Michelin stars in Thailand 5 years ago, and dishes at Konglao were of the same quality. Another reason is that the prices were \textbf{affordable}. Unlike other famous Thailand restaurants, the prices were approximately 30\% lower so I was not \textbf{charged top dollar} for \textbf{wining and dining} my friends there. All in all, I consider it a restaurant which is \textbf{good value for money}.
    \end{qa}

        \begin{VocabExplain}[Part 2]
            \begin{ExplainCard}{come by}[phr.v][C1]
                \EN{(1) to visit or stop at a place briefly; (2) to obtain something, especially something hard to find.}
                \SY{(1) drop by; stop by; (2) obtain; secure}
                \VI{(1) ghé qua, tạt vào; (2) kiếm/giành được (thường là thứ khó tìm).}
                \EX{I often come by that noodle shop after work.}
                \EX{High-quality data can be hard to come by in field studies.}
                \CO{come by the office; come by something; hard-to-come-by resources}
            \end{ExplainCard}

            \begin{ExplainCard}{a stone away}[idiom][C1]
                \EN{very close; only a short distance away (often said as \textit{a stone’s throw away}).}
                \SY{nearby; close by; within walking distance}
                \VI{rất gần; cách một quãng ngắn (thường dùng: \textit{a stone’s throw away}).}
                \EX{The café is just a stone’s throw away from my dorm.}
                \EX{The campus library is a stone’s throw away from the main lecture hall.}
                \CO{a stone’s throw from/away; just a stone’s throw}
            \end{ExplainCard}

            \begin{ExplainCard}{of the first water}[idiom][C2]
                \EN{of the very highest quality or excellence.}
                \SY{first-rate; top-notch; superlative; consummate}
                \VI{hạng nhất, tuyệt hảo.}
                \EX{Their green curry is Thai cuisine of the first water.}
                \EX{Her methodological rigor is scholarship of the first water.}
                \CO{talent/quality/artistry of the first water}
            \end{ExplainCard}

            \begin{ExplainCard}{sample}[v][B2]
                \EN{to try a small amount of food or an experience in order to judge it.}
                \SY{taste; try; test}
                \VI{nếm thử, trải nghiệm thử.}
                \EX{We sampled several dishes before ordering the mains.}
                \EX{Participants sampled each prototype and rated usability.}
                \CO{sample dishes/cuisine; sample a range/variety}
            \end{ExplainCard}

            \begin{ExplainCard}{it is common knowledge}[phrase][C1]
                \EN{a fact that is widely known and generally accepted as true.}
                \SY{widely known; well known; public knowledge}
                \VI{điều ai cũng biết; kiến thức phổ biến.}
                \EX{It’s common knowledge that Thai food is spicy.}
                \EX{It is common knowledge that peer review improves research quality.}
                \CO{It is common knowledge that + clause; common-knowledge fact}
            \end{ExplainCard}

            \begin{ExplainCard}{cannot resist my temptation from}[phrase][B2]
                \EN{(natural collocation: \textit{cannot resist the temptation to do sth}) to find it very hard not to do something enjoyable.}
                \SY{give in to; succumb to; yield to}
                \VI{không cưỡng lại được cám dỗ (thường: \textit{không cưỡng lại cám dỗ làm gì}).}
                \EX{I can’t resist the temptation to order extra chili.}
                \EX{Many users cannot resist the temptation to check notifications during study.}
                \CO{resist the temptation to + V; succumb/give in to temptation}
            \end{ExplainCard}

            \begin{ExplainCard}{on the premises}[phrase][C1]
                \EN{inside or within the building and its grounds.}
                \SY{on-site; in-house; within the grounds}
                \VI{trong khuôn viên/tòa nhà.}
                \EX{Food consumed on the premises is subject to tax.}
                \EX{Only authorized personnel may remain on the premises after hours.}
                \CO{no smoking on the premises; stay/remain on the premises}
            \end{ExplainCard}

            \begin{ExplainCard}{polish off}[phr.v][C1]
                \EN{to finish something, especially food, quickly and completely.}
                \SY{devour; demolish; finish off}
                \VI{chén sạch, ăn/hoàn thành rất nhanh.}
                \EX{We polished off two bowls of tom yum in minutes.}
                \EX{The team polished off the remaining tasks before the deadline.}
                \CO{polish off a meal/plate/dessert; polish off tasks}
            \end{ExplainCard}

            \begin{ExplainCard}{affordable}[adj][B2]
                \EN{reasonably priced and within one’s budget.}
                \SY{reasonably priced; budget-friendly; economical}
                \VI{phải chăng, vừa túi tiền.}
                \EX{The lunch sets here are affordable.}
                \EX{Affordable housing remains a key policy objective.}
                \CO{affordable price/housing/options; make sth affordable}
            \end{ExplainCard}

            \begin{ExplainCard}{charged top dollar}[idiom][C1]
                \EN{to be asked to pay a very high price for something (\textit{charge/pay top dollar}).}
                \SY{charge a premium; cost a fortune; pricey}
                \VI{bị tính giá rất cao; phải trả giá đắt.}
                \EX{We weren’t charged top dollar for the set menu.}
                \EX{Flagship models often command top dollar at launch.}
                \CO{charge/pay/command top dollar for sth}
            \end{ExplainCard}

            \begin{ExplainCard}{wining and dining}[idiom][C1]
                \EN{entertaining someone with food and alcoholic drinks.}
                \SY{treat; entertain; feast}
                \VI{thiết đãi ăn uống (thường có rượu).}
                \EX{He enjoys wining and dining friends on weekends.}
                \EX{Firms spend heavily on wining and dining prospective clients.}
                \CO{wine and dine clients/guests/friends}
            \end{ExplainCard}

            \begin{ExplainCard}{good value for money}[phrase][B2]
                \EN{worth the amount paid; giving satisfactory quality or quantity for the price.}
                \SY{cost-effective; great value; economical}
                \VI{đáng đồng tiền; xứng đáng với số tiền bỏ ra.}
                \EX{This set meal is good value for money.}
                \EX{Open-source tools often provide good value for money in research.}
                \CO{offer/represent/provide good value for money}
            \end{ExplainCard}
        \end{VocabExplain}

    \noindent
    \textbf{Part 3.}
    \begin{qa}{Why do you think people go to restaurants when they want to celebrate something?}
    Booking a favored restaurant to enjoy celebrations is \textbf{the norm} in many countries, to begin with. The reasons why lots of people \textbf{feel up to} this are very simple. Firstly, because of the fast rhythm \textbf{pace of life}, people want to treat themselves to something special, and getting \textbf{dressed to kill} to have a night out is justifiable. On top of that, this event could create an opportunity for \textbf{family gatherings} and relationship strengthening. Last but not least, there is no need for the event holder to do the washing-up or give his house a clean-up afterwards.
    \end{qa}

    \begin{qa}{Which are more popular in your country: fast food restaurants or traditional restaurants? Why do you think that is?}
    Generally speaking, the locals \textbf{frequent} traditional restaurants because these places can provide \textbf{authentic} and distinctive tastes in that area. I believe many would savor their \textbf{comfort food} which reminds themselves of their hometown. By contrast, fast food chains are favoured by oversea tourists if they have \textbf{an allergy} to local food, or for working adults and even youngsters who want to save some time in the kitchen.
    \end{qa}

    \begin{qa}{Some people say that food in an expensive restaurant is always better than food in a cheap restaurant - would you agree?}
    It is not the case. While I acknowledge a \textbf{fine-dining} restaurant often \textbf{render} better services and facilities than the wallet - friendly ones, it \textbf{has nothing to do with} the flavor. I mean, the flavor of food depends much on chefs and recipes. The combination of a \textbf{cordon bleu} chef and good recipe can make customers \textbf{mouth-watering} regardless of whether that is an \textbf{upscale restaurant} or not.
    \end{qa}

    \begin{qa}{Why do you think there will be a greater choice of food available in shops in the future, or will there be less choice?}
    Well, the growing number of food choices is \textbf{foreseeable} in the upcoming future. \textbf{In light of} the fact that many \textbf{bilateral trade agreements} have been signed, commodity exchange will continue at an ever-increasing rate. That is why I suppose customers will enjoy more favorable food, even \textbf{exotic} food that is only native to some regions.
    \end{qa}

    \begin{qa}{What effects has modern technology had on the way food is produced?}
    On the one hand, the success of \textbf{mechanization} has led to \textbf{mass production} which boosts productivity and provides more choices for users. This mean food industry is now able to \textbf{accommodate} the population explosion. On the other hand, the introduction of \textbf{preservatives} to lengthen the \textbf{lifespan} of products has posed a threat to the health of users. Therefore, people should be more cautious in terms of food choice.
    \end{qa}

    \begin{qa}{How important is it for a country to be able to grow all the food it needs, without importing any from other countries?}
    Long time ago, a \textbf{self-sufficient economy} is what many countries strived for, because this would make the country not to fell vulnerable to \textbf{famine}. However, such a mechanism seemed \textbf{impractical} due to many limitations. Nowadays, technological breakthroughs have allowed countries to cultivate \textbf{genetically modified} crops that can be \textbf{resistant} to diseases and weather-related failures. From that point, the governments can expect to self provide their inhabitants and depend less on other economies.
    \end{qa}

        \begin{VocabExplain}[Part 3]
            \begin{ExplainCard}{the norm}[n][C1]
                \EN{a standard or typical pattern of behaviour in a particular group or society.}
                \SY{standard; usual practice; convention}
                \VI{điều thông thường/chuẩn mực; thông lệ.}
                \EX{Celebrating birthdays at restaurants is the norm in my city.}
                \EX{In many industries, remote work has become the norm post-pandemic.}
                \CO{become the norm; the social/cultural norm}
            \end{ExplainCard}

            \begin{ExplainCard}{feel up to (sth)}[phr.v][B2]
                \EN{to have enough energy, confidence, or willingness to do something.}
                \SY{be ready for; be inclined to; be in the mood for}
                \VI{cảm thấy đủ sức/động lực để làm gì.}
                \EX{I don’t feel up to a late night tonight.}
                \EX{Participants may not feel up to completing lengthy surveys.}
                \CO{feel up to doing sth; not feel up to it}
            \end{ExplainCard}

            \begin{ExplainCard}{pace of life}[n][B2]
                \EN{the speed at which daily activities and routines happen in a place or for a person.}
                \SY{rhythm; tempo; speed}
                \VI{nhịp sống; tốc độ sinh hoạt hằng ngày.}
                \EX{City dwellers are used to a faster pace of life.}
                \EX{A slower pace of life is often linked to lower stress indicators.}
                \CO{fast/slow pace of life; adjust to the pace of life}
            \end{ExplainCard}

            \begin{ExplainCard}{dressed to kill}[idiom][C1]
                \EN{wearing very fashionable or striking clothes intended to attract attention.}
                \SY{dressed to the nines; glamorous; sharp}
                \VI{ăn mặc cực kỳ nổi bật, gây ấn tượng mạnh.}
                \EX{Everyone at the gala was dressed to kill.}
                \EX{In hospitality research, patrons dressed to kill were perceived as higher-status customers.}
                \CO{be/turn up dressed to kill}
            \end{ExplainCard}

            \begin{ExplainCard}{family gathering}[n][B2]
                \EN{a meeting or celebration where family members come together.}
                \SY{reunion; get-together; family event}
                \VI{buổi tụ họp gia đình.}
                \EX{We had a family gathering to celebrate grandma’s birthday.}
                \EX{Family gatherings are key sites of intergenerational transmission of traditions.}
                \CO{hold/organize a family gathering; annual family gathering}
            \end{ExplainCard}

            \begin{ExplainCard}{frequent}[v][C1]
                \EN{to visit or go to a place often.}
                \SY{patronize; visit regularly; haunt}
                \VI{hay lui tới; thường xuyên ghé.}
                \EX{Locals frequent the market for breakfast.}
                \EX{Students frequently frequent libraries during exam periods.}
                \CO{frequent a café/bar/venue; a much-frequented spot}
            \end{ExplainCard}

            \begin{ExplainCard}{authentic}[adj][C1]
                \EN{genuine and true to origin; not a copy or imitation.}
                \SY{genuine; real; true-to-tradition}
                \VI{đích thực, chuẩn vị/chuẩn gốc.}
                \EX{This bistro serves authentic regional dishes.}
                \EX{Authentic materials enhance language learners’ pragmatic competence.}
                \CO{authentic cuisine/experience/flavor}
            \end{ExplainCard}

            \begin{ExplainCard}{comfort food}[n][B2]
                \EN{food that provides a feeling of well-being, often because it is familiar from childhood.}
                \SY{home-style fare; soul food; hearty food}
                \VI{món ăn quen thuộc mang lại cảm giác dễ chịu.}
                \EX{Pho is my ultimate comfort food.}
                \EX{Studies link comfort food choices to nostalgia and stress regulation.}
                \CO{eat/seek comfort food; classic comfort food}
            \end{ExplainCard}

            \begin{ExplainCard}{allergy (to sth)}[n][B2]
                \EN{a medical condition causing adverse reactions to a substance.}
                \SY{sensitivity; intolerance; hypersensitivity}
                \VI{dị ứng (với thứ gì).}
                \EX{He has an allergy to peanuts.}
                \EX{Food allergies affect a growing share of the population globally.}
                \CO{have/develop an allergy; allergy to nuts/dairy}
            \end{ExplainCard}

            \begin{ExplainCard}{fine-dining}[adj][C1]
                \EN{relating to expensive restaurants offering high-quality service and refined cuisine.}
                \SY{gourmet; high-end; upscale}
                \VI{thuộc nhà hàng cao cấp, sang trọng.}
                \EX{They chose a fine-dining venue for the anniversary.}
                \EX{Fine-dining establishments emphasize service rituals and presentation.}
                \CO{fine-dining restaurant/experience/scene}
            \end{ExplainCard}

            \begin{ExplainCard}{render}[v][C1]
                \EN{(1) to provide or give (a service); (2) to cause to become.}
                \SY{provide; deliver; make}
                \VI{(1) cung cấp (dịch vụ); (2) khiến/biến thành.}
                \EX{The kitchen renders excellent service during rush hour.}
                \EX{Supply shocks can render forecasts obsolete.}
                \CO{render services/assistance; render sth + adj}
            \end{ExplainCard}

            \begin{ExplainCard}{have nothing to do with}[phrase][C1]
                \EN{to be unrelated or not connected with something.}
                \SY{be unrelated to; be independent of}
                \VI{không liên quan đến; chẳng dính dáng.}
                \EX{Price has nothing to do with flavor for me.}
                \EX{Measurement error often has nothing to do with model choice.}
                \CO{have nothing to do with X; nothing to do with}
            \end{ExplainCard}

            \begin{ExplainCard}{cordon bleu}[adj][C2]
                \EN{(of a cook or dish) of the highest culinary standard.}
                \SY{first-rate; masterful; top-class}
                \VI{(đầu bếp/món ăn) hạng nhất, thượng hạng.}
                \EX{A cordon bleu chef designed the tasting menu.}
                \EX{Cordon bleu training emphasizes classic techniques and precision.}
                \CO{cordon bleu chef/cook/cuisine}
            \end{ExplainCard}

            \begin{ExplainCard}{mouth-watering}[adj][C1]
                \EN{smelling or looking extremely appetizing.}
                \SY{appetizing; delectable; tempting}
                \VI{kích thích vị giác; ngon chảy nước miếng.}
                \EX{The barbecue aroma was mouth-watering.}
                \EX{Menus with mouth-watering descriptions can boost sales.}
                \CO{mouth-watering aroma/photos/dishes}
            \end{ExplainCard}

            \begin{ExplainCard}{upscale restaurant}[n][C1]
                \EN{a high-end restaurant aimed at affluent customers.}
                \SY{high-end eatery; premium venue; posh restaurant}
                \VI{nhà hàng cao cấp.}
                \EX{They booked an upscale restaurant for the proposal.}
                \EX{Upscale restaurants compete on ambience and service quality.}
                \CO{dine at an upscale restaurant; upscale dining scene}
            \end{ExplainCard}

            \begin{ExplainCard}{foreseeable}[adj][C1]
                \EN{able to be predicted or expected in the near future.}
                \SY{predictable; likely; prospective}
                \VI{có thể dự đoán được (trong tương lai gần).}
                \EX{Shortages are not foreseeable this quarter.}
                \EX{Demand growth is foreseeable given demographic trends.}
                \CO{in the foreseeable future; foreseeable effects/consequences}
            \end{ExplainCard}

            \begin{ExplainCard}{in light of}[prep phrase][C1]
                \EN{considering or taking into account a particular fact.}
                \SY{given; considering; because of}
                \VI{xét tới, dựa trên (thông tin/sự kiện).}
                \EX{In light of the weather, we ate indoors.}
                \EX{Policies were revised in light of new evidence.}
                \CO{in light of the fact that ...}
            \end{ExplainCard}

            \begin{ExplainCard}{bilateral trade agreement}[n][C1]
                \EN{a pact between two countries to regulate trade terms and reduce barriers.}
                \SY{two-party trade pact; bilateral accord}
                \VI{hiệp định thương mại song phương.}
                \EX{The bilateral trade agreement lowered tariffs on fruit.}
                \EX{Bilateral trade agreements can reconfigure supply chains.}
                \CO{sign/enter a bilateral trade agreement}
            \end{ExplainCard}

            \begin{ExplainCard}{exotic}[adj][C1]
                \EN{unusual or originating in a distant foreign country; strikingly different.}
                \SY{unfamiliar; outlandish; nonnative}
                \VI{lạ, ngoại lai; khác thường.}
                \EX{The menu features several exotic spices.}
                \EX{Exotic species may disrupt native ecosystems.}
                \CO{exotic food/flavor/species}
            \end{ExplainCard}

            \begin{ExplainCard}{mechanization}[n][C1]
                \EN{the process of using machines to do work previously done by people.}
                \SY{automation; industrialization}
                \VI{cơ giới hoá.}
                \EX{Mechanization sped up rice harvesting.}
                \EX{Mechanization increases output but can displace labor.}
                \CO{agricultural/industrial mechanization; the mechanization of X}
            \end{ExplainCard}

            \begin{ExplainCard}{mass production}[n][B2]
                \EN{manufacturing large quantities of standardized products efficiently.}
                \SY{large-scale production; assembly-line production}
                \VI{sản xuất hàng loạt.}
                \EX{Canning enabled mass production of soups.}
                \EX{Mass production reduces unit costs through economies of scale.}
                \CO{move to mass production; mass-production system}
            \end{ExplainCard}

            \begin{ExplainCard}{accommodate}[v][C1]
                \EN{(1) to provide what is needed for someone; (2) to adapt or adjust to something.}
                \SY{serve; meet; adapt}
                \VI{(1) đáp ứng/chu cấp; (2) điều chỉnh thích nghi.}
                \EX{The plant expanded to accommodate demand.}
                \EX{Models were adjusted to accommodate seasonality.}
                \CO{accommodate demand/growth/needs}
            \end{ExplainCard}

            \begin{ExplainCard}{preservative}[n][B2]
                \EN{a substance used to prevent food or materials from decaying.}
                \SY{additive; stabilizer; antioxidant}
                \VI{chất bảo quản.}
                \EX{Some shoppers prefer food without preservatives.}
                \EX{Excessive preservatives may raise health concerns in studies.}
                \CO{food/chemical preservatives; contain/use preservatives}
            \end{ExplainCard}

            \begin{ExplainCard}{lifespan}[n][C1]
                \EN{the length of time something is expected to last or continue.}
                \SY{longevity; service life; durability}
                \VI{tuổi thọ; thời gian sử dụng.}
                \EX{Packaging extends the lifespan of fresh produce.}
                \EX{Battery lifespan is a key constraint in mobile devices.}
                \CO{extend/prolong lifespan; average/expected lifespan}
            \end{ExplainCard}

            \begin{ExplainCard}{self-sufficient economy}[n][C1]
                \EN{an economy that can provide for its own needs without external imports.}
                \SY{autarky; self-reliant economy}
                \VI{nền kinh tế tự cung tự cấp.}
                \EX{A fully self-sufficient economy is rare today.}
                \EX{Autarkic policies aim for a self-sufficient economy during crises.}
                \CO{move toward a self-sufficient economy; economic self-sufficiency}
            \end{ExplainCard}

            \begin{ExplainCard}{famine}[n][B2]
                \EN{a severe shortage of food leading to widespread hunger.}
                \SY{starvation; food crisis; dearth}
                \VI{nạn đói.}
                \EX{The region suffered a famine after the drought.}
                \EX{Early-warning systems help mitigate famine risk.}
                \CO{risk of famine; prevent/relieve famine}
            \end{ExplainCard}

            \begin{ExplainCard}{impractical}[adj][C1]
                \EN{not sensible or feasible in practice.}
                \SY{unworkable; unrealistic; infeasible}
                \VI{không khả thi; thiếu thực tế.}
                \EX{Total self-sufficiency is impractical.}
                \EX{The proposed algorithm proved impractical at scale.}
                \CO{impractical solution/plan/idea}
            \end{ExplainCard}

            \begin{ExplainCard}{genetically modified}[adj][C1]
                \EN{whose genetic material has been altered using biotechnology.}
                \SY{GM; engineered; bioengineered}
                \VI{biến đổi gen.}
                \EX{Genetically modified corn resists certain pests.}
                \EX{Debates about genetically modified crops involve yield and ethics.}
                \CO{genetically modified crops/organisms/food}
            \end{ExplainCard}

            \begin{ExplainCard}{resistant (to)}[adj][C1]
                \EN{not harmed or affected by something; able to withstand it.}
                \SY{impervious; immune; tolerant}
                \VI{kháng/chịu được (bệnh, hoá chất, điều kiện).}
                \EX{This variety is resistant to mildew.}
                \EX{Antibiotic-resistant strains complicate treatment protocols.}
                \CO{resistant to disease/drought/antibiotics}
            \end{ExplainCard}
        \end{VocabExplain}

    \begin{VocabHighlights}
        \VH{coverage}{(n) the reporting of news and sport in newspapers and on the radio and television}{(danh từ) sự phổ cập (thông tin)}
        \VH{voraciously}{(adv) wanting a lot of new information and knowledge}{(trạng từ) ngấu nghiến}
        \VH{at one’s disposal}{(phrase) at will}{(cụm từ) tùy ý}
        \VH{avid}{(adj) very enthusiastic about something (often a hobby)}{(tính từ) cuồng nhiệt}
        \VH{to grab one’s attention}{(phrase) to attract one’s attention}{(cụm từ) thu hút sự chú ý}
        \VH{to be au courant with}{(adj) aware of what is going on; well informed}{(tính từ) cập nhật được}
        \VH{to sum up}{(phrase) to form a judgment or opinion about someone or something}{(cụm từ) tóm lại là}
        \VH{tons of}{(phrase) a lot of something}{(cụm từ) rất nhiều}
        \VH{familiarize}{(v) to learn about something or teach somebody about something, so that you/they start to understand it}{(động từ) làm quen với}
        \VH{no end}{(idiom) very much}{(thành ngữ) đáng kể}
        \VH{up to a point}{(idiom) to a certain extent}{(thành ngữ) ở 1 chừng mực nhất định}
        \VH{can’t resist my temptation}{(phrase) can’t adjust yourself because of your desire of something}{(cụm từ) không cưỡng nổi}
        \VH{to be on the premises}{(phrase) to be inside a building or on the area of land that it is on}{(cụm từ) ở nhà hàng, ở một tòa nhà nào đó}
        \VH{to polish off}{(phr.v) finish or consume something quickly}{(cụm động từ) ăn thật nhanh}
        \VH{affordable}{(adj) not expensive}{(tính từ) giá cả phải chăng}
        \VH{to be charged top dollar}{(idiom) pay a lot of money}{(thành ngữ) trả nhiều tiền}
        \VH{to wine and dine}{(idiom) to treat someone to an expensive meal of the type that includes fine wines}{(thành ngữ) thiết đãi ai đó}
        \VH{to be good value for money}{(idiom) something that is good value is not expensive, or worth what you pay for it}{(thành ngữ) đáng giá từng đồng}
        \VH{pace of life}{(phrase) used to refer to the speed at which changes and events occur}{(cụm từ) nhịp sống}
        \VH{to be dressed to kill}{(phrase) intentionally wearing clothes that attract sexual attention and admiration}{(cụm từ) diện đồ; ăn diện}
        \VH{a family gathering}{(noun) a party or a meeting when many people of a family come together as a group}{(danh từ) tụ tập, sum họp gia đình}
        \VH{to frequent}{(v) visit (a place) often or habitually}{(động từ) đến thường xuyên}
        \VH{authentic}{(adj) known to be real and genuine and not a copy}{(tính từ) bản địa}
        \VH{to savor}{(v) enjoy food or an experience slowly, in order to appreciate it as much as possible}{(động từ) ăn chậm rãi, thưởng thức}
        \VH{comfort food}{(n) food that provides consolation or a feeling of well-being, typically any with a high sugar or other carbohydrate content and associated with childhood or home cooking}{(danh từ) món ăn yêu thích, thường gắn với kỉ ức}
        \VH{allergy}{(n) a medical condition that causes you to react badly or feel ill/sick when you eat or touch a particular substance}{(danh từ) dị ứng}
        \VH{It’s not the case}{(idiom) It isn’t true}{(thành ngữ) không đúng}
        \VH{fine dining}{(adj) a style of eating that usually takes place in expensive restaurants, where especially good food is served to people, often in a formal way}{(tính từ) (nhà hàng) cao cấp}
        \VH{to render}{(v) to express or perform something}{(động từ) cung cấp}
        \VH{to have nothing to do with}{(phrase) to be unrelated or irrelevant to someone or something}{(cụm từ) không liên quan đến ai/cái gì}
        \VH{cordon bleu}{(adj) used to refer to people who are able to cook food to the highest standard}{(tính từ) chỉ đầu bếp hạng nhất}
        \VH{upscale restaurant}{(idiom) a restaurant designed for rich people}{(thành ngữ) nhà hàng cho người giàu}
        \VH{mouth-watering}{(adj) having a very good appearance or smell that makes you want to eat}{(tính từ) thèm thuồng}
        \VH{foreseeable}{(adj) that you can predict will happen; that can be foreseen}{(tính từ) có thể nhìn thấy trước}
        \VH{in light of}{(phrase) taking (something) into consideration}{(cụm từ) dựa trên thực tế là}
        \VH{bilateral trade agreements}{(phrase) trade exclusively between two states}{(cụm từ) thương mại song phương}
        \VH{exotic}{(adj) from or in another country, especially a tropical one; seeming exciting and unusual because it is connected with foreign countries}{(tính từ) ngoại lai}
        \VH{mechanization}{(n) changes made to a process, so that the work is done by machines rather than people}{(danh từ) cơ giới hóa}
        \VH{mass production}{(phrase) the production of large quantities of a standardized article by an automated mechanical process}{(cụm từ) sản xuất hàng loạt}
        \VH{to accommodate}{(v) to provide enough space for somebody/something}{(động từ) cung cấp đủ}
        \VH{preservative}{(n) a substance used to preserve foodstuffs, wood, or other materials against decay}{(danh từ) chất bảo quản}
        \VH{lifespan}{(n) the length of time that something is likely to live, continue or function}{(danh từ) tuổi thọ}
        \VH{a self-sufficient economy}{(phrase) a system in which the does not trade with other countries because it can produce its goods and services using its natural resources, sustainable agriculture, and renewable energy}{(cụm từ) nền kinh tế tự cung tự cấp}
        \VH{famine}{(n) extreme scarcity of food}{(danh từ) nạn đói}
        \VH{impractical}{(adj) not sensible or realistic}{(tính từ) không thực tế}
        \VH{genetically modified crops}{(phrase) plants used in agriculture, the dna of which has been modified using genetic engineering methods}{(cụm từ) cây trồng biến đổi gen}
        \VH{resistant}{(adj) not affected by something; able to resist something}{(tính từ) có sức chống chịu}
    \end{VocabHighlights}
    \end{test}

    \begin{test}{TEST 3}
    \noindent
    \textbf{Part 1. Flowers}
    \begin{qa}{Do you like to have flowers in your home? [Why/Why not?]}
    \textbf{Fat chance}. Flowers, no matter how cheap or expensive they might be, have a characteristic of \textbf{withering} in a short period of time. Buying something more \textbf{durable}, \textbf{versatile} and helpful like electronic devices is \textbf{my kind of thing} \textbf{in lieu of} purchasing flowers which are of \textbf{ornamental} values only.
    \end{qa}

    \begin{qa}{Where would you go to buy flowers? [Why?]}
    As online shopping is \textbf{the in-thing}, I often ordered flowers from virtual shops \textbf{in the comfort of} my home. I admit that I do not have \textbf{aesthetic appreciation} of flowers so buying flowers online with \textbf{fixed price} tags and reasonable delivery fees benefits me a great deal. Hence, I would not have to arrive at any physical florists' to \textbf{haggle} with the sellers over the price of any bouquet of flowers anymore.
    \end{qa}

    \begin{qa}{On what occasions would you give someone flowers?}
    Well, I will be willing to present someone a bunch of flowers on special occasions such as Mother's Day, the International and Vietnamese Women's Day. \textbf{To a great extent}, flowers are \textbf{geared towards} women because of its decorative and aesthetic values. Thus, only when women are the beneficiaries do I send flowers to them.
    \end{qa}

    \begin{qa}{Are flowers important in your culture? [Why/Why not?]}
    Yes, definitely. That can be illustrated by the case of lotus. Firstly, it acts as a symbol of Vietnam Airline, a \textbf{flag carrier} in the aviation industry established more than half a century ago. Secondly, on a larger scale, it symbolizes the purity and innocence of the Vietnamese people. No matter how hard one's life may be, a Vietnamese person will always \textbf{rise above adversities}, which is somehow reflected by the fact that a lotus reaches out of mud and \textbf{grime} to emerge.
    \end{qa}

        \begin{VocabExplain}[Part 1]
            \begin{ExplainCard}{fat chance}[idiom][C1]
                \EN{almost no possibility of something happening (often ironic).}
                \SY{little chance; highly unlikely; slim odds}
                \VI{khó mà có thể xảy ra; cơ hội gần như bằng không.}
                \EX{Fat chance I’ll keep flowers alive for a week.}
                \EX{Given budget cuts, there’s fat chance the proposal will pass.}
                \CO{fat/slim chance of sth; there’s a fat chance (that) ...}
            \end{ExplainCard}

            \begin{ExplainCard}{withering}[n][C1]
                \EN{the process of drying and shriveling, especially of plants; gradual fading.}
                \SY{wilting; shriveling; decline}
                \VI{sự héo úa, tàn héo.}
                \EX{Cut roses show visible withering after a few days.}
                \EX{Without proper hydration, leaf withering accelerates markedly.}
                \CO{signs of withering; prevent/slow withering}
            \end{ExplainCard}

            \begin{ExplainCard}{durable}[adj][B2]
                \EN{able to withstand wear or damage; lasting for a long time.}
                \SY{long-lasting; hard-wearing; sturdy}
                \VI{bền, dùng được lâu.}
                \EX{I prefer durable gadgets to short-lived bouquets.}
                \EX{Durable materials reduce lifecycle replacement costs.}
                \CO{durable goods/materials/solution; highly/very durable}
            \end{ExplainCard}

            \begin{ExplainCard}{versatile}[adj][C1]
                \EN{able to be used in many different ways or for many different purposes.}
                \SY{adaptable; multifunctional; all-purpose}
                \VI{đa năng, linh hoạt.}
                \EX{A tablet is versatile for study and leisure.}
                \EX{Versatile models generalize better across datasets.}
                \CO{highly/extremely versatile; versatile tool/device}
            \end{ExplainCard}

            \begin{ExplainCard}{my kind of thing}[phrase][B2]
                \EN{something I personally enjoy or prefer.}
                \SY{my cup of tea; to my taste; up my alley}
                \VI{đúng gu/tuýp của tôi.}
                \EX{DIY tech is my kind of thing.}
                \EX{Qualitative interviews are not my kind of thing methodologically.}
                \CO{be/not be my kind of thing}
            \end{ExplainCard}

            \begin{ExplainCard}{in lieu of}[prep phrase][C1]
                \EN{instead of; in place of.}
                \SY{instead of; in place of; as a substitute for}
                \VI{thay vì; thay cho.}
                \EX{He bought plants in lieu of cut flowers.}
                \EX{Participants received gift cards in lieu of cash payments.}
                \CO{in lieu of payment/attendance/flowers}
            \end{ExplainCard}

            \begin{ExplainCard}{ornamental}[adj][C1]
                \EN{intended for decoration rather than practical use.}
                \SY{decorative; aesthetic; nonfunctional}
                \VI{mang tính trang trí.}
                \EX{The vase is purely ornamental.}
                \EX{Ornamental species may lack ecological resilience.}
                \CO{ornamental plants/values/features}
            \end{ExplainCard}

            \begin{ExplainCard}{the in-thing}[n phrase][C1]
                \EN{something currently fashionable or popular.}
                \SY{trend; craze; vogue}
                \VI{mốt/thứ đang thịnh hành.}
                \EX{Ordering bouquets online is the in-thing now.}
                \EX{Micro-credentials have become the in-thing in higher education.}
                \CO{become/remain the in-thing; the latest in-thing}
            \end{ExplainCard}

            \begin{ExplainCard}{in the comfort of}[phrase][B2]
                \EN{within the relaxed, pleasant setting of (a place), usually one’s home.}
                \SY{from; right in; within the ease of}
                \VI{ngay trong sự tiện nghi/thoải mái của (nhà mình...).}
                \EX{She shops in the comfort of her home.}
                \EX{Remote exams allow students to test in the comfort of familiar surroundings.}
                \CO{in the comfort of your home/room}
            \end{ExplainCard}

            \begin{ExplainCard}{aesthetic appreciation}[n][C1]
                \EN{the ability to perceive and value beauty or artistic qualities.}
                \SY{taste; artistic sensibility; aesthetic sense}
                \VI{khả năng thưởng thức thẩm mỹ.}
                \EX{I don’t have much aesthetic appreciation for bouquets.}
                \EX{Courses aim to cultivate students’ aesthetic appreciation of design.}
                \CO{develop/cultivate aesthetic appreciation; lack of aesthetic appreciation}
            \end{ExplainCard}

            \begin{ExplainCard}{fixed price}[n phrase][B2]
                \EN{a set, non-negotiable price.}
                \SY{set price; non-negotiable price; list price}
                \VI{giá cố định, không mặc cả.}
                \EX{Online stores show fixed price tags.}
                \EX{Fixed-price contracts transfer risk to vendors.}
                \CO{fixed price tag/contract; at a fixed price}
            \end{ExplainCard}

            \begin{ExplainCard}{haggle}[v][C1]
                \EN{to bargain persistently about the cost of something.}
                \SY{bargain; negotiate; beat down}
                \VI{trả giá, mặc cả.}
                \EX{I hate haggling at flower stalls.}
                \EX{Informal markets encourage buyers to haggle for discounts.}
                \CO{haggle over/with; haggle the price down}
            \end{ExplainCard}

            \begin{ExplainCard}{to a great extent}[phrase][C1]
                \EN{largely; for the most part.}
                \SY{to a large degree; substantially; largely}
                \VI{ở mức độ lớn; phần lớn.}
                \EX{Taste is, to a great extent, subjective.}
                \EX{Outcomes depend, to a great extent, on prior preparation.}
                \CO{to a great/large/considerable extent}
            \end{ExplainCard}

            \begin{ExplainCard}{be geared towards}[v phrase][C1]
                \EN{to be designed or intended for a particular group or purpose.}
                \SY{aimed at; tailored to; oriented toward}
                \VI{hướng tới, nhắm đến.}
                \EX{The campaign is geared towards young buyers.}
                \EX{The curriculum is geared towards employability skills.}
                \CO{geared towards/for/at + N}
            \end{ExplainCard}

            \begin{ExplainCard}{flag carrier}[n][C1]
                \EN{a nation’s principal airline recognized as its representative.}
                \SY{national airline; state carrier}
                \VI{hãng hàng không quốc gia.}
                \EX{Vietnam Airlines is the flag carrier of Vietnam.}
                \EX{Flag carriers often receive government support during crises.}
                \CO{national/official flag carrier; the flag carrier of X}
            \end{ExplainCard}

            \begin{ExplainCard}{rise above adversities}[phrase][C1]
                \EN{to overcome difficult situations or hardship.}
                \SY{overcome hardship; prevail over difficulties; surmount}
                \VI{vươn lên vượt qua nghịch cảnh.}
                \EX{She rose above adversities to finish college.}
                \EX{Communities can rise above adversities through collective action.}
                \CO{rise above challenges/adversity/obstacles}
            \end{ExplainCard}

            \begin{ExplainCard}{grime}[n][C1]
                \EN{deep-seated dirt or soot, especially that which is hard to remove.}
                \SY{dirt; filth; muck}
                \VI{bụi bẩn, cáu bẩn.}
                \EX{The lotus blooms despite the grime of the pond.}
                \EX{Microscopy revealed layers of grime on the artifact’s surface.}
                \CO{covered in grime; layers/film of grime}
            \end{ExplainCard}
        \end{VocabExplain}

    \noindent
    \textbf{Part 2.}
    \begin{qa}{Describe a meeting you remember going to at work, college or school.}
    \begin{itemize}
    \item Where and why the meeting was held
    \item Who was at the meeting
    \item What the people at the meeting talked about
    \item and explain why you remember going to this meeting.
    \end{itemize}

    There are numerous meetings that I have attended in my life, but there is one that is still \textbf{imprinted in my mind}. It is a meeting about how to \textbf{sharpen} students’ speaking skills that took place at my educational institution 3 years ago. The bottom line is that although some of my students are, hard-working, their speaking skills are not much improved. So my head teacher launched a discussion on how to boost speaking skills for students. All English trainers were supposed to show up to voice out their opinions and \textbf{reach a compromise} in teaching speaking skills. It was a \textbf{heated} discussion because many aspects related to the teaching process were analyzed. Some teachers \textbf{put the blame on} foreign teachers, because some of them were not enthusiastic about practicing with students. Some people held a firm belief that students were not \textbf{industrious} enough, so their speaking skills were so poor. At the same time, teaching methodology was believed to be the root reason. The debate was very interesting, because it was \textbf{a golden chance} for me to \textbf{have a closer look at} the philosophy in education. Several teachers \textbf{placed great emphasis} on hard work, while some \textbf{attached great importance to} passion. It was the teacher that had to \textbf{instill a sense of passion} to students. The discussion lasted for 3 hours, and thanks to it, I could \textbf{broaden my horizons} when all teachers shared their tricks of the trade in teaching.
    \end{qa}
        \begin{VocabExplain}[Part 2]
            \begin{ExplainCard}{imprinted in my mind}[phrase][C1]
                \EN{fixed very firmly in one’s memory; unforgettable.}
                \SY{etched in my memory; engraved on my mind; unforgettable}
                \VI{hằn sâu trong trí nhớ; không thể quên.}
                \EX{That debate is still imprinted in my mind.}
                \EX{Early classroom experiences can be imprinted in students’ minds for years.}
                \CO{be imprinted in/on one’s mind/memory}
            \end{ExplainCard}

            \begin{ExplainCard}{sharpen (skills)}[v][C1]
                \EN{to improve a skill and make it more effective or precise.}
                \SY{hone; refine; polish}
                \VI{mài giũa, rèn luyện (kỹ năng).}
                \EX{I joined a club to sharpen my speaking.}
                \EX{Workshops are designed to sharpen teachers’ assessment skills.}
                \CO{sharpen skills/abilities/focus; sharpen up}
            \end{ExplainCard}

            \begin{ExplainCard}{reach a compromise}[phrase][C1]
                \EN{to agree on a middle course where each side gives up part of its demands.}
                \SY{come to terms; find middle ground; strike a deal}
                \VI{đạt được thoả hiệp.}
                \EX{After an hour we reached a compromise on the schedule.}
                \EX{Committees often reach a compromise after several voting rounds.}
                \CO{reach/arrive at a compromise; a workable compromise}
            \end{ExplainCard}

            \begin{ExplainCard}{heated (discussion)}[adj][C1]
                \EN{full of strong emotions or intense argument.}
                \SY{intense; passionate; fiery}
                \VI{sôi nổi, căng thẳng.}
                \EX{The meeting turned into a heated debate.}
                \EX{Heated discussions frequently precede curricular changes.}
                \CO{heated debate/argument/exchange}
            \end{ExplainCard}

            \begin{ExplainCard}{put the blame on}[phrase][B2]
                \EN{to say or think that someone/something is responsible for a problem.}
                \SY{blame; fault; lay responsibility on}
                \VI{đổ lỗi cho.}
                \EX{They put the blame on the new syllabus.}
                \EX{Media narratives often put the blame on teachers for low scores.}
                \CO{put/place/lay the blame on sb/sth}
            \end{ExplainCard}

            \begin{ExplainCard}{industrious}[adj][C1]
                \EN{regularly working very hard; diligent.}
                \SY{diligent; hardworking; assiduous}
                \VI{chăm chỉ, siêng năng.}
                \EX{She’s industrious and rarely procrastinates.}
                \EX{Industrious students tend to achieve higher term GPAs.}
                \CO{highly/remarkably industrious; an industrious worker/student}
            \end{ExplainCard}

            \begin{ExplainCard}{a golden chance}[n phrase][C1]
                \EN{an excellent and perhaps unique opportunity.}
                \SY{prime opportunity; perfect chance; window of opportunity}
                \VI{cơ hội vàng.}
                \EX{That talk was a golden chance to ask questions.}
                \EX{Scholarships offer a golden chance for first-generation students.}
                \CO{miss/seize a golden chance/opportunity}
            \end{ExplainCard}

            \begin{ExplainCard}{have a closer look at}[phrase][B2]
                \EN{to examine something more carefully or in more detail.}
                \SY{inspect; scrutinize; examine}
                \VI{nhìn/khảo sát kỹ hơn.}
                \EX{Let’s have a closer look at our notes.}
                \EX{The study has a closer look at classroom feedback dynamics.}
                \CO{have/take a closer look at sth}
            \end{ExplainCard}

            \begin{ExplainCard}{place great emphasis on}[phrase][C1]
                \EN{to give particular importance or attention to something.}
                \SY{stress; highlight; underscore}
                \VI{đặt nặng/nhấn mạnh vào.}
                \EX{Our school places great emphasis on speaking practice.}
                \EX{Policy documents place great emphasis on formative assessment.}
                \CO{place/put great emphasis on + N}
            \end{ExplainCard}

            \begin{ExplainCard}{attach great importance to}[phrase][C1]
                \EN{to consider something very significant or valuable.}
                \SY{value highly; regard as crucial; prioritize}
                \VI{coi trọng; gán tầm quan trọng lớn cho.}
                \EX{Teachers attach great importance to motivation.}
                \EX{Governments attach great importance to educational equity.}
                \CO{attach importance/significance/priority to sth}
            \end{ExplainCard}

            \begin{ExplainCard}{instill a sense of passion}[v phrase][C2]
                \EN{to gradually make someone feel strong enthusiasm about something.}
                \SY{imbue with zeal; foster enthusiasm; ignite passion}
                \VI{gieo/nuôi dưỡng niềm say mê.}
                \EX{A mentor instilled a sense of passion for debate in me.}
                \EX{Teacher training aims to instill a sense of passion for lifelong learning.}
                \CO{instill a sense of passion/discipline/responsibility}
            \end{ExplainCard}

            \begin{ExplainCard}{broaden (one's) horizons}[idiom][C1]
                \EN{to increase one’s knowledge or experiences beyond the usual limits.}
                \SY{expand outlook; widen perspective; open one’s eyes}
                \VI{mở rộng tầm nhìn/hiểu biết.}
                \EX{Conferences help me broaden my horizons.}
                \EX{Study-abroad programs significantly broaden students’ horizons.}
                \CO{broaden/widen/expand one’s horizons}
            \end{ExplainCard}
            
            \begin{ExplainCard}{tricks of the trade}[idiom][C1]
                \EN{a clever method used by people who are experienced in a particular type of work or activity.}
                \SY{tips}
                \VI{Tuyệt chiêu của người trong ngành}
                \EX{Magazines often improve photographs before they print them - it's one of the tricks of the trade.}
            \end{ExplainCard}
        \end{VocabExplain}

    \noindent
    \textbf{Part 3.}
    \begin{qa}{What are the different types of meeting that people often go to?}
    Basically, I am not kind of \textbf{meeting-goers}, it seems to me that most of the meetings I have attended are very \textbf{interminable}. But, I cannot deny that the meeting is a precious opportunity for group discussions. In most companies, the meeting will be well scheduled with a full \textbf{agenda}, and all \textbf{attendees} are required to look \textbf{presentable}. Having said that, there are also emergency meetings to tackle immediate problems which could be held at any time without fixed events.
    \end{qa}

    \begin{qa}{Some people say that no-one likes to go to meetings - what do you think?}
    Yes, as I have mentioned above, I am not fond of formal events like meetings. But, it is unfair to say nobody is fond of going to the meetings. To many, especially the \textbf{officials} or \textbf{executives}, meetings are part of their working schedule. To employees, myself included, the meeting is a platform to raise the voices and to \textbf{articulate} the ideas effectively. A meeting’s success depends heavily on the speakers and the discussion contents. If two factors are believed to be \textbf{beneficial} for attendees, they will be more than willing to join these meetings.
    \end{qa}

    \begin{qa}{Why can it sometimes be important to go to meetings?}
    Basically, while some think meetings are a waste of time, I do believe they are of great value to both managers and staff. In some corporations, the meeting is an occasion for \textbf{stakeholders} to resolve \textbf{queries} as well as to examine documents in detail. Moreover, team building and human resource strengthening are obvious benefits of meetings, which provide more \textbf{competent} staff for the company.
    \end{qa}

    \begin{qa}{Why do you think world leaders often have meetings together?}
    There are, of course, a \textbf{profusion} of reasons to explain for this. You know, the whole world is changing all the time, and problems \textbf{arise} on a minute basis. That is where governments and officials will address \textbf{pressing} issues with regard to education or finance. They often vote for compromise proposals and \textbf{veto} controversial ones.
    \end{qa}

    \begin{qa}{What possible difficulties might be involved in organising meetings between world leaders?}
    Well, this is a really \textbf{intriguing} question that is not easy to find an appropriate answer. World leaders, especially political authorities often, have opposite views and interests in many issues. Because leaders are national representatives, they would protect their national interests and \textbf{strike a blow} against unfavorable conditions. Therefore, I guess the most challenge is to resolve \textbf{conflict of interests} among interest groups
    \end{qa}

    \begin{qa}{Do you think that meetings between international leaders will become more frequent in the future? Or will there be less need for world leaders to meet?}
    As \textbf{social unrest} and \textbf{terrorism} are becoming more serious than ever before, summits among global leaders are likely to be convened more often to \textbf{ward off} global disasters. However, apart from physical meetings which encourage face-to-face communication, virtual contacts will be more \textbf{prevalent} and \textbf{dominate} the traditional method. This could reduce the time for travelling while securing effective communication.
    \end{qa}

        \begin{VocabExplain}[Part 3]
            \begin{ExplainCard}{meeting-goer}[n][C1]
                \EN{a person who regularly attends meetings.}
                \SY{attendee; participant; delegate}
                \VI{người thường xuyên dự họp.}
                \EX{I’m not much of a meeting-goer.}
                \EX{Frequent meeting-goers reported higher coordination scores.}
                \CO{regular/reluctant meeting-goer; conference-goer}
            \end{ExplainCard}

            \begin{ExplainCard}{interminable}[adj][C2]
                \EN{seemingly endless and often tedious.}
                \SY{endless; never-ending; wearisome}
                \VI{dài lê thê, không dứt.}
                \EX{That briefing felt interminable.}
                \EX{Interminable meetings reduce task efficiency.}
                \CO{an interminable meeting/wait/debate}
            \end{ExplainCard}

            \begin{ExplainCard}{agenda}[n][B2]
                \EN{(1) a list of items to be discussed at a meeting; (2) a hidden plan or aim.}
                \SY{schedule; docket; (2) motive; purpose}
                \VI{(1) chương trình nghị sự; (2) ý đồ/nghị trình ngầm.}
                \EX{We kept to a tight agenda.}
                \EX{Critics suspected a political agenda behind the reform.}
                \CO{set/follow the agenda; hidden/personal agenda}
            \end{ExplainCard}

            \begin{ExplainCard}{attendee}[n][B2]
                \EN{a person who is present at a meeting or event.}
                \SY{participant; delegate; guest}
                \VI{người tham dự.}
                \EX{All attendees must sign in.}
                \EX{Attendees completed a post-seminar survey.}
                \CO{registered attendee; number of attendees}
            \end{ExplainCard}

            \begin{ExplainCard}{presentable}[adj][B2]
                \EN{neat and suitable in appearance for public view.}
                \SY{tidy; smart; respectable}
                \VI{gọn gàng, chỉnh tề.}
                \EX{Dress presentably for the client visit.}
                \EX{Frontline staff are required to look presentable at all times.}
                \CO{look/appear presentable; presentable attire}
            \end{ExplainCard}

            \begin{ExplainCard}{official}[n][C1]
                \EN{a person holding public office or having authority in an organization.}
                \SY{functionary; authority; officeholder}
                \VI{quan chức; cán bộ; người có thẩm quyền.}
                \EX{City officials attended the briefing.}
                \EX{Officials released the policy draft for consultation.}
                \CO{government/union/party official}
            \end{ExplainCard}

            \begin{ExplainCard}{executive}[n][C1]
                \EN{a senior manager with decision-making authority.}
                \SY{manager; director; administrator}
                \VI{lãnh đạo cấp cao; giám đốc điều hành.}
                \EX{Top executives joined the call.}
                \EX{Executives allocate budgets during quarterly reviews.}
                \CO{chief executive; senior/corporate executive}
            \end{ExplainCard}

            \begin{ExplainCard}{articulate}[v][C1]
                \EN{to express an idea or feeling clearly and effectively.}
                \SY{express; voice; convey}
                \VI{diễn đạt/biểu đạt rõ ràng.}
                \EX{She articulated the plan convincingly.}
                \EX{Leaders must articulate strategic priorities.}
                \CO{articulate a vision/argument/concern}
            \end{ExplainCard}

            \begin{ExplainCard}{beneficial}[adj][B2]
                \EN{producing good results; helpful.}
                \SY{advantageous; helpful; favorable}
                \VI{có lợi, hữu ích.}
                \EX{Workshops are beneficial to interns.}
                \EX{Cross-team meetings proved beneficial for knowledge transfer.}
                \CO{beneficial to/for; mutually beneficial}
            \end{ExplainCard}

            \begin{ExplainCard}{stakeholder}[n][C1]
                \EN{a person or group with an interest in or affected by a decision.}
                \SY{interested party; shareholder; constituent}
                \VI{bên liên quan.}
                \EX{Consult all stakeholders before rollout.}
                \EX{Stakeholder engagement increases project buy-in.}
                \CO{key/primary stakeholders; stakeholder engagement}
            \end{ExplainCard}

            \begin{ExplainCard}{query}[n][C1]
                \EN{a question, especially one expressing doubt or requiring information.}
                \SY{question; inquiry; doubt}
                \VI{thắc mắc; câu hỏi.}
                \EX{Email your queries to HR.}
                \EX{The committee logged queries about data sources.}
                \CO{raise/answer queries; customer queries}
            \end{ExplainCard}

            \begin{ExplainCard}{competent}[adj][C1]
                \EN{having the necessary skills or knowledge to do something well.}
                \SY{capable; proficient; skilled}
                \VI{thành thạo, có năng lực.}
                \EX{She’s a competent facilitator.}
                \EX{Training aims to build a competent workforce.}
                \CO{competent staff/professional; be competent in}
            \end{ExplainCard}

            \begin{ExplainCard}{profusion}[n][C2]
                \EN{a very large quantity of something.}
                \SY{abundance; plenty; multitude}
                \VI{sự dồi dào, vô số.}
                \EX{A profusion of ideas emerged.}
                \EX{Reports noted a profusion of policy options.}
                \CO{a profusion of choices/flowers/data}
            \end{ExplainCard}

            \begin{ExplainCard}{arise}[v][C1]
                \EN{to begin to exist; to occur.}
                \SY{occur; emerge; crop up}
                \VI{nảy sinh, phát sinh.}
                \EX{Issues arise when goals are unclear.}
                \EX{New risks may arise during implementation.}
                \CO{arise from/out of; problems/needs arise}
            \end{ExplainCard}

            \begin{ExplainCard}{pressing}[adj][C1]
                \EN{urgent and needing immediate attention.}
                \SY{urgent; acute; critical}
                \VI{cấp bách, bức thiết.}
                \EX{Budget is the most pressing concern.}
                \EX{Pressing issues were prioritized on the agenda.}
                \CO{pressing need/issue/priority}
            \end{ExplainCard}

            \begin{ExplainCard}{veto}[v][C2]
                \EN{(1) to refuse to allow a decision or proposal to be enacted; (2) (n.) the official power to do this.}
                \SY{(1) overrule; reject; block}
                \VI{(1) phủ quyết/bác bỏ; (2) quyền phủ quyết.}
                \EX{The board vetoed the motion.}
                \EX{Member states can exercise a veto on security matters.}
                \CO{veto a bill/proposal; veto power/right}
            \end{ExplainCard}

            \begin{ExplainCard}{intriguing}[adj][C1]
                \EN{very interesting because it is unusual or mysterious.}
                \SY{fascinating; compelling; thought-provoking}
                \VI{hấp dẫn, gợi tò mò.}
                \EX{That’s an intriguing question.}
                \EX{The findings present an intriguing avenue for research.}
                \CO{an intriguing question/idea/pattern}
            \end{ExplainCard}

            \begin{ExplainCard}{strike a blow (against/for)}[idiom][C2]
                \EN{to take strong action to oppose or support something.}
                \SY{take a stand; fight; champion}
                \VI{ra tay/chống lại hoặc ủng hộ mạnh mẽ điều gì.}
                \EX{They struck a blow against corruption.}
                \EX{New regulations strike a blow for transparency.}
                \CO{strike a blow for/against + N}
            \end{ExplainCard}

            \begin{ExplainCard}{conflict of interest(s)}[n][C1]
                \EN{a situation in which someone’s personal interests could improperly influence decisions.}
                \SY{clash of interests; divided loyalties}
                \VI{xung đột lợi ích.}
                \EX{He recused himself due to a conflict of interest.}
                \EX{Policies require disclosure of potential conflicts of interest.}
                \CO{have/declare a conflict of interest; manage conflicts of interest}
            \end{ExplainCard}

            \begin{ExplainCard}{social unrest}[n][C1]
                \EN{disturbances caused by a group of people protesting or rioting.}
                \SY{civil disorder; turmoil; upheaval}
                \VI{bất ổn xã hội.}
                \EX{Social unrest can disrupt travel plans.}
                \EX{Periods of social unrest correlate with economic shocks.}
                \CO{periods of social unrest; spark/face social unrest}
            \end{ExplainCard}

            \begin{ExplainCard}{terrorism}[n][C1]
                \EN{the use of violence to achieve political aims by creating fear.}
                \SY{extremism; insurgency (contextual)}
                \VI{khủng bố.}
                \EX{Airports tightened security against terrorism.}
                \EX{Studies examine how terrorism affects foreign investment.}
                \CO{combat/condemn terrorism; terrorism threat}
            \end{ExplainCard}

            \begin{ExplainCard}{ward off}[phr.v][C1]
                \EN{to prevent something unpleasant or dangerous from affecting you.}
                \SY{fend off; avert; stave off}
                \VI{ngăn chặn/đẩy lùi.}
                \EX{Leaders met to ward off a crisis.}
                \EX{Vaccination helps ward off seasonal outbreaks.}
                \CO{ward off danger/risks/threats}
            \end{ExplainCard}

            \begin{ExplainCard}{prevalent}[adj][C1]
                \EN{widely existing or happening in a particular place or time.}
                \SY{widespread; common; pervasive}
                \VI{phổ biến, thịnh hành.}
                \EX{Video calls are prevalent now.}
                \EX{Remote collaboration became prevalent across sectors.}
                \CO{become/remain prevalent; prevalent practice}
            \end{ExplainCard}

            \begin{ExplainCard}{dominate}[v][B2]
                \EN{to control or have a lot of influence over something.}
                \SY{control; overshadow; prevail in}
                \VI{chi phối, thống trị; lấn át.}
                \EX{Virtual meetings dominate my week.}
                \EX{A few platforms dominate the videoconferencing market.}
                \CO{dominate the market/agenda/discussion}
            \end{ExplainCard}
        \end{VocabExplain}

    \begin{VocabHighlights}
        \VH{fat chance}{(idiom) definitely not}{(thành ngữ) chắc chắn không}
        \VH{to wither}{(v) to become, or cause something to become, weak, dry, and smaller}{(động từ) lụi tàn dần}
        \VH{durable}{(adj) able to last a long time without becoming damaged}{(tính từ) bền}
        \VH{versatile}{(adj) able to change easily from one activity to another or able to be used for many different purposes}{(tính từ) đa năng}
        \VH{my kind of thing}{(idiom) the type of person, thing, place etc that someone usually likes}{(thành ngữ) thứ ưa thích}
        \VH{in lieu of}{(phrase) instead of}{(cụm từ) thay vì}
        \VH{ornamental}{(adj) beautiful rather than useful}{(tính từ) mang tính trang trí}
        \VH{the in-thing}{(n) to be very fashionable at the moment}{(danh từ) thứ thịnh hành hiện tại}
        \VH{in the comfort of}{(phrase) at}{(cụm từ) ở nơi nào thoải mái}
        \VH{aesthetic appreciation}{(phrase) admiration of beauty}{(cụm từ) sự thẩm mỹ học}
        \VH{to haggle with}{(v) to argue with somebody in order to reach an agreement, especially about the price of something}{(động từ) mặc cả, cò kè}
        \VH{to a great extent}{(idiom) mainly}{(thành ngữ) chủ yếu là}
        \VH{to gear towards}{(phrase) to design something with a focus on a particular audience or objective}{(cụm từ) nhắm đến}
        \VH{flag carrier}{(phrase) an airline owned by or strongly identified with a nation}{(cụm từ) hãng hàng không hàng đầu ở một nước}
        \VH{to rise above adversities}{(phrase) to overcome problems}{(cụm từ) vượt qua khó khăn}
        \VH{grime}{(n) dirt that forms a layer on the surface of something}{(danh từ) bụi bề mặt}
        \VH{to be imprinted in somebody's mind}{(phrase) to be put something firmly and deeply into something else, or to be put into something in this way}{(cụm từ) khắc ghi vào}
        \VH{to sharpen}{(v) make or grow sharp}{(động từ) mài giũa}
        \VH{to reach a compromise}{(phrase) gain/achieve/obtain a compromise}{(cụm từ) đạt được một thỏa hiệp}
        \VH{heated}{(adj) excited or angry}{(tính từ) gay gắt hoặc rất sôi nổi}
        \VH{to put the blame on}{(phrase) blame somebody}{(cụm từ) đổ lỗi cho ai đó}
        \VH{industrious}{(adj) busy and hard-working}{(tính từ) siêng năng}
        \VH{a golden chance}{(phrase) a rare chance that unusually happens}{(cụm từ) cơ hội vàng}
        \VH{to place great emphasis on}{(phrase) to emphasize}{(cụm từ) rất chú trọng vào}
        \VH{to attach great importance to}{(phrase) to think that something is important or true and that it should be considered seriously}{(cụm từ) rất coi trọng}
        \VH{to instill a sense of passion}{(phrase) give somebody a passion}{(cụm từ) truyền đam mê}
        \VH{to broaden somebody's horizons}{(idiom) to widen somebody's knowledge}{(thành ngữ) mở mang đầu óc, mở mang tri thức}
        \VH{tricks of the trade}{(idiom) a skill associated with a particular job that makes one more proficient, often acquired through experience}{(thành ngữ) tuyệt chiêu, bí quyết}
        \VH{a meeting-goer}{(phrase) a person who goes to the meeting}{(cụm từ) người hay đi các cuộc hội thảo, họp mặt}
        \VH{interminable}{(adj) lasting a very long time and therefore boring or annoying}{(tính từ) vô tận, liên miên}
        \VH{agenda}{(n) a list of items to be discussed at a meeting}{(danh từ) nội dung cuộc họp, hội thảo}
        \VH{attendees}{(n) a person who attends a meeting, etc}{(danh từ) người tham dự}
        \VH{presentable}{(adj) looking clean and attractive and suitable to be seen in public}{(tính từ) tươm tất}
        \VH{executives}{(n) a person who has an important job as a manager of a company or an organization}{(danh từ) người quản lý, điều hành}
        \VH{to articulate}{(v) to express or explain your thoughts or feelings clearly in words}{(động từ) thể hiện, trình bày bằng lời nói}
        \VH{stakeholders}{(n) a person or company that is involved in a particular organization, project, system, etc., especially because they have invested money in it}{(danh từ) nhà đầu tư, các bên tham gia}
        \VH{a query}{(n) a question, especially one asking for information or expressing a doubt about something}{(danh từ) câu hỏi, vấn đề}
        \VH{competent}{(adj) having enough skill or knowledge to do something well or to the necessary standard}{(tính từ) thành thạo}
        \VH{profusion}{(n) a very large quantity of something}{(danh từ) dồi dào, phong phú}
        \VH{to arise}{(v) to happen; to start to exist}{(động từ) xuất hiện, nảy sinh}
        \VH{pressing}{(adj) requiring quick or immediate action or attention}{(tính từ) bức bối}
        \VH{veto}{(n) the right to refuse to allow something to be done, especially the right to stop a law from being passed or a decision from being taken}{(danh từ) quyền phủ quyết}
        \VH{intriguing}{(adj) very interesting because of being unusual or not having an obvious answer}{(tính từ) hấp dẫn gây hứng thú}
        \VH{conflict of interests}{(phrase) a situation in which the concerns or aims of two different parties are incompatible}{(cụm từ) xung đột lợi ích}
        \VH{social unrest}{(phrase) disagreements or fighting between different groups of people}{(cụm từ) bất ổn xã hội}
        \VH{terrorism}{(n) violent action for political purposes}{(danh từ) khủng bố}
        \VH{to ward off}{(phr.v) to prevent}{(cụm động từ) phòng tránh}
        \VH{to dominate}{(v) to have control over a place or person}{(động từ) thống trị}
    \end{VocabHighlights}
    \end{test}

    \begin{test}{TEST 4}
    \noindent
    \textbf{Part 1. Television}
    \begin{qa}{How often do you watch television? [Why/Why not?]}
    The \textbf{sole} purpose of my buying a TV is watching football, especially English Premier League at weekends. That’s why I turn on the TV once or twice on weekends only. Sometimes a football match might be \textbf{kicked off} at different time slots on weekdays so I may also \textbf{watch the box} but it is definitely not \textbf{on a daily basis}.
    \end{qa}

    \begin{qa}{Which television channel do you usually watch? [Why?]}
    You see, K+, a branch of VTV, the national TV station of Vietnam, \textbf{had a monopoly of} broadcasting live football matches of the English Premier League in the territory of Vietnam. That’s why I only turn to K+ channels with a view to enjoying thrilling matches at weekends.
    \end{qa}

    \begin{qa}{Do you enjoy the advertisements on television? [Why/Why not?]}
    Definitely not. Advertisements on TV generally are aired at \textbf{commercial breaks} when a TV show or film is approaching the \textbf{climax}. I do consider TV advertisements actually \textbf{gets my goat}. Although I know that the funds generated from ads is the main source of income for all TV stations, I still wish they’d be shorter for sure.
    \end{qa}

    \begin{qa}{Do you think most programmes on television are good? [Why/Why not?]}
    Generally, TV programmes have to \textbf{undergo} strict censorships, especially in Vietnam where the main purpose of broadcasting anything lies in spreading government \textbf{propaganda} to TV viewers across the country. As a result, their contents are generally educational and beneficial.
    \end{qa}

        \begin{VocabExplain}[Part 1]
            \begin{ExplainCard}{sole}[adj][C1]
                \EN{only; single; not shared with anyone else.}
                \SY{only; exclusive; solitary}
                \VI{duy nhất; độc nhất.}
                \EX{The sole reason I switch on the TV is football.}
                \EX{The policy’s sole objective is to reduce emissions.}
                \CO{sole purpose/owner/occupant}
            \end{ExplainCard}

            \begin{ExplainCard}{kick off}[phr.v][B2]
                \EN{(1) (of a match) to start; (2) to begin an event or activity.}
                \SY{start; begin; commence}
                \VI{(1) (trận đấu) bắt đầu; (2) khởi động/mở màn.}
                \EX{The game kicks off at 7 p.m.}
                \EX{The conference will kick off with a keynote address.}
                \CO{kick off at + time; season/opener kicks off}
            \end{ExplainCard}

            \begin{ExplainCard}{watch the box}[idiom][C1]
                \EN{to watch television (informal, esp. BrE).}
                \SY{watch TV; watch telly}
                \VI{xem tivi (khẩu ngữ).}
                \EX{I rarely watch the box on weekdays.}
                \EX{Survey data show fewer teenagers watch the box than before.}
                \CO{sit and watch the box; stop watching the box}
            \end{ExplainCard}

            \begin{ExplainCard}{on a daily basis}[phrase][B2]
                \EN{every day; regularly each day.}
                \SY{every day; daily; day by day}
                \VI{hàng ngày; mỗi ngày.}
                \EX{I don’t use the TV on a daily basis.}
                \EX{The lab records sensor data on a daily basis.}
                \CO{do/use/check on a daily/weekly basis}
            \end{ExplainCard}

            \begin{ExplainCard}{have a monopoly (of/on)}[n][C1]
                \EN{to have exclusive control or possession of the supply of or trade in a service or commodity.}
                \SY{exclusive control; domination; corner (the market)}
                \VI{nắm độc quyền (về cung ứng/thương mại).}
                \EX{K+ once had a monopoly of EPL broadcasting rights.}
                \EX{State firms long had a monopoly on energy distribution.}
                \CO{have/hold a monopoly on/of; state monopoly; break a monopoly}
            \end{ExplainCard}

            \begin{ExplainCard}{commercial break}[n][B2]
                \EN{a short interruption of a TV/radio program for advertisements.}
                \SY{ad break; advertising break}
                \VI{quãng nghỉ quảng cáo trên truyền hình/phát thanh.}
                \EX{The film paused for a commercial break.}
                \EX{Commercial breaks cluster at natural scene transitions.}
                \CO{air during a commercial break; long/short commercial breaks}
            \end{ExplainCard}

            \begin{ExplainCard}{climax}[n][C1]
                \EN{the most exciting or important point of something, especially a story or event.}
                \SY{peak; culmination; high point}
                \VI{cao trào; đỉnh điểm.}
                \EX{Ads popped up right at the climax.}
                \EX{The study reaches its climax in a large-scale field experiment.}
                \CO{reach the climax; build to a climax}
            \end{ExplainCard}

            \begin{ExplainCard}{get (someone's) goat}[idiom][C1]
                \EN{to annoy or anger someone.}
                \SY{irritate; bug; rile}
                \VI{làm ai bực mình/khó chịu.}
                \EX{Long ad blocks really get my goat.}
                \EX{In meetings, interruptions tend to get participants’ goat.}
                \CO{really/always get my/your goat}
            \end{ExplainCard}

            \begin{ExplainCard}{undergo}[v][C1]
                \EN{to experience or be subjected to (something difficult or unpleasant).}
                \SY{go through; experience; endure}
                \VI{trải qua; chịu đựng.}
                \EX{TV programmes must undergo strict review.}
                \EX{The system will undergo maintenance over the weekend.}
                \CO{undergo review/changes/testing}
            \end{ExplainCard}

            \begin{ExplainCard}{propaganda}[n][C1]
                \EN{information, often biased or misleading, used to promote a political cause or viewpoint.}
                \SY{indoctrination; publicity; spin}
                \VI{tuyên truyền (thiên lệch, mang mục đích chính trị).}
                \EX{Some shows feel like pure propaganda.}
                \EX{Researchers analyze propaganda techniques in state media.}
                \CO{government/state propaganda; spread/emit propaganda}
            \end{ExplainCard}
        \end{VocabExplain}

    \noindent
    \textbf{Part 2.}
    \begin{qa}{Describe a friend of your family you remember from your childhood.}
    \begin{itemize}
    \item Who the person was
    \item How your family knew this person
    \item How often this person visited your family
    \item and explain why you remember this person.
    \end{itemize}

    I would like to tell you about a friend of my family that left me with a long-lasting impression when I was \textbf{knee-high to a grasshoper}. He is Mai Thanh Tuan. The first thing I would like to mention is that he was 50 years old, but he was always \textbf{full of beans}. Nobody would guess he was in his 40s at that time. He was working as a businessman and a teacher. It may sound strange, but I still remember vividly that he had a \textbf{true zeal} for teaching languages. Teaching was not a \textbf{financially rewarding} job, so he also did business for financial security. Concerning how my family knew him, he was a business partner. He went into partnership in setting up a language center because in my area, English teachers were \textbf{in great demand}. Although he lived away from my house, he usually \textbf{dropped in} my house on a weekly basis. I guess he wanted to discuss business strategies with my parents, but whenever he \textbf{showed up}, he always gave me candies. I \textbf{have a sweet tooth}, so I love him so much. The reason why I remember him is that he taught me how to swim. During the \textbf{dog days} of the summer, he usually took me to the swimming pool with his son. I adore him because of not only the person himself but also the things I learned from him.
    \end{qa}

        \begin{VocabExplain}[Part 2]
            \begin{ExplainCard}{knee-high to a grasshoper}[idiom][C1]
                \EN{very young; at a very early age.}
                \SY{very young; little; a child}
                \VI{còn rất nhỏ; hồi bé xíu.}
                \EX{I loved cartoons when I was knee-high to a grasshoper.}
                \EX{Many habits are formed when learners are knee-high to a grasshoper.}
                \CO{since/when knee-high to a grasshopper; remember from knee-high}
            \end{ExplainCard}

            \begin{ExplainCard}{full of beans}[idiom][C1]
                \EN{energetic and lively; showing lots of enthusiasm.}
                \SY{energetic; lively; exuberant}
                \VI{tràn đầy năng lượng, hoạt bát.}
                \EX{Grandpa is still full of beans at seventy.}
                \EX{New cohorts arrive full of beans during orientation week.}
                \CO{be/look/feel full of beans}
            \end{ExplainCard}

            \begin{ExplainCard}{true zeal}[n][C1]
                \EN{strong and sincere enthusiasm for a cause or activity.}
                \SY{passion; fervor; enthusiasm}
                \VI{lòng nhiệt huyết chân thành.}
                \EX{She shows a true zeal for teaching kids.}
                \EX{Researchers with true zeal often sustain long-term projects.}
                \CO{show/have true zeal (for); zeal for learning/teaching}
            \end{ExplainCard}

            \begin{ExplainCard}{financially rewarding}[adj phrase][C1]
                \EN{bringing good pay or clear monetary benefits.}
                \SY{well-paid; lucrative; remunerative}
                \VI{mang lại thu nhập tốt; sinh lợi.}
                \EX{Tutoring isn’t always financially rewarding at first.}
                \EX{Financially rewarding careers may not align with public-interest goals.}
                \CO{a financially rewarding job/career/choice}
            \end{ExplainCard}

            \begin{ExplainCard}{in great demand}[phrase][B2]
                \EN{wanted or needed by many people.}
                \SY{sought-after; popular; desired}
                \VI{được săn đón, có nhu cầu cao.}
                \EX{Skilled tutors are in great demand before exams.}
                \EX{Data analysts are in great demand across industries.}
                \CO{be/remain in great/high demand}
            \end{ExplainCard}

            \begin{ExplainCard}{drop in}[phr.v][B2]
                \EN{to visit someone informally, often without arranging it first.}
                \SY{stop by; pop in; come by}
                \VI{ghé qua, tạt vào (không hẹn trước).}
                \EX{He drops in our house every Friday.}
                \EX{Participants could drop in for optional consultation hours.}
                \CO{drop in (on sb); drop into a place}
            \end{ExplainCard}

            \begin{ExplainCard}{show up}[phr.v][B2]
                \EN{to arrive or appear at a place, especially unexpectedly or late.}
                \SY{turn up; appear; arrive}
                \VI{xuất hiện, đến nơi.}
                \EX{He showed up with a bag of candies.}
                \EX{Only half of the registrants showed up for the workshop.}
                \CO{show up late/early; fail to show up}
            \end{ExplainCard}

            \begin{ExplainCard}{have a sweet tooth}[idiom][B2]
                \EN{to like eating sweet foods very much.}
                \SY{love sweets; be fond of desserts}
                \VI{hảo đồ ngọt; thích ăn ngọt.}
                \EX{I have a sweet tooth, so I never refuse chocolate.}
                \EX{Students with a sweet tooth reported higher snack purchases on campus.}
                \CO{have/develop a sweet tooth}
            \end{ExplainCard}

            \begin{ExplainCard}{dog days}[n][C1]
                \EN{the hottest, most sultry period of summer; by extension, a difficult stagnant period.}
                \SY{the height of summer; heatwave period}
                \VI{những ngày nóng nực nhất của mùa hè; thời kỳ trì trệ.}
                \EX{We swam together during the dog days of August.}
                \EX{Markets often slow down in the dog days of summer.}
                \CO{during/in the dog days (of summer)}
            \end{ExplainCard}
        \end{VocabExplain}

    \noindent
    \textbf{Part 3.}
    \begin{qa}{What do you think makes someone a good friend to a whole family?}
    Personally, people wish to make friends with those who \textbf{get on like a house on fire} with them, which is why I have many acquaintances but only a few close friends. Particularly, it usually takes time to \textbf{see through} someone before I can really confide in them. Being a good friend to a family is similar as \textbf{benevolent} and \textbf{amiable} nature might attract family members. Also, another good quality, like a sense of humor, is particularly necessary if somebody wants to \textbf{assert} his or her \textbf{individuality}.
    \end{qa}

    \begin{qa}{Do you think we meet different kinds of friends at different stages of our lives? In what ways are these types of friends different?}
    Definitely, the circle of our friendship always develops and has no end. This is simply because we need to \textbf{grab every chance} to \textbf{meet up} different sorts of people in our life. At an early stage, students often \textbf{speak the same language} with their classmates and their stories are all about studies and hobbies, I guess. Once people reach their \textbf{maturity}, there is a need to build up relationships with distinctive groups of people such as colleagues, neighbours or partners.
    \end{qa}

    \begin{qa}{How easy is it to make friends with people from a different age group?}
    To be honest, a \textbf{generation gap} is expected when people \textbf{socialize with} people from different \textbf{age brackets}. It is quite challenging to befriend with those of different age ranges because everyone has distinctive characters. But, to make it become easier, I guess people should \textbf{put themselves in others’ shoes} to \textbf{uplift} age barriers and to be \textbf{on speaking terms} with their friends.
    \end{qa}

    \begin{qa}{Do you think it is possible to be friends with someone if you never meet them in person? Is this real friendship?}
    Well, I think \textbf{virtual friendship} is not a new \textbf{conception} these days. The \textbf{omnipresence} of the Internet \textbf{has allowed} people to make \textbf{small talk} with strangers from different areas in the absence of face-to-face interaction. Many may underestimate friendships established through online platforms, but I \textbf{credit} making friends with people I have never met in person is possible. Sometimes, how people make friends does not justify how strong the relationship might be.
    \end{qa}

    \begin{qa}{What kind of influences can friends have on our lives?}
    In my \textbf{recollection}, a \textbf{prominent} figure said “a true friend is the best \textbf{possession}”, and I think this \textbf{proverb} is very meaningful. For one, close friends could be the secret to \textbf{longevity} because sharing the concerns with their \textbf{confidants} might \textbf{take a load off} their mind. Furthermore, \textbf{like-minded} people can support each other to reach their goals. On the other hand, bad friends may set \textbf{egregious} examples for one to follow, which may lead to unintended consequences later on.
    \end{qa}

    \begin{qa}{How important would you say it is to have friends from different cultures?}
    It would be amazing if people can make friends with foreigners. In other words, \textbf{cross-cultural friendship} is a fantastic way to \textbf{acquire} vast knowledge of different places and to avoid \textbf{embarrassing} mistakes. This is a \textbf{contributing factor} to \textbf{bridge} that gap among nations and bring people become closer.
    \end{qa}

        \begin{VocabExplain}[Part 3]
            \begin{ExplainCard}{get on like a house on fire}[idiom][C1]
                \EN{to become very friendly with someone very quickly.}
                \SY{hit it off; get along well; be on great terms}
                \VI{hợp nhau ngay; nhanh chóng thân thiết.}
                \EX{We got on like a house on fire at the first meetup.}
                \EX{New teammates who gel quickly often get on like a house on fire.}
                \CO{get on like a house on fire with sb}
            \end{ExplainCard}

            \begin{ExplainCard}{see through (someone)}[phr.v][C1]
                \EN{to understand a person’s true character or intentions.}
                \SY{see past; discern; detect}
                \VI{nhìn thấu bản chất/ý đồ của ai.}
                \EX{It took months to see through his bragging.}
                \EX{Skilled interviewers can see through rehearsed answers.}
                \CO{see through lies/pretence/someone}
            \end{ExplainCard}

            \begin{ExplainCard}{benevolent}[adj][C1]
                \EN{kind and helpful; showing goodwill.}
                \SY{kindly; charitable; magnanimous}
                \VI{nhân hậu, tốt bụng.}
                \EX{Her benevolent manner won everyone over.}
                \EX{Benevolent leadership correlates with higher team morale.}
                \CO{benevolent attitude/leader/act}
            \end{ExplainCard}

            \begin{ExplainCard}{amiable}[adj][C1]
                \EN{pleasant and friendly in manner.}
                \SY{affable; genial; friendly}
                \VI{dễ mến, hoà nhã.}
                \EX{He’s an amiable neighbor.}
                \EX{An amiable tone helps de-escalate disagreements.}
                \CO{amiable personality/smile/disposition}
            \end{ExplainCard}

            \begin{ExplainCard}{assert}[v][C1]
                \EN{to state or demand firmly; to make your rights or opinions recognized.}
                \SY{maintain; affirm; insist}
                \VI{khẳng định; thể hiện (quyền/quan điểm).}
                \EX{She asserted her need for boundaries.}
                \EX{Writers must assert a clear thesis throughout the paper.}
                \CO{assert rights/authority/independence}
            \end{ExplainCard}

            \begin{ExplainCard}{individuality}[n][C1]
                \EN{the qualities that make a person different from others.}
                \SY{distinctiveness; identity; uniqueness}
                \VI{cá tính; bản sắc riêng.}
                \EX{His fashion shows his individuality.}
                \EX{Curricula should nurture students’ individuality alongside teamwork.}
                \CO{express/preserve individuality; sense of individuality}
            \end{ExplainCard}

            \begin{ExplainCard}{grab every chance}[phrase][B2]
                \EN{to take every opportunity that appears.}
                \SY{seize opportunities; make the most of; capitalize on}
                \VI{chớp/không bỏ lỡ cơ hội.}
                \EX{Grab every chance to meet new people.}
                \EX{Graduates should grab every chance for internships.}
                \CO{grab/seize every/any chance/opportunity}
            \end{ExplainCard}

            \begin{ExplainCard}{meet up}[phr.v][B2]
                \EN{to meet someone, often by arrangement.}
                \SY{get together; link up; rendezvous}
                \VI{gặp gỡ; tụ họp.}
                \EX{We meet up on Fridays.}
                \EX{Alumni meet up annually to expand networks.}
                \CO{meet up with sb; plan/arrange a meet-up}
            \end{ExplainCard}

            \begin{ExplainCard}{speak the same language}[idiom][B2]
                \EN{to have similar ideas or ways of thinking; to understand each other easily.}
                \SY{be on the same wavelength; see eye to eye}
                \VI{cùng quan điểm; hiểu ý nhau.}
                \EX{My best friend and I speak the same language.}
                \EX{Cross-functional teams succeed when they speak the same language.}
                \CO{truly/clearly speak the same language}
            \end{ExplainCard}

            \begin{ExplainCard}{maturity}[n][C1]
                \EN{the state of being fully developed physically or mentally.}
                \SY{adulthood; ripeness; sophistication}
                \VI{sự trưởng thành, chín chắn.}
                \EX{With maturity, priorities change.}
                \EX{Cognitive maturity affects decision-making under risk.}
                \CO{reach/gain maturity; emotional/social maturity}
            \end{ExplainCard}

            \begin{ExplainCard}{generation gap}[n][C1]
                \EN{differences in opinions or habits between younger and older people.}
                \SY{age gap; intergenerational divide}
                \VI{khoảng cách thế hệ.}
                \EX{There’s a clear generation gap in music taste.}
                \EX{The generation gap shapes workplace expectations.}
                \CO{bridge/narrow the generation gap}
            \end{ExplainCard}

            \begin{ExplainCard}{socialize with}[v][B2]
                \EN{to spend time with other people in a friendly way.}
                \SY{mingle with; mix with; interact with}
                \VI{giao lưu; giao tiếp với.}
                \EX{She likes to socialize with older colleagues.}
                \EX{Clubs help freshmen socialize with peers.}
                \CO{socialize with friends/colleagues/locals}
            \end{ExplainCard}

            \begin{ExplainCard}{age bracket}[n][C1]
                \EN{a range of ages considered as a group.}
                \SY{age group; cohort; demographic band}
                \VI{nhóm tuổi; dải tuổi.}
                \EX{The 18–25 age bracket prefers apps.}
                \EX{Responses varied sharply by age bracket.}
                \CO{in/within the X–Y age bracket}
            \end{ExplainCard}

            \begin{ExplainCard}{put oneself in others’ shoes}[idiom][C1]
                \EN{to imagine how someone else feels or thinks.}
                \SY{empathize; see from others’ perspective}
                \VI{đặt mình vào vị trí của người khác.}
                \EX{Try to put yourself in her shoes.}
                \EX{Empathy training teaches staff to put themselves in clients’ shoes.}
                \CO{put oneself in sb’s shoes}
            \end{ExplainCard}

            \begin{ExplainCard}{uplift}[v][C1]
                \EN{to raise or improve something; figuratively, to remove barriers or lift spirits.}
                \SY{elevate; boost; lift}
                \VI{nâng lên; cải thiện; (bóng) khích lệ, gỡ bỏ rào cản.}
                \EX{Shared hobbies uplift friendships.}
                \EX{Targeted policies can uplift barriers faced by minorities.}
                \CO{uplift mood/spirits; uplift barriers/standards}
            \end{ExplainCard}

            \begin{ExplainCard}{on speaking terms}[idiom][C1]
                \EN{friendly enough to talk to someone; not estranged.}
                \SY{on good terms; in contact; civil}
                \VI{nói chuyện bình thường; không cạch mặt.}
                \EX{We’re finally on speaking terms again.}
                \EX{Maintaining colleagues on speaking terms prevents conflict escalation.}
                \CO{be/keep/remain on speaking terms (with)}
            \end{ExplainCard}

            \begin{ExplainCard}{virtual friendship}[n][C1]
                \EN{a friendship maintained mainly through online interaction.}
                \SY{online friendship; digital tie}
                \VI{tình bạn trực tuyến.}
                \EX{Many teens form virtual friendships through games.}
                \EX{Virtual friendships can provide real social support, studies show.}
                \CO{build/maintain virtual friendships}
            \end{ExplainCard}

            \begin{ExplainCard}{conception}[n][C1]
                \EN{an idea or understanding of something.}
                \SY{notion; concept; perception}
                \VI{khái niệm; quan niệm.}
                \EX{His conception of friendship is broad.}
                \EX{The paper refines our conception of social capital.}
                \CO{conception of/that; traditional/modern conception}
            \end{ExplainCard}

            \begin{ExplainCard}{omnipresence}[n][C2]
                \EN{the state of being present or widespread everywhere.}
                \SY{ubiquity; pervasiveness}
                \VI{sự hiện diện khắp nơi; tính phổ biến.}
                \EX{The omnipresence of Wi-Fi changes habits.}
                \EX{Platform omnipresence reshapes media ecosystems.}
                \CO{the omnipresence/ubiquity of X}
            \end{ExplainCard}

            \begin{ExplainCard}{allow}[v][B2]
                \EN{to make it possible for something to happen.}
                \SY{enable; permit; let}
                \VI{cho phép; tạo điều kiện.}
                \EX{The Internet allows small talk across borders.}
                \EX{APIs allow systems to interoperate efficiently.}
                \CO{allow sb to do sth; allow for sth}
            \end{ExplainCard}

            \begin{ExplainCard}{small talk}[n][B2]
                \EN{polite conversation about unimportant topics.}
                \SY{chit-chat; casual talk}
                \VI{trò chuyện xã giao.}
                \EX{He’s good at small talk at parties.}
                \EX{Ice-breakers help learners practise small talk in L2.}
                \CO{make small talk; small-talk topics}
            \end{ExplainCard}

            \begin{ExplainCard}{credit (sth)}[v][C1]
                \EN{to believe or acknowledge something; to attribute.}
                \SY{believe; acknowledge; attribute}
                \VI{tin/cho là; quy cho.}
                \EX{I credit online ties as real friendships.}
                \EX{The study credits peer support with higher retention.}
                \CO{credit sb/sth with sth; credit that + clause}
            \end{ExplainCard}

            \begin{ExplainCard}{recollection}[n][C1]
                \EN{the ability to remember; a memory of something.}
                \SY{memory; remembrance; recall}
                \VI{hồi ức; ký ức.}
                \EX{To my recollection, we met in 2019.}
                \EX{Witness recollection degrades over time, research shows.}
                \CO{to my recollection; have no recollection of}
            \end{ExplainCard}

            \begin{ExplainCard}{prominent}[adj][C1]
                \EN{important and well-known; easily noticeable.}
                \SY{notable; distinguished; conspicuous}
                \VI{nổi bật; có tiếng.}
                \EX{A prominent blogger praised the book.}
                \EX{Prominent figures influence public discourse.}
                \CO{prominent role/figure/feature}
            \end{ExplainCard}

            \begin{ExplainCard}{proverb}[n][B2]
                \EN{a short well-known saying that states a general truth.}
                \SY{adage; maxim; saying}
                \VI{tục ngữ, châm ngôn.}
                \EX{“A friend in need is a friend indeed” is a proverb.}
                \EX{Proverbs encapsulate cultural values succinctly.}
                \CO{ancient/common proverb; a proverb says}
            \end{ExplainCard}

            \begin{ExplainCard}{longevity}[n][C1]
                \EN{long life or long duration.}
                \SY{long life; durability; endurance}
                \VI{tuổi thọ; sự bền lâu.}
                \EX{Friendship may boost longevity.}
                \EX{Network longevity predicts community stability.}
                \CO{human/product longevity; longevity of relationships}
            \end{ExplainCard}

            \begin{ExplainCard}{confidant}[n][C1]
                \EN{a person you trust and share private matters with.}
                \SY{trusted friend; intimate; sounding board}
                \VI{người tâm giao/tin cậy.}
                \EX{She’s my closest confidant.}
                \EX{Mentors often become students’ confidants.}
                \CO{a close/trusted confidant; confide in a confidant}
            \end{ExplainCard}

            \begin{ExplainCard}{take a load off (one's) mind}[idiom][C1]
                \EN{to make someone stop worrying about something.}
                \SY{relieve; ease; lighten}
                \VI{cất gánh lo; làm ai yên tâm.}
                \EX{Talking to friends takes a load off my mind.}
                \EX{Clear grading rubrics take a load off students’ minds.}
                \CO{really/instantly take a load off sb’s mind}
            \end{ExplainCard}

            \begin{ExplainCard}{like-minded}[adj][C1]
                \EN{having similar opinions, interests, or goals.}
                \SY{kindred; aligned; of the same mind}
                \VI{đồng chí hướng; cùng quan điểm.}
                \EX{She found like-minded friends in the club.}
                \EX{Like-minded peers facilitate collaborative learning.}
                \CO{like-minded friends/peers/communities}
            \end{ExplainCard}

            \begin{ExplainCard}{egregious}[adj][C2]
                \EN{extremely bad and noticeable.}
                \SY{outrageous; flagrant; shocking}
                \VI{tệ hại; quá đáng.}
                \EX{That was an egregious lie.}
                \EX{Egregious errors undermine data credibility.}
                \CO{egregious mistake/violation/example}
            \end{ExplainCard}

            \begin{ExplainCard}{cross-cultural friendship}[n][C1]
                \EN{friendship between people from different cultures.}
                \SY{intercultural friendship; international tie}
                \VI{tình bạn xuyên văn hoá.}
                \EX{Cross-cultural friendships broaden horizons.}
                \EX{Programs foster cross-cultural friendship on campus.}
                \CO{build/foster cross-cultural friendships}
            \end{ExplainCard}

            \begin{ExplainCard}{acquire}[v][B2]
                \EN{to gain knowledge or a skill by learning or experience.}
                \SY{gain; obtain; pick up}
                \VI{tiếp thu; đạt được (kiến thức/kỹ năng).}
                \EX{You can acquire slang from friends.}
                \EX{Learners acquire vocabulary faster through immersion.}
                \CO{acquire knowledge/skills/experience}
            \end{ExplainCard}

            \begin{ExplainCard}{embarrassing}[adj][B2]
                \EN{making you feel ashamed or uncomfortable.}
                \SY{awkward; cringe-worthy; humiliating}
                \VI{gây xấu hổ; ngượng nghịu.}
                \EX{Misusing idioms can be embarrassing.}
                \EX{Embarrassing errors often stem from false friends in L2.}
                \CO{find sth embarrassing; an embarrassing mistake/moment}
            \end{ExplainCard}

            \begin{ExplainCard}{contributing factor}[n][C1]
                \EN{one of the causes that helps bring about a result.}
                \SY{driver; cause; determinant}
                \VI{yếu tố đóng góp/tác nhân.}
                \EX{Sleep is a contributing factor to good mood.}
                \EX{Socioeconomic status is a major contributing factor in outcomes.}
                \CO{major/key contributing factor to sth}
            \end{ExplainCard}

            \begin{ExplainCard}{bridge (a gap)}[v][C1]
                \EN{to reduce the differences or distance between groups or ideas.}
                \SY{close; narrow; span}
                \VI{thu hẹp/khắc phục khoảng cách.}
                \EX{Shared projects bridge the cultural gap.}
                \EX{Exchange programs bridge gaps between institutions.}
                \CO{bridge the gap/divide/distance}
            \end{ExplainCard}
        \end{VocabExplain}

    \begin{VocabHighlights}
        \VH{sole}{(adj) only, single}{(tính từ) độc nhất}
        \VH{to kick off}{(v) start}{(động từ) bắt đầu}
        \VH{to watch the box}{(idiom) to watch TV}{(thành ngữ) xem TV}
        \VH{on a daily basis}{(phrase) daily and regularly}{(cụm từ) thường xuyên hằng ngày}
        \VH{to have a monopoly of}{(phrase) to have the complete control of trade in particular goods or the supply of a particular service}{(cụm từ) độc quyền}
        \VH{aired}{(p2) to be broadcast}{(phần từ 2) được phát sóng}
        \VH{commercial break}{(phrase) an interruption in the transmission of broadcast programming during which advertisements are broadcast}{(cụm từ) quảng cáo giữa chương trình / phim}
        \VH{climax}{(n) the most exciting or important event or point in time}{(danh từ) lúc cao trào}
        \VH{to get one’s goat}{(idiom) to irritate somebody}{(thành ngữ) gây khó chịu cho ai}
        \VH{to undergo}{(v) to experience something, especially a change or something unpleasant}{(động từ) trải qua}
        \VH{propaganda}{(n) ideas or statements that may be false or exaggerated and that are used in order to gain support for a political leader, party, etc}{(danh từ) chương trình tuyên truyền}
        \VH{to be knee-high to a grasshopper}{(idiom) to be very small or young}{(thành ngữ) rất nhỏ hoặc trẻ tuổi}
        \VH{to be full of beans}{(idiom) to be energetic}{(thành ngữ) giàu năng lượng}
        \VH{vividly}{(adv) lively}{(trạng từ) sống động}
        \VH{to have a true zeal for}{(phrase) have a passion for}{(cụm từ) có đam mê}
        \VH{financially rewarding}{(phrase) well-paid}{(cụm từ) trả lương cao}
        \VH{lucrative}{(adj) well-paid}{(tính từ) hậu hĩnh}
        \VH{in (great) demand}{(phrase) very popular and wanted by many people}{(cụm từ) có nhu cầu lớn}
        \VH{to drop in}{(phr.v) visit somebody}{(cụm động từ) thăm ai đó}
        \VH{to show up}{(phr.v) arrive or appear}{(cụm động từ) xuất hiện}
        \VH{to have a sweet tooth}{(idiom) like eating something sweet}{(thành ngữ) thích ăn đồ ngọt}
        \VH{the dog days}{(idiom) hot days}{(thành ngữ) ngày nóng nực}
        \VH{to get on like a house on fire}{(idiom) if two people get on like a house on fire, they like each other very much and become friends very quickly}{(thành ngữ) hợp cạ}
        \VH{to see through somebody}{(phr.v) to realize that someone is trying to deceive you to get an advantage, or that someone’s behaviour is intended to deceive you, and to understand the truth about the situation}{(cụm động từ) nhìn thấu bản chất, tâm can ai}
        \VH{benevolent}{(adj) kind, helpful and generous; used in the names of some organizations that give help and money to people in need}{(tính từ) nhân đạo; từ thiện}
        \VH{amiable}{(adj) pleasant; friendly and easy to like}{(tính từ) hòa nhã, hòa đồng}
        \VH{to assert}{(v) to state clearly and firmly that something is true}{(động từ) khẳng định}
        \VH{individuality}{(n) the qualities that make somebody/something different from other people or things}{(danh từ) cá tính, cái tôi cá nhân}
        \VH{to speak the same language}{(idiom) understand one another as a result of shared opinions or values}{(thành ngữ) thấu hiểu lẫn nhau}
        \VH{maturity}{(n) the quality of thinking and behaving in a sensible, adult manner}{(danh từ) sự trưởng thành}
        \VH{a generation gap}{(phrase) a difference of opinions between one generation and another regarding beliefs, politics, or values}{(cụm từ) khoảng cách thế hệ}
        \VH{to socialize with}{(v) to meet and spend time with people in a friendly way, in order to enjoy yourself}{(động từ) hòa nhập, kết bạn với}
        \VH{age brackets}{(n) people of a similar age, considered as a group}{(danh từ) nhóm tuổi}
        \VH{to put oneself in somebody’s shoes}{(idiom) to imagine oneself in the situation or circumstances of another person, so as to understand or empathize with their perspective, opinion, or point of view}{(thành ngữ) đặt bản thân vào vị trí của người khác}
        \VH{to uplift}{(v) to remove}{(động từ) gỡ bỏ}
        \VH{on speaking terms with}{(idiom) friendly enough to talk}{(thành ngữ) hòa hợp}
        \VH{virtual friendship}{(phrase) used to describe a friendship that exists in essence but not in actuality}{(cụm từ) tình bạn ảo}
        \VH{conception}{(n) the process of forming an idea or a plan}{(danh từ) khái niệm}
        \VH{omnipresence}{(n) the fact of being present everywhere}{(danh từ) sự phổ biến}
        \VH{to credit}{(v) to believe something that seems unlikely to be true}{(động từ) công nhận, cho rằng}
        \VH{recollection}{(n) a thing that you remember from the past}{(danh từ) kí ức, trí nhớ}
        \VH{possession}{(n) the state of having or owning something}{(danh từ) tài sản}
        \VH{proverb}{(n) a well-known phrase or sentence that gives advice or says something that is generally true, for example ‘waste not, want not’}{(danh từ) tục ngữ}
        \VH{longevity}{(n) long life; the fact of lasting a long time}{(danh từ) tuổi thọ}
        \VH{confidant}{(n) a person that you trust and who you talk to about private or secret things}{(danh từ) bạn tâm giao}
        \VH{to take a load off somebody’s mind}{(idiom) to relieve one’s mind of a problem or a worry}{(thành ngữ) giúp ai nhẹ nhõm}
        \VH{like-minded}{(adj) having similar ideas and interests}{(tính từ) cùng chung ý tưởng, chung sở thích}
        \VH{cross-cultural friendship}{(phrase) friendship across cultures}{(cụm từ) tình bạn giữa nhiều nền văn hóa}
        \VH{vast}{(adj) extremely large in area, size, amount, etc}{(tính từ) bao la; rộng lớn}
        \VH{embarrassing}{(adj) making you feel shy, awkward or ashamed}{(tính từ) lúng túi; bối rối}
        \VH{to bridge}{(v) to make the difference or division between two things smaller or less severe}{(động từ) thu hẹp khoảng cách}
    \end{VocabHighlights}
    \end{test}
\end{glossarymc}