\begin{glossarymc}[Cambridge 9]
    \begin{test}{TEST 1}
    \noindent
    \textbf{Part 1. Games}
    \begin{qa}{What games are popular in your country? [Why?]}
    In the past, traditional games such as “\textbf{Hide and Seek}”, “\textbf{Tug's War}”, and “\textbf{Blind Man's Bluff}”, etc. used to \textbf{prevail} but these games might be considered \textbf{a thing of the past} in my country. However, in this day and age, video games in the form of online ones such as League of Legends, Dota, \textbf{battle royale} ones like PUBG (Player's Unknown Battlegrounds), Fortnite, and so forth are really \textbf{catching on}.
    \end{qa}

    \begin{qa}{Do you play any games? [Why?/Why not?]}
    Yes, indeed. Regarding sports games, I would play \textbf{physically demanding} ones like football and tennis. However, I have been \textbf{out of practice} for a long time, for I \textbf{am having a lot on my plate} at the moment. Now, I can only play offline games like PES and FIFA for a short amount of time, say 15 to 20 minutes, per day as it is not time-consuming to play these.
    \end{qa}

    \begin{qa}{How do you people learn to play games in your country?}
    In the past, concerning traditional games, people need to learn the rules taught by their peers. In this modern world, computer and mobile games can be played after players got some \textbf{first-hand experience} after grabbing hold of the controllers. They may \textbf{get the hang of} using the buttons on gamepads, keyboards or mouses to master the games.
    \end{qa}

    \begin{qa}{Do you think it's important for people to play games? [Why?/Why not?]}
    In my point of view, it is necessary for people to play games. It is an excellent chance for players to \textbf{break the ice} and get to know each other via some battles or matches. Another advantage I would bring is its recreation. It may create leisure time for those involved and relieve their stress as well.
    \end{qa}

        \begin{VocabExplain}[Part 1]
            \begin{ExplainCard}{Hide and Seek}[n][A2]
            \EN{a children's game in which one player searches for the others who are hiding.}
            \SY{—}
            \VI{trò trốn tìm.}
            \EX{The kids played hide and seek in the yard.}
            \EX{In developmental studies, hide-and-seek is used to examine spatial memory in children.}
            \CO{play \textit{hide and seek}; a game of \textit{hide and seek}}
            \end{ExplainCard}

            \begin{ExplainCard}{Tug's War}[n][B1]
            \EN{a contest in which two teams pull on opposite ends of a rope.}
            \SY{tug-of-war}
            \VI{kéo co.}
            \EX{Our class won the school's tug-of-war.}
            \EX{The festival featured a traditional tug-of-war competition between villages.}
            \CO{a \textit{tug-of-war} match; hold/organize a \textit{tug-of-war}}
            \end{ExplainCard}

            \begin{ExplainCard}{Blind Man's Bluff}[n][B1]
            \EN{a game where a blindfolded player tries to catch the others.}
            \SY{—}
            \VI{bịt mắt bắt dê.}
            \EX{We used to play blind man's bluff at parties.}
            \EX{Historical accounts describe blind man's bluff as a popular parlour game in Victorian England.}
            \CO{play \textit{blind man's bluff}; wear a \textit{blindfold}}
            \end{ExplainCard}

            \begin{ExplainCard}{prevail}[v][C1]
            \EN{to be common or dominant; to win out over others.}
            \SY{predominate; dominate; triumph}
            \VI{thịnh hành; chiếm ưu thế.}
            \EX{Traditional values still prevail in rural areas.}
            \EX{In the long run, evidence-based practices tend to prevail over anecdotal methods.}
            \CO{\textit{prevail} in/among; \textit{prevail over} sth}
            \end{ExplainCard}

            \begin{ExplainCard}{a thing of the past}[idiom][C1]
            \EN{something that no longer exists or happens.}
            \SY{obsolete; bygone}
            \VI{điều đã lỗi thời/không còn tồn tại.}
            \EX{Paper tickets will soon be a thing of the past.}
            \EX{Many predict that cash transactions will become a thing of the past in the digital economy.}
            \CO{\textit{be}/\textit{become} a \textit{thing of the past}}
            \end{ExplainCard}

            \begin{ExplainCard}{battle royale}[n][B2]
            \EN{a multiplayer game mode where many players fight until only one (or one team) remains.}
            \SY{survival mode}
            \VI{chế độ sinh tồn nhiều người chơi; “battle royale”.}
            \EX{He spends weekends playing battle royale with friends.}
            \EX{Battle-royale mechanics have reshaped monetization strategies in the gaming industry.}
            \CO{\textit{battle royale} game/mode; drop into a \textit{battle royale}}
            \end{ExplainCard}

            \begin{ExplainCard}{catch on}[phr.v][B2]
            \EN{to become popular; (also) to understand after a while.}
            \SY{take off; gain traction}
            \VI{trở nên phổ biến; (cũng) hiểu ra.}
            \EX{That dance trend really caught on last summer.}
            \EX{The platform caught on with users after the update improved usability.}
            \CO{\textit{quickly/slowly} catch on; catch on \textit{with} consumers}
            \end{ExplainCard}

            \begin{ExplainCard}{physically demanding}[adj][B2]
            \EN{requiring a lot of physical effort or strength.}
            \SY{strenuous; taxing}
            \VI{đòi hỏi thể lực cao; nặng về thể chất.}
            \EX{Rock climbing is physically demanding.}
            \EX{The study examined injury risks in physically demanding occupations.}
            \CO{\textit{physically demanding} job/sport/training}
            \end{ExplainCard}

            \begin{ExplainCard}{out of practice}[idiom][B2]
            \EN{not as skilled as before because you have not done something for a while.}
            \SY{rusty}
            \VI{mất tay nghề/không quen tay vì lâu không làm.}
            \EX{I'm out of practice with the piano.}
            \EX{Participants reported feeling out of practice after a prolonged training hiatus.}
            \CO{\textit{be/feel} out of practice; get back \textit{into practice}}
            \end{ExplainCard}

            \begin{ExplainCard}{have a lot on one's plate}[idiom][B2]
            \EN{to be very busy or have many responsibilities to deal with.}
            \SY{be snowed under; swamped}
            \VI{bận ngập đầu; có quá nhiều việc phải lo.}
            \EX{Sorry I can't join—I've got a lot on my plate.}
            \EX{Managers often have a lot on their plate during peak seasons.}
            \CO{\textit{have/get} a lot on your plate; too much on one's plate}
            \end{ExplainCard}

            \begin{ExplainCard}{first-hand experience}[n][B2]
            \EN{knowledge gained by directly doing or seeing something yourself.}
            \SY{direct experience}
            \VI{trải nghiệm trực tiếp; kinh nghiệm tận tay.}
            \EX{Volunteering gave me first-hand experience of teaching.}
            \EX{Researchers sought first-hand experience through field observations.}
            \CO{\textit{gain/obtain} first-hand experience; a first-hand account}
            \end{ExplainCard}

            \begin{ExplainCard}{get the hang of}[idiom][B2]
            \EN{to learn how to do something well, especially when it is not easy at first.}
            \SY{get the knack of; master (basics)}
            \VI{nắm được cách làm; quen tay.}
            \EX{Give it a week and you'll get the hang of it.}
            \EX{Students quickly got the hang of the new interface after a short tutorial.}
            \CO{\textit{get the hang of} doing sth; quickly \textit{get the hang of}}
            \end{ExplainCard}

            \begin{ExplainCard}{break the ice}[idiom][B2]
            \EN{to do or say something to make people feel more relaxed in a social situation.}
            \SY{ease the tension}
            \VI{phá băng; làm không khí bớt ngại ngùng.}
            \EX{A silly game helped us break the ice.}
            \EX{Facilitators use introductions to break the ice at workshops.}
            \CO{\textit{break the ice} with sb; an \textit{ice-breaker} activity}
            \end{ExplainCard}
        \end{VocabExplain}

    \noindent
    \textbf{Part 2.}
    \begin{qa}{Describe an open-air or street market which you enjoyed visiting. You should say:}
    \begin{itemize}
    \item Where the market is
    \item What the market sells
    \item How big the market is
    \item and explain why you enjoyed visiting this market.
    \end{itemize}

    I'm going to describe a street market that I've enjoyed visiting many times in Ha Long. It's Cai Dam Market, and it is open only on weekends. The market stalls are spread across several sites in the city centre, but the \textbf{centrepiece} is the large market on Tran Hung Dao street, which is \textbf{within easy walking distance} from Royal hotel, a four-star one. Cai Dam Market stalls sell \textbf{an array} of nice clothes, \textbf{mouth-watering} food, drink and stuff like that. It's a great place to find handmade crafts such as jewellery, ornaments, wooden toys and other souvenirs, but it's clothes that seem to be most popular. Probably most of clothes are imported from Thailand. To the best of my knowledge, the market was originally quite small, but it's grown quickly in recent years, spilling over into a few other squares. Apparently, there are over 300 stalls now, so it's become a \textbf{must-visit place} among tourists. The main reason I've always enjoyed visiting Cai Dam Market is the prices of items. Clothes are \textbf{good value for money}, so I might \textbf{shop until I drop}. A special thing is that some stalls reduce prices if I buy items \textbf{in bulk}. Another reason is its fantastic atmosphere. Ha Long seemed to come alive when the market opened. I will never forget looking out of my window and it felt like I was in a scene.
    \end{qa}

        \begin{VocabExplain}[Part 2]
            \begin{ExplainCard}{centrepiece}[n][C1]
            \EN{the most important or attractive part of something.}
            \SY{focal point; highlight; showpiece}
            \VI{tâm điểm; điểm nhấn chính.}
            \EX{The fountain is the centrepiece of the square.}
            \EX{In the exhibition, the restored manuscript served as the centrepiece of the collection.}
            \CO{the \textit{centrepiece of} sth; serve as the \textit{centrepiece}}
            \end{ExplainCard}

            \begin{ExplainCard}{within easy walking distance}[phrase][B2]
            \EN{close enough to reach comfortably on foot.}
            \SY{a short walk from; close to}
            \VI{ở khoảng cách có thể đi bộ một cách dễ dàng.}
            \EX{The beach is within easy walking distance of our hotel.}
            \EX{Urban planners recommend student housing within walking distance of key facilities.}
            \CO{\textit{within (easy) walking distance of/from}}
            \end{ExplainCard}

            \begin{ExplainCard}{an array}[n][C1]
            \EN{a large and impressive variety or selection of things.}
            \SY{assortment; range; spectrum}
            \VI{một loạt; nhiều loại đa dạng.}
            \EX{The deli offers an array of cheeses and cold cuts.}
            \EX{Visitors can observe an array of species across multiple habitats.}
            \CO{\textit{an array of} options/products/services}
            \end{ExplainCard}

            \begin{ExplainCard}{mouth-watering}[adj][B2]
            \EN{looking or smelling extremely good and making you want to eat.}
            \SY{appetizing; delectable; tempting}
            \VI{ngon “chảy nước miếng”; kích thích vị giác.}
            \EX{We ordered a plate of mouth-watering barbecued ribs.}
            \EX{Food photography often relies on mouth-watering imagery to influence consumer choices.}
            \CO{\textit{mouth-watering} dishes/food/aromas}
            \end{ExplainCard}

            \begin{ExplainCard}{must-visit place}[phrase][B2]
            \EN{a destination so good or important that people should visit it.}
            \SY{unmissable spot; not-to-be-missed destination}
            \VI{điểm đến đáng ghé; không nên bỏ lỡ.}
            \EX{The night market is a must-visit place for food lovers.}
            \EX{Guides list the temple complex as a must-visit place for cultural tourism.}
            \CO{a \textit{must-visit place}/destination/attraction}
            \end{ExplainCard}

            \begin{ExplainCard}{good value for money}[idiom][B2]
            \EN{worth the amount paid; offering quality or quantity at a fair price.}
            \SY{cost-effective; economical; a good buy}
            \VI{đáng đồng tiền; xứng đáng với số tiền bỏ ra.}
            \EX{This jacket is good value for money and lasts for years.}
            \EX{Customers rated the service as good value for money compared with rivals.}
            \CO{\textit{offer/represent} good value (for money); best value for money}
            \end{ExplainCard}

            \begin{ExplainCard}{shop until I drop}[idiom][B2]
            \EN{to go shopping for a long time, often until you are exhausted.}
            \SY{go on a shopping spree}
            \VI{mua sắm đến kiệt sức; “tới bến”.}
            \EX{On sale day I could shop until I drop.}
            \EX{Many tourists intend to shop until they drop at outlet malls.}
            \CO{\textit{shop until/till you drop}; a \textit{shopping spree}}
            \end{ExplainCard}

            \begin{ExplainCard}{in bulk}[adv][B2]
            \EN{in large quantities, often at a reduced unit price.}
            \SY{wholesale; en masse}
            \VI{(mua/bán) với số lượng lớn; mua sỉ.}
            \EX{We buy rice in bulk to save money.}
            \EX{Restaurants typically order staple ingredients in bulk from wholesalers.}
            \CO{\textit{buy/order} in bulk; ship/store \textit{in bulk}}
            \end{ExplainCard}
        \end{VocabExplain}

    \noindent
    \textbf{Part 3.}
    \begin{qa}{Do people in your country enjoy going to open-air markets that sell things like food or clothes or old objects? Which type of market is more popular? Why?}
    Oh, yes. \textbf{Market square} or \textbf{toad market}, which are well-known among Vietnamese people, have become a \textbf{staple} of daily life in my country. People in all age groups enjoy \textbf{shopping around} these places because not only is the food \textbf{decent} and fresh but used items like clothes and household items, many of which are still in \textbf{pristine conditions}, are on display there. But, I do not think this would be the first choice for people who are \textbf{booked solid} because they'd prefer to buy frozen products in supermarkets for a whole week.
    \end{qa}

    \begin{qa}{Do you think markets are more suitable places for selling certain types of things? Which ones? Why do you think this is?}
    Although the widespread popularity of shopping malls has \textbf{marginalized} the role of traditional markets in the modern world, I do believe the markets still serve their own function as many products, especially home-made and local ones, are mainly sold there. \textbf{Perishable} food such as vegetables and fish are often displayed in the market for customers to \textbf{sample} before they \textbf{snap these dishes up}. However, this type of food is rarely sold in supermarkets because of its \textbf{pungent} smell.
    \end{qa}

    \begin{qa}{Do you think young people feel the same about shopping at markets as older people? Why is that?}
    It is dependable. Some young people still consider that buying things at markets is a tradition, and they love the culturally \textbf{vibrant} atmosphere and \textbf{superb} food in the morning market. In Vietnam, many housewives and their daughters often go to outdoor markets to prepare for \textbf{whipping up} family meals. But, for other young people, they would \textbf{lean towards} online shopping if they can because they may \textbf{while away} their free time going to the market.
    \end{qa}

    \begin{qa}{What do you think are the advantages of buying things from shops rather than markets?}
    Undeniably, the main positive aspect of shops or department stores is \textbf{adequate} facilities. With air-conditions, elevators and the like, shopping \textbf{inside} malls is more pleasurable than touring outdoor markets, especially in hot \textbf{sticky} months. Besides, most commodities have gone through product \textbf{inspections} before they are got into supermarkets, so customers could save lots of time \textbf{scouting around} quality products.
    \end{qa}

    \begin{qa}{How does advertising influence what people choose to buy? Is this true for everyone?}
    Honesty, advertising industry has \textbf{deliberately manipulated} the way people do shopping. Since the \textbf{advent} of advertisements like TV commercials or social media ads, customers, especially young people, can get themselves \textbf{genned up on} latest products through media with ease. But, it seems to me that the elderly still like getting \textbf{word-of-mouth recommendations} because they are more reliable.
    \end{qa}

    \begin{qa}{Do you think that any recent changes in the way people live have affected general shopping habits? Why is this?}
    In the fast-changing world today, the fluctuation in shopping habits is an \textbf{inevitable} trend. Few decades ago, retail stores were \textbf{du jour} and shoppers \textbf{flocked to} these places every day. But, the exploration of Internet shopping has put many physical stores \textbf{on the verge of bankruptcy}. For example, Forever 21, one of the most world-famous fashion brands worldwide, was \textbf{closed down} due to their inability to compete with online rivals. Customers nowadays would \textbf{opt for} online choices because of time efficiency and irresistible available online bargains.
    \end{qa}

        \begin{VocabExplain}[Part 3]
            \begin{ExplainCard}{Market square}[n][B2]
            \EN{an open public space in a town where markets are held.}
            \SY{town square; marketplace}
            \VI{quảng trường chợ (nơi họp chợ ngoài trời).}
            \EX{Vendors set up in the market square every weekend.}
            \EX{Urban studies highlight how the market square fosters civic interaction.}
            \CO{historic \textit{market square}; stall in the \textit{market square}}
            \end{ExplainCard}

            \begin{ExplainCard}{toad market}[n][C1]
            \EN{a makeshift, informal street market that pops up temporarily (Vietnamese context).}
            \SY{pop-up market; makeshift street market}
            \VI{\textit{chợ cóc}, chợ tạm họp ven đường.}
            \EX{I grab fruit at the toad market near my alley.}
            \EX{Informal “toad markets” illustrate grey-economy retail in Vietnamese cities.}
            \CO{\textit{street-side} toad market; a \textit{temporary} toad market}
            \end{ExplainCard}

            \begin{ExplainCard}{staple}[n][C1]
            \EN{a main or important feature of something; also a basic food or product.}
            \SY{mainstay; cornerstone}
            \VI{thành phần chủ đạo; vật phẩm thiết yếu.}
            \EX{Street food is a staple of city life here.}
            \EX{Tourism has become a staple of the regional economy.}
            \CO{\textit{a staple of} daily life; local \textit{staples}}
            \end{ExplainCard}

            \begin{ExplainCard}{shop around}[phr.v][B2]
            \EN{to compare prices or options in different shops before buying.}
            \SY{browse; compare prices}
            \VI{đi dò giá; xem nhiều nơi trước khi mua.}
            \EX{We shopped around for a cheaper blender.}
            \EX{Consumers who shop around achieve better price–quality outcomes.}
            \CO{\textit{shop around for} deals/prices}
            \end{ExplainCard}

            \begin{ExplainCard}{decent}[adj][B2]
            \EN{of satisfactory quality or standard; good enough.}
            \SY{respectable; acceptable; adequate}
            \VI{tươm tất; ổn; chấp nhận được.}
            \EX{This café serves decent noodles.}
            \EX{The survey reports a decent level of customer satisfaction.}
            \CO{\textit{decent} meal/wage/quality}
            \end{ExplainCard}

            \begin{ExplainCard}{pristine}[adj][C1]
            \EN{in perfect condition; as if new or not spoiled.}
            \SY{immaculate; unspoiled}
            \VI{như mới; nguyên vẹn; tinh tươm.}
            \EX{The second-hand jacket was in pristine condition.}
            \EX{Pristine environments are increasingly rare in urban ecosystems.}
            \CO{\textit{pristine} condition/beach/packaging}
            \end{ExplainCard}

            \begin{ExplainCard}{booked solid}[idiom][C1]
            \EN{having no available time or space; fully booked.}
            \SY{fully booked; packed}
            \VI{kín lịch/kín chỗ.}
            \EX{The dentist is booked solid this week.}
            \EX{Conference hotels were booked solid during the summit.}
            \CO{\textit{be} booked solid; schedule booked solid}
            \end{ExplainCard}

            \begin{ExplainCard}{marginalize}[v][C1]
            \EN{to treat something as less important or to push it to the edge of attention.}
            \SY{sideline; diminish}
            \VI{gạt ra rìa; làm cho kém quan trọng.}
            \EX{Small vendors feel marginalized by big malls.}
            \EX{Digitization can marginalize traditional retail formats.}
            \CO{\textit{marginalize} communities/roles/traditions}
            \end{ExplainCard}

            \begin{ExplainCard}{perishable}[adj][C1]
            \EN{(of food) likely to decay or go bad quickly.}
            \SY{short-lived; easily spoiled}
            \VI{dễ hỏng; mau hư (thực phẩm).}
            \EX{Keep perishable goods in the fridge.}
            \EX{Cold chains reduce waste in perishable supply networks.}
            \CO{\textit{perishable} goods/produce/items}
            \end{ExplainCard}

            \begin{ExplainCard}{sample}[v][B2]
            \EN{to try a small amount of something to judge its quality.}
            \SY{taste; try}
            \VI{nếm/thử mẫu.}
            \EX{We sampled cheese at the stall.}
            \EX{Participants sampled new products during the pilot study.}
            \CO{\textit{sample} dishes/products/a selection}
            \end{ExplainCard}

            \begin{ExplainCard}{snap (sth) up}[phr.v][B2]
            \EN{to buy or take something quickly because it is cheap or available.}
            \SY{grab; seize}
            \VI{chộp/lượm mua ngay khi có cơ hội.}
            \EX{Tickets were snapped up in minutes.}
            \EX{Consumers snap up discounted items during flash sales.}
            \CO{\textit{snap up} a bargain/tickets/deals}
            \end{ExplainCard}

            \begin{ExplainCard}{pungent}[adj][C1]
            \EN{having a strong, sharp smell or taste.}
            \SY{sharp; acrid; piquant}
            \VI{nồng; hăng; mùi mạnh.}
            \EX{Fish sauce has a pungent aroma.}
            \EX{Pungent compounds in onions trigger lachrymation.}
            \CO{\textit{pungent} smell/aroma/flavour}
            \end{ExplainCard}

            \begin{ExplainCard}{vibrant}[adj][B2]
            \EN{full of energy and life; (of colors/places) bright and lively.}
            \SY{lively; bustling; vivid}
            \VI{sôi động; rực rỡ.}
            \EX{The market is vibrant at dawn.}
            \EX{A vibrant street culture contributes to urban vitality.}
            \CO{\textit{vibrant} atmosphere/community/scene}
            \end{ExplainCard}

            \begin{ExplainCard}{superb}[adj][C1]
            \EN{of the highest quality; excellent.}
            \SY{excellent; outstanding; first-rate}
            \VI{tuyệt hảo; xuất sắc.}
            \EX{We had a superb breakfast there.}
            \EX{The museum offers superb curatorial practice.}
            \CO{\textit{superb} quality/performance/meal}
            \end{ExplainCard}

            \begin{ExplainCard}{whip up}[phr.v][B2]
            \EN{to quickly prepare (especially food); also to excite (feelings).}
            \SY{rustle up; throw together}
            \VI{chế biến nhanh; khuấy động.}
            \EX{She can whip up dinner in 20 minutes.}
            \EX{The campaign whipped up enthusiasm among volunteers.}
            \CO{\textit{whip up} a meal/support/interest}
            \end{ExplainCard}

            \begin{ExplainCard}{lean towards}[phr.v][B2]
            \EN{to prefer or be inclined to choose something.}
            \SY{favor; incline to}
            \VI{nghiêng về; thiên về.}
            \EX{I lean towards buying online.}
            \EX{Policymakers lean toward market-based solutions.}
            \CO{\textit{lean towards} a choice/policy/opinion}
            \end{ExplainCard}

            \begin{ExplainCard}{while away}[phr.v][C1]
            \EN{to spend time in a relaxed, pleasant way when you have nothing to do.}
            \SY{idle away; pass (the) time}
            \VI{giết thời gian một cách thư thái.}
            \EX{We whiled away the afternoon people-watching.}
            \EX{Passengers while away layovers in lounge areas.}
            \CO{\textit{while away} the hours/time/evening}
            \end{ExplainCard}

            \begin{ExplainCard}{adequate}[adj][B2]
            \EN{good enough in quality or quantity for a particular purpose.}
            \SY{sufficient; satisfactory}
            \VI{đầy đủ; tương xứng.}
            \EX{The mall has adequate parking.}
            \EX{Adequate facilities improve retail experience metrics.}
            \CO{\textit{adequate} facilities/resources/support}
            \end{ExplainCard}

            \begin{ExplainCard}{inside}[adv][A2]
            \EN{in or into the interior of a place.}
            \SY{indoors; within}
            \VI{ở bên trong; trong nhà.}
            \EX{Let's wait inside the mall.}
            \EX{Shoppers spend more during inside promotional events.}
            \CO{\textit{inside} the building/mall/shop}
            \end{ExplainCard}

            \begin{ExplainCard}{sticky}[adj][B2]
            \EN{(of weather) hot and humid; making you feel sweaty.}
            \SY{muggy; humid}
            \VI{oi bức; ẩm nóng.}
            \EX{It's too sticky to walk around the market today.}
            \EX{Heatwaves cause prolonged periods of sticky conditions in the tropics.}
            \CO{\textit{sticky} weather/night/summer}
            \end{ExplainCard}

            \begin{ExplainCard}{inspection}[n][B2]
            \EN{a careful check or examination of something.}
            \SY{examination; audit}
            \VI{kiểm tra; giám định.}
            \EX{Food passes safety inspections before sale.}
            \EX{Regulatory inspections ensure compliance with hygiene standards.}
            \CO{carry out/conduct an \textit{inspection}; routine \textit{inspections}}
            \end{ExplainCard}

            \begin{ExplainCard}{scout around}[phr.v][B2]
            \EN{to look in various places for something.}
            \SY{look around; hunt for; search}
            \VI{lục tìm/đi xem khắp nơi để tìm.}
            \EX{We scouted around for the best price.}
            \EX{Researchers scouted around local shops to source materials.}
            \CO{\textit{scout around for} bargains/venues/options}
            \end{ExplainCard}

            \begin{ExplainCard}{manipulate}[v][C1]
            \EN{to control or influence someone or something, often unfairly or cleverly.}
            \SY{exploit; sway; engineer}
            \VI{thao túng; điều khiển khéo (thường tiêu cực).}
            \EX{Ads can deliberately manipulate consumer choices.}
            \EX{The study shows how framing effects manipulate preference formation.}
            \CO{\textit{deliberately} manipulate; \textit{manipulate} public opinion/data}
            \end{ExplainCard}

            \begin{ExplainCard}{advent}[n][C1]
            \EN{the arrival or beginning of something important.}
            \SY{arrival; onset; emergence}
            \VI{sự ra đời; sự xuất hiện.}
            \EX{With the advent of smartphones, shopping changed.}
            \EX{The advent of e-commerce disrupted traditional retail.}
            \CO{the \textit{advent of} technology/era/platforms}
            \end{ExplainCard}

            \begin{ExplainCard}{be/get genned up on}[phr.v][C1]
            \EN{(BrE, informal) to learn about or become well-informed on a subject.}
            \SY{get clued up on; get up to speed on}
            \VI{tìm hiểu kỹ; cập nhật kiến thức về.}
            \EX{Teens get genned up on gadgets via social media.}
            \EX{New hires are genned up on policy changes during orientation.}
            \CO{\textit{get/be} genned up on sth}
            \end{ExplainCard}

            \begin{ExplainCard}{word-of-mouth recommendation}[n][C1]
            \EN{advice shared orally from one person to another, not through ads.}
            \SY{personal referral; testimonial}
            \VI{khuyến nghị truyền miệng.}
            \EX{I chose the café via word-of-mouth recommendations.}
            \EX{Word-of-mouth recommendations strongly affect service adoption.}
            \CO{\textit{rely on} word-of-mouth; positive \textit{word-of-mouth}}
            \end{ExplainCard}

            \begin{ExplainCard}{inevitable}[adj][C1]
            \EN{certain to happen and impossible to avoid.}
            \SY{unavoidable; inescapable}
            \VI{không thể tránh khỏi; tất yếu.}
            \EX{Price rises feel inevitable this summer.}
            \EX{Automation is an inevitable outcome of digital transformation.}
            \CO{\textit{seem/become} inevitable; the \textit{inevitable} result}
            \end{ExplainCard}

            \begin{ExplainCard}{du jour}[adj][C2]
            \EN{fashionable or popular at the present time (from French).}
            \SY{trendy; in vogue; fashionable}
            \VI{thời thượng; “đang mốt”.}
            \EX{Minimalist sneakers were du jour last year.}
            \EX{The platform became the du jour choice among start-ups.}
            \CO{\textit{topic/brand/trend} du jour}
            \end{ExplainCard}

            \begin{ExplainCard}{flock to}[phr.v][C1]
            \EN{to go somewhere in large numbers.}
            \SY{throng; stream to}
            \VI{đổ xô/ùn ùn kéo đến.}
            \EX{Tourists flock to the night market at weekends.}
            \EX{Shoppers flocked to outlets during clearance events.}
            \CO{\textit{flock to} malls/attractions/events}
            \end{ExplainCard}

            \begin{ExplainCard}{on the verge of}[phrase][C1]
            \EN{very close to experiencing or doing something.}
            \SY{on the brink of; nearing}
            \VI{sắp/đang bên bờ; gần như.}
            \EX{Several shops were on the verge of closing.}
            \EX{The firm is on the verge of bankruptcy after losses.}
            \CO{\textit{on the verge of} collapse/bankruptcy/tears}
            \end{ExplainCard}

            \begin{ExplainCard}{bankruptcy}[n][C1]
            \EN{the state of not having enough money to pay debts.}
            \SY{insolvency; collapse}
            \VI{phá sản.}
            \EX{The chain filed for bankruptcy last year.}
            \EX{Retail bankruptcy rates rose during the downturn.}
            \CO{\textit{file for} bankruptcy; risk of \textit{bankruptcy}}
            \end{ExplainCard}

            \begin{ExplainCard}{close down}[phr.v][B2]
            \EN{(of a business) to stop operating permanently.}
            \SY{shut down; cease trading}
            \VI{đóng cửa; ngừng hoạt động.}
            \EX{The old bookstore closed down in June.}
            \EX{Several unprofitable outlets were closed down after restructuring.}
            \CO{\textit{close down} a store/factory; be \textit{closed down}}
            \end{ExplainCard}

            \begin{ExplainCard}{opt for}[phr.v][B2]
            \EN{to choose one thing instead of another.}
            \SY{choose; go for; prefer}
            \VI{chọn; ưu tiên cho.}
            \EX{Many customers opt for delivery now.}
            \EX{Respondents opted for online channels due to convenience.}
            \CO{\textit{opt for} online shopping/a plan/an option}
            \end{ExplainCard}
        \end{VocabExplain}

    \begin{VocabHighlights}
        \VH{hide and seek}{(n) a game in which any number of players (ideally at least three) conceal themselves in a set environment, to be found by one or more seekers}{(danh từ) trò trốn tìm}
        \VH{tug's war}{(n) a contest in which two teams pull against each other at opposite ends of a rope with the object of pulling the middle of the rope over a mark on the ground}{(danh từ) trò kéo co}
        \VH{blind man's bluff}{(phrase) a game in which one player is blindfolded and gropes around attempting to touch the other players without being able to see them, while the other players scatter}{(danh từ) trò bịt mắt bắt dê}
        \VH{to prevail}{(v) to exist or be very common at a particular time or in a particular place}{(động từ) phổ biến}
        \VH{a thing of the past}{(idiom) a thing that no longer happens or exists}{(thành ngữ) thứ không còn tồn tại}
        \VH{battle royale}{(n) online multiplayer video game genre that blends the survival, exploration, and scavenging elements of a survival game with last-man-standing gameplay}{(danh từ) trò chơi đấu tranh sinh tồn online}
        \VH{to catch on}{(phr.v) to become popular or fashionable}{(cụm động từ) trở nên phổ biến}
        \VH{physically demanding}{(adj) strenuous}{(tính từ) tốn nhiều sức lực}
        \VH{out of practice}{(idiom) not currently proficient in a particular activity or skill due to not having exercised or performed it for some time}{(thành ngữ) yếu, kém do ít tập luyện}
        \VH{to have a lot on one's plate}{(idiom) to have something, usually a large amount of important work, to deal with}{(thành ngữ) có nhiều mối bận tâm}
        \VH{to get the hang of}{(idiom) to learn how to do something when it is not simple or obvious}{(thành ngữ) học cách dùng, sử dụng cái gì}
        \VH{to break the ice}{(idiom) do or say something to relieve tension or get conversation going at the start of a party or when people meet for the first time}{(thành ngữ) nói hoặc làm gì để xóa nhòa ngại ngùng khi lần đầu gặp mặt}
        \VH{centrepiece}{(n) the most important or attractive part or feature of something}{(danh từ) trung tâm}
        \VH{within easy walking distance}{(idiom) close enough to walk/drive to in a short time}{(thành ngữ) gần}
        \VH{an array of}{(phrase) an impressive display or range of a particular type of thing}{(cụm từ) đủ chủng loại}
        \VH{mouth-watering}{(adj) having a very good appearance or smell that makes you want to eat}{(tính từ) gây thèm ăn}
        \VH{be good value for money}{(idiom) something makes the best out of money}{(thành ngữ) đáng giá tiền}
        \VH{to shop until I drop}{(idiom) do a large amount of shopping}{(thành ngữ) mua rất nhiều}
        \VH{in bulk}{(phrase) in huge amounts}{(cụm từ) số lượng lớn}
        \VH{superb}{(adj) excellent; of very good quality}{(tính từ) xuất sắc}
        \VH{to lean}{(v) to bend or move from a vertical position}{(động từ) nghiêng về phía}
        \VH{to whip up}{(phr.v) to make food or a meal very quickly and easily}{(cụm động từ) nấu ăn nhanh, dễ dàng}
        \VH{to while away something}{(phr.v) to spend time in a relaxed way because you have nothing to do or you are waiting for something else to happen}{(cụm động từ) dành thời gian tiêu khiển, giết thời gian}
        \VH{adequate}{(adj) enough in quantity, or good enough in quality, for a particular purpose or need}{(tính từ) đầy đủ, tốt}
        \VH{hot sticky}{(adj) feeling hot and uncomfortable}{(tính từ) nóng nực}
        \VH{inspection}{(n) careful examination or scrutiny}{(danh từ) sự kiểm tra, kiểm duyệt}
        \VH{to scout around}{(v) to search, inspect, or look around an area for someone or something}{(động từ) đi ngắm nghía}
        \VH{deliberately}{(adv) done in a way that was planned, not by chance}{(trạng từ) cố tình}
        \VH{to manipulate}{(v) to control or influence somebody/something, often in a dishonest way so that they do not realize it}{(động từ) điều khiển, tác động}
        \VH{to be genned up on}{(phr.v) to become fully conversant with}{(cụm động từ) có đủ thông tin về}
        \VH{word-of-mouth recommendations}{(phrase) an official suggestion about the products from customers that have used them}{(cụm từ) lời truyền miệng}
        \VH{inevitable}{(adj) that you cannot avoid or prevent}{(tính từ) không thể tránh khỏi}
        \VH{du jour}{(adj) available and being served on this day}{(tính từ) có sẵn}
        \VH{to flock}{(v) to go or gather together somewhere in large numbers}{(động từ) tụ tập; tụ họp}
        \VH{bankruptcy}{(n) the state of being bankrupt}{(danh từ) sự phá sản}
        \VH{to close down}{(phr.v) to stop operating}{(cụm động từ) đóng cửa}
    \end{VocabHighlights}
    \end{test}

    \begin{test}{TEST 2}
    \noindent
    \textbf{Part 1. Giving Gifts}
    \begin{qa}{When do people give gifts or presents in your country?}
    Gifts may be given \textbf{on the occasions of} birthdays or special anniversaries such as weddings, a baby's \textbf{full month celebration}, \textbf{longevity wishing ceremony}, etc. Sometimes, friends and relatives can offer surprising presents \textbf{out of the blue}. For example, I sometimes buy accessories for my wife as long as she \textbf{has an affinity for} them. Her \textbf{bliss} is also my joy and I believe the main purpose of giving somebody gifts is to spread the happiness, isn't it?
    \end{qa}

    \begin{qa}{Do you ever take a gift when you visit someone in their home? [Why/Why not?]}
    There was a time when I happened to \textbf{drop by} my friend's house a year ago or maybe. He had just returned from a \textbf{self-planned} tour across Europe. When he gave me a souvenir, a Swiss chocolate bar, I knew \textbf{I stroke it lucky} and felt honored to be given such a present. \textbf{I have a sweet tooth}, you know.
    \end{qa}

    \begin{qa}{When did you last receive a gift? [What was is?]}
    I still remember my last birthday \textbf{vividly}. My wife and my younger brother threw a surprising birthday party for me. almost a year ago. When I came back home, they had set up everything \textbf{in advance}. Once the lights were on, I was moved by how they had prepared. I got a pair of PS4 gamepads to hook up to my computer via its software and thanks to that, I've been able to play games in my spare time.
    \end{qa}

    \begin{qa}{Do you enjoy looking for gifts for people? [Why/ Why not?]}
    Yes indeed. I adore the feeling of \textbf{scouring} the market for something that my friend or my beloved is \textbf{craving for}. Others say that it is time-consuming to find an item that \textbf{caters for} everyone's liking but their exclamation of happiness is \textbf{rewarding} enough.
    \end{qa}

        \begin{VocabExplain}[Part 1]
            \begin{ExplainCard}{on the occasions of}[phrase][B2]
            \EN{at times of particular events or celebrations.}
            \SY{on; at the time of; upon}
            \VI{vào dịp của (các sự kiện/ lễ kỷ niệm).}
            \EX{We exchange gifts on the occasions of birthdays and weddings.}
            \EX{Donations peak on the occasions of national festivals.}
            \CO{\textit{on the occasion(s) of} + event}
            \end{ExplainCard}

            \begin{ExplainCard}{full month celebration}[n][C1]
            \EN{a ceremony marking a baby's first full month (esp. in some Asian cultures).}
            \SY{one-month celebration; first-month rite}
            \VI{lễ đầy tháng cho em bé.}
            \EX{Relatives gathered for the baby's full month celebration.}
            \EX{Anthropologists document gifts given at the full month celebration in Vietnam.}
            \CO{\textit{hold/attend} a full month celebration}
            \end{ExplainCard}

            \begin{ExplainCard}{longevity wishing ceremony}[n][C1]
            \EN{a ceremony to honor elders and wish them long life.}
            \SY{longevity rite; birthday tribute}
            \VI{lễ chúc thọ.}
            \EX{My family organized a longevity wishing ceremony for my grandmother.}
            \EX{Communities host a longevity wishing ceremony during spring festivals.}
            \CO{\textit{hold} a longevity wishing ceremony; \textit{offer} wishes of longevity}
            \end{ExplainCard}

            \begin{ExplainCard}{out of the blue}[idiom][B2]
            \EN{happening unexpectedly and without warning.}
            \SY{unexpectedly; all of a sudden}
            \VI{bất ngờ; đột ngột.}
            \EX{She called me out of the blue after ten years.}
            \EX{Market shocks can arrive out of the blue and unsettle investors.}
            \CO{\textit{come/appear} out of the blue}
            \end{ExplainCard}

            \begin{ExplainCard}{have an affinity for}[phrase][C1]
            \EN{to have a natural liking for or connection with something.}
            \SY{be fond of; have a penchant for}
            \VI{có cảm tình/thiên hướng với.}
            \EX{He has an affinity for hand-crafted gifts.}
            \EX{Designers with an affinity for minimalism favor clean lines.}
            \CO{\textit{have/show} an affinity for/with}
            \end{ExplainCard}

            \begin{ExplainCard}{bliss}[n][C1]
            \EN{perfect happiness; a state of profound joy.}
            \SY{ecstasy; felicity; rapture}
            \VI{phúc lạc; hạnh phúc tột độ.}
            \EX{A surprise gift can bring pure bliss.}
            \EX{Psychology literature links gratitude practices to reported bliss.}
            \CO{\textit{wedded/marital} bliss; a moment of \textit{bliss}}
            \end{ExplainCard}

            \begin{ExplainCard}{drop by}[phr.v][B2]
            \EN{to visit briefly and informally.}
            \SY{stop by; pop in}
            \VI{ghé qua; tạt vào thăm.}
            \EX{I'll drop by your place after work.}
            \EX{Participants could drop by the lab to collect materials.}
            \CO{\textit{drop/stop} by + place}
            \end{ExplainCard}

            \begin{ExplainCard}{self-planned}[adj][B2]
            \EN{arranged or organized by oneself.}
            \SY{self-organized; DIY; independent}
            \VI{tự lên kế hoạch; tự tổ chức.}
            \EX{They took a self-planned trip through Europe.}
            \EX{A self-planned schedule increased students' autonomy.}
            \CO{\textit{self-planned} tour/itinerary/project}
            \end{ExplainCard}

            \begin{ExplainCard}{strike it lucky}[idiom][C1]
            \EN{to have unexpected good luck or success.}
            \SY{get lucky; hit the jackpot}
            \VI{gặp may; trúng vận đỏ.}
            \EX{I struck it lucky and won a voucher.}
            \EX{Early investors struck it lucky as the shares surged.}
            \CO{\textit{strike it} lucky/rich}
            \end{ExplainCard}

            \begin{ExplainCard}{have a sweet tooth}[idiom][B2]
            \EN{to like eating sweet foods very much.}
            \SY{be fond of sweets}
            \VI{hảo ngọt; mê đồ ngọt.}
            \EX{I have a sweet tooth so chocolate gifts thrill me.}
            \EX{Dietary surveys show that many adolescents have a sweet tooth.}
            \CO{\textit{have} a sweet tooth; \textit{craving} for sweets}
            \end{ExplainCard}

            \begin{ExplainCard}{vividly}[adv][C1]
            \EN{in a clear, powerful, and detailed way.}
            \SY{clearly; distinctly; graphically}
            \VI{một cách sống động/rõ nét.}
            \EX{I vividly remember opening the present.}
            \EX{Respondents vividly recalled key moments from the campaign.}
            \CO{\textit{remember/recall/describe} vividly}
            \end{ExplainCard}

            \begin{ExplainCard}{in advance}[adv][B2]
            \EN{before an event happens; ahead of time.}
            \SY{ahead of time; beforehand}
            \VI{trước; sớm hơn dự định.}
            \EX{Please order the cake in advance.}
            \EX{Seats must be reserved in advance for capacity planning.}
            \CO{\textit{book/prepare/pay} in advance}
            \end{ExplainCard}

            \begin{ExplainCard}{scour}[v][C1]
            \EN{(1) to search thoroughly for something. (2) to clean by rubbing hard.}
            \SY{(1) comb; ransack \quad (2) scrub}
            \VI{(1) lùng sục/tìm kỹ; (2) cọ rửa mạnh.}
            \EX{(1) We scoured the market for the perfect gift.}
            \EX{(2) The lab scoured the equipment before reuse.}
            \CO{\textit{scour} the market/internet/area \quad \textit{scour} pans/surfaces}
            \end{ExplainCard}

            \begin{ExplainCard}{crave (for)}[v][C1]
            \EN{to want something very much.}
            \SY{long for; yearn for; desire}
            \VI{khao khát; thèm muốn.}
            \EX{She's craving for a handmade present.}
            \EX{Consumers often crave novelty in product design.}
            \CO{\textit{crave (for)} attention/sweets/recognition}
            \end{ExplainCard}

            \begin{ExplainCard}{cater for}[phr.v][B2]
            \EN{to provide what is needed or wanted by a particular group.}
            \SY{serve; accommodate}
            \VI{đáp ứng/phục vụ nhu cầu của.}
            \EX{This shop caters for people who love crafts.}
            \EX{Policies must cater for diverse learner needs.}
            \CO{\textit{cater for/to} tastes/needs/clients}
            \end{ExplainCard}

            \begin{ExplainCard}{rewarding}[adj][B2]
            \EN{giving satisfaction, benefit, or a sense of achievement.}
            \SY{fulfilling; gratifying; worthwhile}
            \VI{xứng đáng; mang lại cảm giác hài lòng/bổ ích.}
            \EX{Choosing a thoughtful gift is rewarding.}
            \EX{Volunteering can be rewarding in terms of skills and wellbeing.}
            \CO{\textit{find} sth rewarding; a \textit{highly} rewarding experience}
            \end{ExplainCard}
        \end{VocabExplain}

    \noindent
    \textbf{Part 2.}
    \begin{qa}{Describe something you did that was new or exciting. You should say:}
    \begin{itemize}
    \item What you did
    \item Where and when you did this
    \item Who you shared the activity with
    \item and explain why this activity was new or exciting for you.
    \end{itemize}

    If I have to describe something I did that was exciting, I would probably talk about a motorbike trip. It has been the longest motorbike journey in my life. We took a motorbike trip to Da Nang, which is a coastal city in Central Vietnam. Visiting Da Nang should be on the top of everyone's list because of its \textbf{sun-kissed} beaches and \textbf{crystal clear} water, let alone other \textbf{mind-blowing} landscapes such as Marble Mountains, Ba Na Hills and so on. It is over 400 kilometers away from Hanoi, so we spent 9 consecutive hours reaching it two years ago. At that time, it was in the summer that the flight tickets \textbf{cost a bomb} because of \textbf{peak season}. We were \textbf{hard up}, so we decided to travel by motorbike instead. It was more fun and affordable. Another reason is that I had just graduated from university at that time, so the trip to Da Nang helped me to \textbf{let my hair down}. We set up a group of 6 people, most of whom were my close friends. \textbf{There was no point} in going alone because having company was fun. I believe that it was a \textbf{golden opportunity} to strengthen our relationship. More importantly, we could support each other if there were some dangers such as illness or loss of money when we were underway. The trip was exciting because it left me with unforgettable experiences. While we were riding our motorbikes, one of my friends stopped \textbf{on the hard shoulder} \textbf{out of the blue}. We \textbf{had no clue} at that time, and everyone was genuinely worried. When we approached him, he said that we should stop to take a \textbf{selfie} together.
    \end{qa}

        \begin{VocabExplain}[Part 2]
            \begin{ExplainCard}{sun-kissed}[adj][C1]
            \EN{bright with sunshine; pleasantly warmed or tanned by the sun.}
            \SY{sun-drenched; radiant}
            \VI{ngập nắng; nhuốm màu nắng.}
            \EX{We relaxed on a sun-kissed beach all afternoon.}
            \EX{Tourism boards often promote sun-kissed coasts to signal ideal holiday climates.}
            \CO{sun-kissed beaches/fields/skin}
            \end{ExplainCard}

            \begin{ExplainCard}{crystal-clear}[adj][C1]
            \EN{(1) perfectly transparent and clean; (2) extremely easy to understand.}
            \SY{(1) limpid \quad (2) explicit; unmistakable}
            \VI{(1) trong vắt; (2) rõ như ban ngày, dễ hiểu.}
            \EX{(1) The lake water was crystal-clear.}
            \EX{(2) Instructions must be crystal-clear to avoid errors in experiments.}
            \CO{\textit{crystal-clear} water/sea; a \textit{crystal-clear} message}
            \end{ExplainCard}

            \begin{ExplainCard}{mind-blowing}[adj][C1]
            \EN{extremely impressive, surprising, or exciting.}
            \SY{breathtaking; astonishing; awe-inspiring}
            \VI{choáng ngợp; kinh ngạc.}
            \EX{The view from Ba Na Hills was mind-blowing.}
            \EX{The dataset revealed mind-blowing growth in user adoption.}
            \CO{\textit{mind-blowing} view/experience/figure}
            \end{ExplainCard}

            \begin{ExplainCard}{cost a bomb}[idiom][C1]
            \EN{to be very expensive.}
            \SY{cost a fortune; cost an arm and a leg}
            \VI{đắt cắt cổ.}
            \EX{Flights during Tet often cost a bomb.}
            \EX{Peak-demand pricing can make festival tickets cost a bomb.}
            \CO{tickets/hotels \textit{cost a bomb}; \textit{cost a bomb} to buy}
            \end{ExplainCard}

            \begin{ExplainCard}{peak season}[n][B2]
            \EN{the busiest time of the year when demand and prices are highest.}
            \SY{high season; busy season}
            \VI{mùa cao điểm.}
            \EX{We avoid traveling in peak season to save money.}
            \EX{Room rates typically surge in peak season due to limited supply.}
            \CO{\textit{during/in} peak season; peak-season prices}
            \end{ExplainCard}

            \begin{ExplainCard}{hard up}[adj][C1]
            \EN{(1) short of money; (2) lacking or in need of something.}
            \SY{(1) broke; strapped \quad (2) short of}
            \VI{(1) túng thiếu; (2) thiếu thốn (cái gì đó).}
            \EX{(1) We were hard up after paying tuition.}
            \EX{(2) The team was hard up for volunteers during the event.}
            \CO{\textit{be} hard up; hard up \textit{for} cash/time/ideas}
            \end{ExplainCard}

            \begin{ExplainCard}{let one's hair down}[idiom][C1]
            \EN{to relax and enjoy yourself without worrying about rules or expectations.}
            \SY{unwind; loosen up; kick back}
            \VI{xả hơi; thư giãn hết mình.}
            \EX{After finals, we went to the beach to let our hair down.}
            \EX{Retreats give staff a chance to let their hair down and build rapport.}
            \CO{\textit{let your hair down} at/for the weekend}
            \end{ExplainCard}

            \begin{ExplainCard}{there is no point (in) doing sth}[phrase][B2]
            \EN{it is not worth doing something because it brings no benefit.}
            \SY{pointless; futile}
            \VI{không có ích; vô nghĩa khi làm việc gì.}
            \EX{There's no point in arguing about the schedule.}
            \EX{The study concludes there is no point in duplicating datasets already available.}
            \CO{\textit{no point (in) + V-ing}; \textit{there's no point} at all}
            \end{ExplainCard}

            \begin{ExplainCard}{golden opportunity}[n][C1]
            \EN{an excellent chance that should not be missed.}
            \SY{perfect chance; window of opportunity}
            \VI{cơ hội vàng.}
            \EX{The internship was a golden opportunity for me.}
            \EX{Scholarships provide a golden opportunity for underrepresented students.}
            \CO{\textit{seize/miss} a golden opportunity}
            \end{ExplainCard}

            \begin{ExplainCard}{hard shoulder}[n][C1]
            \EN{the narrow strip beside a motorway where vehicles stop in emergencies.}
            \SY{shoulder; breakdown lane}
            \VI{làn dừng khẩn cấp (ven đường cao tốc).}
            \EX{We pulled over on the hard shoulder to fix a tire.}
            \EX{Drivers are fined for non-emergency stops on the hard shoulder.}
            \CO{\textit{on the} hard shoulder; \textit{pull over/stop} on the hard shoulder}
            \end{ExplainCard}

            \begin{ExplainCard}{out of the blue}[idiom][B2]
            \EN{unexpectedly and without warning.}
            \SY{unexpectedly; out of nowhere; all of a sudden}
            \VI{bất ngờ, không báo trước.}
            \EX{He called me out of the blue after years.}
            \EX{Service outages sometimes occur out of the blue during peak loads.}
            \CO{\textit{come/appear} out of the blue}
            \end{ExplainCard}

            \begin{ExplainCard}{have no clue}[idiom][B2]
            \EN{to not know or understand anything about something.}
            \SY{have no idea; be clueless about}
            \VI{không biết gì; hoàn toàn mù tịt.}
            \EX{We had no clue why he stopped.}
            \EX{Many novices have no clue how memory management works initially.}
            \CO{\textit{have no clue} what/why/how; be \textit{clueless about} sth}
            \end{ExplainCard}

            \begin{ExplainCard}{selfie}[n][B1]
            \EN{a photograph that you take of yourself, typically with a smartphone.}
            \SY{self-portrait (digital)}
            \VI{ảnh “tự sướng”; ảnh tự chụp.}
            \EX{We took a quick selfie at the viewpoint.}
            \EX{The study analyzes how travel selfies shape destination branding online.}
            \CO{\textit{take/post} a selfie; a \textit{group} selfie}
            \end{ExplainCard}
        \end{VocabExplain}

    \noindent
    \textbf{Part 3.}
    \begin{qa}{Why do you think some people like doing new things?}
    For the most part, creativity is a human nature, which is why people are into doing \textbf{novel} things. Once they come up with new inventions, they will obtain a greater sense of \textbf{fulfilment} and \textbf{gratification}, which could further their motivation for continued research.
    \end{qa}

    \begin{qa}{What problems can people have when they try new activities for the first time?}
    It has never been easy for people to \textbf{have a crack at} something they have not tried before. Everything should be done carefully and step by step, just like how \textbf{a toddler} learns to crawl. Some may suffer from depression if they try to achieve the desirable results, but \textbf{to no avail}, while the others are lucky enough to overcome their difficulties. On the other hand, some could feel a sense of frustration on the way to succes. However, regardless of the situation, practice is key to success.
    \end{qa}

    \begin{qa}{Do you think it's best to do new things on your own or with other people? Why?}
    There is no \textbf{rigid} rule that you need to collaborate with someone or do it yourself when it comes to doing novel things. It totally depends. The thing is all about \textbf{instinct} and the environment which are decisive factors to \textbf{inspire creativity}, I guess. Some want to work with others to \textbf{work out the kinks} more quickly, but others enjoy their \textbf{solitude} when trying something new. Personally, I do not mind cooperating with other people , or do it myself if necessary.
    \end{qa}

    \begin{qa}{What kinds of things do children learn to do when they are very young? How important are these things?}
    In my \textbf{hazy recollection}, what I learned at kindergardens or primary schools were core subjects and I had to sit exams. This is not the case for the majority of the children nowadays since personal and \textbf{interpersonal skills} like greeting and teamwork are added into their coursework to build up their social skills. That are very basic but really greatly benefit the cognitive development of the children.
    \end{qa}

    \begin{qa}{Do you think children and adults learn to do new things in the same way? How is their learning style different?}
    Basically, there are \textbf{distinctions} between the way grown-up people and children encounter new things. Compared to youngsters, adults have a \textbf{wealth} of experience to draw on, therefore, they are less \textbf{fearful} of examining unknown issues. Adults have \textbf{preconceived} notions about education, learning styles and subject matter, so they prefer to learn what really benefit themselves. Unlike adults, children will try most new tasks and \textbf{see them through}, regardless of how well they do. In other words, children and adults generally do not learn to do novel things in a similar fashion.
    \end{qa}

    \begin{qa}{Some people say that it is more important to be able to learn new things now than it was in the past. Do you agree or disagree with that? Why?}
    I don't think so. Obviously, learning is a must \textbf{irrespective} of the time we are living in. Having said that, in a knowledge-based society nowadays, people are trained to adapt to more \textbf{meticulous} information, requiring both \textbf{intellectual} and logical skills. Also, the way we perfect ourself needs to be faster as global catastrophes such as global warming are \textbf{escalating}.
    \end{qa}

        \begin{VocabExplain}[Part 3]
            \begin{ExplainCard}{novel}[adj][C1]
            \EN{new and original; not like anything seen before.}
            \SY{original; innovative; fresh}
            \VI{mới lạ; độc đáo.}
            \EX{She's always chasing novel experiences.}
            \EX{The lab proposed a novel approach to reduce training time.}
            \CO{\textit{novel} idea/method/approach}
            \end{ExplainCard}

            \begin{ExplainCard}{fulfilment}[n][C1]
            \EN{a deep sense of satisfaction from achieving something meaningful.}
            \SY{contentment; satisfaction}
            \VI{sự mãn nguyện; cảm giác trọn vẹn.}
            \EX{Teaching gives him a sense of fulfilment.}
            \EX{Career-fulfilment strongly predicts long-term retention in surveys.}
            \CO{\textit{a sense of} fulfilment; personal/professional fulfilment}
            \end{ExplainCard}

            \begin{ExplainCard}{gratification}[n][C1]
            \EN{pleasure gained from satisfying a desire or need.}
            \SY{pleasure; satisfaction; reward}
            \VI{sự thỏa mãn; niềm vui.}
            \EX{Finishing the project brought real gratification.}
            \EX{Studies link delayed gratification to better academic outcomes.}
            \CO{\textit{instant/delayed} gratification; seek/derive \textit{gratification}}
            \end{ExplainCard}

            \begin{ExplainCard}{have a crack at}[phrase][C1]
            \EN{to attempt or try something, especially for the first time.}
            \SY{give (sth) a try; attempt}
            \VI{thử làm; thử sức.}
            \EX{I'll have a crack at baking this weekend.}
            \EX{New interns were encouraged to have a crack at the prototype.}
            \CO{\textit{have/take} a crack at + V-ing/sth}
            \end{ExplainCard}

            \begin{ExplainCard}{toddler}[n][B2]
            \EN{a young child who is just learning to walk (aged about 1–3).}
            \SY{young child; tot}
            \VI{trẻ mới biết đi.}
            \EX{A toddler needs constant supervision.}
            \EX{Language acquisition in toddlers accelerates via interaction.}
            \CO{\textit{toddler} years/stage; \textit{toddler} development}
            \end{ExplainCard}

            \begin{ExplainCard}{to no avail}[idiom][C1]
            \EN{without success; with no useful result.}
            \SY{in vain; unsuccessfully}
            \VI{vô ích; không có kết quả.}
            \EX{We searched for hours to no avail.}
            \EX{Remedial policies were implemented, to no avail, during the recession.}
            \CO{\textit{try/search/appeal} to no avail}
            \end{ExplainCard}

            \begin{ExplainCard}{rigid}[adj][C1]
            \EN{not flexible or adaptable; very strict.}
            \SY{inflexible; strict; unyielding}
            \VI{cứng nhắc; khắt khe.}
            \EX{Their rigid schedule leaves little room for breaks.}
            \EX{Rigid protocols can hinder creative problem-solving.}
            \CO{\textit{rigid} rules/structure/approach}
            \end{ExplainCard}

            \begin{ExplainCard}{instinct}[n][C1]
            \EN{an innate tendency to act in a certain way; a natural feeling.}
            \SY{intuition; gut feeling}
            \VI{bản năng; linh cảm.}
            \EX{Trust your instinct when choosing topics.}
            \EX{Decision models often incorporate human instinct alongside data.}
            \CO{\textit{follow/trust} one's instinct; natural \textit{instinct for}}
            \end{ExplainCard}

            \begin{ExplainCard}{inspire creativity}[phrase][B2]
            \EN{to make someone feel encouraged to produce original ideas or work.}
            \SY{spark imagination; stimulate innovation}
            \VI{truyền cảm hứng sáng tạo.}
            \EX{Workshops are designed to inspire creativity in students.}
            \EX{Open office layouts may inspire creativity through spontaneous exchange.}
            \CO{\textit{inspire} creativity/innovation/ideas}
            \end{ExplainCard}

            \begin{ExplainCard}{work out the kinks}[idiom][C1]
            \EN{to solve the small problems in a plan or system.}
            \SY{iron out issues; debug; refine}
            \VI{gỡ rối; chỉnh các lỗi nhỏ.}
            \EX{Give us a week to work out the kinks.}
            \EX{The beta phase helps teams work out the kinks before launch.}
            \CO{\textit{work/iron} out the kinks in sth}
            \end{ExplainCard}

            \begin{ExplainCard}{solitude}[n][C1]
            \EN{the state of being alone, often by choice.}
            \SY{seclusion; privacy}
            \VI{sự cô tịnh; ở một mình.}
            \EX{He enjoys a bit of solitude after work.}
            \EX{Writers often seek solitude to sustain deep work.}
            \CO{\textit{enjoy/seek} solitude; peaceful \textit{solitude}}
            \end{ExplainCard}

            \begin{ExplainCard}{hazy recollection}[n][C1]
            \EN{a memory that is unclear or vague.}
            \SY{vague memory; faint recollection}
            \VI{ký ức mơ hồ; nhớ lờ mờ.}
            \EX{I have a hazy recollection of that lecture.}
            \EX{Witnesses reported only a hazy recollection of events.}
            \CO{\textit{have} a hazy recollection (of)}
            \end{ExplainCard}

            \begin{ExplainCard}{interpersonal skills}[n][B2]
            \EN{abilities that help people interact effectively with others.}
            \SY{people skills; social skills}
            \VI{kỹ năng giao tiếp giữa người với người.}
            \EX{Team projects build interpersonal skills.}
            \EX{Employers consistently rate interpersonal skills as critical competencies.}
            \CO{\textit{develop/build} interpersonal skills}
            \end{ExplainCard}

            \begin{ExplainCard}{distinction}[n][C1]
            \EN{a clear difference or contrast between similar things.}
            \SY{difference; contrast; demarcation}
            \VI{sự khác biệt; ranh giới phân biệt.}
            \EX{There's a distinction between hobby and work.}
            \EX{The paper draws a key distinction between correlation and causation.}
            \CO{\textit{draw/make} a distinction (between)}
            \end{ExplainCard}

            \begin{ExplainCard}{a wealth (of)}[n][B2]
            \EN{a large amount of something useful or valuable.}
            \SY{abundance; profusion}
            \VI{một kho/nguồn dồi dào.}
            \EX{She has a wealth of teaching experience.}
            \EX{Archives offer a wealth of primary data for researchers.}
            \CO{\textit{a wealth of} experience/data/resources}
            \end{ExplainCard}

            \begin{ExplainCard}{fearful}[adj][B2]
            \EN{feeling or showing fear; worried about something.}
            \SY{afraid; apprehensive}
            \VI{sợ hãi; e ngại.}
            \EX{Many are fearful of speaking in public.}
            \EX{Respondents were less fearful after targeted training.}
            \CO{\textit{fearful of/about} sth; become \textit{less} fearful}
            \end{ExplainCard}

            \begin{ExplainCard}{preconceived}[adj][C1]
            \EN{(of ideas/opinions) formed before having enough information or experience.}
            \SY{preformed; prejudged; preconceived}
            \VI{định kiến; hình thành trước khi có bằng chứng.}
            \EX{Try to challenge your preconceived notions.}
            \EX{The study controls for preconceived attitudes among participants.}
            \CO{\textit{preconceived} ideas/notions/biases}
            \end{ExplainCard}

            \begin{ExplainCard}{see (sth) through}[phr.v][B2]
            \EN{to continue doing something until it is finished.}
            \SY{persevere; carry through; complete}
            \VI{làm đến cùng; theo đuổi đến hết.}
            \EX{She promised to see the project through.}
            \EX{Grant funding helped the team see the trial through to completion.}
            \CO{\textit{see} a task/project \textit{through}}
            \end{ExplainCard}

            \begin{ExplainCard}{irrespective (of)}[prep][C1]
            \EN{without considering; regardless of.}
            \SY{regardless of; independent of}
            \VI{bất kể; không kể.}
            \EX{The rules apply to everyone irrespective of age.}
            \EX{Access remains limited irrespective of recent reforms.}
            \CO{\textit{irrespective of} race/age/background}
            \end{ExplainCard}

            \begin{ExplainCard}{meticulous}[adj][C1]
            \EN{very careful and with great attention to detail.}
            \SY{scrupulous; thorough; painstaking}
            \VI{tỉ mỉ; kỹ lưỡng.}
            \EX{She keeps meticulous notes for revision.}
            \EX{Meticulous data curation improves reproducibility.}
            \CO{\textit{meticulous} planning/research/records}
            \end{ExplainCard}

            \begin{ExplainCard}{intellectual}[adj][C1]
            \EN{relating to the ability to think and understand ideas; involving serious thought.}
            \SY{cerebral; academic}
            \VI{thuộc trí tuệ; thiên về suy nghĩ.}
            \EX{He enjoys intellectual debates.}
            \EX{The fellowship supports intellectual exploration across disciplines.}
            \CO{\textit{intellectual} ability/curiosity/work}
            \end{ExplainCard}

            \begin{ExplainCard}{escalate / escalating}[v/adj][C1]
            \EN{to increase rapidly in intensity or seriousness; becoming more severe.}
            \SY{intensify; mount; amplify}
            \VI{leo thang; ngày càng nghiêm trọng.}
            \EX{Costs escalated after the delay.}
            \EX{Reports show escalating climate risks over the decade.}
            \CO{\textit{escalate} quickly; \textit{escalating} costs/tensions}
            \end{ExplainCard}
        \end{VocabExplain}

    \begin{VocabHighlights}
        \VH{on the occasion of}{(phrase) a particular time, especially as marked by certain circumstances or occurrences}{(cụm từ) nhân dịp}
        \VH{full month celebration}{(phrase) when a baby turns one month old, a ceremony is held to celebrate her first full month of life}{(cụm từ) lễ đầy tháng}
        \VH{longevity wishing ceremony}{(phrase) to wish an elderly person longevity}{(cụm từ) lễ mừng thọ}
        \VH{out of the blue}{(idiom) completely unexpected}{(thành ngữ) bất ngờ, không định trước}
        \VH{to have an affinity for}{(phrase) to have a spontaneous or natural liking or sympathy for someone or something}{(cụm từ) có cảm tình với}
        \VH{bliss}{(n) great joy}{(danh từ) niềm hạnh phúc}
        \VH{to drop by}{(phr.v) to pay an informal visit to a person or a place}{(cụm động từ) tạt qua thăm bất chợt}
        \VH{a self-planned tour}{(phrase) a self-governing tour where one navigates a route oneself}{(cụm từ) chuyến đi du lịch tự túc}
        \VH{to strike it lucky}{(idiom) to have some good luck suddenly}{(thành ngữ) gặp may bất chợt}
        \VH{to have a sweet tooth}{(idiom) to like eating things that are sugary or taste sweet}{(thành ngữ) thích ăn đồ ngọt}
        \VH{vividly}{(adv) in a way that produces very clear pictures in your mind}{(trạng từ) rõ rệt}
        \VH{in advance}{(phrase) ahead in time}{(cụm từ) trước đó}
        \VH{to scour}{(v) to search a place or thing thoroughly in order to find somebody/something}{(động từ) lùng sục, sục sạo}
        \VH{to crave for}{(v) want to have something very much}{(động từ) khát khao cái gì rất nhiều}
        \VH{to cater for}{(v) provide all the things that they need or want}{(động từ) cung cấp mọi thứ ai cần}
        \VH{rewarding}{(adj) worth doing; that makes you happy because you think it is useful or important}{(tính từ) đáng làm}
        \VH{sun-kissed}{(adj) made warm or brown by the sun}{(tính từ) đầy nắng}
        \VH{crystal clear}{(idiom) extremely clear}{(thành ngữ) rất rõ}
        \VH{mind-blowing}{(adj) extremely exciting or surprising}{(tính từ) cực hay, tuyệt vời}
        \VH{cost a bomb}{(idiom) something costs a lot of money}{(thành ngữ) đắt đỏ}
        \VH{peak season}{(idiom) the time of year when a lot of people travel and prices are usually at their highest}{(thành ngữ) mùa cao điểm}
        \VH{hard up}{(idiom) having very little money}{(thành ngữ) có rất ít tiền}
        \VH{let my hair down}{(idiom) have a rest}{(thành ngữ) nghỉ ngơi}
        \VH{There was no point in}{(phrase) it's not worth}{(cụm từ) không đáng}
        \VH{out of the blue}{(idiom) unexpectedly}{(thành ngữ) bất ngờ}
        \VH{on the hard shoulder}{(phrase) a hard area at the side of a main road where a driver can stop if there is a serious problem}{(cụm từ) vạch dừng cách xe lộ để dắt xe}
        \VH{had no clue}{(idiom) have no idea about something}{(thành ngữ) không biết gì}
        \VH{novel}{(adj) new}{(tính từ) mới}
        \VH{a sense of triumph}{(phrase) a very great success}{(cụm từ) cảm giác chiến thắng}
        \VH{have a crack at}{(idiom) to have a talent for something}{(thành ngữ) có tài năng về cái gì}
        \VH{a toddler}{(n) a child who has only recently learnt to walk}{(danh từ) trẻ sơ sinh}
        \VH{to no avail}{(phrase) in vain}{(cụm từ) vô ích}
        \VH{rigid}{(adj) very strict and difficult to change}{(tính từ) cứng nhắc}
        \VH{instinct}{(n) a natural tendency for people and animals to behave in a particular way using the knowledge and abilities that they were born with rather than thought or training}{(danh từ) bản năng}
        \VH{to inspire creativity}{(phrase) to enhance creativity}{(cụm từ) nâng cao tính sáng tạo}
        \VH{to work out the kinks}{(idiom) to solve the problem}{(thành ngữ) giải quyết vấn đề}
        \VH{solitude}{(n) the state of being alone, especially when you find this pleasant}{(danh từ) trạng thái một mình}
        \VH{hazy}{(adj) not clear because of haze}{(tính từ) mờ hồ}
        \VH{interpersonal}{(adj) connected with relationships between people}{(tính từ) mối liên hệ giữa các cá nhân}
        \VH{greeting}{(n) something that you say or do to greet somebody}{(danh từ) lời chào hỏi}
        \VH{distinction}{(n) distinction (between a and b) a clear difference or contrast especially between people or things that are similar or related}{(danh từ) sự khác biệt}
        \VH{a wealth of something}{(phrase) having a lot of something}{(cụm từ) rất nhiều}
        \VH{fearful}{(adj) nervous and afraid}{(tính từ) lo sợ, sợ hãi}
        \VH{preconceived}{(adj) formed before having the evidence for its truth or usefulness}{(tính từ) định kiến, hiểu biết có sẵn}
        \VH{to see through}{(phrase) to continue until something is finished}{(cụm từ) làm đến khi hoàn thành}
        \VH{irrespective}{(adv) without considering something or being influenced by it}{(trạng từ) bất kể}
        \VH{meticulous}{(adj) paying careful attention to every detail}{(tính từ) tỉ mỉ, kĩ lưỡng}
        \VH{intellectual}{(adj) connected with or using a person's ability to think in a logical way and understand things}{(tính từ) thuộc về trí tuệ}
        \VH{to escalate}{(v) to increase rapidly}{(động từ) tăng nhanh chóng}
    \end{VocabHighlights}
    \end{test}

    \begin{test}{TEST 3}
    \noindent
    \textbf{Part 1. Telephoning}
    \begin{qa}{How often do you make telephone calls? [Why?/Why not?]}
    Making phone calls is what I do on a daily basis. Although I'm not an extrovert type, I still \textbf{can't help getting ahold of} others as it is \textbf{part and parcel} of my job. I need to notify my relatives, assistants, students and their parents of the schedules, academic results and progress. In other words, hardly a day goes by without my being on the phone.
    \end{qa}

    \begin{qa}{Who do you spend most time talking to on the telephone? [Why?]}
    My wife is the \textbf{recipient} of most of my phone calls. We need to cooperate to put our work and family life \textbf{in equilibrium} with the ultimate goal of maintaining a loving family. Sometimes I feel it's a real \textbf{hassle} answering her repeated phone calls each day but maybe that's what running a family is like.
    \end{qa}

    \begin{qa}{When do you think you'll next make a telephone call? [Why?]}
    The next time I make a phone call depends on the purpose of the conversation. Perhaps I will \textbf{give} my uncle who lives overseas \textbf{a ring}. It is just that we need to keep in touch regularly to \textbf{cement} our distant relationship. \textbf{Shooting the breeze} with him somehow boosts my morale, too as he can still \textbf{cheer me up} when I am \textbf{in a pickle}. For example, I did not feel \textbf{bummed out} anymore after he had given me advice on how to deal with friends' \textbf{bad-mouthing} my lifestyle a long time ago.
    \end{qa}

    \begin{qa}{Do you sometimes prefer to send a text message instead of telephoning? [Why?/Why not?]}
    I express a strong preference for text message. It is short but mostly long enough to convey my ideas completely. It is also cheaper to deliver a text message than get onto the phone. Even if the recipient is not in the mood for answering the sender's direct phone calls due to previous argument, a text message carrying excuses may serve the purpose of \textbf{healing the rift} between them.
    \end{qa}

        \begin{VocabExplain}[Part 1]
            \begin{ExplainCard}{can't help getting ahold of}[idiom][C1]
            \EN{to be unable to stop oneself from contacting others; feel compelled to reach someone.}
            \SY{can't resist (contacting); can't help but call}
            \VI{không thể không liên lạc; không kiềm được việc gọi/nhắn.}
            \EX{Even on holiday, she can't help getting ahold of her team.}
            \EX{Managers often can't help getting ahold of stakeholders during critical phases.}
            \CO{can't help (but) + V; get ahold of sb}
            \end{ExplainCard}

            \begin{ExplainCard}{part and parcel}[idiom][C1]
            \EN{an essential or unavoidable part of something.}
            \SY{integral part; inherent element}
            \VI{phần tất yếu/không thể tách rời.}
            \EX{Late nights are part and parcel of running a startup.}
            \EX{Peer review is part and parcel of academic publishing.}
            \CO{\textit{be} part and parcel of sth}
            \end{ExplainCard}

            \begin{ExplainCard}{recipient}[n][C1]
            \EN{a person who receives something such as a message, gift, or payment.}
            \SY{receiver; beneficiary}
            \VI{người nhận.}
            \EX{Please check the recipient before sending the text.}
            \EX{Grant recipients must submit quarterly reports.}
            \CO{\textit{intended} recipient; payment/mail \textit{recipient}}
            \end{ExplainCard}

            \begin{ExplainCard}{in equilibrium}[phrase][C1]
            \EN{in a state of balance between different forces, influences, or needs.}
            \SY{in balance; in steady state}
            \VI{ở trạng thái cân bằng/hài hòa.}
            \EX{We try to keep work and family life in equilibrium.}
            \EX{Markets tend toward prices where supply and demand are in equilibrium.}
            \CO{\textit{keep/maintain} in equilibrium; reach equilibrium}
            \end{ExplainCard}

            \begin{ExplainCard}{hassle}[n][B2]
            \EN{a situation that causes difficulty or irritation; a nuisance.}
            \SY{inconvenience; bother}
            \VI{sự phiền phức; rắc rối.}
            \EX{Calling customer service can be a real hassle.}
            \EX{Excessive paperwork creates administrative hassle for researchers.}
            \CO{\textit{a real/needless} hassle; \textit{avoid} the hassle}
            \end{ExplainCard}

            \begin{ExplainCard}{give (sb) a ring}[phrase][B2]
            \EN{to call someone on the phone.}
            \SY{phone; call; ring (sb)}
            \VI{gọi điện cho ai.}
            \EX{I'll give you a ring tonight.}
            \EX{Participants were advised to give the coordinator a ring with questions.}
            \CO{\textit{give} sb \textit{a ring}; \textit{a quick} ring}
            \end{ExplainCard}

            \begin{ExplainCard}{cement}[v][C1]
            \EN{to make a relationship, idea, or position stronger and more certain.}
            \SY{strengthen; solidify; consolidate}
            \VI{củng cố; thắt chặt.}
            \EX{Regular chats cement our friendship.}
            \EX{The partnership cemented the university's role in regional innovation.}
            \CO{\textit{cement} ties/relationship/reputation}
            \end{ExplainCard}

            \begin{ExplainCard}{shoot the breeze}[idiom][C1]
            \EN{to chat informally about unimportant things.}
            \SY{chit-chat; shoot the bull (infml)}
            \VI{tán gẫu; tám chuyện.}
            \EX{We met for coffee to shoot the breeze.}
            \EX{Unstructured check-ins let remote teams shoot the breeze and build rapport.}
            \CO{\textit{sit and} shoot the breeze; \textit{just} shooting the breeze}
            \end{ExplainCard}

            \begin{ExplainCard}{cheer (sb) up}[phr.v][B2]
            \EN{to make someone feel happier; to become happier.}
            \SY{lift (sb's) spirits; buoy}
            \VI{làm ai đó vui lên; phấn chấn.}
            \EX{Her call really cheered me up.}
            \EX{Peer support programs can cheer up students during exam stress.}
            \CO{\textit{cheer up} a friend; \textit{be} cheered up by sth}
            \end{ExplainCard}

            \begin{ExplainCard}{in a pickle}[idiom][C1]
            \EN{in a difficult or troublesome situation.}
            \SY{in a bind; in a jam}
            \VI{gặp tình thế khó xử; rắc rối.}
            \EX{I'm in a pickle—my phone just died.}
            \EX{Firms are in a pickle when supply chains suddenly stall.}
            \CO{\textit{be/find yourself} in a pickle}
            \end{ExplainCard}

            \begin{ExplainCard}{bummed out}[adj][B2]
            \EN{informal: unhappy, disappointed, or depressed.}
            \SY{down; dejected}
            \VI{chán nản; buồn rầu.}
            \EX{He was bummed out after the call.}
            \EX{Participants felt bummed out by repeated rejections in the process.}
            \CO{\textit{feel/get} bummed out; \textit{pretty} bummed out}
            \end{ExplainCard}

            \begin{ExplainCard}{bad-mouth}[v][C1]
            \EN{to criticize someone or something unfairly or maliciously.}
            \SY{disparage; slander; run down}
            \VI{nói xấu; dèm pha.}
            \EX{Stop bad-mouthing your classmates online.}
            \EX{Competitors sometimes bad-mouth new entrants to sway customers.}
            \CO{\textit{bad-mouth} sb/sth; \textit{publicly} bad-mouth}
            \end{ExplainCard}

            \begin{ExplainCard}{heal the rift}[phrase][C1]
            \EN{to repair a serious disagreement or break in a relationship.}
            \SY{mend fences; patch things up; reconcile}
            \VI{hàn gắn rạn nứt (mối quan hệ).}
            \EX{A heartfelt text helped heal the rift.}
            \EX{Mediation aims to heal the rift between departments after conflict.}
            \CO{\textit{heal/mend} the rift \textit{between} A and B}
            \end{ExplainCard}
        \end{VocabExplain}

    \noindent
    \textbf{Part 2.}
    \begin{qa}{Describe a journey [e.g. by car, plane, boat] that you remember well. You should say:}
    \begin{itemize}
    \item Where you went
    \item How you travelled
    \item Why you went on the journey
    \item and explain why you remember this journey well.
    \end{itemize}

    Travelling is \textbf{right up my alley}, so I've been to many places so far. For me, the most memorable journey is when I traveled from Newcastle, a city in the northeast of the U.K where my university is based, to Manchester, a city which is home to Manchester United, my favorite team. Before I travelled, I had had several travel options to consider. Finally, I chose to travel by coach which departed \textbf{bright and early}, at around 5.30 a.m I guess as the fare was \textbf{going for a song}, less than 10 pounds perhaps. Frankly speaking, I couldn't wait to arrive in Manchester to \textbf{see the sights} as Old Trafford, the home stadium of Manchester United, the team I have supported all my life. The scenario of setting foot at Old Trafford \textbf{pumped me up} before I landed in the U.K and the thought of entering this legendary stadium in the next few hours made me sleepless during the whole journey. After I had entered the stadium, an \textbf{ecstasy} poured through me as my ultimate dream finally came true. Later that day, I also \textbf{relished} attending a live concert by Linkin Park, a renowned rock band that is also my childhood idol. However, when I returned to the hostel, I realized that I had booked at the wrong place so it was way farther than I thought. The taxi fares \textbf{set me back} \textbf{an awful lot} of money, at around 60 pounds, but the \textbf{sheer joy} that day somehow \textbf{made up for} that incident. What a journey to remember!
    \end{qa}

        \begin{VocabExplain}[Part 2]
            \begin{ExplainCard}{right up (one's) alley}[idiom][B2]
            \EN{exactly suitable for someone's interests or skills.}
            \SY{be one's thing; suit someone down to the ground}
            \VI{đúng “gu”/hợp sở thích hoặc sở trường.}
            \EX{This travel vlog is right up my alley.}
            \EX{Project-based courses are right up the alley of hands-on learners.}
            \CO{\textit{be} right up sb's alley; find sth \textit{right up} your alley}
            \end{ExplainCard}

            \begin{ExplainCard}{bright and early}[idiom][B2]
            \EN{very early in the morning.}
            \SY{at first light; at the crack of dawn}
            \VI{sáng sớm tinh mơ.}
            \EX{We left bright and early to beat the traffic.}
            \EX{Fieldwork began bright and early to catch low-tide conditions.}
            \CO{\textit{leave/arrive/start} bright and early}
            \end{ExplainCard}

            \begin{ExplainCard}{go (for) a song}[idiom][C1]
            \EN{to be sold very cheaply.}
            \SY{be a steal; cost next to nothing}
            \VI{bán/rẻ bèo; mua được với giá hời.}
            \EX{I got the ticket for a song.}
            \EX{Off-season packages often go for a song in the tourism market.}
            \CO{\textit{buy/get} sth for a song; tickets \textit{go} for a song}
            \end{ExplainCard}

            \begin{ExplainCard}{see the sights}[phrase][B2]
            \EN{to visit famous or interesting places in a city.}
            \SY{go sightseeing; tour around}
            \VI{đi ngắm cảnh; tham quan danh thắng.}
            \EX{We spent the weekend seeing the sights of Manchester.}
            \EX{New arrivals typically see the sights during orientation week.}
            \CO{\textit{see the sights of} + place; \textit{go} sightseeing}
            \end{ExplainCard}

            \begin{ExplainCard}{pump (sb) up}[phr.v][B2]
            \EN{to make someone excited and full of enthusiasm.}
            \SY{fire up; hype up; energize}
            \VI{kích thích/tăng hứng khởi cho ai.}
            \EX{The anthem pumped the fans up before kickoff.}
            \EX{Motivational briefings pump teams up ahead of high-stakes tasks.}
            \CO{\textit{pump} the crowd \textit{up}; feel \textit{pumped up}}
            \end{ExplainCard}

            \begin{ExplainCard}{ecstasy}[n][C1]
            \EN{an overwhelming feeling of great happiness or joyful excitement.}
            \SY{rapture; elation; bliss}
            \VI{niềm hân hoan tột độ; ngây ngất.}
            \EX{He shouted in ecstasy when his team scored.}
            \EX{Crowd-emotion studies document moments of collective ecstasy at concerts.}
            \CO{\textit{in} ecstasy; cries/shouts of \textit{ecstasy}}
            \end{ExplainCard}

            \begin{ExplainCard}{relish}[v][C1]
            \EN{to enjoy or take great pleasure in something.}
            \SY{enjoy; savor; delight in}
            \VI{thưởng thức; thích thú.}
            \EX{She relished every minute of the show.}
            \EX{Participants relished the opportunity to present their findings.}
            \CO{\textit{relish} the chance/prospect; \textit{relish} doing sth}
            \end{ExplainCard}

            \begin{ExplainCard}{set (sb) back}[phr.v][C1]
            \EN{to cost someone a particular amount of money (often a lot).}
            \SY{cost; run sb; put sb out}
            \VI{ngốn của ai một khoản tiền; tốn kém.}
            \EX{The taxi ride set me back sixty pounds.}
            \EX{Last-minute bookings can set travelers back significantly.}
            \CO{\textit{set sb back} \$X/a tidy sum; expenses that \textit{set you back}}
            \end{ExplainCard}

            \begin{ExplainCard}{an awful lot (of)}[phrase][B2]
            \EN{a very large amount or number (of).}
            \SY{a great deal (of); loads of}
            \VI{rất nhiều; vô số.}
            \EX{That trip cost an awful lot of money.}
            \EX{An awful lot of data was discarded during cleaning.}
            \CO{\textit{an awful lot of} time/money/work}
            \end{ExplainCard}

            \begin{ExplainCard}{sheer joy}[n phrase][B2]
            \EN{pure and intense happiness.}
            \SY{pure delight; unalloyed joy}
            \VI{niềm vui thuần khiết; vui sướng tột độ.}
            \EX{He cried out of sheer joy.}
            \EX{Spectators reported sheer joy at the final whistle.}
            \CO{\textit{out of} sheer joy; feel/experience \textit{sheer joy}}
            \end{ExplainCard}

            \begin{ExplainCard}{make up for}[phr.v][B2]
            \EN{to compensate for something bad or unpleasant.}
            \SY{compensate for; offset}
            \VI{bù đắp/đền bù cho.}
            \EX{The view made up for the long hike.}
            \EX{Service recovery can make up for initial service failures.}
            \CO{\textit{make up for} a loss/delay/inconvenience}
            \end{ExplainCard}
        \end{VocabExplain}

    \noindent
    \textbf{Part 3.}
    \begin{qa}{Why do people need to travel every day?}
    Of course, travelling has been \textbf{part and parcel} of daily life, and there are \textbf{heaps} of reasons to going out night and day. For the most part, \textbf{daily commute} is mainly for school or work, which is definitely the top priority. Besides, going out for shopping and \textbf{running errands} for \textbf{daily necessities} are not unheard of. Travelling to popular tourist destinations or \textbf{going off the beaten track} on holidays is also favored nowadays because it is also a good way for tourists or backpackers to take a break from the , \textbf{hubbub} of the city life.
    \end{qa}

    \begin{qa}{What problems can people have when they are on their daily journey, for example to work or school? Why?}
    I must say people could be \textbf{subject} to many \textbf{incidents} during their commutes. Traffic congestion is one of the noticeable problems which prevent people from being \textbf{punctual}, especially in rush hours. Road accident is another alarming issue caused by \textbf{reckless driving}. In my opinion, many of the problems that people may encounter every day often stem from human behaviours, so if people change their habits, things are \textbf{looking up}.
    \end{qa}

    \begin{qa}{Some people say daily journeys like these will not be so common in the future. Do you agree or disagree? Why?}
    That is probably true as \textbf{teleworking} is becoming popular, which allow employees to work in the comfort of their houses more often. Technological advances have made the workplace accessible to everybody without setting off for anywhere. In addition, many universities are now offering online courses with \textbf{affordable} tuition fees to encourage distance learning, which helps to remove \textbf{geographical constraints}. That is why we may live in the future when travelling is unnecessary.
    \end{qa}

    \begin{qa}{What do you think people can learn from travelling to other countries? Why?}
    \textbf{Frankly} speaking, I am an \textbf{avid} traveler, and based on my experience, travel basically has \textbf{dual} purposes. Firstly, going abroad teaches any travellers or tourists the outside world, which they probably cannot learn \textbf{from} school. I bet there is no textbook that covers all ethnic cultures or geographic diversity, but travelling is a bridge and it might broaden one's horizons. Travel also teaches about one's homeland and the capability to \textbf{differentiate} one's national identity from the other, which then leads to increased \textbf{patriotism}.
    \end{qa}

    \begin{qa}{Can travel make a positive difference to the economy of a country? How?}
    Absolutely, travel directly support the economy through tourism industry. Coupled with other sectors like transport, tourism is a source of income and job creation in many countries. Every year, the revenue from services for tourists has \textbf{financed} many governmental programs such as infrastructure enhancement in \textbf{remote} areas. The unemployment rate is also reduced as job \textbf{vacancies} in tourism like marketers or salesmen might be \textbf{filled} by the locals.
    \end{qa}

    \begin{qa}{Do you think a society can benefit if its members have experience of travelling to other countries? In what ways?}
    Beyond a doubt, I agree that travel experience is for the sake of \textbf{community}. On a national perspective, the experience is \textbf{constructive} to international \textbf{mutual understanding} and to avoid \textbf{confrontation}, leading to more international cooperation. Moreover, if one has a curious mind, he or she may try to apply what he or she has observed in \textbf{foreign} countries to his or her own country in the furtherance of the citizens. For example, Phung Khac Khoan, a \textbf{feudal mandarin} in the 16th century, imported the corn seeds which had been \textbf{unbeknown} to the Vietnamese people at that time from China when he was travelling there as an envoy. The rest is history. Besides, this also helps to promote the preservation and \textbf{maintenance} of cultural heritage as people are generally guided to be aware of the significance of cultural values through travelling.
    \end{qa}

        \begin{VocabExplain}[Part 3]
            \begin{ExplainCard}{part and parcel}[idiom][C1]
            \EN{an essential or unavoidable part of something.}
            \SY{integral part; inherent element}
            \VI{phần tất yếu, không thể tách rời.}
            \EX{Night shifts are part and parcel of nursing.}
            \EX{Peer review is part and parcel of academic publishing.}
            \CO{be \textit{part and parcel of} sth}
            \end{ExplainCard}

            \begin{ExplainCard}{heaps}[n][B2]
            \EN{a large amount or number of something (informal).}
            \SY{loads; a ton; a wealth}
            \VI{rất nhiều; vô số.}
            \EX{There are heaps of cafés near campus.}
            \EX{The dataset contains heaps of user interactions to analyze.}
            \CO{\textit{heaps of} time/work/options}
            \end{ExplainCard}

            \begin{ExplainCard}{daily commute}[n][B2]
            \EN{the regular journey between home and work/school.}
            \SY{regular travel; commute}
            \VI{chuyến đi lại hằng ngày (đi làm/đi học).}
            \EX{Her daily commute takes forty minutes.}
            \EX{Reducing the daily commute improves employee productivity.}
            \CO{\textit{long/short} daily commute; commute \textit{by} bus/train}
            \end{ExplainCard}

            \begin{ExplainCard}{run errands}[phrase][B2]
            \EN{to make short trips to do tasks like shopping or paying bills.}
            \SY{do errands; dash about}
            \VI{chạy việc lặt vặt; đi làm việc vặt.}
            \EX{I ran errands for my parents on Saturday.}
            \EX{Gig couriers are often hired to run errands for customers.}
            \CO{\textit{run} errands; \textit{errand}-running}
            \end{ExplainCard}

            \begin{ExplainCard}{daily necessities}[n][B2]
            \EN{basic items needed for everyday life.}
            \SY{essentials; staples}
            \VI{nhu yếu phẩm hằng ngày.}
            \EX{The shop sells daily necessities at low prices.}
            \EX{Inflation in daily necessities affects low-income households most.}
            \CO{\textit{buy/provide} daily necessities}
            \end{ExplainCard}

            \begin{ExplainCard}{go off the beaten track}[phrase][C1]
            \EN{to visit places that are not popular or crowded.}
            \SY{take the road less travelled; explore hidden spots}
            \VI{đi đến nơi ít người biết/không phổ biến.}
            \EX{We went off the beaten track in the mountains.}
            \EX{Eco-tours encourage travelers to go off the beaten track to reduce overtourism.}
            \CO{\textit{go/venture} off the beaten track}
            \end{ExplainCard}

            \begin{ExplainCard}{hubbub}[n][C1]
            \EN{a loud, busy noise and activity of a place.}
            \SY{bustle; commotion; din}
            \VI{sự ồn ào náo nhiệt.}
            \EX{We escaped the hubbub of downtown.}
            \EX{Fieldnotes describe the hubbub of the morning market.}
            \CO{the \textit{hubbub of} the city/street}
            \end{ExplainCard}

            \begin{ExplainCard}{subject (to)}[adj][C1]
            \EN{likely to be affected by something; under the effect of.}
            \SY{prone to; liable to}
            \VI{dễ bị; chịu tác động bởi.}
            \EX{Commuters are subject to delays in bad weather.}
            \EX{Results are subject to sampling error in small cohorts.}
            \CO{\textit{be} subject to change/delay/approval}
            \end{ExplainCard}

            \begin{ExplainCard}{incident}[n][B2]
            \EN{an event, especially one that is unusual or unpleasant.}
            \SY{occurrence; episode}
            \VI{sự cố; vụ việc.}
            \EX{There were several minor incidents on the road.}
            \EX{Safety logs recorded three incidents during the trial.}
            \CO{traffic/security \textit{incident}; report an \textit{incident}}
            \end{ExplainCard}

            \begin{ExplainCard}{punctual}[adj][B2]
            \EN{happening or doing something at the agreed time.}
            \SY{on time; timely}
            \VI{đúng giờ.}
            \EX{Trains are rarely punctual at rush hour.}
            \EX{Punctual attendance correlates with higher course grades.}
            \CO{\textit{be} punctual; \textit{punctual} service}
            \end{ExplainCard}

            \begin{ExplainCard}{reckless driving}[n][C1]
            \EN{driving without care, creating danger to others.}
            \SY{careless driving; dangerous driving}
            \VI{lái xe ẩu; coi thường an toàn.}
            \EX{Reckless driving causes many accidents.}
            \EX{Law reforms impose heavier penalties for reckless driving.}
            \CO{\textit{charge/fine} for reckless driving}
            \end{ExplainCard}

            \begin{ExplainCard}{look up}[phr.v][B2]
            \EN{(of a situation) to improve; become better.}
            \SY{improve; pick up}
            \VI{khởi sắc; tiến triển tốt.}
            \EX{Things are looking up after the repair.}
            \EX{Economic indicators began to look up in Q3.}
            \CO{\textit{things} look up; prospects \textit{are} looking up}
            \end{ExplainCard}

            \begin{ExplainCard}{teleworking}[n][B2]
            \EN{working from home or outside the office using technology.}
            \SY{remote work; working from home}
            \VI{làm việc từ xa.}
            \EX{Teleworking lets parents manage school pick-ups.}
            \EX{Firms adopted teleworking policies during public-health emergencies.}
            \CO{\textit{adopt/allow} teleworking; teleworking policy}
            \end{ExplainCard}

            \begin{ExplainCard}{affordable}[adj][B2]
            \EN{cheap enough for most people to buy or pay for.}
            \SY{reasonably priced; cost-effective}
            \VI{giá phải chăng.}
            \EX{This course is affordable for students.}
            \EX{Affordable tuition improves access to higher education.}
            \CO{\textit{affordable} housing/tuition/option}
            \end{ExplainCard}

            \begin{ExplainCard}{geographical constraints}[n][C1]
            \EN{limits or barriers caused by distance, location, or terrain.}
            \SY{spatial barriers; location limits}
            \VI{rào cản địa lý.}
            \EX{Online classes remove geographical constraints.}
            \EX{Service delivery faces geographical constraints in remote regions.}
            \CO{\textit{remove/overcome} geographical constraints}
            \end{ExplainCard}

            \begin{ExplainCard}{Frankly}[adv][B2]
            \EN{honestly and directly, sometimes in a way that may seem blunt.}
            \SY{honestly; candidly}
            \VI{nói thẳng, thành thật mà nói.}
            \EX{Frankly, I don't like that plan.}
            \EX{Frankly stated limitations enhance research transparency.}
            \CO{\textit{Frankly}, ... ; speak \textit{frankly}}
            \end{ExplainCard}

            \begin{ExplainCard}{avid}[adj][C1]
            \EN{extremely interested and enthusiastic about something.}
            \SY{keen; ardent; enthusiastic}
            \VI{rất mê; nhiệt thành.}
            \EX{She's an avid traveler.}
            \EX{Avid readers often perform better in vocabulary tests.}
            \CO{\textit{avid} fan/reader/traveler}
            \end{ExplainCard}

            \begin{ExplainCard}{dual}[adj][C1]
            \EN{having two parts, uses, or aspects.}
            \SY{twofold; double}
            \VI{kép; hai mặt.}
            \EX{The trip had dual aims: study and leisure.}
            \EX{Bilingualism confers dual cognitive and social benefits.}
            \CO{\textit{dual} purpose/role/citizenship}
            \end{ExplainCard}

            \begin{ExplainCard}{from}[prep][A1]
            \EN{showing the source, origin, or starting point.}
            \SY{out of; originating in}
            \VI{từ; xuất phát từ.}
            \EX{We learned a lot from the tour.}
            \EX{Insights from prior studies guided the methodology.}
            \CO{\textit{learn from}; \textit{come from}}
            \end{ExplainCard}

            \begin{ExplainCard}{differentiate}[v][C1]
            \EN{to show or recognize how things are different.}
            \SY{distinguish; tell apart}
            \VI{phân biệt; làm cho khác biệt.}
            \EX{It's hard to differentiate real from fake goods.}
            \EX{The model differentiates classes with high precision.}
            \CO{\textit{differentiate} A \textit{from} B; \textit{differentiate between}}
            \end{ExplainCard}

            \begin{ExplainCard}{patriotism}[n][C1]
            \EN{love for and loyalty to one's country.}
            \SY{national pride; civic loyalty}
            \VI{lòng yêu nước.}
            \EX{Sporting victories can spark patriotism.}
            \EX{Civic education programs aim to foster informed patriotism.}
            \CO{foster/express \textit{patriotism}}
            \end{ExplainCard}

            \begin{ExplainCard}{finance}[v][C1]
            \EN{to provide money for a project or activity.}
            \SY{fund; bankroll; underwrite}
            \VI{tài trợ; cấp vốn.}
            \EX{Tourist taxes finance local services.}
            \EX{Grants financed infrastructure in underserved areas.}
            \CO{\textit{finance} programs/projects; publicly \textit{financed}}
            \end{ExplainCard}

            \begin{ExplainCard}{remote}[adj][B2]
            \EN{far away from main centers; distant.}
            \SY{far-flung; outlying}
            \VI{xa xôi; hẻo lánh.}
            \EX{They built clinics in remote villages.}
            \EX{Connectivity remains limited in remote regions.}
            \CO{\textit{remote} areas/communities/sites}
            \end{ExplainCard}

            \begin{ExplainCard}{vacancy}[n][B2]
            \EN{an unoccupied job or position.}
            \SY{opening; position}
            \VI{vị trí tuyển dụng trống.}
            \EX{There are several vacancies in hospitality.}
            \EX{Tourism growth creates seasonal vacancies for locals.}
            \CO{job \textit{vacancy}; fill/advertise a \textit{vacancy}}
            \end{ExplainCard}

            \begin{ExplainCard}{fill}[v][B2]
            \EN{to occupy a job or position; to appoint someone to it.}
            \SY{staff; occupy}
            \VI{lấp đầy; tuyển vào vị trí.}
            \EX{Local workers filled the vacancies quickly.}
            \EX{The university filled three tenure-track posts this year.}
            \CO{\textit{fill} a role/vacancy/post}
            \end{ExplainCard}

            \begin{ExplainCard}{community}[n][B2]
            \EN{a group of people living in the same place or sharing interests.}
            \SY{society; public}
            \VI{cộng đồng.}
            \EX{The local community benefits from tourism.}
            \EX{Community engagement strengthens policy uptake.}
            \CO{\textit{local} community; community \textit{benefit}}
            \end{ExplainCard}

            \begin{ExplainCard}{constructive}[adj][C1]
            \EN{helpful or useful for achieving a good result.}
            \SY{productive; beneficial}
            \VI{mang tính xây dựng; hữu ích.}
            \EX{Let's keep feedback constructive.}
            \EX{Constructive dialogue reduces cross-border tensions.}
            \CO{\textit{constructive} feedback/dialogue/role}
            \end{ExplainCard}

            \begin{ExplainCard}{mutual understanding}[n][C1]
            \EN{shared recognition of each other's views or needs.}
            \SY{reciprocal understanding; common ground}
            \VI{sự thấu hiểu lẫn nhau.}
            \EX{Travel fosters mutual understanding between cultures.}
            \EX{Programs aim to build mutual understanding among stakeholders.}
            \CO{\textit{promote/build} mutual understanding}
            \end{ExplainCard}

            \begin{ExplainCard}{confrontation}[n][C1]
            \EN{a hostile or argumentative meeting or situation.}
            \SY{conflict; clash}
            \VI{xung đột; đối đầu.}
            \EX{Dialogue helps avoid confrontation.}
            \EX{Preventing confrontation is central to peacebuilding research.}
            \CO{\textit{avoid/defuse} confrontation}
            \end{ExplainCard}

            \begin{ExplainCard}{foreign}[adj][B2]
            \EN{from or in a country that is not your own.}
            \SY{overseas; external}
            \VI{nước ngoài; ngoại quốc.}
            \EX{She studies foreign languages.}
            \EX{Foreign investment accelerates regional growth.}
            \CO{\textit{foreign} country/policy/language}
            \end{ExplainCard}

            \begin{ExplainCard}{feudal mandarin}[n][C1]
            \EN{a scholar-official serving in a Confucian bureaucracy (historical).}
            \SY{scholar-official; court official}
            \VI{quan lại thời phong kiến.}
            \EX{He wrote about a feudal mandarin's mission to China.}
            \EX{Texts describe the role of feudal mandarins in regional diplomacy.}
            \CO{\textit{a} feudal mandarin; imperial/royal \textit{mandarin}}
            \end{ExplainCard}

            \begin{ExplainCard}{unbeknown (to sb)}[adj][C2]
            \EN{unknown to someone; without their knowledge.}
            \SY{unknown; unnoticed}
            \VI{không hay biết; không ai biết.}
            \EX{Unbeknown to me, they planned a party.}
            \EX{Unbeknown to officials, the practice persisted in rural areas.}
            \CO{\textit{unbeknown to} sb}
            \end{ExplainCard}

            \begin{ExplainCard}{maintenance}[n][B2]
            \EN{the process of preserving or keeping something in good condition.}
            \SY{upkeep; preservation}
            \VI{bảo tồn; bảo trì, duy trì.}
            \EX{Museum maintenance protects artifacts.}
            \EX{Regular maintenance is vital for heritage conservation projects.}
            \CO{\textit{heritage} maintenance; \textit{routine} maintenance}
            \end{ExplainCard}
        \end{VocabExplain}

    \begin{VocabHighlights}
        \VH{can't help V-ing}{(phrase) used for saying that someone cannot stop themselves doing something}{(cụm từ) không thể ngừng}
        \VH{to get ahold of}{(idiom) to communicate with someone by phone}{(thành ngữ) liên lạc với ai qua điện thoại}
        \VH{part and parcel of}{(idiom) to be a feature of something, especially a feature that cannot be avoided}{(thành ngữ) là đặc điểm của}
        \VH{recipient}{(n) a person who receives something}{(danh từ) người nhận}
        \VH{in equilibrium with}{(phrase) in a state of balance}{(cụm từ) trong trạng thái cân bằng}
        \VH{hassle}{(n) a situation that is annoying because it involves doing something difficult or complicated that needs a lot of effort}{(danh từ) điều rắc rối, phiền muộn}
        \VH{to cement}{(v) to strengthen/make a relationship, an agreement, etc. stronger}{(động từ) củng cố, làm vững chắc thêm}
        \VH{to shoot the breeze}{(idiom) to chat, to waste time talking}{(thành ngữ) nói chuyện phiếm với ai}
        \VH{to cheer somebody up}{(phr.v) to make somebody happier}{(cụm động từ) khiến ai trở nên vui hơn}
        \VH{in a pickle}{(idiom) in a difficult/unpleasant situation}{(thành ngữ) trong tình cảnh khó khăn}
        \VH{bummed out}{(adj) depressed}{(tính từ) cảm thấy chán nản}
        \VH{to badmouth}{(v) to say unpleasant things about somebody}{(động từ) nói xấu ai}
        \VH{to heal the rift}{(idiom) make an unfriendly situation friendly again}{(thành ngữ) hàn gắn quan hệ}
        \VH{to be right up one's alley}{(idiom) be what somebody likes or good at doing}{(thành ngữ) đúng sở thích}
        \VH{bright and early}{(idiom) very early in the morning}{(thành ngữ) sớm tinh mơ}
        \VH{to go for a song}{(idiom) very cheap}{(thành ngữ) rất rẻ}
        \VH{to see the sights}{(idiom) to visit or view noteworthy things or locations, especially those frequented by tourists}{(thành ngữ) đi thăm những nơi nổi tiếng}
        \VH{to pump somebody up}{(idiom) to make someone feel more excited}{(thành ngữ) khuyến khích ai đó}
        \VH{an ecstasy}{(n) a state of extreme happiness}{(danh từ) sự cuồng dại}
        \VH{to relish}{(v) to like or enjoy something}{(động từ) thích hoặc thưởng thức cái gì đó}
        \VH{to set somebody back}{(phr.v) to cost somebody}{(cụm động từ) tốn của ai}
        \VH{an awful lot of}{(phrase) a lot of}{(cụm từ) cực nhiều}
        \VH{sheer}{(adj) used to emphasize how very great, important, or powerful a quality or feeling is}{(tính từ) đến tột độ}
        \VH{to make up for}{(phr.v) to take the place of something lost or damaged or to compensate for something bad with something good}{(cụm động từ) bù đắp lại}
        \VH{part and parcel of}{(phrase) to be a feature of something, especially a feature that cannot be avoided}{(cụm từ) là một phần không thể thiếu}
        \VH{heap of something}{(phrase) a lot of something}{(cụm từ) rất nhiều}
        \VH{daily commute}{(phrase) a regular journey of some distance to and from your place of work}{(cụm từ) di chuyển hằng ngày}
        \VH{to go off the beaten track}{(idiom) go to a place where few people go, far from any main roads and towns}{(thành ngữ) đi những nơi ít người biết}
        \VH{hubbub}{(n) a situation in which there is a lot of noise, excitement and activity}{(danh từ) sự ồn ào}
        \VH{constructive}{(adj) having a useful and helpful effect rather than being negative or with no purpose}{(tính từ) có lợi, có ích cho}
        \VH{confrontation}{(n) a situation in which there is an angry disagreement between people or groups who have different opinions}{(danh từ) sự đối chất; đối đầu}
        \VH{a feudal mandarin}{(phrase) an official in the history}{(cụm từ) quan lại phong kiến}
        \VH{unbeknown}{(adj) without a particular person knowing}{(tính từ) không ai biết đến}
        \VH{maintenance}{(n) the act of making a state or situation continue}{(danh từ) sự bảo trì}
    \end{VocabHighlights}
    \end{test}

    \begin{test}{TEST 4}
    \noindent
    \textbf{Part 1. Bicycles}
    \begin{qa}{How popular are bicycles in your home town? [Why?]}
    \textbf{As far as I'm concerned}, bicycles are still \textbf{all the rage} nowadays. However, there is a striking difference between the two major types of bicycles: road or electric ones. The former is gradually losing popularity as riders are likely to suffer from \textbf{fatigue}; whereas, it is much faster and less physically demanding to ride the latter.
    \end{qa}

    \begin{qa}{How often do you ride a bicycle? [Why?/Why not?]}
    It \textbf{dates back} to the time when I was in junior high school. At that time, travelling by bike to school was my daily routine. Since high school, I've \textbf{mounted} on my motorbike and now enjoyed the comfort of the front car seat. As a matter of fact, I haven't \textbf{pedalled in aeons} a bicycle.
    \end{qa}

    \begin{qa}{Do you think that bicycles are suitable for all ages? [Why?/Why not?]}
    Definitely yes. Everyone aged from five to eighty can get on their bikes to stay in shape and keep fit as well. However, in Hanoi, riding a bike is sometimes unsuitable for commuters as their health might be \textbf{put in jeopardy} after breathing in extremely polluted air on their way to work.
    \end{qa}

    \begin{qa}{What are the advantages of a bicycle compared to a car? [Why?]}
    There are several advantages I can think of. Firstly, in terms of monthly expenses, owning a car is by far more expensive than riding a bike. It's quite \textbf{pricey} to \textbf{maintain} a car monthly as such costs related to fuels, parking lots, and so on are taken into consideration. The bicycle also gains the advantage of being able to move freely regardless of traffic congestion where cars might \textbf{come to a standstill}.
    \end{qa}

        \begin{VocabExplain}[Part 1]
            \begin{ExplainCard}{as far as I'm concerned}[phrase][B2]
            \EN{used to introduce your personal opinion or view.}
            \SY{in my opinion; to me}
            \VI{theo quan điểm của tôi.}
            \EX{As far as I'm concerned, cycling is the best way to commute.}
            \EX{As far as I'm concerned, the data supports a cautious conclusion.}
            \CO{\textit{as far as I'm concerned}, ...}
            \end{ExplainCard}

            \begin{ExplainCard}{all the rage}[idiom][C1]
            \EN{very fashionable or popular at a particular time.}
            \SY{in vogue; trendy; the craze}
            \VI{đang rất thịnh hành.}
            \EX{Foldable bikes are all the rage this summer.}
            \EX{At the time of the study, short-form videos were all the rage among teens.}
            \CO{\textit{be} all the rage; become all the rage}
            \end{ExplainCard}

            \begin{ExplainCard}{fatigue}[n][C1]
            \EN{extreme tiredness; reduced physical strength from exertion.}
            \SY{exhaustion; weariness}
            \VI{sự mệt mỏi; kiệt sức.}
            \EX{Long hill climbs cause serious fatigue.}
            \EX{Driver fatigue is a significant risk factor in transport safety research.}
            \CO{\textit{combat/reduce} fatigue; mental/physical \textit{fatigue}}
            \end{ExplainCard}

            \begin{ExplainCard}{date back (to)}[phr.v][B2]
            \EN{to have existed since a particular time in the past.}
            \SY{go back to; originate from}
            \VI{bắt nguồn/tồn tại từ (thời điểm).}
            \EX{My cycling habit dates back to middle school.}
            \EX{The bridge dates back to the colonial period, archival records show.}
            \CO{\textit{date back to/from} + time}
            \end{ExplainCard}

            \begin{ExplainCard}{mount}[v][B2]
            \EN{to get on a horse, bicycle, or motorcycle to ride it.}
            \SY{get on; climb on}
            \VI{leo/lên (xe, ngựa) để lái/cưỡi.}
            \EX{He mounted his bike and rode off.}
            \EX{Participants mounted stationary bicycles for the VO\textsubscript{2} test.}
            \CO{\textit{mount} a bike/horse/motorbike}
            \end{ExplainCard}

            \begin{ExplainCard}{(not) in aeons}[idiom][C1]
            \EN{for a very long time; in ages.}
            \SY{in ages; for donkey's years}
            \VI{(đã) rất lâu rồi.}
            \EX{I haven't pedalled in aeons.}
            \EX{The village hasn't seen such floods in aeons, according to local reports.}
            \CO{\textit{not} in aeons/ages}
            \end{ExplainCard}

            \begin{ExplainCard}{put (sb/sth) in jeopardy}[phrase][C1]
            \EN{to put someone or something at risk of harm or loss.}
            \SY{endanger; imperil}
            \VI{đặt vào tình trạng nguy hiểm.}
            \EX{Cycling without a helmet puts you in jeopardy.}
            \EX{Cutting maintenance budgets may put public safety in jeopardy.}
            \CO{\textit{put} health/safety \textit{in jeopardy}}
            \end{ExplainCard}

            \begin{ExplainCard}{pricey}[adj][B2]
            \EN{informal: expensive; costing a lot of money.}
            \SY{costly; expensive; steep}
            \VI{đắt đỏ.}
            \EX{Downtown parking is pretty pricey.}
            \EX{Electric vehicles remain pricey for many households, surveys indicate.}
            \CO{\textit{pricey} fees/tickets/brands}
            \end{ExplainCard}

            \begin{ExplainCard}{maintain}[v][B2]
            \EN{to keep a machine/vehicle in good condition by regular checks and repair.}
            \SY{service; upkeep; preserve}
            \VI{bảo dưỡng; duy trì hoạt động.}
            \EX{It costs a lot to maintain an old car.}
            \EX{Transit agencies must maintain fleets to meet reliability targets.}
            \CO{\textit{maintain} a car/bike/system; routine \textit{maintenance}}
            \end{ExplainCard}

            \begin{ExplainCard}{come to a standstill}[idiom][C1]
            \EN{to stop completely, usually because of a problem.}
            \SY{grind to a halt; stop dead}
            \VI{đứng lại/đình trệ hoàn toàn.}
            \EX{Traffic came to a standstill after the accident.}
            \EX{Supply chains came to a standstill during the strike, the report notes.}
            \CO{\textit{come/grind} to a standstill}
            \end{ExplainCard}
        \end{VocabExplain}

    \noindent
    \textbf{Part 2.}
    \begin{qa}{Describe a person who has done a lot of work to help people. You should say:}
    \begin{itemize}
    \item Who this person is/was
    \item Where this person lives/lived
    \item What he/she has done to help people
    \item and explain how you know this person.
    \end{itemize}

    I have met so many people that have done \textbf{no end of} work to support people financially and mentally, but one person that I \textbf{look up to} is My Tam. She is a \textbf{household name} in music industry. I would like to \textbf{highlight the fact} that she is 39 years old now, but she is still a \textbf{fresh-faced} young woman and \textbf{as pretty as a picture}. If you see her \textbf{in the flesh}, you will be surprised by how young she is. I have been a huge fan of her for over 10 years because of her contribution to society. First, she started a charity for the homeless children, so they could be \textbf{brought up} to become well-educated citizens. As far as I'm concerned, her charity is home to over 300 homeless children. Second, she has launched some fund-raising events to offer scholarships to \textbf{academically gifted} but \textbf{impoverished} children. These scholarships enabled them to pursue their passion because in Vietnam, \textbf{needy} children are forced into early employment rather than go on with their education. I should not forget to mention that she is collaborating with some non-profit organizations in \textbf{putting an end} to violence in family. Domestic violence is still running \textbf{rampant}, especially in rural areas where literacy rate is poor and the husbands are somewhat patriarchal. Her influence helps to \textbf{heighten} the citizens' awareness of the severity of violence, and to \textbf{call for} more activities to protect women from violence. I adore her \textbf{from the bottom of my heart}.
    \end{qa}

        \begin{VocabExplain}[Part 2]
            \begin{ExplainCard}{no end of}{idiom}
            \EN{a very large amount of; a great many.}
            \SY{loads of; a wealth of}
            \VI{rất nhiều; vô số.}
            \EX{She's done no end of work for charity.}
            \EX{The campaign generated no end of useful data for analysis.}
            \CO{\textit{no end of} trouble/work/help}
            \end{ExplainCard}

            \begin{ExplainCard}{look up to}{phr.v}
            \EN{to admire and respect someone.}
            \SY{respect; idolize}
            \VI{ngưỡng mộ; kính trọng.}
            \EX{Many young singers look up to her.}
            \EX{Students often look up to mentors in professional programs.}
            \CO{\textit{look up to} a role model/leader}
            \end{ExplainCard}

            \begin{ExplainCard}{household name}{n}
            \EN{a person or thing that is very well known.}
            \SY{famous name; celebrity}
            \VI{tên tuổi ai cũng biết; nổi tiếng.}
            \EX{She's a household name across Vietnam.}
            \EX{After the breakthrough paper, the researcher became a household name in the field.}
            \CO{\textit{become} a household name}
            \end{ExplainCard}

            \begin{ExplainCard}{highlight the fact}{phrase}
            \EN{to emphasize that something is true or important.}
            \SY{underscore; stress}
            \VI{nhấn mạnh rằng; làm nổi bật thực tế rằng.}
            \EX{I must highlight the fact that she never seeks praise.}
            \EX{Reports highlight the fact that early aid is cost-effective.}
            \CO{\textit{highlight the fact that} + clause}
            \end{ExplainCard}

            \begin{ExplainCard}{fresh-faced}{adj}
            \EN{looking young and healthy; with a youthful appearance.}
            \SY{youthful; dewy}
            \VI{trẻ trung; khuôn mặt tươi trẻ.}
            \EX{She still looks fresh-faced after long tours.}
            \EX{Fresh-faced recruits often bring new perspectives to teams.}
            \CO{\textit{fresh-faced} teenager/performer}
            \end{ExplainCard}

            \begin{ExplainCard}{as pretty as a picture}{idiom}
            \EN{very attractive or beautiful.}
            \SY{gorgeous; lovely}
            \VI{xinh như tranh vẽ.}
            \EX{In person she's as pretty as a picture.}
            \EX{The village, as pretty as a picture, draws cultural tourists.}
            \CO{\textit{be} as pretty as a picture}
            \end{ExplainCard}

            \begin{ExplainCard}{in the flesh}{idiom}
            \EN{in real life, not on TV or in a photo.}
            \SY{in person; face to face}
            \VI{bằng xương bằng thịt; gặp trực tiếp.}
            \EX{I finally met her in the flesh.}
            \EX{Seeing the artifact in the flesh changed students' interpretations.}
            \CO{\textit{see/meet} sb \textit{in the flesh}}
            \end{ExplainCard}

            \begin{ExplainCard}{bring up}{phr.v}
            \EN{(1) to raise a child. (2) to mention a subject for discussion.}
            \SY{(1) raise \quad (2) mention}
            \VI{(1) nuôi dạy; (2) nêu ra (vấn đề).}
            \EX{(1) The charity helps children get brought up well.}
            \EX{(2) She brought up the need for transparency at the meeting.}
            \CO{\textit{be} well/badly \textit{brought up}; \textit{bring up} an issue}
            \end{ExplainCard}

            \begin{ExplainCard}{academically gifted}{adj}
            \EN{having exceptional ability or achievement in academic study.}
            \SY{high-achieving; talented}
            \VI{có năng khiếu/thiên hướng học thuật.}
            \EX{Scholarships support academically gifted students from poor families.}
            \EX{Programs for the academically gifted can accelerate learning outcomes.}
            \CO{\textit{academically gifted} children/students}
            \end{ExplainCard}

            \begin{ExplainCard}{impoverished}{adj}
            \EN{very poor; lacking money and resources.}
            \SY{destitute; underprivileged}
            \VI{nghèo túng; thiếu thốn.}
            \EX{Aid targeted impoverished communities.}
            \EX{Impoverished households are vulnerable to school dropout.}
            \CO{\textit{impoverished} areas/families/backgrounds}
            \end{ExplainCard}

            \begin{ExplainCard}{needy}{adj}
            \EN{lacking basic necessities; poor and requiring help.}
            \SY{disadvantaged; deprived}
            \VI{nghèo khó; cần được trợ giúp.}
            \EX{The fund buys textbooks for needy children.}
            \EX{Policies prioritize healthcare access for the needy.}
            \CO{\textit{needy} children/families/communities}
            \end{ExplainCard}

            \begin{ExplainCard}{put an end to}{phrase}
            \EN{to make something stop happening.}
            \SY{end; abolish; stamp out}
            \VI{chấm dứt; đặt dấu chấm hết.}
            \EX{They work to put an end to domestic violence.}
            \EX{Strict enforcement helped put an end to illegal logging.}
            \CO{\textit{put an end to} abuse/violence/practices}
            \end{ExplainCard}

            \begin{ExplainCard}{rampant}{adj}
            \EN{(of something bad) spreading quickly and hard to control.}
            \SY{unchecked; widespread}
            \VI{lan tràn; khó kiểm soát.}
            \EX{Rumors were rampant on social media.}
            \EX{Rampant inequality undermines social cohesion, studies show.}
            \CO{\textit{run} rampant; \textit{rampant} crime/corruption}
            \end{ExplainCard}

            \begin{ExplainCard}{heighten}{v}
            \EN{to increase the degree or intensity of something.}
            \SY{intensify; amplify; raise}
            \VI{tăng cường; nâng cao.}
            \EX{Campaigns heighten public awareness.}
            \EX{Heightened vigilance reduces relapse rates in clinical trials.}
            \CO{\textit{heighten} awareness/sensitivity/tension}
            \end{ExplainCard}

            \begin{ExplainCard}{call for}{phr.v}
            \EN{(1) to publicly demand or request. (2) to require/need.}
            \SY{(1) urge; appeal for \quad (2) require}
            \VI{(1) kêu gọi; (2) đòi hỏi/cần.}
            \EX{(1) Activists called for stronger protections.}
            \EX{(2) This emergency calls for swift action, say reports.}
            \CO{\textit{call for} action/reform; situation \textit{calls for} sth}
            \end{ExplainCard}

            \begin{ExplainCard}{from the bottom of my heart}{idiom}
            \EN{with deep sincerity; very genuinely.}
            \SY{wholeheartedly; sincerely}
            \VI{từ tận đáy lòng; chân thành.}
            \EX{Thank you from the bottom of my heart.}
            \EX{Survivors expressed gratitude from the bottom of their hearts in interviews.}
            \CO{\textit{thank/apologize} from the bottom of my heart}
            \end{ExplainCard}
        \end{VocabExplain}

    \noindent
    \textbf{Part 3.}
    \begin{qa}{What are some of the ways people can help others in the community? Which is most important?}
    I believe we can \textbf{lend a hand} to \textbf{the needy} in countless ways. Today, there are an increasing number of charities which people can donate their everyday items like clothes or footwear or even a small sum of money. Other ways can be participating in \textbf{disaster relief} programs to support \textbf{vulnerable} people. In my opinion, there is not a scale of importance with regard to helping people as all support is encouraged.
    \end{qa}

    \begin{qa}{Why do you think some people like to help other people?}
    Personally, helping out people is a \textbf{noble} gesture, and this should be \textbf{instilled} into as many people as possible. Most people are \textbf{sympathetic to} towards unfavourable conditions that many people are suffering and they have their \textbf{heart set on} joining hands to change the situation. Moreover, they feel a sense of community when it comes to assisting disadvantaged people, especially those are \textbf{living under the poverty line}.
    \end{qa}

    \begin{qa}{Some people say that people help others in the community more now than they did in the past. Do you agree or disagree? Why?}
    It is unfair to \textbf{generalise} how much people have helped their \textbf{fellows} through years as each generation has its own problems. Many years ago, most people were in \textbf{abject poverty} due to the war and diseases, so the majority of support came from the government \textbf{subsidies}. However, because of increased living standard nowadays, more and more people are living in good conditions which grant them chances to aid each other. But, I still believe that there's no point in judging how \textbf{altruistic} people are now or were in the past.
    \end{qa}

    \begin{qa}{What types of services, such as libraries or health centres, are available to the people who live in your area? Do you think there are enough of them?}
    I have been living in \textbf{downtown} Hanoi since I was knee-high to a grasshopper, and I enjoy most of \textbf{amenities} I have expected. Apart from retail stores, there is a gymnasium, a private \textbf{clinic} and a shopping mall nearby my house which perfectly meet the demands of the residents there. However, there is no public library in the \textbf{vicinity} of my place, just a few book shops with specialized books and magazines. Anyway, I think this does not trouble us at all because it is just a short drive to the library in the city center.
    \end{qa}

    \begin{qa}{Which groups of people generally need most support in a community? Why?}
    I remember an official \textbf{census} conducted few months ago that has revealed some interesting facts. First, disabled people usually suffer from social \textbf{discrimination} and anxiety because they are unable to live independently. Besides, children and senior citizens are two other groups which are \textbf{defenseless} because they may not be healthy enough to resist harmful impacts.
    \end{qa}

    \begin{qa}{Who do you think should pay for the services that are available to the people in the community? Should it be government or individual people?}
    It depends. On the one hand, the government and local councils ought to render social welfares to their citizens as they have their income \textbf{deducted} for the national tax system. In turn, part of the state budget should be expended in public services like insurance or medical \textbf{check-up}. On the other hand, pursuing education is individual obligation; therefore, the parents are also responsible for the tuition fees of their children. The government cannot \textbf{subsidize} everything, I believe.
    \end{qa}

        \begin{VocabExplain}[Part 3]
            \begin{ExplainCard}{lend a hand}[phr.v][B2]
            \EN{to help someone with a task.}
            \SY{assist; help out}
            \VI{giúp một tay.}
            \EX{Neighbors lent a hand after the storm.}
            \EX{Volunteers regularly lend a hand at community clinics.}
            \CO{\textit{lend/give} sb a hand; \textit{lend a hand with} sth}
            \end{ExplainCard}

            \begin{ExplainCard}{the needy}[n][B2]
            \EN{people who lack basic necessities or money.}
            \SY{the poor; the disadvantaged}
            \VI{những người túng thiếu.}
            \EX{The drive collects food for the needy.}
            \EX{Policies target the needy through means-tested benefits.}
            \CO{\textit{aid/support} the needy}
            \end{ExplainCard}

            \begin{ExplainCard}{disaster relief}[n][C1]
            \EN{assistance given to people after natural or man-made disasters.}
            \SY{emergency aid; humanitarian relief}
            \VI{cứu trợ thảm hoạ.}
            \EX{Students raised funds for disaster relief.}
            \EX{International disaster-relief efforts prioritize water and shelter.}
            \CO{\textit{provide/coordinate} disaster relief; disaster-relief \textit{program}}
            \end{ExplainCard}

            \begin{ExplainCard}{vulnerable}[adj][C1]
            \EN{at risk of harm or exploitation; easily affected.}
            \SY{at risk; exposed}
            \VI{dễ bị tổn thương.}
            \EX{Elderly people can be vulnerable to scams.}
            \EX{Reports map climate-vulnerable communities in coastal areas.}
            \CO{\textit{vulnerable} groups/populations; \textit{become} vulnerable to}
            \end{ExplainCard}

            \begin{ExplainCard}{noble}[adj][C1]
            \EN{morally good and admirable.}
            \SY{honorable; worthy}
            \VI{cao đẹp; đáng kính.}
            \EX{Volunteering is a noble thing to do.}
            \EX{Noble intentions alone rarely ensure policy success.}
            \CO{\textit{noble} cause/gesture/aim}
            \end{ExplainCard}

            \begin{ExplainCard}{instill}[v][C1]
            \EN{to gradually teach someone to accept a feeling or idea.}
            \SY{inculcate; implant}
            \VI{thấm nhuần; gieo vào.}
            \EX{Parents instill respect for others in children.}
            \EX{Civic education aims to instill pro-social values.}
            \CO{\textit{instill} confidence/discipline/values \textit{in} sb}
            \end{ExplainCard}

            \begin{ExplainCard}{sympathetic to}[adj][B2]
            \EN{showing that you understand and care about someone's suffering; supportive of.}
            \SY{compassionate; supportive}
            \VI{thông cảm; ủng hộ.}
            \EX{People were sympathetic to the flood victims.}
            \EX{Stakeholders were sympathetic to the proposed reforms.}
            \CO{\textit{be} sympathetic to/towards sb/sth}
            \end{ExplainCard}

            \begin{ExplainCard}{have one's heart set on}[idiom][C1]
            \EN{to be determined to get or do something.}
            \SY{be intent on; be set on}
            \VI{quyết tâm; nhất quyết muốn.}
            \EX{She has her heart set on volunteering abroad.}
            \EX{Many applicants have their hearts set on scholarship funding.}
            \CO{\textit{have your heart set on} + noun/V-ing}
            \end{ExplainCard}

            \begin{ExplainCard}{(live) under the poverty line}[phrase][B2]
            \EN{to have income below the level considered adequate for basic needs.}
            \SY{in poverty; impoverished}
            \VI{sống dưới mức nghèo.}
            \EX{Many families still live under the poverty line.}
            \EX{The report measures households under the poverty line by region.}
            \CO{\textit{live/fall} under the poverty line}
            \end{ExplainCard}

            \begin{ExplainCard}{generalise}[v][C1]
            \EN{to make a broad statement from limited cases.}
            \SY{overgeneralize; extrapolate}
            \VI{khái quát; đánh đồng.}
            \EX{Don't generalise from one story.}
            \EX{The study warns not to generalise beyond the sample.}
            \CO{\textit{too} quick to generalise; \textit{generalise} about/from}
            \end{ExplainCard}

            \begin{ExplainCard}{fellow(s)}[n][B2]
            \EN{people in the same group or community; companions.}
            \SY{peers; compatriots}
            \VI{đồng bào; bạn đồng nghiệp/người cùng cảnh.}
            \EX{He always helps his fellows at work.}
            \EX{Programs support refugees and their host-community fellows.}
            \CO{\textit{help/support} one's fellows; fellow citizens}
            \end{ExplainCard}

            \begin{ExplainCard}{abject poverty}[n][C1]
            \EN{extreme and hopeless poverty.}
            \SY{destitution; penury}
            \VI{cảnh nghèo đói cùng cực.}
            \EX{They grew up in abject poverty.}
            \EX{Policies reduced abject poverty over two decades.}
            \CO{\textit{live in/escape} abject poverty}
            \end{ExplainCard}

            \begin{ExplainCard}{subsidy / subsidize}[n/v][C1]
            \EN{money given by a government to support costs; to support with such money.}
            \SY{grant; funding; bankroll (v)}
            \VI{trợ cấp; trợ giá / trợ cấp.}
            \EX{Transport subsidies help rural students.}
            \EX{The state subsidizes essential medicines to widen access.}
            \CO{\textit{government} subsidies; \textit{subsidize} housing/tuition}
            \end{ExplainCard}

            \begin{ExplainCard}{altruistic}[adj][C1]
            \EN{showing a selfless concern for others' well-being.}
            \SY{selfless; philanthropic}
            \VI{vị tha; vì người khác.}
            \EX{Her altruistic acts inspired the team.}
            \EX{Donor behavior often stems from altruistic motives.}
            \CO{\textit{altruistic} motives/behavior}
            \end{ExplainCard}

            \begin{ExplainCard}{downtown}[n/adj/adv][B2]
            \EN{(in) the central business district of a city.}
            \SY{city centre; central district}
            \VI{(khu) trung tâm thành phố.}
            \EX{They moved to a flat downtown.}
            \EX{Downtown retail rebounded after transit upgrades.}
            \CO{\textit{downtown} area/shops; live \textit{downtown}}
            \end{ExplainCard}

            \begin{ExplainCard}{amenities}[n][B2]
            \EN{useful or desirable features of a place or building.}
            \SY{facilities; conveniences}
            \VI{tiện ích; cơ sở tiện nghi.}
            \EX{The neighborhood has great amenities.}
            \EX{Access to amenities correlates with quality-of-life indices.}
            \CO{\textit{local/public} amenities; lack of amenities}
            \end{ExplainCard}

            \begin{ExplainCard}{clinic}[n][B2]
            \EN{a place where people receive medical treatment.}
            \SY{health center; surgery (BrE)}
            \VI{phòng khám; cơ sở y tế.}
            \EX{She works at a private clinic.}
            \EX{Community clinics deliver primary care in low-income areas.}
            \CO{dental/eye \textit{clinic}; run a \textit{clinic}}
            \end{ExplainCard}

            \begin{ExplainCard}{vicinity}[n][C1]
            \EN{the area near or surrounding a particular place.}
            \SY{neighborhood; proximity}
            \VI{vùng lân cận; khu vực gần.}
            \EX{There's no library in the vicinity.}
            \EX{Noise levels were measured in the vicinity of airports.}
            \CO{\textit{in the} vicinity (of); \textit{nearby} vicinity}
            \end{ExplainCard}

            \begin{ExplainCard}{census}[n][C1]
            \EN{an official count or survey of a population.}
            \SY{population survey; enumeration}
            \VI{cuộc điều tra dân số.}
            \EX{The census runs every ten years.}
            \EX{Census data inform social-service allocation.}
            \CO{\textit{conduct} a census; census \textit{data}}
            \end{ExplainCard}

            \begin{ExplainCard}{discrimination}[n][C1]
            \EN{unfair treatment of people based on identity or condition.}
            \SY{bias; prejudice}
            \VI{sự phân biệt đối xử.}
            \EX{They campaign against disability discrimination.}
            \EX{Experiments measure labor-market discrimination.}
            \CO{\textit{fight/face} discrimination; \textit{anti-}discrimination law}
            \end{ExplainCard}

            \begin{ExplainCard}{defenseless}[adj][C1]
            \EN{unable to protect oneself from harm.}
            \SY{vulnerable; unprotected}
            \VI{yếu thế; không tự bảo vệ được.}
            \EX{Children are defenseless against abuse.}
            \EX{Defenseless groups require robust social safeguards.}
            \CO{\textit{defenseless} children/civilians; leave sb \textit{defenseless}}
            \end{ExplainCard}

            \begin{ExplainCard}{deduct}[v][B2]
            \EN{to take an amount away from a total, especially from money owed.}
            \SY{subtract; withhold}
            \VI{khấu trừ.}
            \EX{Taxes are deducted from your salary.}
            \EX{Fees were deducted before the grant was disbursed.}
            \CO{\textit{deduct} tax/points/fees \textit{from}}
            \end{ExplainCard}

            \begin{ExplainCard}{check-up}[n][B2]
            \EN{a medical examination to test general health.}
            \SY{health exam; screening}
            \VI{kiểm tra sức khỏe định kỳ.}
            \EX{Book a dental check-up twice a year.}
            \EX{Annual check-ups improve early-detection rates.}
            \CO{annual/routine \textit{check-up}; go for a \textit{check-up}}
            \end{ExplainCard}

            \begin{ExplainCard}{subsidize}[v][C1]
            \EN{to support financially in order to lower costs for users.}
            \SY{fund; underwrite}
            \VI{trợ cấp; bao cấp.}
            \EX{The city subsidizes bus fares for students.}
            \EX{Subsidized childcare increases labor-force participation.}
            \CO{\textit{subsidize} costs/tuition/transport}
            \end{ExplainCard}
        \end{VocabExplain}

    \begin{VocabHighlights}
        \VH{As far as I'm concerned}{(phrase) in my opinion}{(cụm từ) theo ý tôi thì}
        \VH{be all the rage}{(idiom) to be very popular at a particular time}{(thành ngữ) phổ biến}
        \VH{to date back to}{(phr.v) to have existed for a particular length of time or since a particular time}{(cụm động từ) có, bắt nguồn từ}
        \VH{to mount on}{(v) to get on a horse, bicycle, etc. in order to ride}{(động từ) cưỡi lên}
        \VH{to pedal}{(v) to push the pedals of a bicycle with your feet}{(động từ) đạp xe}
        \VH{in aeons}{(phrase) an immeasurably or indefinitely long period of time}{(cụm từ) từ rất lâu}
        \VH{to be put in jeopardy}{(phrase) to be at risk}{(cụm từ) gặp hiểm nguy}
        \VH{pricey}{(adj) expensive}{(tính từ) đắt đỏ}
        \VH{to come to a standstill}{(idiom) to slow down and finally stop}{(thành ngữ) dừng hẳn lại}
        \VH{no end of}{(idiom) a lot of}{(thành ngữ) rất nhiều}
        \VH{to look up to}{(phr.v) to admire and respect someone}{(cụm động từ) ngưỡng mộ ai đó}
        \VH{a household name}{(idiom) a famous person or organization}{(thành ngữ) người nổi tiếng, tổ chức nổi tiếng}
        \VH{to highlight the fact}{(phrase) emphasize the fact that}{(cụm từ) nhấn mạnh 1 điều rằng}
        \VH{fresh-faced}{(adj) having a young, healthy-looking face}{(tính từ) có gương mặt tươi trẻ}
        \VH{as pretty as a picture}{(idiom) very pretty}{(thành ngữ) rất xinh đẹp}
        \VH{in the flesh}{(idiom) in real life, and not on TV, in a film, in a picture}{(thành ngữ) bằng xương bằng thịt}
        \VH{to bring up}{(phr.v) raise somebody}{(cụm động từ) nuôi dạy}
        \VH{impoverished}{(adj) poor}{(tính từ) nghèo khó, nghèo khổ}
        \VH{needy}{(adj) not having enough money, food, clothes, etc.}{(tính từ) nghèo đói}
        \VH{to collaborate with}{(phr.v) to work with someone else for a special purpose}{(cụm động từ) hợp tác}
        \VH{to run rampant}{(idiom) getting worse quickly and in an uncontrolled way}{(thành ngữ) hoành hành, ác liệt}
        \VH{to heighten awareness}{(phrase) raise awareness}{(cụm từ) nâng cao nhận thức}
        \VH{to call for}{(phr.v) demand, require}{(cụm động từ) đòi hỏi, yêu cầu}
        \VH{from the bottom of my heart}{(idiom) sincerely}{(thành ngữ) từ tận đáy lòng}
        \VH{to lend a hand}{(phrase) to help somebody}{(cụm từ) giúp đỡ (ai đó)}
        \VH{the needy}{(n) (plural) people who do not have enough money, food, etc.}{(danh từ) người nghèo, túng thiếu}
        \VH{disaster relief}{(phrase) aid provided for alleviating the suffering of domestic disaster victims}{(cụm từ) cứu trợ thiên tai}
        \VH{vulnerable}{(adj) weak and easily hurt physically or emotionally}{(tính từ) dễ bị tổn thương}
        \VH{noble}{(adj) moral in an honest, brave, and kind way}{(tính từ) cao đẹp}
        \VH{to instill}{(v) to put a feeling, idea, or principle gradually into someone's mind, so that it has a strong influence on the way that person thinks or behaves}{(động từ) truyền đạt}
        \VH{sympathetic}{(adj) kind to somebody who is hurt or sad; showing that you understand and care about their problems}{(tính từ) thông cảm, đồng cảm}
        \VH{live under the poverty line}{(phrase) below the official level of income that is needed to achieve a basic living standard with enough money for things such as food, clothing, and a place to live}{(cụm từ) mức sống thấp; nghèo}
        \VH{to generalise}{(v) to use a particular set of facts or ideas in order to form an opinion that is considered valid for a different situation}{(động từ) khái quát hóa}
        \VH{fellows}{(n) a way of referring to a man or boy}{(danh từ) bạn bè; hội viên; đồng bào}
        \VH{abject poverty}{(phrase) poverty without hope}{(cụm từ) nghèo đói cùng cực}
        \VH{subsidy}{(n) money that is paid by a government or an organization to reduce the costs of services or of producing goods so that their prices can be kept low}{(danh từ) trợ cấp}
        \VH{altruistic}{(adj) caring about the needs and happiness of other people more than your own}{(tính từ) tính vị tha}
        \VH{downtown}{(adj) in, towards or typical of the centre of a city, especially its main business area}{(tính từ) trung tâm thành phố}
        \VH{amenity}{(n) a feature that makes a place pleasant, comfortable or easy to live in}{(danh từ) khu tiện nghi, tổ hợp}
        \VH{clinic}{(n) a building or part of a hospital where people can go for special medical treatment or advice}{(danh từ) phòng khám bệnh}
        \VH{vicinity}{(n) the area around a place or where the speaker is}{(danh từ) lân cận}
        \VH{census}{(n) the process of officially counting something, especially a country's population, and recording various facts}{(danh từ) điều tra dân số}
        \VH{discrimination}{(n) the ability to recognize a difference between one thing and another; a difference that is recognized}{(danh từ) phân biệt (chủng tộc, xã hội)}
        \VH{senior}{(adj) high in rank or status; higher in rank or status than others}{(tính từ) già}
        \VH{defenseless}{(adj) weak; not able to protect yourself; having no protection}{(tính từ) không phòng bị, yếu ớt}
        \VH{to deduct}{(v) to take away money, points, etc. from a total amount}{(động từ) khấu trừ}
        \VH{check-up}{(n) an examination of something, especially a medical one to make sure that you are healthy}{(danh từ) kiểm tra (sức khỏe)}
        \VH{obligation}{(n) the state of being forced to do something because it is your duty, or because of a law, etc.}{(danh từ) nghĩa vụ}
        \VH{to subsidize}{(v) to give money to a person or an organization in order to pay part of the cost of something that they do or make}{(động từ) bao cấp, trợ cấp}
    \end{VocabHighlights}
    \end{test}
\end{glossarymc}