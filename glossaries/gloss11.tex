\begin{glossarymc}[Cambridge 13]
    \begin{test}{TEST 1}
    \noindent
    \textbf{Part 1. Television Programmes}
    \begin{qa}{Where do you usually watch TV programmes/shows? [Why/Why not?]}
    As TV shows and programmes are \textbf{in abundance} on TV, I am in the habit of sitting on my couch watching them \textbf{in the comfort of} my home.
    \end{qa}

    \begin{qa}{What's your favourite TV programme/show? [Why?]}
    Well, I’m not an \textbf{avid} TV viewer so what I watch on TV is rather limited, except football matches at weekends. Maybe I am in favor of “The Voice”, a reality show with a view to discovering \textbf{latent} singers for the showbiz industry. I prefer its format, particularly Blind Auditions. This is a round when musical experts use their ears to select appropriate team members based on their voices only. To be admitted to any teams, a contestant must provide an \textbf{electrifying} performance to captivate their potential music trainers or else he or she would be disqualified.
    \end{qa}

    \begin{qa}{Are there any programmes/shows you don’t like watching? [Why/Why not?]}
    A show that I never bother to watch is “Bolero Idol” in Vietnam based on the show “Nation’s best voice” in the U.K. A metal fan like me finds bolero extremely boring. I tried turning on the TV and watching the show but I couldn’t stand watching this show for more than 5 minutes.
    \end{qa}

    \begin{qa}{Do you think you will watch more TV or fewer TV programmes/shows in the future? [Why/Why not?]}
    Maybe yes. The time I generally spend on watching shows may be replaced by that on football matches instead. In particular, I’m \textbf{inclined} to watch English Football Premier League at weekends. This is considered the best football league on this planet and its thrilling matches never fail to \textbf{enthral} me.
    \end{qa}

        \begin{VocabExplain}[Part 1]
            \begin{ExplainCard}{in abundance}[phrase][C1]
            \EN{in large quantities; more than enough.}
            \VI{dồi dào, có nhiều.}
            \SY{plentiful, ample, profuse}
            \EX{There are flowers in abundance during spring.}
            \EX{The TV offers entertainment shows in abundance.}
            \CO{exist/occur in abundance; resources in abundance}
            \end{ExplainCard}

            \begin{ExplainCard}{in the comfort of}[phrase][B2]
            \EN{in a state or situation where one feels relaxed and free from difficulties.}
            \VI{trong sự thoải mái, dễ chịu.}
            \SY{ease, relaxation}
            \EX{She enjoyed reading in the comfort of her living room.}
            \EX{I prefer studying in the comfort of my own home.}
            \CO{in the comfort of home/bed}
            \end{ExplainCard}

            \begin{ExplainCard}{avid}[adj][C1]
            \EN{very enthusiastic about something, often to an extreme degree.}
            \VI{hết sức nhiệt tình, đam mê.}
            \SY{keen, passionate, ardent}
            \EX{He is an avid football fan.}
            \EX{She is an avid reader of science fiction.}
            \CO{avid fan/reader/collector}
            \end{ExplainCard}

            \begin{ExplainCard}{latent}[adj][C1]
            \EN{existing but not yet developed or obvious; hidden.}
            \VI{tiềm ẩn, chưa bộc lộ.}
            \SY{hidden, dormant, undeveloped}
            \EX{He discovered his latent talent for singing.}
            \EX{There is a latent risk in ignoring climate change.}
            \CO{latent talent/ability/risk}
            \end{ExplainCard}

            \begin{ExplainCard}{electrifying}[adj][C1]
            \EN{very exciting; making people feel full of energy and enthusiasm.}
            \VI{gây hứng khởi mãnh liệt, đầy phấn khích.}
            \SY{thrilling, exciting, exhilarating}
            \EX{The band gave an electrifying performance.}
            \EX{Her speech was so electrifying that everyone stood up to applaud.}
            \CO{electrifying speech/performance/moment}
            \end{ExplainCard}

            \begin{ExplainCard}{inclined}[adj][C1]
            \EN{likely or tending to do something; having a tendency.}
            \VI{có xu hướng, có khuynh hướng.}
            \SY{disposed, prone, likely}
            \EX{She was inclined to accept the offer.}
            \EX{I am inclined to believe his story is true.}
            \CO{inclined to do sth; inclined towards}
            \end{ExplainCard}

            \begin{ExplainCard}{enthral}[v][C1]
            \EN{to keep someone completely interested and excited.}
            \VI{cuốn hút, mê hoặc.}
            \SY{captivate, fascinate, mesmerize}
            \EX{The magician enthralled the children with his tricks.}
            \EX{The match enthralled millions of viewers worldwide.}
            \CO{enthral sb with sth; be enthralled by}
            \end{ExplainCard}
        \end{VocabExplain}

    \noindent
    \textbf{Part 2.}
    \begin{qa}{Describe someone you know who has started a business. You should say: 
    \begin{itemize}
        \item Who this person is
        \item What work this person does
        \item Why this person decided to start a business
        \item and explain whether you would like to do the same kind of work as this person.
    \end{itemize}}

    I am an \textbf{alumnus} from Foreign Trade University, which has \textbf{stellar} reputation for \textbf{nurturing entrepreneurship}. I have had the opportunity to meet \textbf{a myriad of} businessmen with successful start-ups, but a person that I take inspiration from is Quang Huy. He is one year younger than me, but he is more mature for his age.

    We have known each other for a long time because we both worked for an English center. Hence, I know him \textbf{like the back of my hand}. If my memory serves me right, one year ago, he wanted me to \textbf{go into a business partnership} with him in \textbf{setting up} a language center. To my surprise, I asked him the reasons for this bold decision. First of all, he wanted to \textbf{earn good money}. If he worked for someone else, there were limited opportunities for \textbf{making a small fortune}. Another remarkable reason is that he wanted to build his own reputation. On second thought, I \textbf{turned down} his offer. At that time, I was not a \textbf{risk-taker} because I did not \textbf{have expertise in} management. But he has proved that he is a successful businessman when his start-up is \textbf{plain sailing}. His success has inspired me to \textbf{think big} and \textbf{from the bottom of my heart}, he is truly a \textbf{shining example} for me to follow.
    \end{qa}

        \begin{VocabExplain}[Part 2]
            \begin{ExplainCard}{alumnus}[n][C1]
            \EN{A former student of a particular school, college, or university.}
            \SY{graduate; former student}
            \VI{cựu sinh viên, cựu học sinh.}
            \EX{He is an alumnus of Harvard University.}
            \EX{Many alumni contribute financially to their alma mater.}
            \CO{distinguished alumnus; university alumnus}
            \end{ExplainCard}

            \begin{ExplainCard}{stellar}[adj][C1]
            \EN{Extremely high in quality; outstanding or excellent.}
            \SY{exceptional; superb; remarkable}
            \VI{xuất sắc, tuyệt vời.}
            \EX{She gave a stellar performance in the play.}
            \EX{The company has built a stellar reputation for customer service.}
            \CO{stellar performance; stellar reputation; stellar achievement}
            \end{ExplainCard}

            \begin{ExplainCard}{nurture entrepreneurship}[phrase][C1]
            \EN{To support and encourage the growth and development of entrepreneurial activities.}
            \SY{foster business spirit; cultivate innovation}
            \VI{nuôi dưỡng tinh thần khởi nghiệp.}
            \EX{The university runs programs to nurture entrepreneurship among students.}
            \EX{Government policies are designed to nurture entrepreneurship and innovation.}
            \CO{nurture entrepreneurship spirit; programs to nurture entrepreneurship}
            \end{ExplainCard}

            \begin{ExplainCard}{a myriad of}[phrase][C1]
            \EN{A very large number of something.}
            \SY{a multitude of; countless; innumerable}
            \VI{vô số, rất nhiều.}
            \EX{There are a myriad of stars visible in the night sky.}
            \EX{The city offers a myriad of opportunities for young professionals.}
            \CO{a myriad of opportunities; a myriad of choices}
            \end{ExplainCard}

            \begin{ExplainCard}{like the back of one’s hand}[idiom][C1]
            \EN{To know something or someone very well.}
            \SY{be very familiar with; know thoroughly}
            \VI{biết rõ như lòng bàn tay.}
            \EX{I know this neighborhood like the back of my hand.}
            \EX{He knows the company’s policies like the back of his hand.}
            \CO{know a place/person like the back of one’s hand}
            \end{ExplainCard}

            \begin{ExplainCard}{go into a business partnership}[phrase][C1]
            \EN{To start working with someone together in running a business.}
            \SY{team up; collaborate; establish a partnership}
            \VI{bắt đầu hợp tác kinh doanh.}
            \EX{They decided to go into a business partnership to open a café.}
            \EX{Going into a business partnership requires trust and clear agreements.}
            \CO{go into partnership with; establish a business partnership}
            \end{ExplainCard}

            \begin{ExplainCard}{set up}[phr.v][B2]
            \EN{To start or establish something, especially a business or organization.}
            \SY{establish; start; found}
            \VI{khởi nghiệp, thành lập.}
            \EX{They set up a non-profit organization to help children.}
            \EX{He set up a small business after graduating from university.}
            \CO{set up a company; set up a project; set up an organization}
            \end{ExplainCard}

            \begin{ExplainCard}{earn good money}[phrase][B2]
            \EN{To receive a large or satisfactory income from one’s work.}
            \SY{make a decent living; have a high income}
            \VI{kiếm được nhiều tiền.}
            \EX{She earns good money as a software engineer.}
            \EX{Doctors usually earn good money after years of training.}
            \CO{earn good money from; earn good money doing sth}
            \end{ExplainCard}

            \begin{ExplainCard}{make a small fortune}[phrase][C1]
            \EN{To earn or acquire a large amount of money.}
            \SY{earn a lot; amass wealth}
            \VI{kiếm bộn tiền, làm giàu.}
            \EX{He made a small fortune in the stock market.}
            \EX{She made a small fortune selling her artwork online.}
            \CO{make a small fortune from; make a small fortune doing sth}
            \end{ExplainCard}

            \begin{ExplainCard}{turn down}[phr.v][B2]
            \EN{To refuse or reject an offer or request.}
            \SY{reject; decline; refuse}
            \VI{từ chối.}
            \EX{He turned down the job offer.}
            \EX{The committee turned down the proposal due to high costs.}
            \CO{turn down an invitation; turn down an offer}
            \end{ExplainCard}

            \begin{ExplainCard}{risk-taker}[n][C1]
            \EN{A person who is willing to take risks in order to achieve something.}
            \SY{adventurer; daredevil; entrepreneur}
            \VI{người thích mạo hiểm.}
            \EX{Entrepreneurs are often risk-takers.}
            \EX{Scientific progress sometimes depends on risk-takers challenging old assumptions.}
            \CO{be a risk-taker; natural risk-taker}
            \end{ExplainCard}

            \begin{ExplainCard}{have expertise in}[phrase][C1]
            \EN{To have specialized knowledge or skills in a particular field.}
            \SY{be skilled in; have proficiency in}
            \VI{có chuyên môn về.}
            \EX{She has expertise in digital marketing.}
            \EX{The professor has expertise in quantum physics.}
            \CO{have expertise in management; expertise in engineering}
            \end{ExplainCard}

            \begin{ExplainCard}{plain sailing}[idiom][C1]
            \EN{Something that is easy and without problems.}
            \SY{smooth; effortless; straightforward}
            \VI{thuận buồm xuôi gió, dễ dàng.}
            \EX{The exam was plain sailing for her.}
            \EX{The negotiations were plain sailing compared to what we expected.}
            \CO{be plain sailing; everything is plain sailing}
            \end{ExplainCard}

            \begin{ExplainCard}{think big}[phrase][C1]
            \EN{To set ambitious goals or make grand plans.}
            \SY{be ambitious; aim high; dream big}
            \VI{nghĩ lớn, có tham vọng.}
            \EX{He always encourages his students to think big about their future.}
            \EX{Companies must think big to succeed in a global market.}
            \CO{encourage sb to think big; think big about the future}
            \end{ExplainCard}

            \begin{ExplainCard}{from the bottom of my heart}[idiom][B2]
            \EN{Used to emphasize that one means what one is saying very sincerely.}
            \SY{sincerely; genuinely; wholeheartedly}
            \VI{tận đáy lòng.}
            \EX{I want to thank you from the bottom of my heart.}
            \EX{From the bottom of my heart, I wish you success in your career.}
            \CO{thank sb from the bottom of one’s heart; apologize from the bottom of one’s heart}
            \end{ExplainCard}

            \begin{ExplainCard}{shining example}[n][C1]
            \EN{A person or thing that is a very good example of something.}
            \SY{model; role model; inspiration}
            \VI{tấm gương sáng.}
            \EX{She is a shining example of hard work and dedication.}
            \EX{The project is a shining example of international cooperation.}
            \CO{a shining example of success; a shining example for others}
            \end{ExplainCard}
        \end{VocabExplain}

    \noindent
    \textbf{Part 3.}
    \begin{qa}{What kinds of jobs do young people not want to do in your country?}
    To begin with, \textbf{underpaid} jobs might not sound appealing to most young people because these jobs can only provide meager income or a living wage for employees to lead a stable life. For example, \textbf{menial work} such as \textbf{janitors} is often suitable for those who are unskilled. Besides, dirty or dangerous jobs like \textbf{roofers} are not highly appreciated as well.
    \end{qa}

    \begin{qa}{Who is best at advising young people about choosing a job, teachers or parents?}
    It is \textbf{inequitable} to compare the role of parents and teachers as they can affect the way young people earning a living in equal measure. Teachers can link their subjects to the world of work, and apply their \textbf{pedagogic} skills to the delivery of career learning. Meanwhile, parents who are accustomed to the characteristics or even the \textbf{quirk} of their children can really give sound career advice.
    \end{qa}

    \begin{qa}{Is money always the most important thing when choosing a job?}
    Not really. Although almost every jobseeker would \textbf{gravitate} towards a guaranteed income when looking for a job to be free from financial worry, money is not \textbf{the name of the game}. I think \textbf{career prospect} is more important than money because it will directly affect their \textbf{tenure} and devotion. A job might not offer its candidates a promising income at first but in the long run, as long as it has \textbf{promotion prospects}, employees may benefit from more lucrative income in the long run.
    \end{qa}

    \begin{qa}{Do you agree that many people nowadays are under pressure to work longer hours and take less holiday?}
    I cannot agree more. Obviously, we cannot deny that the labour market is increasingly becoming \textbf{selective}, which means only outstanding employees can \textbf{flourish}. Therefore, employees are likely to be exploited by excessive workload and constant pressure to prove themselves professionally. For that reason, people these days will be \textbf{swarmed with} their work and \textbf{chances are that} they will work for more, \textbf{being entitled to} less holiday.
    \end{qa}

    \begin{qa}{What is the impact on society of people having a poor work-life balance?}
    Well, if this problem \textbf{persists} for a long time, many undesirable effects could be seen on a social perspective. Firstly, working adults do not have enough time to take care of their family and friends, which may \textbf{loosen} or even ruin the relationship. Secondly, if there is no break, a poor work-life balance will seriously \textbf{undermine} employees’ health, and cause health problems such as \textbf{insomnia} and even depression.
    \end{qa}

    \begin{qa}{Could you recommend some effective strategies for governments and employers to ensure people have a good work-life balance?}
    My suggestions are as follows. Firstly, it may sound too controlling but the government could impose harsher restrictions on the working time limits to ensure that no employee has to follow \textbf{grueling} work schedule. Secondly, \textbf{offering} an occasional company outing can boost employees’ \textbf{morale} and help workers get to know each other in a non-stressful capacity. This can be \textbf{low-key} or \textbf{extravagant}, depending on the company’s culture and budget for such things.
    \end{qa}

        \begin{VocabExplain}[Part 3]
            \begin{ExplainCard}{underpaid}[adj][C1]
            \EN{Receiving less money than is deserved for work done.}
            \SY{low-paid; poorly compensated}
            \VI{bị trả lương thấp.}
            \EX{Teachers are often underpaid for the hard work they do.}
            \EX{Underpaid jobs can cause dissatisfaction and high turnover.}
            \CO{be underpaid; underpaid workers; underpaid job}
            \end{ExplainCard}

            \begin{ExplainCard}{menial work}[phrase][C1]
            \EN{Work that does not require much skill and is often considered boring or unpleasant.}
            \SY{unskilled job; routine task}
            \VI{công việc tay chân, không đòi hỏi kỹ năng.}
            \EX{He was tired of doing menial work in the factory.}
            \EX{Menial work is usually given to untrained employees.}
            \CO{menial work such as; perform menial work}
            \end{ExplainCard}

            \begin{ExplainCard}{janitor}[n][B2]
            \EN{A person employed to take care of a building, including cleaning and maintenance.}
            \SY{caretaker; custodian}
            \VI{người lao công, bảo vệ tòa nhà.}
            \EX{The school janitor locked the doors every evening.}
            \EX{Janitors are essential to maintaining clean public spaces.}
            \CO{school janitor; building janitor}
            \end{ExplainCard}

            \begin{ExplainCard}{roofer}[n][B2]
            \EN{A person whose job is to build or repair roofs.}
            \SY{roof builder; roof mechanic}
            \VI{thợ lợp mái.}
            \EX{The roofer repaired the tiles after the storm.}
            \EX{Roofers often face dangerous working conditions.}
            \CO{hire a roofer; experienced roofer}
            \end{ExplainCard}

            \begin{ExplainCard}{inequitable}[adj][C1]
            \EN{Unfair or unjust.}
            \SY{unjust; unfair; biased}
            \VI{bất công, không công bằng.}
            \EX{It is inequitable to give special treatment to certain groups.}
            \EX{The policy was criticized for being inequitable towards minorities.}
            \CO{inequitable system; inequitable distribution}
            \end{ExplainCard}

            \begin{ExplainCard}{pedagogic}[adj][C1]
            \EN{Related to teaching or education.}
            \SY{educational; didactic; instructional}
            \VI{thuộc về giáo dục, sư phạm.}
            \EX{She used a pedagogic approach to engage her students.}
            \EX{The book provides pedagogic guidance for trainee teachers.}
            \CO{pedagogic skills; pedagogic methods}
            \end{ExplainCard}

            \begin{ExplainCard}{quirk}[n][C1]
            \EN{A peculiar aspect of a person’s character or behavior.}
            \SY{peculiarity; oddity; eccentricity}
            \VI{thói quen lạ, nét riêng.}
            \EX{His biggest quirk is always wearing mismatched socks.}
            \EX{Cultural quirks can influence business communication.}
            \CO{quirk of personality; cultural quirk}
            \end{ExplainCard}

            \begin{ExplainCard}{gravitate}[v][C1]
            \EN{To be attracted or drawn towards something, often naturally.}
            \SY{be drawn to; move toward}
            \VI{hướng về, bị cuốn hút bởi.}
            \EX{Children often gravitate towards friendly teachers.}
            \EX{Jobseekers gravitate towards companies with good benefits.}
            \CO{gravitate towards sth; gravitate naturally}
            \end{ExplainCard}

            \begin{ExplainCard}{the name of the game}[idiom][C1]
            \EN{The most important aspect of a situation.}
            \SY{the key point; the priority}
            \VI{điều quan trọng nhất.}
            \EX{In this industry, innovation is the name of the game.}
            \EX{Survival is the name of the game in competitive markets.}
            \CO{be the name of the game}
            \end{ExplainCard}

            \begin{ExplainCard}{career prospect}[n][C1]
            \EN{The chance of future success in a career.}
            \SY{job opportunity; professional outlook}
            \VI{triển vọng nghề nghiệp.}
            \EX{Good career prospects attract graduates to large firms.}
            \EX{Career prospects are limited in small towns.}
            \CO{excellent career prospects; long-term career prospect}
            \end{ExplainCard}

            \begin{ExplainCard}{tenure}[n][C1]
            \EN{The period of time that someone holds a job or position.}
            \SY{term; incumbency; occupancy}
            \VI{nhiệm kỳ, thời gian giữ chức.}
            \EX{His tenure as CEO lasted five years.}
            \EX{Professors with tenure cannot easily be dismissed.}
            \CO{long tenure; tenure of office}
            \end{ExplainCard}

            \begin{ExplainCard}{promotion prospects}[n][C1]
            \EN{The likelihood of being promoted in one’s job.}
            \SY{career advancement opportunities}
            \VI{cơ hội thăng tiến.}
            \EX{He chose that company because of its promotion prospects.}
            \EX{Poor promotion prospects can reduce employee motivation.}
            \CO{excellent promotion prospects; lack of promotion prospects}
            \end{ExplainCard}

            \begin{ExplainCard}{selective}[adj][C1]
            \EN{Carefully chosen; only allowing a limited group.}
            \SY{exclusive; choosy; particular}
            \VI{có chọn lọc, khắt khe.}
            \EX{The university is highly selective in admissions.}
            \EX{Selective hiring improves the company’s performance.}
            \CO{highly selective; selective admission}
            \end{ExplainCard}

            \begin{ExplainCard}{flourish}[v][C1]
            \EN{To grow or develop in a successful way.}
            \SY{thrive; prosper; succeed}
            \VI{phát triển mạnh mẽ.}
            \EX{Her business began to flourish after the new strategy.}
            \EX{Art and culture flourished during the Renaissance.}
            \CO{flourish in; flourish under}
            \end{ExplainCard}

            \begin{ExplainCard}{swarmed with}[phrase][C1]
            \EN{To be crowded or filled with a large number of people or things.}
            \SY{overrun; flooded; packed}
            \VI{tràn ngập, đầy ắp.}
            \EX{The station was swarmed with commuters.}
            \EX{She was swarmed with work emails.}
            \CO{swarmed with tourists; swarmed with tasks}
            \end{ExplainCard}

            \begin{ExplainCard}{chances are that}[phrase][B2]
            \EN{It is likely that something will happen.}
            \SY{likely; probably}
            \VI{có khả năng là.}
            \EX{Chances are that it will rain tomorrow.}
            \EX{Chances are that he will get the promotion.}
            \CO{chances are that + clause}
            \end{ExplainCard}

            \begin{ExplainCard}{being entitled to}[phrase][C1]
            \EN{Having the right to do or have something.}
            \SY{eligible for; authorized to}
            \VI{có quyền làm gì.}
            \EX{Employees are entitled to sick leave.}
            \EX{Citizens are entitled to free education.}
            \CO{be entitled to benefits; entitled to compensation}
            \end{ExplainCard}

            \begin{ExplainCard}{persist}[v][C1]
            \EN{To continue to exist or happen despite difficulty.}
            \SY{continue; endure; last}
            \VI{kéo dài, dai dẳng.}
            \EX{If the pain persists, see a doctor.}
            \EX{The problem persisted for many years.}
            \CO{persist in doing sth; persist for years}
            \end{ExplainCard}

            \begin{ExplainCard}{loosen}[v][B2]
            \EN{To make less tight or firm.}
            \SY{relax; slacken; untighten}
            \VI{nới lỏng, làm lỏng.}
            \EX{He loosened his tie after work.}
            \EX{Strict regulations should not be loosened carelessly.}
            \CO{loosen the grip; loosen restrictions}
            \end{ExplainCard}

            \begin{ExplainCard}{undermine}[v][C1]
            \EN{To weaken or damage gradually.}
            \SY{weaken; erode; sabotage}
            \VI{làm suy yếu, phá hoại.}
            \EX{Gossip can undermine trust in the workplace.}
            \EX{Poor diet will undermine your health.}
            \CO{undermine confidence; undermine authority}
            \end{ExplainCard}

            \begin{ExplainCard}{insomnia}[n][C1]
            \EN{The condition of being unable to sleep.}
            \SY{sleeplessness; restlessness}
            \VI{chứng mất ngủ.}
            \EX{She suffers from chronic insomnia.}
            \EX{Stress often leads to insomnia in students.}
            \CO{chronic insomnia; suffer from insomnia}
            \end{ExplainCard}

            \begin{ExplainCard}{grueling}[adj][C1]
            \EN{Extremely tiring and demanding.}
            \SY{exhausting; arduous; punishing}
            \VI{gian khổ, mệt nhoài.}
            \EX{They endured a grueling marathon.}
            \EX{Medical interns work grueling hours.}
            \CO{grueling schedule; grueling task}
            \end{ExplainCard}

            \begin{ExplainCard}{morale}[n][C1]
            \EN{The confidence, enthusiasm, and discipline of a group.}
            \SY{spirit; confidence; determination}
            \VI{tinh thần, nhuệ khí.}
            \EX{Winning the match boosted team morale.}
            \EX{Poor management can damage employee morale.}
            \CO{high morale; boost morale}
            \end{ExplainCard}

            \begin{ExplainCard}{low-key}[adj][C1]
            \EN{Not elaborate, modest, or simple in style.}
            \SY{simple; restrained; modest}
            \VI{giản dị, kín đáo.}
            \EX{They had a low-key wedding with only family present.}
            \EX{The meeting was low-key and informal.}
            \CO{low-key event; keep it low-key}
            \end{ExplainCard}

            \begin{ExplainCard}{extravagant}[adj][C1]
            \EN{Spending or costing too much money; excessive.}
            \SY{lavish; excessive; wasteful}
            \VI{xa hoa, hoang phí.}
            \EX{He bought an extravagant gift for her birthday.}
            \EX{Extravagant spending can lead to debt.}
            \CO{extravagant lifestyle; extravagant spending}
            \end{ExplainCard}
        \end{VocabExplain}

    \begin{VocabHighlights}
        \VH{to be in abundance}{(idiom) in large amounts}{(thành ngữ) có số lượng lớn}
        \VH{in the comfort of one’s home}{(idiom) at home}{(thành ngữ) ở nhà}
        \VH{avid}{(adj) very eager and enthusiastic}{(tính từ) nhiệt thành}
        \VH{latent}{(adj) existing, but not yet very noticeable, active or well developed}{(tính từ) có tiềm năng}
        \VH{electrifying}{(adj) very exciting}{(tính từ) rất hay}
        \VH{to be inclined to V-inf}{(adj) to be wanting to do something}{(tính từ) có xu hướng, mong muốn làm gì}
        \VH{to enthral}{(v) to charm, attract somebody}{(động từ) cuốn hút ai}
        \VH{alumnus}{(n) a graduate or former student, especially a male one, of a particular school, college, or university}{(danh từ) cựu học sinh/ cựu sinh viên}
        \VH{stellar}{(adj) excellent}{(tính từ) tuyệt vời}
        \VH{to nurture}{(v) to help someone or something develop by encouraging that person or thing}{(động từ) nuôi dưỡng}
        \VH{a myriad of}{(phrase) a very large number of something}{(cụm từ) một lượng lớn}
        \VH{to know somebody like the back of my hand}{(idiom) to have very good and detailed knowledge of someone}{(thành ngữ) biết ai đó như lòng bàn tay}
        \VH{to go into a business partnership}{(phrase) co-operate with somebody to do business}{(cụm từ) hợp tác kinh doanh}
        \VH{to somebody’s surprise}{(phrase) something makes me surprised}{(cụm từ) trước sự ngạc nhiên của ai}
        \VH{to earn good money}{(idiom) earn a lot of money}{(thành ngữ) kiếm nhiều tiền}
        \VH{to make a small fortune}{(idiom) to earn a great deal of money}{(thành ngữ) kiếm nhiều tiền}
        \VH{to turn down}{(phr.v) refuse}{(cụm động từ) từ chối}
        \VH{a risk-taker}{(n) someone who often takes risks}{(danh từ) người chấp nhận rủi ro}
        \VH{to have expertise in}{(phrase) have particular knowledge about particular field}{(cụm từ) có chuyên môn về}
        \VH{to be plain sailing}{(idiom) to be easy and without problems}{(thành ngữ) dễ dàng không thành vấn đề}
        \VH{to think big}{(idiom) to have plans to be very successful or powerful}{(thành ngữ) có kế hoạch để thành công}
        \VH{to get down to business}{(phrase) to start talking about the subject to be discussed}{(cụm từ) bắt tay vào làm việc gì đó}
        \VH{from the bottom of my heart}{(idiom) sincerely, frankly}{(thành ngữ) từ tận đáy lòng của tôi}
        \VH{a shining example}{(phrase) an excellent example}{(cụm từ) ví dụ điển hình}
        \VH{menial work}{(phrase) a task is anything that takes very little training, skill, or talent}{(cụm từ) công việc tay chân}
        \VH{a janitor}{(n) a person whose job is to clean and take care of a building}{(danh từ) người lau dọn}
        \VH{a roofer}{(n) a person whose job is to put new roofs on buildings or to repair damaged roofs}{(danh từ) thợ sửa mái nhà}
        \VH{inequitable}{(adj) not fair; not the same for everyone}{(tính từ) không công bằng}
        \VH{pedagogic}{(adj) relating to teaching}{(tính từ) thuộc về sư phạm}
        \VH{quirk}{(n) an aspect of somebody’s personality or behaviour that is a little strange}{(danh từ) thói quen riêng}
        \VH{the name of the game}{(idiom) the main purpose or most important aspect of a situation}{(thành ngữ) mục đích chính, điều quan trọng nhất}
        \VH{to gravitate}{(v) to be attracted to or move toward something}{(động từ) nghiêng về}
        \VH{career prospect}{(phrase) the probability or chance for future success in a profession}{(cụm từ) thăng tiến sự nghiệp}
        \VH{tenure}{(n) the period of time when somebody holds an important job, especially a political one; the act of holding an important job}{(danh từ) nhiệm kỳ}
        \VH{promotion prospect}{(phrase) the chances or prospects an employee has for promotion or for gaining a better position, often in the same company}{(cụm từ) triển vọng thăng tiến}
        \VH{selective}{(adj) affecting or concerned with only a small number of people or things from a larger group}{(tính từ) tính chọn lọc}
        \VH{surface}{(v) to appear at the surface of something}{(động từ) nổi trội}
        \VH{to be swamped with}{(phrase) to be extremely busy with}{(cụm từ) bận bịu}
        \VH{(the) chances are that}{(idiom) it is likely that}{(thành ngữ) rất có thể là}
        \VH{to be entitled to}{(phrase) be allowed}{(cụm từ) được quyền làm gì}
        \VH{to persist}{(v) to continue to do something despite difficulties or opposition, in a way that can seem unreasonable}{(động từ) kéo dài}
        \VH{loosen}{(v) to make something less tight or firmly fixed; to become less tight or firmly fixed}{(động từ) xa cách, nới lỏng}
        \VH{to undermine}{(v) to make something, especially somebody’s confidence or authority, gradually weaker or less effective}{(động từ) mài mòn}
        \VH{insomnia}{(n) the condition of being unable to sleep}{(danh từ) tình trạng khó ngủ}
        \VH{grueling}{(adj) very difficult and tiring, needing great effort for a long time}{(tính từ) gay go, vất vả}
        \VH{low-key}{(adj) not intended to attract a lot of attention}{(tính từ) có chừng mực}
        \VH{extravagant}{(adj) spending a lot more money or using a lot more of something than you can afford or than is necessary}{(tính từ) hoang phí}
    \end{VocabHighlights}
    \end{test}

    \begin{test}{TEST 2}
    \noindent
    \textbf{Part 1. Age}
    \begin{qa}{Are you happy to be the age you are now? [Why/Why not?]}
    Truth be told, I’m content with myself at my age now. I have \textbf{fulfilled} every childhood dream such as visiting Old Trafford, the home stadium of Manchester United, my favorite football team and attending concerts of my idols such as X Japan, Linkin Park and Buc Tuong, etc. I’ve also been granted a Master degree in addition to landing a stable job and living in a loving family. I definitely \textbf{couldn’t ask for more}.
    \end{qa}

    \begin{qa}{When you were a child, did you think a lot about your future? [Why/Why not?]}
    I didn’t. In fact, I learned to \textbf{take it one day at a time}. Instead of worrying too much about my future, I focused on doing what I liked and what I could do best to \textbf{have a whale of a time}. Predicting the future is the duty of a fortune teller, not an \textbf{atheist like} me.
    \end{qa}

    \begin{qa}{Do you think you have changed as you have got older? [Why/Why not?]}
    I have changed dramatically since I was getting older. In particular, I have been more \textbf{realistic}, maybe even more \textbf{pragmatic}, than I used to be. I once used to consider studying Japanese to know more about Japanese cultures in the forms of manga and music. However, as \textbf{time flies like an arrow}, I realize that I am \textbf{not cut out for} this language as it is too complicated to learn. I switch to English and then I can proudly call myself an \textbf{accomplished} teacher. This is only a \textbf{self-proclaimed} title but all in all, I’m satisfied with these changes.
    \end{qa}

    \begin{qa}{What will be different about your life in the future? [Why]}
    It depends. In terms of my career, although it is not a \textbf{high-powered job in the absence of promotion potential}, I’ll still \textbf{stick to it} in the long run. Regarding my personal life, I wish to have a penthouse and hopefully this dream will come true. I’ll have to \textbf{shoulder} more responsibilities but I’m always \textbf{willing to} take on new challenges.
    \end{qa}

        \begin{VocabExplain}[Part 1]
            \begin{ExplainCard}{fulfill}[v][B2]
            \EN{To achieve or realize something that you have desired or promised.}
            \SY{accomplish; achieve; carry out}
            \VI{hoàn thành, thực hiện.}
            \EX{She fulfilled her ambition to become a doctor.}
            \EX{The government fulfilled its promise to lower taxes.}
            \CO{fulfill a dream; fulfill an ambition; fulfill a requirement}
            \end{ExplainCard}

            \begin{ExplainCard}{couldn't ask for more}[idiom][C1]
            \EN{To be extremely satisfied with what one has; nothing better could be desired.}
            \SY{be content; be fully satisfied}
            \VI{không còn mong gì hơn.}
            \EX{The service was excellent—I couldn’t ask for more.}
            \EX{With supportive friends and a stable job, she really couldn’t ask for more.}
            \CO{couldn’t ask for more from; couldn’t ask for more in life}
            \end{ExplainCard}

            \begin{ExplainCard}{take it one day at a time}[idiom][C1]
            \EN{To deal with things as they happen, without worrying about the future.}
            \SY{live in the moment; go step by step}
            \VI{sống từng ngày, không lo xa.}
            \EX{After the accident, he learned to take it one day at a time.}
            \EX{In stressful times, it helps to take things one day at a time.}
            \CO{take it one day at a time; live one day at a time}
            \end{ExplainCard}

            \begin{ExplainCard}{have a whale of a time}[idiom][C1]
            \EN{To have a very enjoyable experience.}
            \SY{enjoy oneself; have a blast}
            \VI{có một khoảng thời gian vui vẻ.}
            \EX{We had a whale of a time at the party.}
            \EX{Travelers often have a whale of a time exploring new cultures.}
            \CO{have a whale of a time doing sth}
            \end{ExplainCard}

            \begin{ExplainCard}{atheist}[n][C1]
            \EN{A person who does not believe in the existence of God or gods.}
            \SY{non-believer; skeptic}
            \VI{người vô thần.}
            \EX{He declared himself an atheist at the age of 20.}
            \EX{Atheists argue that morality can exist without religion.}
            \CO{atheist belief; atheist like sb}
            \end{ExplainCard}

            \begin{ExplainCard}{realistic}[adj][B2]
            \EN{Accepting things as they are and able to deal with them sensibly.}
            \SY{practical; rational; down-to-earth}
            \VI{thực tế, hợp lý.}
            \EX{We need a realistic plan for the budget.}
            \EX{She is realistic about her chances of success.}
            \CO{realistic expectations; realistic approach}
            \end{ExplainCard}

            \begin{ExplainCard}{pragmatic}[adj][C1]
            \EN{Dealing with problems in a practical and reasonable way.}
            \SY{practical; rational; sensible}
            \VI{thực dụng, thực tế.}
            \EX{He took a pragmatic approach to management.}
            \EX{Policy decisions must be pragmatic, not ideological.}
            \CO{pragmatic approach; be pragmatic about}
            \end{ExplainCard}

            \begin{ExplainCard}{time flies like an arrow}[idiom][C1]
            \EN{Time passes very quickly.}
            \SY{time passes quickly; time slips away}
            \VI{thời gian trôi nhanh như tên bắn.}
            \EX{Time flies like an arrow when you’re on vacation.}
            \EX{As people age, they often feel that time flies like an arrow.}
            \CO{time flies like an arrow; time flies}
            \end{ExplainCard}

            \begin{ExplainCard}{not cut out for}[idiom][C1]
            \EN{Not suitable for a particular task or job.}
            \SY{unsuited; unfit; inadequate}
            \VI{không phù hợp, không có khả năng.}
            \EX{I’m not cut out for teaching young kids.}
            \EX{He realized he was not cut out for politics.}
            \CO{not cut out for sth}
            \end{ExplainCard}

            \begin{ExplainCard}{accomplished}[adj][C1]
            \EN{Highly skilled or proficient in something.}
            \SY{skilled; expert; proficient}
            \VI{tài giỏi, xuất sắc.}
            \EX{She is an accomplished pianist.}
            \EX{He became an accomplished scholar in his field.}
            \CO{accomplished teacher; accomplished writer}
            \end{ExplainCard}

            \begin{ExplainCard}{self-proclaimed}[adj][C1]
            \EN{Described as such by the person themselves, often without official recognition.}
            \SY{self-declared; self-styled}
            \VI{tự xưng, tự nhận.}
            \EX{He is a self-proclaimed expert on the subject.}
            \EX{The self-proclaimed leader made a speech.}
            \CO{self-proclaimed expert; self-proclaimed title}
            \end{ExplainCard}

            \begin{ExplainCard}{high-powered job}[phrase][C1]
            \EN{A very important and influential job.}
            \SY{prestigious job; top position}
            \VI{công việc quyền lực, danh giá.}
            \EX{She holds a high-powered job in finance.}
            \EX{High-powered jobs often come with stress.}
            \CO{high-powered executive/job/career}
            \end{ExplainCard}

            \begin{ExplainCard}{absence of promotion potential}[phrase][C1]
            \EN{The lack of opportunity for career advancement.}
            \SY{no career growth; limited prospects}
            \VI{thiếu cơ hội thăng tiến.}
            \EX{He left because of the absence of promotion potential.}
            \EX{Absence of promotion potential leads to low morale.}
            \CO{absence of promotion potential in a job}
            \end{ExplainCard}

            \begin{ExplainCard}{stick to it}[idiom][B2]
            \EN{To continue doing something despite difficulty.}
            \SY{persevere; persist; keep at it}
            \VI{kiên trì, bền bỉ theo đuổi.}
            \EX{He stuck to it until he mastered the guitar.}
            \EX{Success requires people to stick to it.}
            \CO{stick to it despite challenges}
            \end{ExplainCard}

            \begin{ExplainCard}{shoulder}[v][C1]
            \EN{To take responsibility for something.}
            \SY{take on; bear; assume}
            \VI{gánh vác, chịu trách nhiệm.}
            \EX{He shouldered the blame for the accident.}
            \EX{Parents often shoulder the cost of education.}
            \CO{shoulder responsibility; shoulder the cost}
            \end{ExplainCard}

            \begin{ExplainCard}{willing to}[phrase][B2]
            \EN{Ready and prepared to do something.}
            \SY{prepared; ready; inclined}
            \VI{sẵn lòng, sẵn sàng.}
            \EX{She is always willing to help her friends.}
            \EX{Employers look for people who are willing to learn.}
            \CO{be willing to do sth}
            \end{ExplainCard}
        \end{VocabExplain}

    \noindent
    \textbf{Part 2.}
    \begin{qa}{Describe a time when you started using a new technological device. You should say: 
    \begin{itemize}
        \item What device you started using
        \item Why you started using this device
        \item How easy or difficult it was to use
        \item and explain how helpful this device was to you.
    \end{itemize}}

    I am a \textbf{tech-savvy} man so I’m \textbf{in the habit of shelling out} money to purchase high-tech gadgets. If I had to choose a technological device, I’d choose to talk about my new gaming laptop, Asus ROG FX504. It was \textbf{a bargain} at the price of \$1,000 because I \textbf{snapped it up} in the sales. No laptop of such great specifications could be cheaper than that. \textbf{As its name suggests}, it is built for gamers to play games smoothly at the highest setting. I am not a hardcore gamer but I needed a new laptop that can handle heavy tasks such as editing photos on Photoshop to replace my old one which had \textbf{played up} earlier. I did not have any difficulty \textbf{getting to grips with} this laptop as it runs on Windows 10, an operating system that everyone is \textbf{familiar with}, let alone I am a \textbf{techie}. Since the day I owned it 4 years ago until now, it has never disappointed me. It has handled both light and heavy tasks \textbf{with ease}, which surely \textbf{boosts my work efficiency}. It weighs just 2.2kg, which is lighter than other gaming laptops so it is \textbf{portable} enough for me to bring it to my workplace. If I had the right to choose again, I would still buy this one \textbf{without a second thought}.
    \end{qa}

        \begin{VocabExplain}[Part 2]
            \begin{ExplainCard}{tech-savvy}[adj][C1]
            \EN{Having a good knowledge and understanding of modern technology, especially computers.}
            \SY{technologically skilled; computer-literate}
            \VI{rành công nghệ.}
            \EX{Young people today are generally more tech-savvy.}
            \EX{Employers look for tech-savvy candidates in IT roles.}
            \CO{tech-savvy user; tech-savvy generation}
            \end{ExplainCard}

            \begin{ExplainCard}{in the habit of shelling out}[phrase][C1]
            \EN{To regularly spend a large amount of money on something.}
            \SY{spend; pay out; fork out}
            \VI{có thói quen chi nhiều tiền.}
            \EX{He is in the habit of shelling out money for luxury items.}
            \EX{They are in the habit of shelling out on expensive gadgets.}
            \CO{in the habit of shelling out money/cash}
            \end{ExplainCard}

            \begin{ExplainCard}{a bargain}[n][B2]
            \EN{Something bought cheaply or for less than its usual price.}
            \SY{good deal; discount; value buy}
            \VI{món hời, mua được giá rẻ.}
            \EX{That dress was a real bargain at only \$20.}
            \EX{They found a bargain in the electronics sale.}
            \CO{real bargain; bargain price}
            \end{ExplainCard}

            \begin{ExplainCard}{snap up}[phr.v][C1]
            \EN{To quickly buy or take advantage of something that is in short supply.}
            \SY{grab; seize; take quickly}
            \VI{chộp lấy, mua nhanh.}
            \EX{Tickets for the concert were snapped up in minutes.}
            \EX{He snapped up a new phone during the Black Friday sales.}
            \CO{snap up bargains; snap up tickets}
            \end{ExplainCard}

            \begin{ExplainCard}{as its name suggests}[phrase][B2]
            \EN{Used to explain that something is exactly what its name implies.}
            \SY{as implied; as indicated}
            \VI{như tên gọi đã gợi ý.}
            \EX{The White House, as its name suggests, is painted white.}
            \EX{The ‘Smart Watch,’ as its name suggests, can connect to your phone.}
            \CO{as its name suggests + clause}
            \end{ExplainCard}

            \begin{ExplainCard}{play up}[phr.v][C1]
            \EN{When a machine or device does not work properly.}
            \SY{malfunction; act up; fail}
            \VI{trục trặc, hỏng hóc.}
            \EX{My phone is playing up again.}
            \EX{The printer played up during the exam.}
            \CO{machine/device plays up}
            \end{ExplainCard}

            \begin{ExplainCard}{get to grips with}[idiom][C1]
            \EN{To begin to understand or deal with something difficult.}
            \SY{tackle; handle; come to terms with}
            \VI{nắm bắt, bắt đầu quen với.}
            \EX{She finally got to grips with the new software.}
            \EX{It takes time to get to grips with a new language.}
            \CO{get to grips with a task; get to grips with technology}
            \end{ExplainCard}

            \begin{ExplainCard}{familiar with}[phrase][B2]
            \EN{Having knowledge or experience of something.}
            \SY{accustomed to; knowledgeable about}
            \VI{quen thuộc, biết về.}
            \EX{I am familiar with this area of town.}
            \EX{The students are familiar with the exam format.}
            \CO{be familiar with sth; familiar with the system}
            \end{ExplainCard}

            \begin{ExplainCard}{techie}[n][C1]
            \EN{A person who is very knowledgeable or enthusiastic about technology.}
            \SY{tech enthusiast; IT specialist}
            \VI{người đam mê công nghệ.}
            \EX{He is a techie who loves coding.}
            \EX{Techies often attend gadget fairs.}
            \CO{self-proclaimed techie; true techie}
            \end{ExplainCard}

            \begin{ExplainCard}{with ease}[phrase][B2]
            \EN{Easily and without difficulty.}
            \SY{easily; effortlessly; smoothly}
            \VI{dễ dàng.}
            \EX{She passed the exam with ease.}
            \EX{The athlete completed the race with ease.}
            \CO{do sth with ease}
            \end{ExplainCard}

            \begin{ExplainCard}{boost one’s work efficiency}[phrase][C1]
            \EN{To increase productivity and effectiveness in work.}
            \SY{improve performance; enhance productivity}
            \VI{tăng hiệu suất công việc.}
            \EX{Using modern tools can boost work efficiency.}
            \EX{Flexible schedules often boost employee efficiency.}
            \CO{boost efficiency; efficiency boost}
            \end{ExplainCard}

            \begin{ExplainCard}{portable}[adj][B2]
            \EN{Easy to carry or move, especially because of being lighter or smaller.}
            \SY{lightweight; movable; handy}
            \VI{cơ động, dễ mang theo.}
            \EX{Laptops are more portable than desktop computers.}
            \EX{Portable devices are essential for remote work.}
            \CO{portable computer; portable device}
            \end{ExplainCard}

            \begin{ExplainCard}{without a second thought}[idiom][C1]
            \EN{Immediately and without any hesitation.}
            \SY{instantly; without hesitation}
            \VI{không do dự, ngay lập tức.}
            \EX{He agreed to help without a second thought.}
            \EX{She spent the money without a second thought.}
            \CO{do sth without a second thought}
            \end{ExplainCard}
        \end{VocabExplain}

    \noindent
    \textbf{Part 3.}
    \begin{qa}{What is the best age for children to start computer lessons?}
    Becoming \textbf{computer literate} is definitely important in the digital world today. In my opinion, computer lessons should be conducted in secondary schools because this is the age that children have developed basic \textbf{literacy} and \textbf{numeracy}. As a result, the foundation will facilitate the way students learn how to program computers.
    \end{qa}

    \begin{qa}{Do you think schools should use more technology to help children learn?}
    Schools are \textbf{on the fence} about the use of certain technological devices in classroom as this has both advantages and disadvantages. On the one hand, \textbf{one redeeming feature} is that \textbf{deploying} mobile technology into campus allows students to access the most up-to-date information more quickly and easily than ever before. Therefore, the traditional passive learning model is broken, and a teacher becomes a supporter and coach \textbf{as opposed to} an information provider. However, school regulations should be imposed to constrain the time that students can use mobile devices to \textbf{shun} technology addiction.
    \end{qa}

    \begin{qa}{Do you agree or disagree that computers will replace teachers one day?}
    Fat chance! Personally, computers can be a multi-purpose tool to aid teachers in classroom. One positive aspect is that computers can look for a piece of information more quickly than teachers do, and they do not drop the ball whereas human beings could sometimes be at fault. Having said that, the presence of a teacher in a classroom is still exceptionally important as they can adjust their \textbf{pace} according to children’s capabilities and provide invaluable feedback and advice to their students. In addition, teachers can act as role models for students to follow, partly shaping students’ characters to mould them into useful citizens of the future. This is the aspect an \textbf{automaton} would fail to cover.
    \end{qa}

    \begin{qa}{How much has technology improved how we communicate with each other?}
    For one thing, we can communicate faster and more \textbf{cost-effectively} thanks to the appearance of technological advances such as the Internet or smart phones. As the speed of communicating has \textbf{ramped up}, costs have been dramatically reduced, hence we do not need to \textbf{rack up} a big long-distance phone bill. Moreover, \textbf{information overload} has become a reality, with the Internet providing much more knowledge \textbf{at the push of a button} than could even be imagined in the past. This means there is much more data that can be communicated about any topic than ever before.
    \end{qa}

    \begin{qa}{Do you agree that there are still many more major technological innovations to be made?}
    Remarkably, technological creations have greatly assisted many life aspects, and there will be more \textbf{advances} and breakthroughs for as long as human societies exist. For example, \textbf{driverless cars} like those made by Tesla have grown in popularity recently. In particular, self-driving cars are capable of \textbf{sensing} their environment and moving with little or no \textbf{human input}. However, I might \textbf{feel intimidated} to let machines take the wheel as any errors from the automatic system could cause \textbf{horrific} car crashes.
    \end{qa}

    \begin{qa}{Could you suggest some reasons why some people are deciding to reduce their use of technology?}
    Although technology is advantageous to life, there are grave \textbf{implications} that deter people from using hi-tech devices more often. Firstly, the excessive use of mobile devices has been linked to anxiety and depression because the constant \textbf{bombardment} of information can leave users \textbf{numb} to the real world and \textbf{indifferent to} other relationships. Besides, technological dependence can become so extreme that it causes severe anxiety whenever technology is unavailable.
    \end{qa}

        \begin{VocabExplain}[Part 3]
            \begin{ExplainCard}{computer literate}[adj][B2]
            \EN{Having enough knowledge and skills to use computers effectively.}
            \SY{technologically skilled; computer-competent}
            \VI{thành thạo máy tính.}
            \EX{Most jobs today require employees to be computer literate.}
            \EX{Schools should ensure students are computer literate before graduation.}
            \CO{be computer literate; computer-literate workforce}
            \end{ExplainCard}

            \begin{ExplainCard}{literacy}[n][B2]
            \EN{The ability to read and write.}
            \SY{reading ability; education}
            \VI{khả năng đọc viết.}
            \EX{Improving literacy is a priority for developing countries.}
            \EX{Literacy skills are vital for lifelong learning.}
            \CO{basic literacy; literacy rate}
            \end{ExplainCard}

            \begin{ExplainCard}{numeracy}[n][B2]
            \EN{The ability to do basic mathematics.}
            \SY{mathematical ability; arithmetic skill}
            \VI{khả năng tính toán.}
            \EX{Numeracy is as important as literacy in education.}
            \EX{Good numeracy helps with everyday tasks like budgeting.}
            \CO{basic numeracy; numeracy skills}
            \end{ExplainCard}

            \begin{ExplainCard}{on the fence}[idiom][C1]
            \EN{Undecided or unsure about something.}
            \SY{hesitant; uncertain; undecided}
            \VI{chưa quyết định, lưỡng lự.}
            \EX{She’s on the fence about accepting the new job.}
            \EX{Many voters are still on the fence before the election.}
            \CO{be on the fence about sth}
            \end{ExplainCard}

            \begin{ExplainCard}{one redeeming feature}[phrase][C1]
            \EN{A good aspect of an otherwise poor or negative situation.}
            \SY{positive aspect; saving grace}
            \VI{điểm sáng duy nhất.}
            \EX{The hotel’s one redeeming feature was its friendly staff.}
            \EX{The story’s redeeming feature is its humor.}
            \CO{one redeeming feature of sth}
            \end{ExplainCard}

            \begin{ExplainCard}{deploy}[v][C1]
            \EN{To use something effectively; to position resources strategically.}
            \SY{utilize; employ; implement}
            \VI{triển khai, sử dụng.}
            \EX{The software was deployed across all company computers.}
            \EX{Teachers deploy new methods to engage students.}
            \CO{deploy technology; deploy resources}
            \end{ExplainCard}

            \begin{ExplainCard}{as opposed to}[phrase][C1]
            \EN{Used to show contrast between two things.}
            \SY{instead of; rather than}
            \VI{trái ngược với.}
            \EX{He prefers coffee as opposed to tea.}
            \EX{Online classes are interactive as opposed to traditional lectures.}
            \CO{as opposed to sth}
            \end{ExplainCard}

            \begin{ExplainCard}{shun}[v][C1]
            \EN{To avoid something deliberately.}
            \SY{avoid; steer clear of; eschew}
            \VI{tránh xa, né tránh.}
            \EX{He shuns publicity and prefers a quiet life.}
            \EX{Young people should shun unhealthy habits.}
            \CO{shun responsibility; shun technology}
            \end{ExplainCard}

            \begin{ExplainCard}{pace}[n][C1]
            \EN{The speed at which something happens.}
            \SY{speed; rate; tempo}
            \VI{nhịp độ, tốc độ.}
            \EX{The pace of life in the city is fast.}
            \EX{Technology changes at a rapid pace.}
            \CO{at a fast pace; adjust the pace}
            \end{ExplainCard}

            \begin{ExplainCard}{automaton}[n][C1]
            \EN{A machine that performs tasks automatically; metaphorically, a person acting mechanically.}
            \SY{robot; machine}
            \VI{người/máy tự động.}
            \EX{He worked like an automaton, without emotion.}
            \EX{Automatons are used in manufacturing.}
            \CO{behave like an automaton}
            \end{ExplainCard}

            \begin{ExplainCard}{cost-effective}[adj][C1]
            \EN{Giving good results without costing a lot of money.}
            \SY{economical; efficient}
            \VI{hiệu quả với chi phí thấp.}
            \EX{Online courses are a cost-effective way of learning.}
            \EX{This solution is both eco-friendly and cost-effective.}
            \CO{cost-effective method; cost-effective measure}
            \end{ExplainCard}

            \begin{ExplainCard}{ramp up}[phr.v][C1]
            \EN{To increase or cause to increase in speed, power, or intensity.}
            \SY{increase; boost; accelerate}
            \VI{tăng tốc, đẩy mạnh.}
            \EX{They ramped up production to meet demand.}
            \EX{Marketing efforts ramped up before the launch.}
            \CO{ramp up production; ramp up efforts}
            \end{ExplainCard}

            \begin{ExplainCard}{rack up}[phr.v][C1]
            \EN{To accumulate or achieve something, often a large number.}
            \SY{accumulate; amass; gather}
            \VI{tích lũy, đạt được.}
            \EX{He racked up a lot of debt.}
            \EX{The team racked up five wins in a row.}
            \CO{rack up debt; rack up points}
            \end{ExplainCard}

            \begin{ExplainCard}{information overload}[n][C1]
            \EN{The difficulty of understanding an issue due to too much information.}
            \SY{data flood; excessive information}
            \VI{quá tải thông tin.}
            \EX{Information overload makes it hard to make decisions.}
            \EX{The internet can lead to information overload.}
            \CO{suffer from information overload}
            \end{ExplainCard}

            \begin{ExplainCard}{at the push of a button}[idiom][C1]
            \EN{Something that can be done quickly and easily with very little effort.}
            \SY{instantly; effortlessly}
            \VI{ngay lập tức, rất dễ dàng.}
            \EX{You can shop online at the push of a button.}
            \EX{Modern devices can start at the push of a button.}
            \CO{do sth at the push of a button}
            \end{ExplainCard}

            \begin{ExplainCard}{advance}[n][B2]
            \EN{A new development or improvement in something.}
            \SY{progress; innovation; breakthrough}
            \VI{sự tiến bộ, cải tiến.}
            \EX{Medical advances have saved many lives.}
            \EX{Recent advances in AI are remarkable.}
            \CO{major advance; technological advance}
            \end{ExplainCard}

            \begin{ExplainCard}{driverless car}[n][C1]
            \EN{A car that operates without a human driver, using AI and sensors.}
            \SY{self-driving car; autonomous car}
            \VI{xe tự lái.}
            \EX{Driverless cars are being tested in big cities.}
            \EX{Many believe driverless cars will reduce accidents.}
            \CO{driverless car technology; test driverless cars}
            \end{ExplainCard}

            \begin{ExplainCard}{sense}[v][C1]
            \EN{To detect or become aware of something.}
            \SY{detect; perceive; notice}
            \VI{cảm nhận, phát hiện.}
            \EX{The system can sense movement in the room.}
            \EX{Humans can sense danger instinctively.}
            \CO{sense danger; sense changes}
            \end{ExplainCard}

            \begin{ExplainCard}{human input}[n][C1]
            \EN{Participation or action provided by humans rather than machines.}
            \SY{manual intervention; human involvement}
            \VI{sự can thiệp của con người.}
            \EX{The system requires little human input.}
            \EX{The process is automated, with no human input needed.}
            \CO{require human input; without human input}
            \end{ExplainCard}

            \begin{ExplainCard}{feel intimidated}[phrase][C1]
            \EN{To feel frightened or nervous, often by something powerful or advanced.}
            \SY{be daunted; feel uneasy; be afraid}
            \VI{cảm thấy sợ hãi, e dè.}
            \EX{She felt intimidated on her first day at work.}
            \EX{Many people feel intimidated by advanced technology.}
            \CO{feel intimidated by sth}
            \end{ExplainCard}

            \begin{ExplainCard}{horrific}[adj][C1]
            \EN{Extremely shocking or disturbing.}
            \SY{terrible; dreadful; appalling}
            \VI{kinh hoàng, khủng khiếp.}
            \EX{The accident was absolutely horrific.}
            \EX{Horrific crimes shocked the community.}
            \CO{horrific accident; horrific scene}
            \end{ExplainCard}

            \begin{ExplainCard}{implication}[n][C1]
            \EN{A possible effect or consequence of an action or decision.}
            \SY{consequence; repercussion; outcome}
            \VI{hệ quả, tác động.}
            \EX{The new law has serious implications for businesses.}
            \EX{Technological advances carry ethical implications.}
            \CO{serious implication; political implication}
            \end{ExplainCard}

            \begin{ExplainCard}{bombardment}[n][C1]
            \EN{The continuous flow of information or questions.}
            \SY{onslaught; flood; deluge}
            \VI{sự dồn dập, bủa vây.}
            \EX{The politician faced a bombardment of questions.}
            \EX{Modern users face a bombardment of data daily.}
            \CO{bombardment of information; media bombardment}
            \end{ExplainCard}

            \begin{ExplainCard}{numb}[adj][C1]
            \EN{Unable to feel, think, or react normally.}
            \SY{insensible; unresponsive; apathetic}
            \VI{tê liệt, vô cảm.}
            \EX{He was numb with shock after the news.}
            \EX{Constant exposure to violence made him numb.}
            \CO{feel numb; go numb}
            \end{ExplainCard}

            \begin{ExplainCard}{indifferent to}[phrase][C1]
            \EN{Having no interest or concern about something.}
            \SY{unconcerned about; apathetic towards}
            \VI{thờ ơ với.}
            \EX{She is indifferent to politics.}
            \EX{He seemed indifferent to his own success.}
            \CO{indifferent to pain; indifferent to criticism}
            \end{ExplainCard}
        \end{VocabExplain}

    \begin{VocabHighlights}
        \VH{to fulfill}{(v) to do or achieve what was hoped for or expected}{(động từ) hiện thực hóa, thực hiện}
        \VH{couldn't ask for more}{(phrase) used for saying that something is so good, you cannot imagine anything better}{(cụm từ) cái gì quá tốt, không thể đòi hỏi thêm}
        \VH{to take it one day at a time}{(idiom) to deal with things as they happen, and not to make plans or to worry about the future}{(thành ngữ) xử lý mọi thứ từng bước, không quá lo ngại tương lai}
        \VH{to have a whale of a time}{(idiom) to enjoy oneself}{(thành ngữ) tận hưởng cuộc sống}
        \VH{atheist}{(n) a person who believes that God does not exist}{(danh từ) người vô thần}
        \VH{realistic}{(adj) accepting in a sensible way what it is actually possible to do or achieve in a particular situation}{(tính từ) thực tế}
        \VH{pragmatic}{(adj) solving problems in a practical and sensible way rather than by having fixed ideas or theories}{(tính từ) thực dụng}
        \VH{time flies like an arrow}{(proverb) time flies fast}{(tục ngữ) thời gian trôi quá nhanh}
        \VH{to be cut out for V-ing}{(phr. v) to be suitable for V-ing}{(cụm động từ) phù hợp làm gì}
        \VH{accomplished}{(adj) very good at a particular thing; having a lot of skills}{(tính từ) tương đối giỏi, thành công}
        \VH{self-proclaimed}{(adj) giving yourself a particular title, job, etc. without the agreement or permission of other people}{(tính từ) tự xưng}
        \VH{high-powered job}{(phrase) powerful job}{(cụm từ) công việc giàu quyền lực}
        \VH{in the absence of}{(phrase) without}{(cụm từ) thiếu vắng đi}
        \VH{promotion potential}{(phrase) the highest grade to which a person may be promoted without additional competition for the position}{(cụm từ) triển vọng thăng tiến}
        \VH{to stick to something}{(phr. v) to continue doing or using something and not want to change it}{(cụm động từ) gắn bó}
        \VH{to shoulder}{(v) to accept the responsibility for something}{(động từ) gánh vác (trách nhiệm)}
        \VH{a tech-savvy}{(n) a person having a practical and deep understanding of something}{(danh từ) dân sành công nghệ}
        \VH{to shell out}{(phr. v) pay a lot of money for}{(cụm động từ) trả tiền}
        \VH{a bargain}{(n) something at a lower price}{(danh từ) món hời}
        \VH{to snap up a bargain}{(idiom) grab a deal}{(thành ngữ) chộp ngay cơ hội giảm giá}
        \VH{as its name suggests}{(idiom) according to the meaning of the name}{(thành ngữ) đúng như cái tên đã nói}
        \VH{to play up}{(phr. v) something is in disorder}{(cụm động từ) hỏng}
        \VH{to get to grips with}{(idiom) begin to understand or deal with something}{(thành ngữ) bắt đầu để hiểu hoặc xử lý cái gì đó}
        \VH{a techie}{(n) a person who is good at technology}{(danh từ) dân kỹ thuật}
        \VH{with ease}{(phrase) easily}{(cụm từ) dễ dàng}
        \VH{to boost my work efficiency}{(phrase) improve work performance}{(cụm từ) nâng cao hiệu quả công việc}
        \VH{portable}{(adj) easy to bring along}{(tính từ) dễ cầm theo}
        \VH{without a second thought}{(idiom) without careful thinking}{(thành ngữ) không nghĩ ngợi nhiều}
        \VH{computer literate}{(phrase) having sufficient knowledge and skill to be able to use computers; familiar with the operation of computers}{(cụm từ) thành thạo kỹ năng máy tính}
        \VH{literacy}{(n) the ability to read and write}{(danh từ) khả năng đọc, viết}
        \VH{numeracy}{(n) a good basic knowledge of mathematics; the ability to understand and work with numbers}{(danh từ) khả năng làm việc với các con số}
        \VH{on the fence}{(idiom) not able to decide something}{(thành ngữ) do dự, thiếu quyết đoán}
        \VH{redeeming feature(s)}{(phrases) advantage}{(cụm từ) điểm tốt}
        \VH{to deploy}{(v) to use something effectively}{(động từ) áp dụng, triển khai}
        \VH{as opposed to}{(phrase) instead of}{(cụm từ) thay vì}
        \VH{to shun}{(v) to avoid something}{(động từ) tránh cái gì}
        \VH{pace}{(n) the speed at which someone or something moves, or with which something happens or changes}{(danh từ) tốc độ}
        \VH{automaton}{(n) a small robot that can perform a particular range of functions}{(danh từ) máy tự động, robot có thể thay thế con người làm một số việc}
        \VH{cost-effectively}{(adj) effective or productive in relation to its cost}{(tính từ) hiệu quả chi phí}
        \VH{to ramp up}{(phr. v) to make something increase in amount}{(cụm động từ) tăng số lượng cái gì đó}
        \VH{to rack up}{(phr. v) to collect something, such as profits or losses in a business, or points in a competition}{(cụm động từ) tăng thêm chi phí}
        \VH{information overload}{(phrase) exposure to or provision of too much information or data}{(cụm từ) quá tải thông tin}
        \VH{at the push of a button}{(idiom) very easily}{(thành ngữ) dễ dàng}
        \VH{driverless car}{(n) a vehicle that can guide itself without human conduction}{(danh từ) xe tự lái}
        \VH{to sense}{(v) to become aware of something even though you cannot see it, hear it, etc.}{(động từ) cảm nhận}
        \VH{intimidated}{(adj) frightened}{(tính từ) lo sợ}
        \VH{implication}{(n) the effect that an action or decision will have on something else in the future}{(danh từ) hệ quả, kết quả}
        \VH{bombardment}{(n) a situation in which so many questions or other things are directed at someone, that they find it difficult to deal with them}{(danh từ) sự dồn dập}
        \VH{numb}{(adj) unable to feel anything}{(tính từ) không quan tâm, không cảm thấy gì}
        \VH{indifferent}{(adj) having or showing no interest in somebody/something}{(tính từ) thờ ơ; lãnh đạm}
    \end{VocabHighlights}
    \end{test}

    \begin{test}{TEST 3}
    \noindent
    \textbf{Part 1. Money}
    \begin{qa}{When you go shopping, do you prefer to pay for things in cash or by card? [Why?]}
    It depends on each type of shops I pay a visit to. If I go to department stores and malls, a debit or \textbf{credit} card is ideal for fear of the fact that a big sum of my cash might be lost on the way. However, if I \textbf{run errands}, my pockets are often filled with cash because some shops might not have a POS (Point of Sale) available.
    \end{qa}

    \begin{qa}{Do you ever save money to buy special things? [Why/Why not?]}
    Yes, I do. The feeling of accumulating enough money to purchase something special is fantastic. 8 years ago, I \textbf{picked up} my first smartphone, Samsung Galaxy S2, a \textbf{flagship} at that time, for 13.5 million VND (\$600) to replace my broken one. The money used on this one had been accrued over 4 years of my college. I myself engaged in a lot of activities during college to \textbf{make ends meet} and own a smartphone, which was still considered a luxurious item by then.
    \end{qa}

    \begin{qa}{Would you ever take a job which had low pay? [Why/Why not?]}
    It depends. If that job had promotion potential or helped me gain valuable soft skills, I wouldn’t mind doing one. For example, I wouldn’t hesitate to ask any foreigners I \textbf{bump into} in the streets to join a walking tour around Hanoi’s particular places of interests like Sword Lake, etc. I will assure them that they are not compelled to pay anything at all. If they feel my service is worthwhile, they can pay me anything \textbf{at their discretion}. This job will surely \textbf{hone} my communication skills in English.
    \end{qa}

    \begin{qa}{Would winning a lot of money make a big difference to your life? [Why/Why not?]}
    I don’t think my life would be altered in anyway. When I was little, my parents \textbf{instilled a sense of thrift} in my mind so I have never spent money \textbf{lavishly} on \textbf{extravagances} so far. Instead, I would still use this big sum of money on running the family and this would be a lesson for my \textbf{children} in terms of using money sensibly.
    \end{qa}

        \begin{VocabExplain}[Part 1]
            \begin{ExplainCard}{credit card}[n][B1]
            \EN{A small plastic card issued by a bank, allowing the holder to purchase goods or services on credit.}
            \SY{charge card; debit card (contrast)}
            \VI{thẻ tín dụng.}
            \EX{She paid for the flight with her credit card.}
            \EX{Credit cards allow convenient payments worldwide.}
            \CO{pay by credit card; credit card balance}
            \end{ExplainCard}

            \begin{ExplainCard}{run errands}[phrase][B2]
            \EN{To make short trips to do various small tasks.}
            \SY{do chores; perform tasks}
            \VI{chạy việc vặt.}
            \EX{I spent the morning running errands for my mom.}
            \EX{Running errands takes up a lot of her free time.}
            \CO{run errands for sb; spend time running errands}
            \end{ExplainCard}

            \begin{ExplainCard}{pick up}[phr.v][B2]
            \EN{To buy or obtain something casually or unexpectedly.}
            \SY{buy; get; purchase}
            \VI{mua, tậu.}
            \EX{I picked up a new book on my way home.}
            \EX{He picked up a car at a great price.}
            \CO{pick up a bargain; pick up an item}
            \end{ExplainCard}

            \begin{ExplainCard}{flagship}[n][C1]
            \EN{The best or most important product, building, etc. that a company owns or produces.}
            \SY{top product; leading model}
            \VI{sản phẩm chủ lực, hàng đầu.}
            \EX{The iPhone is Apple’s flagship product.}
            \EX{This store is the company’s flagship in Asia.}
            \CO{flagship model; flagship store}
            \end{ExplainCard}

            \begin{ExplainCard}{make ends meet}[idiom][B2]
            \EN{To have just enough money to pay for the things you need.}
            \SY{get by; survive financially}
            \VI{đủ tiền trang trải cuộc sống.}
            \EX{After losing his job, he struggled to make ends meet.}
            \EX{Students often work part-time jobs to make ends meet.}
            \CO{struggle to make ends meet}
            \end{ExplainCard}

            \begin{ExplainCard}{bump into}[phr.v][B2]
            \EN{To meet someone unexpectedly.}
            \SY{run into; come across}
            \VI{tình cờ gặp.}
            \EX{I bumped into an old friend at the market.}
            \EX{Tourists might bump into celebrities in LA.}
            \CO{bump into sb}
            \end{ExplainCard}

            \begin{ExplainCard}{at one’s discretion}[phrase][C1]
            \EN{According to one’s own judgment or choice.}
            \SY{freely; optionally; voluntarily}
            \VI{tùy ý quyết định.}
            \EX{You may leave the class at your discretion.}
            \EX{The judge may impose fines at his discretion.}
            \CO{at the discretion of sb; leave to sb’s discretion}
            \end{ExplainCard}

            \begin{ExplainCard}{hone}[v][C1]
            \EN{To improve or sharpen a skill through practice.}
            \SY{sharpen; develop; perfect}
            \VI{trau dồi, mài giũa.}
            \EX{He honed his public speaking skills.}
            \EX{Training programs help hone leadership ability.}
            \CO{hone a skill; hone one’s ability}
            \end{ExplainCard}

            \begin{ExplainCard}{instill a sense of thrift}[phrase][C1]
            \EN{To gradually teach someone to save money and avoid waste.}
            \SY{inculcate frugality; teach thriftiness}
            \VI{gieo vào ý thức tiết kiệm.}
            \EX{Parents should instill a sense of thrift in their children.}
            \EX{My grandparents instilled thrift and hard work in us.}
            \CO{instill a sense of thrift; instill values}
            \end{ExplainCard}

            \begin{ExplainCard}{lavishly}[adv][C1]
            \EN{In a rich, elaborate, or luxurious way.}
            \SY{extravagantly; luxuriously; opulently}
            \VI{xa hoa, phung phí.}
            \EX{They lived lavishly in a mansion by the sea.}
            \EX{The party was lavishly decorated.}
            \CO{lavishly spend; lavishly decorated}
            \end{ExplainCard}

            \begin{ExplainCard}{extravagance}[n][C1]
            \EN{Something expensive or unnecessary that is bought for pleasure.}
            \SY{luxury; indulgence; wastefulness}
            \VI{sự phung phí, xa hoa.}
            \EX{Buying that car was an extravagance he could not afford.}
            \EX{Small extravagances make life enjoyable.}
            \CO{an extravagance; avoid extravagance}
            \end{ExplainCard}

            \begin{ExplainCard}{children}[n][A1]
            \EN{Young human beings below the age of puberty.}
            \SY{kids; youngsters}
            \VI{trẻ em, con cái.}
            \EX{Children need love and care.}
            \EX{Parents often worry about their children’s future.}
            \CO{raise children; care for children}
            \end{ExplainCard}
        \end{VocabExplain}

    \noindent
    \textbf{Part 2.}
    \begin{qa}{Describe an interesting discussion you had as part of your work or studies. You should say:
    \begin{itemize}
        \item What the subject of the discussion was
        \item Who you discussed the subject with
        \item What opinions were expressed
        \item and explain why you found the discussion interesting.
    \end{itemize}}

    \textbf{In all honesty}, I am working as a teacher at Hanoi – Amsterdam High School for the Gifted, my \textbf{alma mater}. During an academic year, there are several meetings I need to attend to discuss with my colleagues about any work-related subjects. \textbf{To the best of my recollection}, the one that has a \textbf{vivid} impression on me is the meeting between the teachers and the club leaders at my school at the beginning of the school year. My school has a wide variety of \textbf{in-school} clubs, \textbf{ranging from} academic ones such as Ams Advisor, which asks \textbf{academically high achievers} to tutor weaker students to \textbf{arts-related} clubs like Glee Ams and HAT, \textbf{standing for} Hanoi – Amsterdam Arts Team which specializes in delivering musical performances on the stage. 

    However, to be responsible for a club, a teacher is \textbf{under compulsion to} get \textbf{tenure} so the number of eligible ones at my school is rather limited. \textbf{In that sense}, a teacher frequently \textbf{shoulders} the responsibilities of 2 or 3 clubs so it is \textbf{of paramount importance} to set up a meeting to have the students oriented towards organizing activities in a new school year. In the last meeting, I discussed with not only the vice principals but also the club leaders about what to do to \textbf{professionalizing} the activities. For example, I asked the club representatives to notify the teachers in charge of their club of any incoming events \textbf{pronto} instead of keeping my colleagues posted \textbf{at the eleventh hour}. 

    I find the discussion \textbf{intriguing} because it partly \textbf{bridges the gap} between teachers and students who are free to raise their voice. The outcome of the meeting was that, the teachers would need to lend a sympathetic ear to the students with a view to organizing activities more effectively, making students’ life an unforgettable one. In return, the club leaders also found it necessary to cooperate with teachers to \textbf{smooth away} the differences for the sake of the club.
    \end{qa}

        \begin{VocabExplain}[Part 2]
            \begin{ExplainCard}{in all honesty}[phrase][C1]
            \EN{Used to emphasize that one is telling the truth or being very sincere.}
            \SY{frankly; truthfully; sincerely}
            \VI{thành thật mà nói.}
            \EX{In all honesty, I don’t think this plan will work.}
            \EX{She admitted, in all honesty, that she was nervous.}
            \CO{in all honesty, + clause}
            \end{ExplainCard}

            \begin{ExplainCard}{alma mater}[n][C1]
            \EN{The school, college, or university that someone attended.}
            \SY{former school; university; institution}
            \VI{trường cũ, trường đã học.}
            \EX{He returned to his alma mater to give a speech.}
            \EX{Cambridge University is her alma mater.}
            \CO{return to one’s alma mater}
            \end{ExplainCard}

            \begin{ExplainCard}{to the best of my recollection}[phrase][C1]
            \EN{As far as I can remember.}
            \SY{as far as I recall; from memory}
            \VI{theo như tôi nhớ.}
            \EX{To the best of my recollection, we met in 2010.}
            \EX{To the best of my recollection, she never visited us again.}
            \CO{to the best of my recollection, + clause}
            \end{ExplainCard}

            \begin{ExplainCard}{vivid}[adj][C1]
            \EN{Producing strong, clear images or memories in the mind.}
            \SY{clear; graphic; distinct}
            \VI{sống động, rõ ràng.}
            \EX{She has vivid memories of her childhood.}
            \EX{The description was so vivid that I could imagine it clearly.}
            \CO{vivid memory; vivid impression}
            \end{ExplainCard}

            \begin{ExplainCard}{in-school}[adj][B2]
            \EN{Happening within a school or related to school activities.}
            \SY{school-based; on-campus}
            \VI{trong trường học.}
            \EX{In-school programs help students build soft skills.}
            \EX{The in-school competition will be held next month.}
            \CO{in-school activities; in-school clubs}
            \end{ExplainCard}

            \begin{ExplainCard}{range from}[phr.v][B2]
            \EN{To vary between two limits.}
            \SY{stretch from; extend from}
            \VI{dao động từ, trải dài từ.}
            \EX{Prices range from \$50 to \$200.}
            \EX{The workshops range from dance to photography.}
            \CO{range from A to B}
            \end{ExplainCard}

            \begin{ExplainCard}{academically high achiever}[n][C1]
            \EN{A student who performs exceptionally well in academic studies.}
            \SY{excellent student; top performer}
            \VI{học sinh xuất sắc trong học tập.}
            \EX{Academically high achievers are offered scholarships.}
            \EX{He was recognized as an academically high achiever.}
            \CO{academically high achiever student}
            \end{ExplainCard}

            \begin{ExplainCard}{arts-related}[adj][B2]
            \EN{Connected with the arts such as music, theatre, or painting.}
            \SY{artistic; cultural}
            \VI{liên quan đến nghệ thuật.}
            \EX{The school offers arts-related courses.}
            \EX{Arts-related events are popular among students.}
            \CO{arts-related subject; arts-related club}
            \end{ExplainCard}

            \begin{ExplainCard}{standing for}[phrase][B2]
            \EN{To represent or mean something, usually by abbreviation.}
            \SY{representing; symbolizing}
            \VI{viết tắt, tượng trưng cho.}
            \EX{UN stands for United Nations.}
            \EX{HAT standing for Hanoi Arts Team.}
            \CO{stand for sth}
            \end{ExplainCard}

            \begin{ExplainCard}{under compulsion to}[phrase][C1]
            \EN{Being forced or required to do something.}
            \SY{obliged to; required to}
            \VI{bị bắt buộc, bị ép buộc.}
            \EX{He signed the contract under compulsion.}
            \EX{Teachers are under compulsion to meet targets.}
            \CO{under compulsion to do sth}
            \end{ExplainCard}

            \begin{ExplainCard}{tenure}[n][C1]
            \EN{The period of time a person holds a position or job, especially with security.}
            \SY{term; incumbency}
            \VI{nhiệm kỳ, thời gian giữ chức vụ.}
            \EX{He was granted tenure as a professor.}
            \EX{Her tenure as director lasted five years.}
            \CO{academic tenure; job tenure}
            \end{ExplainCard}

            \begin{ExplainCard}{in that sense}[phrase][C1]
            \EN{From that point of view; considering it in that way.}
            \SY{in that respect; in that regard}
            \VI{theo nghĩa đó, ở khía cạnh đó.}
            \EX{In that sense, he was correct.}
            \EX{It is useful, in that sense, to review the basics.}
            \CO{in that sense, + clause}
            \end{ExplainCard}

            \begin{ExplainCard}{shoulder}[v][C1]
            \EN{To accept or take responsibility for something.}
            \SY{bear; take on; assume}
            \VI{gánh vác, chịu trách nhiệm.}
            \EX{He shouldered the responsibility of leading the team.}
            \EX{Parents often shoulder financial burdens.}
            \CO{shoulder responsibility; shoulder duties}
            \end{ExplainCard}

            \begin{ExplainCard}{of paramount importance}[phrase][C1]
            \EN{More important than anything else; supreme importance.}
            \SY{vital; crucial; essential}
            \VI{cực kỳ quan trọng.}
            \EX{Safety is of paramount importance.}
            \EX{Education is of paramount importance for the future.}
            \CO{be of paramount importance}
            \end{ExplainCard}

            \begin{ExplainCard}{professionalize}[v][C1]
            \EN{To make an activity or job more professional in standards or methods.}
            \SY{standardize; systematize}
            \VI{chuyên nghiệp hóa.}
            \EX{They aim to professionalize the sports industry.}
            \EX{The club seeks to professionalize its management.}
            \CO{professionalize activities; professionalize a field}
            \end{ExplainCard}

            \begin{ExplainCard}{pronto}[adv][C1]
            \EN{Quickly, without delay.}
            \SY{immediately; promptly; at once}
            \VI{ngay lập tức.}
            \EX{Come here pronto!}
            \EX{They need to fix the issue pronto.}
            \CO{do sth pronto}
            \end{ExplainCard}

            \begin{ExplainCard}{at the eleventh hour}[idiom][C1]
            \EN{At the last possible moment.}
            \SY{last minute; just in time}
            \VI{vào phút chót.}
            \EX{The contract was signed at the eleventh hour.}
            \EX{He always finishes his work at the eleventh hour.}
            \CO{arrive at the eleventh hour; sign at the eleventh hour}
            \end{ExplainCard}

            \begin{ExplainCard}{intriguing}[adj][C1]
            \EN{Very interesting because it is unusual or mysterious.}
            \SY{fascinating; captivating; engaging}
            \VI{hấp dẫn, thú vị.}
            \EX{This is an intriguing idea.}
            \EX{The plot of the film was highly intriguing.}
            \CO{find sth intriguing; an intriguing story}
            \end{ExplainCard}

            \begin{ExplainCard}{bridge the gap}[phrase][C1]
            \EN{To reduce differences between groups or people.}
            \SY{reconcile; connect; link}
            \VI{thu hẹp khoảng cách.}
            \EX{The new policy aims to bridge the gap between rich and poor.}
            \EX{Education helps bridge the gap between generations.}
            \CO{bridge the gap between A and B}
            \end{ExplainCard}

            \begin{ExplainCard}{smooth away}[phr.v][C1]
            \EN{To remove or reduce difficulties or differences.}
            \SY{resolve; iron out; settle}
            \VI{xóa bỏ, làm dịu đi.}
            \EX{He tried to smooth away the conflict.}
            \EX{Diplomatic talks smoothed away their differences.}
            \CO{smooth away difficulties; smooth away conflicts}
            \end{ExplainCard}
        \end{VocabExplain}

    \noindent
    \textbf{Part 3.}
    \begin{qa}{Why is it good to discuss problems with other people?}
    In my opinion, it is \textbf{sensible} to share my problems or \textbf{misery} with those who I can \textbf{take into my confidence}, not all people. Not wanting to look bad in the eyes of the person I admire may keep me from sharing what is on my mind. The truth is that if someone \textbf{dotes on} me, he or she will help me deal with my dilemmas. Talking about it can help to \textbf{shed light on} how to get through a problem and that is also how this therapy works.
    \end{qa}

    \begin{qa}{Do you think that it’s better to talk to friends and not family about problems?}
    Actually, the payoff we get for sharing our feelings with others almost entirely on the quality of the response we get. If the other person listens well, shows empathy, and \textbf{validates} our feelings, we are likely to feel much better. But they just sit there while we \textbf{spill our guts} and their only response is to \textbf{mumble}, which would be terrible.
    \end{qa}

    \begin{qa}{Is it always a good idea to tell lots of people about a problem?}
    Well, I don’t think it is a good idea to \textbf{blurt out} their problems. For the most part, people usually \textbf{crave} for connection and sharing as a way of relaxation. The point is not all people can really comprehend the problems somebody got, and maybe \textbf{make fun of} his or her story. That is why people should not be too \textbf{naive} to open up to many people; otherwise, they might get hurt.
    \end{qa}

    \begin{qa}{Which communication skills are most important when taking part in meetings with colleagues?}
    Obviously, there is an eclectic mixture of communication skills that can help people in meetings in their company. Speaking with \textbf{discretion} is the top habit that every single person should possess because it can prevent any misunderstandings and help to build trust and openness. Also, offering constructive criticism is essential as it would not harm colleagueship. If someone did a great job, the boss could offer positive \textbf{reinforcement} and also give him improvement tips without being mean or \textbf{bossy}.
    \end{qa}

    \begin{qa}{What are the possible effects of poor written communication skills at work?}
    It is true that poor writing skills can impair one’s business \textbf{to some degree}. Employees who can’t clearly express themselves are unlikely to get ahead as their poor skills often becomes an \textbf{obstacle}. Additionally, this may cause revenue loss because unclear or badly written marketing materials could make potential customers take their businesses elsewhere.
    \end{qa}

    \begin{qa}{What do you think will be the future impact of technology on communication in the workplace?}
    There is no denying that technology has had an enormous impact on business communication. Before the advent of the cell phone, there was \textbf{virtually} no way to find someone who was \textbf{incommunicado}. Nowadays, with the availability of smart phones and the Internet \textbf{at will}, the way people communicate has been levelled up dramatically. In the near future, I guess the space communication is the next destination for researchers to come up with more breakthroughs.
    \end{qa}

        \begin{VocabExplain}[Part 3]
            \begin{ExplainCard}{sensible}[adj][B2]
            \EN{Having or showing good sense or judgment.}
            \SY{reasonable; practical; rational}
            \VI{hợp lý, khôn ngoan.}
            \EX{It’s sensible to carry an umbrella on rainy days.}
            \EX{A sensible decision can save time and money.}
            \CO{a sensible idea; be sensible about sth}
            \end{ExplainCard}

            \begin{ExplainCard}{misery}[n][B2]
            \EN{A state of great suffering or unhappiness.}
            \SY{distress; suffering; hardship}
            \VI{nỗi khổ, sự đau đớn.}
            \EX{He lived in misery after losing his job.}
            \EX{The war brought misery to millions of people.}
            \CO{live in misery; bring misery to}
            \end{ExplainCard}

            \begin{ExplainCard}{take into one’s confidence}[phrase][C1]
            \EN{To share secrets or private matters with someone you trust.}
            \SY{confide in; trust with secrets}
            \VI{tâm sự, chia sẻ bí mật.}
            \EX{She took me into her confidence about her plans.}
            \EX{Only a few close friends were taken into his confidence.}
            \CO{take sb into your confidence}
            \end{ExplainCard}

            \begin{ExplainCard}{dote on}[phr.v][C1]
            \EN{To show a lot of love and attention to someone.}
            \SY{adore; cherish; lavish attention on}
            \VI{yêu thương, cưng chiều.}
            \EX{Grandparents often dote on their grandchildren.}
            \EX{She doted on her pet cat.}
            \CO{dote on sb}
            \end{ExplainCard}

            \begin{ExplainCard}{shed light on}[idiom][C1]
            \EN{To make something clearer or easier to understand.}
            \SY{clarify; explain; illuminate}
            \VI{làm sáng tỏ.}
            \EX{The research sheds light on how the brain works.}
            \EX{Her testimony shed light on the case.}
            \CO{shed light on a problem/issue}
            \end{ExplainCard}

            \begin{ExplainCard}{validate}[v][C1]
            \EN{To confirm or recognize the value or truth of something.}
            \SY{confirm; affirm; acknowledge}
            \VI{công nhận, xác nhận.}
            \EX{The study validates his theory.}
            \EX{Teachers should validate students’ efforts.}
            \CO{validate a feeling; validate results}
            \end{ExplainCard}

            \begin{ExplainCard}{spill one’s guts}[idiom][C1]
            \EN{To reveal everything, especially personal or private matters.}
            \SY{confess; open up; disclose}
            \VI{thổ lộ hết, nói hết ra.}
            \EX{He spilled his guts about what happened.}
            \EX{Under pressure, she spilled her guts to the police.}
            \CO{spill your guts to sb}
            \end{ExplainCard}

            \begin{ExplainCard}{mumble}[v][B2]
            \EN{To speak quietly and not clearly enough for others to understand.}
            \SY{mutter; whisper; murmur}
            \VI{lẩm bẩm, nói lí nhí.}
            \EX{He mumbled something under his breath.}
            \EX{The students mumbled answers to the teacher.}
            \CO{mumble to oneself; mumble an apology}
            \end{ExplainCard}

            \begin{ExplainCard}{blurt out}[phr.v][C1]
            \EN{To say something suddenly without thinking.}
            \SY{exclaim; blab; utter}
            \VI{thốt ra, buột miệng nói.}
            \EX{She blurted out the secret without realizing it.}
            \EX{He blurted out the answer in class.}
            \CO{blurt out a secret; blurt out words}
            \end{ExplainCard}

            \begin{ExplainCard}{crave}[v][C1]
            \EN{To have a strong desire for something.}
            \SY{desire; long for; yearn}
            \VI{khao khát, thèm muốn.}
            \EX{She craves chocolate after dinner.}
            \EX{Humans crave social interaction.}
            \CO{crave attention; crave success}
            \end{ExplainCard}

            \begin{ExplainCard}{make fun of}[phrase][B2]
            \EN{To laugh at or mock someone.}
            \SY{ridicule; tease; mock}
            \VI{chế giễu, cười nhạo.}
            \EX{The kids made fun of his accent.}
            \EX{Don’t make fun of people’s mistakes.}
            \CO{make fun of sb}
            \end{ExplainCard}

            \begin{ExplainCard}{naive}[adj][C1]
            \EN{Showing a lack of experience, wisdom, or judgment.}
            \SY{innocent; inexperienced; gullible}
            \VI{ngây thơ, cả tin.}
            \EX{She was naive to believe him.}
            \EX{Naive investors lost their money.}
            \CO{be naive about sth; politically naive}
            \end{ExplainCard}

            \begin{ExplainCard}{discretion}[n][C1]
            \EN{The ability to decide what should be done in a particular situation.}
            \SY{judgment; prudence; caution}
            \VI{sự thận trọng, toàn quyền quyết định.}
            \EX{Use discretion when handling sensitive information.}
            \EX{The manager has discretion over hiring decisions.}
            \CO{exercise discretion; at one’s discretion}
            \end{ExplainCard}

            \begin{ExplainCard}{reinforcement}[n][C1]
            \EN{The act of strengthening or encouraging a behavior or idea.}
            \SY{strengthening; encouragement; support}
            \VI{sự củng cố, sự khích lệ.}
            \EX{Praise acts as positive reinforcement for good behavior.}
            \EX{Military reinforcement arrived in the city.}
            \CO{positive reinforcement; reinforcement learning}
            \end{ExplainCard}

            \begin{ExplainCard}{bossy}[adj][B2]
            \EN{Always telling people what to do in a way that annoys them.}
            \SY{domineering; overbearing; controlling}
            \VI{hách dịch, hay ra lệnh.}
            \EX{She’s so bossy, always ordering people around.}
            \EX{A bossy attitude can create conflicts.}
            \CO{a bossy person; sound bossy}
            \end{ExplainCard}

            \begin{ExplainCard}{to some degree}[phrase][B2]
            \EN{Partly, not completely.}
            \SY{partially; somewhat; in part}
            \VI{ở một mức độ nào đó.}
            \EX{His success was due to some degree of luck.}
            \EX{To some degree, I agree with you.}
            \CO{to some degree, + clause}
            \end{ExplainCard}

            \begin{ExplainCard}{obstacle}[n][B2]
            \EN{Something that makes it difficult to do something.}
            \SY{barrier; hindrance; difficulty}
            \VI{trở ngại, chướng ngại.}
            \EX{Lack of money is a major obstacle to success.}
            \EX{She overcame many obstacles to achieve her goals.}
            \CO{face an obstacle; overcome obstacles}
            \end{ExplainCard}

            \begin{ExplainCard}{virtually}[adv][B2]
            \EN{Almost or nearly true; practically.}
            \SY{nearly; almost; practically}
            \VI{hầu như, gần như.}
            \EX{The project is virtually complete.}
            \EX{There’s virtually no difference between the two products.}
            \CO{virtually impossible; virtually identical}
            \end{ExplainCard}

            \begin{ExplainCard}{incommunicado}[adj][C1]
            \EN{Not able, wanting, or allowed to communicate with others.}
            \SY{isolated; cut off; unreachable}
            \VI{bị cô lập, không liên lạc được.}
            \EX{The prisoner was held incommunicado.}
            \EX{He went incommunicado for weeks on his trip.}
            \CO{be incommunicado; held incommunicado}
            \end{ExplainCard}

            \begin{ExplainCard}{at will}[phrase][C1]
            \EN{Whenever or as often as one likes.}
            \SY{freely; as one wishes; voluntarily}
            \VI{tùy ý, bất cứ khi nào muốn.}
            \EX{He can leave the job at will.}
            \EX{You may use the gym at will.}
            \CO{resign at will; access at will}
            \end{ExplainCard}
        \end{VocabExplain}

    \begin{VocabHighlights}
        \VH{to run errands}{(idiom) to go out to buy or do something}{(thành ngữ) ra ngoài mua đồ, đi chợ}
        \VH{to pick up something for}{(phrasal verb) to buy something at a price of}{(cụm động từ) mua gì giá bao nhiêu}
        \VH{flagship}{(noun) the most important product, service, building, etc. that an organization owns or produces}{(danh từ) sản phẩm mũi nhọn, quan trọng nhất mà một doanh nghiệp làm ra}
        \VH{to make ends meet}{(idiom) earn enough money to live without getting into debt}{(thành ngữ) kiếm đủ tiền trang trải}
        \VH{at somebody's discretion}{(idiom) done if, how, when, etc., someone chooses to do it}{(thành ngữ) tùy ý}
        \VH{to hone}{(v) to develop and improve something, especially a skill, over a period of time}{(động từ) mài giũa}
        \VH{to instill}{(v) to gradually make somebody feel, think or behave in a particular way over a period of time}{(động từ) làm thấm nhuần}
        \VH{a sense of thrift}{(phrase) the habit of saving money and spending it carefully so that none is wasted}{(cụm từ) cảm giác/khái niệm tiết kiệm}
        \VH{lavishly}{(adv) in a way that is impressive and usually costs a lot of money}{(trạng từ) tốn kém}
        \VH{extravagances}{(n) something that you buy although it costs a lot of money, perhaps more than you can afford or than is necessary}{(danh từ) thứ tốn kém, phù phiếm}
        \VH{in all honesty}{(idiom) to be honest}{(thành ngữ) thành thật mà nói}
        \VH{alma mater}{(idiom) the school, college, or university that one once attended}{(thành ngữ) trường cũ}
        \VH{to the best of my recollection}{(phrase) according to one's genuine knowledge or opinion}{(cụm từ) theo trí nhớ của tôi}
        \VH{vivid}{(adj) (remember) clearly}{(tính từ) nhớ rõ ràng}
        \VH{academically high achievers}{(phrase) people who achieve high scores}{(cụm từ) người học giỏi, đạt thành tích cao}
        \VH{under compulsion}{(phrase) be under pressure}{(cụm từ) bị áp lực làm gì đó}
        \VH{get tenure}{(phrase) the right to stay permanently in a job}{(cụm từ) ổn định trong công việc}
        \VH{to shoulder the responsibilities of}{(phrase) be responsible for}{(cụm từ) chịu trách nhiệm}
        \VH{to be of paramount importance}{(phrase) be very important}{(cụm từ) rất quan trọng}
        \VH{to professionalize}{(v) to make an activity more professional}{(động từ) chuyên nghiệp hóa}
        \VH{pronto}{(adv) quickly, immediately}{(trạng từ) ngay lập tức, nhanh chóng}
        \VH{at the eleventh hour}{(idiom) in the last moment}{(thành ngữ) vào phút chót}
        \VH{intriguing}{(adj) interesting}{(tính từ) thú vị}
        \VH{to bridge the gap}{(phrase) to connect two things or make the difference smaller}{(cụm từ) xóa bỏ khoảng cách}
        \VH{to smooth away}{(phrasal verb) to make problems or difficulties disappear}{(cụm động từ) xóa bỏ khó khăn, làm mọi việc suôn sẻ}
        \VH{sensible}{(adj) able to make good judgements based on reason and experience rather than emotion}{(tính từ) hợp lý}
        \VH{misery}{(n) great suffering of the mind or body}{(danh từ) khốn khổ}
        \VH{to take somebody into somebody’s confidence}{(phrase) to tell something secret or personal to someone you trust}{(cụm từ) kể bí mật cho}
        \VH{to shed light on}{(phrase) help to explain something by providing further information}{(cụm từ) giải thích}
        \VH{to validate}{(v) to prove that something is true}{(động từ) trân trọng}
        \VH{to spill our guts}{(idiom) to tell someone all about yourself, especially your problems}{(thành ngữ) trút bầu tâm sự}
        \VH{to mumble}{(v) to speak in a quiet voice in a way that is not clear}{(động từ) nói lẩm bẩm}
        \VH{to blurt out}{(phrasal verb) to say something suddenly and without thinking carefully}{(cụm động từ) buột miệng nói ra}
        \VH{to crave for}{(v) have a very strong desire for something}{(động từ) khao khát điều gì}
        \VH{make fun of}{(phrase) to be unkind and laugh at someone}{(cụm từ) làm trò đùa}
        \VH{naive}{(adj) showing a lack of experience or judgment}{(tính từ) ngây thơ}
        \VH{discretion}{(n) the quality of behaving or speaking in such a way as to avoid causing offense or revealing private information}{(danh từ) nói năng nhỏ nhẹ}
        \VH{reinforcement}{(n) the act of making something stronger, especially a feeling or an idea}{(danh từ) sự củng cố}
        \VH{bossy}{(adj) always telling people what to do}{(tính từ) hách dịch, thích chỉ huy}
        \VH{to some degree}{(idiom) to some extent}{(thành ngữ) ở một chừng mực nào đó}
        \VH{obstacle}{(n) a situation or event that makes it difficult to do or achieve something}{(danh từ) trở ngại}
        \VH{virtually}{(adv) almost or very nearly, so that any slight difference is not important}{(trạng từ) hầu như}
        \VH{incommunicado}{(adj) without communicating with other people, either by choice or restriction}{(tính từ) không thể giao tiếp với người khác}
        \VH{at will}{(phrase) available for use as you prefer or somebody prefers}{(cụm từ) có sẵn, tùy ý sử dụng}
    \end{VocabHighlights}
    \end{test}

    \begin{test}{TEST 4}
    \noindent
    \textbf{Part 1. Animals}
    \begin{qa}{Are there many animals or birds where you live? [Why/Why not?]}
    No, there aren’t. I can only spot some cats and dogs but they are quite limited in my block. Birds \textbf{held captive} in their cages as pets are also rare in my neighborhood.
    \end{qa}

    \begin{qa}{How often do you watch programmes or read articles about wild animals? [Why?]}
    By subscribing to Discovery channel, I may observe wild animals’ lives via the lens of the cameramen and the explorers who \textbf{have the guts to} explore nature. I rarely read articles on wild animals as these are \textbf{often ignored} in the mass media. However, the last one I remembered reading was about the extinction of rhinoceros in Vietnamese forests. It seems that not until one species dies out does everyone respect their presence.
    \end{qa}

    \begin{qa}{Have you ever been to a zoo or a wildlife park? [Why/Why not?]}
    Yes, I have. However, I haven’t been to a zoo \textbf{from way back}. I last took my younger brother to the zoo when he was in \textbf{primary} school. Now he is a \textbf{sophomore}. I believe Thu Le Zoo is more suitable for kids as there are not many interesting activities to \textbf{spark} adults’ interests.
    \end{qa}

    \begin{qa}{Would you like to have a job working with animals? [Why/Why not?]}
    I don’t think so. Animals’ behaviors are generally \textbf{erratic} so I \textbf{stand a chance} of being hurt after working with them. 

    Besides, one must have a strong affection for animals, which I do lack. I do not mind cuddling pets like cats and dogs when arriving at my friends’ houses but working with them on a daily basis would be possible only if pigs could fly.
    \end{qa}

        \begin{VocabExplain}[Part 1]
            \begin{ExplainCard}{held captive}[phrase][C1]
            \EN{Kept in a confined space and not free to leave.}
            \SY{imprisoned; confined; detained}
            \VI{bị giam giữ, nuôi nhốt.}
            \EX{The animals were held captive in the zoo.}
            \EX{He was held captive for three months.}
            \CO{hold captive; kept captive}
            \end{ExplainCard}

            \begin{ExplainCard}{have the guts to}[idiom][C1]
            \EN{To have the courage to do something difficult or risky.}
            \SY{dare to; be brave enough to}
            \VI{có gan, dám làm gì.}
            \EX{She had the guts to speak out against injustice.}
            \EX{He didn’t have the guts to ask her out.}
            \CO{have the guts to do sth}
            \end{ExplainCard}

            \begin{ExplainCard}{often ignored}[phrase][B2]
            \EN{Frequently neglected or disregarded.}
            \SY{overlooked; disregarded; neglected}
            \VI{thường bị bỏ qua.}
            \EX{The issue of mental health is often ignored.}
            \EX{Small details are often ignored in big projects.}
            \CO{often ignored by; often ignored issue}
            \end{ExplainCard}

            \begin{ExplainCard}{from way back}[idiom][C1]
            \EN{Since a long time ago.}
            \SY{long ago; ages ago}
            \VI{từ rất lâu rồi.}
            \EX{We’ve been friends from way back.}
            \EX{He knew her from way back in college.}
            \CO{from way back in time; friends from way back}
            \end{ExplainCard}

            \begin{ExplainCard}{primary}[adj][A2]
            \EN{Relating to the first years of formal education, usually for children.}
            \SY{elementary; basic}
            \VI{tiểu học.}
            \EX{She is a primary school teacher.}
            \EX{They have two kids in primary education.}
            \CO{primary school; primary education}
            \end{ExplainCard}

            \begin{ExplainCard}{sophomore}[n][B2]
            \EN{A student in the second year of high school or college.}
            \SY{second-year student}
            \VI{sinh viên năm hai / học sinh lớp 10.}
            \EX{She is a sophomore at Harvard.}
            \EX{As a sophomore, he joined the debate club.}
            \CO{college sophomore; sophomore year}
            \end{ExplainCard}

            \begin{ExplainCard}{spark}[v][B2]
            \EN{To cause something to start or develop.}
            \SY{trigger; stimulate; provoke}
            \VI{khơi dậy, châm ngòi.}
            \EX{The news sparked anger among citizens.}
            \EX{Her speech sparked my interest in politics.}
            \CO{spark interest; spark a debate}
            \end{ExplainCard}

            \begin{ExplainCard}{erratic}[adj][C1]
            \EN{Not regular or consistent; unpredictable.}
            \SY{unpredictable; inconsistent; irregular}
            \VI{thất thường, khó lường.}
            \EX{Her breathing became erratic.}
            \EX{The stock market has been very erratic lately.}
            \CO{erratic behavior; erratic movements}
            \end{ExplainCard}

            \begin{ExplainCard}{stand a chance}[idiom][B2]
            \EN{To have a possibility of achieving something.}
            \SY{have a possibility; have an opportunity}
            \VI{có cơ hội, khả năng.}
            \EX{He doesn’t stand a chance of winning.}
            \EX{You might stand a chance if you work hard.}
            \CO{stand a chance of doing sth}
            \end{ExplainCard}
        \end{VocabExplain}

    \noindent
    \textbf{Part 2.}
    \begin{qa}{Describe a website you use that helps you a lot in your work or studies. You should say:
    \begin{itemize}
        \item What the website is
        \item How often do you use the website
        \item What information the website gives you
        \item and explain how your work or studies would change if this website didn’t exist.
    \end{itemize}}

    In the 4.0 era today, with the advent of a social networking site like Facebook, I have had my work efficiency enhanced. Facebook was set up by Mark Zuckerberg in 2004 with the \textbf{sole} purpose of creating a platform for students of an Ivy-League institution like Harvard to exchange ideas when the founder was still a \textbf{sophomore} at Harvard University. However, thanks to its \textbf{preeminence} in connecting everyone together without \textbf{incurring} any costs, it has made possible the \textbf{meteoric rise from} a social network of a few hundreds of scholars in America to more than a billion registered ones from \textbf{all walks of life} across the globe so far. 

    I have a large circle of friends on Facebook and I frequently contact them on a daily basis. Besides, by using Facebook and reading what my friends share, I can be \textbf{up to speed with} any events happening in the world without browsing news websites or reading newspapers like before. Most importantly, when it comes to academic work, I would be \textbf{worse off but for} Facebook. To be more specific, I am a member of several study groups on Facebook where \textbf{everyone and his brother} willingly shares free materials related to standardized tests like FCE, CAE, SAT, IELTS, and TOEFL. I need such sources of materials to compile my practice tests for my students. Without Facebook, it would be challenging for me to find such valuable documents. Downloading materials to develop my English proficiency to \textbf{meet up with} the demands of my students is necessary as I am a teacher.
    \end{qa}

        \begin{VocabExplain}[Part 2]
            \begin{ExplainCard}{sole}[adj][C1]
            \EN{Being the only one; single.}
            \SY{only; single; exclusive}
            \VI{duy nhất.}
            \EX{He was the sole survivor of the accident.}
            \EX{She is the sole owner of the company.}
            \CO{sole purpose; sole responsibility}
            \end{ExplainCard}

            \begin{ExplainCard}{sophomore}[n][B2]
            \EN{A student in the second year of high school or university.}
            \SY{second-year student}
            \VI{sinh viên năm hai / học sinh lớp 10.}
            \EX{He is a sophomore at Stanford University.}
            \EX{As a sophomore, she joined the debate club.}
            \CO{college sophomore; sophomore year}
            \end{ExplainCard}

            \begin{ExplainCard}{preeminence}[n][C1]
            \EN{The state of being more important, skillful, or successful than others.}
            \SY{superiority; prominence; distinction}
            \VI{sự vượt trội, xuất chúng.}
            \EX{The company’s preeminence in technology is undisputed.}
            \EX{Her preeminence as a scientist won her many awards.}
            \CO{preeminence in sth}
            \end{ExplainCard}

            \begin{ExplainCard}{incur}[v][C1]
            \EN{To bring something upon oneself, usually something undesirable.}
            \SY{suffer; bring upon; attract}
            \VI{gánh chịu, mắc phải.}
            \EX{He incurred heavy debts due to his gambling.}
            \EX{The company incurred huge losses last year.}
            \CO{incur debts; incur costs}
            \end{ExplainCard}

            \begin{ExplainCard}{meteoric rise from}[phrase][C1]
            \EN{A rapid and dramatic increase in success or popularity.}
            \SY{rapid growth; swift ascent}
            \VI{sự thăng tiến nhanh chóng.}
            \EX{The band had a meteoric rise from obscurity to fame.}
            \EX{The company saw a meteoric rise from startup to industry leader.}
            \CO{meteoric rise from poverty; meteoric rise to power}
            \end{ExplainCard}

            \begin{ExplainCard}{all walks of life}[idiom][C1]
            \EN{All social, economic, and cultural backgrounds.}
            \SY{all backgrounds; all classes}
            \VI{mọi tầng lớp xã hội.}
            \EX{Volunteers came from all walks of life.}
            \EX{People from all walks of life attended the festival.}
            \CO{from all walks of life}
            \end{ExplainCard}

            \begin{ExplainCard}{up to speed with}[idiom][C1]
            \EN{Fully informed or updated about something.}
            \SY{informed; up to date; aware}
            \VI{theo kịp, cập nhật.}
            \EX{I need to be up to speed with the latest news.}
            \EX{He is up to speed with current events.}
            \CO{up to speed with sth}
            \end{ExplainCard}

            \begin{ExplainCard}{worse off but for}[phrase][C1]
            \EN{In a more difficult or poorer situation if something did not exist.}
            \SY{be in trouble without; disadvantaged without}
            \VI{tệ hơn nếu không có.}
            \EX{We would be worse off but for their help.}
            \EX{The economy would be worse off but for tourism.}
            \CO{worse off but for sth}
            \end{ExplainCard}

            \begin{ExplainCard}{everyone and his brother}[idiom][C1]
            \EN{Used to emphasize that a very large number of people are involved.}
            \SY{everybody; a large number of people}
            \VI{mọi người, đông đảo người.}
            \EX{Everyone and his brother came to the concert.}
            \EX{Nowadays, everyone and his brother owns a smartphone.}
            \CO{everyone and his brother + verb}
            \end{ExplainCard}

            \begin{ExplainCard}{meet up with}[phr.v][B2]
            \EN{To satisfy or fulfill demands, needs, or expectations.}
            \SY{satisfy; fulfill; live up to}
            \VI{đáp ứng.}
            \EX{The results did not meet up with our expectations.}
            \EX{The program meets up with students’ needs.}
            \CO{meet up with demands; meet up with standards}
            \end{ExplainCard}
        \end{VocabExplain}

    \noindent
    \textbf{Part 3.}
    \begin{qa}{Why do some people find the Internet addictive?}
    The Internet can be an enormous \textbf{escape hatch} when life becomes too stressful or when relationships become too \textbf{unfulfilling}. It is really helpful for people to immerse themselves in the virtual world and seem \textbf{oblivious to} their life in a moment. Escape hatches, though, are meant to be used \textbf{sparingly} and only at great need, so they are not the ideal way to make an exit.
    \end{qa}

    \begin{qa}{What would the world be like without the Internet?}
    As far as we know, without the Internet, people may \textbf{live under a rock}. Although it may seem crazy, they were all disconnected from each other, as if they were living in \textbf{isolation}, and our life will be \textbf{messy}. We would need to look at maps to get to the places or go out to buy food when it is frozen outside. This scenario seems terrible if it befalls users, I believe.
    \end{qa}

    \begin{qa}{Do you think that the way people use the Internet may change in the future?}
    Possibly, the Internet will become \textbf{akin to} electrical service \textbf{along the line}. This means people will enjoy the Internet \textbf{heart and soul}. Apart from essential functions such as browsing websites, Internet may start tracking our grocery usage, tablets and other commodities at home and be aware of the stock available, or it could connect and ask household appliances to \textbf{deputize for} daily tasks such as sweeping the floor.
    \end{qa}

    \begin{qa}{What are the ways that social media can be used for positive purposes?}
    They offer enormous benefits. Firstly, social media can be used as a voice for the voiceless such as the \textbf{handicap} or the \textbf{destitute}. Various social networking sites such as Facebook and Twitter are being used by the youth to donate money and possessions to the needy, which is really useful. Social media has given teens the ability to hone different skills, both verbal and non-verbal ones, and interpret different situations \textbf{contextually} to \textbf{gear themselves} towards the future. For example, some teenagers got the hang of playing the guitar thanks to free lessons uploaded on a Facebook page.
    \end{qa}

    \begin{qa}{Why do some individuals post highly negative comments about other people on social media?}
    Well, social media bullying is one of the most \textbf{distressing} problems that many might encounter while they are online. If I post something to cause offence or \textbf{bully} people, by going against their beliefs for example, then they might leave a negative comment as a result of their \textbf{hatred}. Many cannot help leaving hostile comments behind the screen because they can hurt others while running a lower risk of identity leak. Hiding behind fake usernames, these aggressive users may use this anonymity to be mean to others without any fear of being prosecuted.
    \end{qa}

    \begin{qa}{Do you think that companies’ main form of advertising will be via social media in the future?}
    I am not sure about that. Social media will still gain in popularity thanks to its nature of bypassing the need to pay a large sum of money for advertising on traditional media. This generally suits younger population who can adapt themselves to technological changes. In fact, marketers are using traditional forms of marketing like radio, television and print to reach out to customers and potential customers. Even though some are \textbf{skeptical} about the \textbf{glory days} for traditional media, I still believe that they are popular with \textbf{senior citizens}, who are not too computer literate.
    \end{qa}

        \begin{VocabExplain}[Part 3]
            \begin{ExplainCard}{escape hatch}[n][C1]
            \EN{Something that provides a way out of a difficult situation.}
            \SY{way out; exit; loophole}
            \VI{lối thoát, giải pháp thoát hiểm.}
            \EX{Humor can be an escape hatch in tense situations.}
            \EX{He used work as an escape hatch from his problems.}
            \CO{an escape hatch from sth}
            \end{ExplainCard}

            \begin{ExplainCard}{unfulfilling}[adj][C1]
            \EN{Not satisfying or rewarding.}
            \SY{dissatisfying; disappointing}
            \VI{không thỏa mãn, không đáng giá.}
            \EX{Many find routine office jobs unfulfilling.}
            \EX{Her unfulfilling career led her to change paths.}
            \CO{an unfulfilling job/life}
            \end{ExplainCard}

            \begin{ExplainCard}{oblivious to}[phrase][C1]
            \EN{Not aware of or not concerned about what is happening.}
            \SY{unaware of; ignorant of}
            \VI{không nhận ra, không để ý.}
            \EX{She was oblivious to the noise outside.}
            \EX{He seemed oblivious to the danger.}
            \CO{be oblivious to sth}
            \end{ExplainCard}

            \begin{ExplainCard}{sparingly}[adv][C1]
            \EN{In small amounts; not often.}
            \SY{economically; moderately}
            \VI{tiết kiệm, dè sẻn.}
            \EX{Use the cream sparingly on your skin.}
            \EX{Water should be used sparingly in dry areas.}
            \CO{use sparingly; applied sparingly}
            \end{ExplainCard}

            \begin{ExplainCard}{live under a rock}[idiom][C1]
            \EN{To be unaware of what is happening in the world.}
            \SY{uninformed; ignorant}
            \VI{không biết gì về thế giới xung quanh.}
            \EX{You must live under a rock if you haven’t heard of this.}
            \EX{She lived under a rock, unaware of world events.}
            \CO{live under a rock about sth}
            \end{ExplainCard}

            \begin{ExplainCard}{isolation}[n][B2]
            \EN{The state of being separated from others.}
            \SY{seclusion; loneliness; detachment}
            \VI{sự cô lập.}
            \EX{The prisoner spent years in isolation.}
            \EX{Isolation can damage mental health.}
            \CO{in isolation; isolation chamber}
            \end{ExplainCard}

            \begin{ExplainCard}{messy}[adj][B2]
            \EN{Untidy, disorganized, or complicated.}
            \SY{disordered; chaotic}
            \VI{lộn xộn, rối rắm.}
            \EX{The divorce was messy.}
            \EX{He has a messy desk.}
            \CO{messy situation; messy room}
            \end{ExplainCard}

            \begin{ExplainCard}{akin to}[phrase][C1]
            \EN{Of similar character or related to.}
            \SY{similar to; comparable to}
            \VI{tương tự như.}
            \EX{His music is akin to jazz.}
            \EX{This style is akin to traditional art.}
            \CO{akin to sth}
            \end{ExplainCard}

            \begin{ExplainCard}{along the line}[idiom][C1]
            \EN{At some point in the future or past.}
            \SY{sometime; eventually}
            \VI{vào một thời điểm nào đó.}
            \EX{Somewhere along the line, we lost contact.}
            \EX{He must have lied along the line.}
            \CO{somewhere along the line}
            \end{ExplainCard}

            \begin{ExplainCard}{heart and soul}[idiom][C1]
            \EN{With complete energy and enthusiasm.}
            \SY{wholeheartedly; passionately}
            \VI{hết lòng, toàn tâm toàn ý.}
            \EX{She put her heart and soul into the project.}
            \EX{He is heart and soul devoted to his work.}
            \CO{put heart and soul into sth}
            \end{ExplainCard}

            \begin{ExplainCard}{deputize for}[phr.v][C1]
            \EN{To act as a substitute for someone or something.}
            \SY{substitute for; fill in for}
            \VI{thay thế, làm thay.}
            \EX{She deputized for the manager during his absence.}
            \EX{Robots may deputize for humans in dangerous jobs.}
            \CO{deputize for sb}
            \end{ExplainCard}

            \begin{ExplainCard}{handicap}[n][C1]
            \EN{A condition that restricts a person’s ability, often physical or mental.}
            \SY{disability; impairment}
            \VI{khuyết tật, sự bất lợi.}
            \EX{He didn’t let his handicap stop him.}
            \EX{People with handicaps need support.}
            \CO{overcome a handicap; born with a handicap}
            \end{ExplainCard}

            \begin{ExplainCard}{destitute}[adj][C1]
            \EN{Without the basic necessities of life.}
            \SY{impoverished; penniless}
            \VI{cùng khổ, nghèo túng.}
            \EX{The flood left families destitute.}
            \EX{He grew up destitute in the countryside.}
            \CO{destitute family; left destitute}
            \end{ExplainCard}

            \begin{ExplainCard}{contextually}[adv][C1]
            \EN{In a way that depends on or relates to the surrounding circumstances.}
            \SY{in context; situation-based}
            \VI{dựa trên bối cảnh.}
            \EX{The word must be understood contextually.}
            \EX{Contextually appropriate examples are important.}
            \CO{analyze contextually; interpret contextually}
            \end{ExplainCard}

            \begin{ExplainCard}{gear oneself}[phr.v][C1]
            \EN{To prepare or adjust oneself for something.}
            \SY{prepare for; adapt to}
            \VI{tự chuẩn bị cho, thích nghi.}
            \EX{She geared herself for the exam.}
            \EX{Workers geared themselves for the changes.}
            \CO{gear oneself for sth}
            \end{ExplainCard}

            \begin{ExplainCard}{distressing}[adj][C1]
            \EN{Causing anxiety, sorrow, or pain.}
            \SY{upsetting; troubling; disturbing}
            \VI{gây đau buồn, khó chịu.}
            \EX{It was distressing to see her cry.}
            \EX{The report contained distressing details.}
            \CO{distressing news; distressing experience}
            \end{ExplainCard}

            \begin{ExplainCard}{bully}[v][B2]
            \EN{To frighten or hurt someone weaker, often repeatedly.}
            \SY{intimidate; harass; torment}
            \VI{bắt nạt.}
            \EX{He was bullied at school.}
            \EX{Don’t let anyone bully you into silence.}
            \CO{bully sb into doing sth; school bullying}
            \end{ExplainCard}

            \begin{ExplainCard}{hatred}[n][C1]
            \EN{Intense dislike or ill will.}
            \SY{loathing; hostility; animosity}
            \VI{lòng căm ghét, thù hận.}
            \EX{He felt hatred towards his enemies.}
            \EX{Hatred often leads to violence.}
            \CO{deep hatred; hatred for sb}
            \end{ExplainCard}

            \begin{ExplainCard}{skeptical}[adj][C1]
            \EN{Doubting the truth or value of something.}
            \SY{doubtful; suspicious; unconvinced}
            \VI{nghi ngờ, hoài nghi.}
            \EX{She was skeptical about his promises.}
            \EX{Experts remain skeptical of the theory.}
            \CO{skeptical about sth; skeptical attitude}
            \end{ExplainCard}

            \begin{ExplainCard}{glory days}[idiom][C1]
            \EN{A period of time when someone or something was most successful.}
            \SY{heyday; peak; prime}
            \VI{thời hoàng kim.}
            \EX{He often talks about the glory days of his youth.}
            \EX{Print media had its glory days in the 20th century.}
            \CO{the glory days of sth}
            \end{ExplainCard}

            \begin{ExplainCard}{senior citizen}[n][B2]
            \EN{An elderly person, usually over the age of 60 or 65.}
            \SY{elderly person; retiree}
            \VI{người cao tuổi.}
            \EX{The bus offers discounts to senior citizens.}
            \EX{Senior citizens often attend community events.}
            \CO{senior citizen discount; retired senior citizens}
            \end{ExplainCard}
        \end{VocabExplain}

    \begin{VocabHighlights}
        \VH{to be held captive}{(phrase) to be held in a cage}{(cụm từ) bị nuôi nhốt}
        \VH{to have the guts to V-inf}{(idiom) to have the courage to V}{(thành ngữ) có đủ dũng cảm để}
        \VH{from way back}{(idiom) since long ago}{(thành ngữ) từ thời xưa}
        \VH{sophomore}{(n) a second-year college or high school student}{(danh từ) sinh viên năm 2}
        \VH{to spark}{(v) to cause something to start or develop, especially suddenly}{(động từ) khuấy động lên}
        \VH{erratic}{(adj) not happening at regular times; not following any plan or regular pattern}{(tính từ) bất thường}
        \VH{to stand a chance of V-ing}{(idiom) to be likely to V-inf}{(thành ngữ) có nhiều cơ hội làm gì}
        \VH{preeminence}{(n) having more quality}{(danh từ) tính ưu việt, điểm nổi trội}
        \VH{to incur}{(v) (cost) arise}{(động từ) phát sinh chi phí}
        \VH{meteoric rise}{(phrase) gaining more recognition from people}{(cụm từ) nổi như cồn, nổi tiếng rất nhanh}
        \VH{to be up to speed with}{(phrase) update}{(cụm từ) cập nhật}
        \VH{to be worse off}{(idiom) be unhappier}{(thành ngữ) bất hạnh hơn}
        \VH{everyone and his brother}{(idiom) a large number of people}{(thành ngữ) rất nhiều người}
        \VH{to meet up with}{(phrase) satisfy the needs of}{(cụm từ) đáp ứng nhu cầu}
        \VH{escape hatch}{(n) a way of getting out of a difficult or unwanted situation}{(danh từ) cách trốn tránh}
        \VH{unfulfilling}{(adj) not causing somebody to feel satisfied and useful}{(tính từ) không thỏa mãn}
        \VH{to be oblivious to}{(adj) not conscious of something, especially what is happening around you}{(tính từ) không quan tâm}
        \VH{sparingly}{(adv) in a way that is careful to use or give only a little of something}{(trạng từ) tiết kiệm}
        \VH{to live under a rock}{(idiom) lack basic knowledge of current events or popular culture}{(thành ngữ) lạc hậu}
        \VH{isolation}{(n) the act of separating somebody/something; the state of being separate}{(danh từ) sự cô lập}
        \VH{messy}{(adj) dirty and/or untidy}{(tính từ) lộn xộn}
        \VH{to be akin to}{(adj) (to something) similar to}{(tính từ) giống với cái gì}
        \VH{along the line}{(idiom) at a further, later, or unspecified point}{(thành ngữ) sau đó thì}
        \VH{the heart and soul of}{(phrase) do something with a great deal of enthusiasm and energy}{(cụm từ) làm gì với lòng nhiệt huyết}
        \VH{to deputize}{(v) to do something that somebody in a higher position than you would usually do}{(động từ) phân quyền}
        \VH{the handicap}{(n) a permanent physical or mental condition that makes it difficult to use a part of your body or mind}{(danh từ) người khuyết tật}
        \VH{the destitute}{(n) people without money, food and the other things necessary for life}{(danh từ) người nghèo}
        \VH{contextually}{(adv) in a way that is connected with a particular context}{(trạng từ) theo ngữ cảnh}
        \VH{to gear}{(v) to make something ready or suitable for a particular purpose}{(động từ) chuẩn bị, trang bị}
        \VH{distressing}{(adj) making you feel extremely upset, especially because of somebody's suffering}{(tính từ) áp lực, đau khổ}
        \VH{to bully}{(v) to hurt or frighten someone, often over a period of time, and often force that person to do something they do not want to do}{(động từ) bắt nạt}
        \VH{hatred}{(n) a very strong feeling of dislike for somebody/something}{(danh từ) sự thù ghét}
        \VH{skeptical}{(adj) having doubts that a claim or statement is true or something will happen}{(tính từ) hoài nghi, nghi ngờ}
        \VH{glory days}{(phrase) a time in the past regarded as being better than the present}{(cụm từ) thời kỳ hoàng kim}
    \end{VocabHighlights}
    \end{test}
\end{glossarymc}