\refstepcounter{section}

\section*{GIỚI THIỆU VỀ KỸ NĂNG SPEAKING TRONG IELTS}
\addcontentsline{toc}{section}{GIỚI THIỆU VỀ KỸ NĂNG SPEAKING TRONG IELTS}
\label{sec:Intro}

\subsection*{I. Format (Cấu trúc đề thi)}

Mỗi bài thi IELTS Speaking sẽ gồm 3 phần chính (tổng thời lượng 11–14 phút/thí sinh) 
nhằm mục đích đánh giá khả năng nói tiếng Anh của thí sinh qua việc truyền tải thông tin, 
ý tưởng qua các chủ đề, tình huống quen thuộc hằng ngày. 
Giám khảo có ghi âm lại bài nói.

\bigskip
\textbf{Phần 1:}  
Giám khảo sẽ hỏi những câu hỏi chung về bản thân và một loạt chủ đề quen thuộc như: nhà cửa, gia đình, công việc, học tập, sở thích… (4–5 phút nói).  
Thông thường, thí sinh sẽ phải trả lời một số câu hỏi liên quan tới 3 chủ đề.

\bigskip
\textbf{Phần 2:}  
Thí sinh sẽ được yêu cầu thảo luận về một chủ đề cụ thể dựa trên những gợi ý (``cue card'').  
Thí sinh có 1 phút chuẩn bị và 2 phút để nói. Trong quá trình chuẩn bị, thí sinh có thể dùng giấy và bút để liệt kê ý tưởng ra.  
Giám khảo có thể đặt một – hai câu hỏi về cùng một chủ đề.  
Phần này không quá 4 phút.

\bigskip
\textbf{Phần 3:}  
Thí sinh sẽ được hỏi thêm các câu hỏi liên quan đến chủ đề của phần 2, các câu hỏi mang tính chất rộng mở hơn.  
Điều này sẽ khiến thí sinh có thể thảo luận thêm cùng giám khảo những ý tưởng, vấn đề trừu tượng hơn.  
Phần này kéo dài từ 4–5 phút.

\bigskip
Phần nói được những giám khảo IELTS có đủ bằng cấp chấm. 
Tất cả các giám khảo đều có bằng cấp tương đương và được các trung tâm được ủy quyền tổ chức IELTS phân công chấm.  
Điểm của thí sinh được Hội đồng Anh (British Council) hoặc IDP (IELTS Australia) công nhận.

\bigskip
Điểm có thể được gửi về dưới dạng chẵn (.0) hoặc (.5). 
Mô tả chi tiết những phần điểm từng kỹ năng để đi đến tổng điểm cuối cùng sẽ được đưa ra trên thang điểm 9.

\subsection*{II. Tiêu chí chấm IELTS}

\textbf{1. Độ lưu loát và tính mạch lạc (Fluency and coherence)}  

\vspace{.25cm}
Tiêu chí này được đặt ra để đo khả năng nói theo mức độ liên tục, sự nỗ lực khi nói và liên kết ý tưởng lẫn ngôn ngữ cùng nhau để tạo thành bài nói liền mạch, gắn kết.  

\vspace{.25cm}
Sự lưu loát được đánh giá qua khả năng nói ở tốc độ bình thường, đủ nghe khi người nói không quá ngắc ngứ hay lặp từ. Thí sinh có thể nói nhanh hơn ở những chủ đề thân quen và nói chậm lại ở những chủ đề bản thân thấy xa lạ để tránh dùng từ sai ngữ cảnh. 

\vspace{.25cm}
Tiêu chuẩn chính để đánh giá tính mạch lạc là sự liên kết chuỗi câu sao cho thật logic, có đánh dấu rõ ràng từng bước trong thảo luận, kể chuyện hoặc tranh luận và việc dùng các liên từ gắn kết câu lại với nhau để người nghe dễ hiểu. Dùng liên từ thích hợp sẽ khiến bài nói trở nên thuyết phục hơn nhiều vì người chấm sẽ hiểu được cách phát triển bài nói của thí sinh: liệu đang muốn đổi chủ đề hay định hướng chủ đề liền mạch từ đầu theo một hướng nhất định.  

\vspace{.25cm}
Để giành được điểm cao, thì sinh cần cho giám khảo thấy khả năng nói dài hơi, phát triển chủ đề, dùng từ nối hợp lí và hạn chế ngắc ngứ hết sức có thể. Tất nhiên, ngay cả khi chúng ta nói tiếng mẹ đẻ, việc mất thời gian suy nghĩ, tìm từ thích hợp để diễn tả ý trong đầu là khó tránh khỏi. Tuy nhiên, nếu có ngắc ngứ thì thí sinh nên hạn chế nói nửa câu. Trong hoàn cảnh đó, giám khảo sẽ cho rằng thí sinh đang cố gắng tìm cách nói hoặc ngữ pháp đúng để hoàn thành câu nói. Nếu ngắc ngứ xảy ra ở đầu câu, điều này sẽ bớt nghiêm trọng hơn vì giám khảo có thiên hướng cho rằng người nói đang tìm ý tưởng phù hợp để nói chứ không phải là tìm ngữ pháp hay từ vựng thích hợp. 

\vspace{.25cm}
\textbf{2. Từ vựng (Lexical resource)}  

\vspace{.25cm}
Tiêu chí này liên quan đến độ đa dạng của từ vựng thí sinh sử dụng cùng với sự chính xác và sắc thái ý nghĩa được thể hiện. Tiêu chuẩn chính để đánh giá từ vựng là sự đa dạng ngôn từ, sự đầy đủ và phù hợp của từ được dùng và khả năng dùng nhiều từ khác nhau để diễn đạt cùng một chủ đề mà không cần do dự nhiều. Để giành được điểm cao phần này, thí sinh cần dùng những từ ở cấp độ cao hơn bình thường. Cụ thể là, thí sinh cần dùng đúng những từ hay đi liền với nhau (collocation), kết thúc từng từ (word endings), từ thông tục trong văn nói (colloquial language) thường là dưới dạng cụm động từ (phrasal verbs) hay thành ngữ (idioms)…  

\vspace{.25cm}
Sự lựa chọn ngôn từ cần phải thích hợp. Giám khảo có thể trừ điểm dù đó là lỗi sai đơn lẻ hoặc lặp đi lặp lại có hệ thống và đặc biệt là ảnh hưởng đến việc truyền tải thông tin. Nên tránh dùng các cụm từ mà bản thân không chắc chắn. Chỉ nên dùng từ vựng nâng cao nếu cần thiết.  

\vspace{.25cm}
Sự sai lầm về chọn dạng từ cũng phần nào ảnh hưởng đến ý trong câu. Ví dụ, câu đúng là:  
\begin{quote}
``I need to check the authenticity of these paintings’’  
\end{quote}  
nhưng sai khi nói:  
\begin{quote}
``I need to check the authentic of these paintings’’  
\end{quote}  

Ý tưởng vẫn được truyền tải, giám khảo vẫn hiểu nhưng khó mà giành điểm tối đa.  

\vspace{.25cm}
Để vượt qua điểm 6.0 IELTS, thí sinh cần cho giám khảo thấy khả năng sử dụng từ thuộc về thành ngữ (idiomatic language) và từ vựng liên quan trực tiếp đến chủ đề được hỏi (topic-based language). Thí sinh nên tìm các từ vựng liên quan đến các nhóm chủ đề lớn của IELTS để ôn luyện trước khi thi thật. Từ vựng thành ngữ thì thí sinh nên học thật chắc một vài cụm để dùng trong ngữ cảnh thích hợp, tránh trường hợp dùng mà không hiểu, sai ngữ cảnh thì còn nghiêm trọng hơn.  

\vspace{.25cm}
Chung quy lại, qua cuốn sách này chúng tôi muốn cung cấp cho người đọc một lượng từ vựng lớn được đặt vào trong ngữ cảnh lúc nói dưới dạng cụm động từ, thành ngữ, từ hay đi liền với nhau. Mong mọi người có thể sử dụng sách thật hiệu quả.  


\vspace{.25cm}
\textbf{3. Độ rộng và chính xác của ngữ pháp (Grammatical Range and Accuracy)}  

\vspace{.25cm}
Tiêu chí này hướng đến độ rộng và việc sử dụng chính xác, cụ thể ngữ pháp của thí sinh. Tiêu chuẩn này được xác định theo cấu trúc câu, độ dài và phức tạp của câu nói, việc sử dụng câu phức và câu ghép dưới dạng mệnh đề phụ (relative clause), mệnh đề trạng từ (adverb clause)… đặc biệt là nhấn mạnh ý trong câu. Dùng được nhiều câu phức chuẩn xác là điều vô cùng quan trọng. Tiêu chuẩn để đánh giá sự chuẩn xác trong ngữ pháp là số lỗi sai ngữ pháp trong thời gian hạn định và liệu xem việc giao tiếp có bị ảnh hưởng nhiều hay không.  

\vspace{.25cm}
Về câu phức, đây là yếu tố rất quan trọng để giành điểm cao trong tiêu chí này. Câu phức là những câu có 1 mệnh đề phụ thuộc (dependent clause) và 1 mệnh đề không phụ thuộc (independent clause). Nói chung, để phân tích ra đầy đủ thì sẽ tốn rất nhiều thời gian. Tốt nhất, thí sinh chỉ nên tập trung vào vài dạng câu chính, ví dụ như câu điều kiện và đặc biệt là mệnh đề quan hệ để bổ sung thêm ý chi tiết cho câu.  

\vspace{.25cm}
Về thì động từ, đây cũng là tiêu chí đánh giá ngữ pháp của thí sinh. Do đó, thí sinh cần hạn chế số lỗi sai về ngữ pháp bằng cách để ý tới thì trong câu hỏi. Giả sử, nếu tác giả hỏi việc mô tả 1 công việc mơ ước trong tương lai (``Describe a future job’’), thì thì tương lai rõ ràng phải dùng xuyên suốt cả bài nói.  

\vspace{.25cm}
Về trật tự từ, đây cũng là lỗi thí sinh thường mắc, nhất là thí sinh Việt Nam. Đây là 1 lỗi nghiêm trọng có thể ảnh hưởng lớn đến điểm số vì nó dễ khiến thông tin trở nên khó nắm bắt. Thông thường trong tiếng Việt, tính từ đứng sau bổ nghĩa cho danh từ và đại từ. Ví dụ, trong tiếng Việt ta nói: ``Cô ấy là 1 con người tốt bụng’’ thì rõ ràng, tính từ ``tốt bụng’’ đứng sau danh từ ``con người’’. Tuy nhiên, trong tiếng Anh, tính từ lại đứng trước danh từ và đại từ. Ví dụ, câu trên ta có thể tạm dịch là:  
\begin{quote}
``She is a kind-hearted person.’’   ``Kind-hearted’’ là tính từ bổ nghĩa cho danh từ ``person’’.  
\end{quote}  


\vspace{.25cm}
Về giới từ và mạo từ, nếu dùng sai thì thí sinh có thể được giám khảo châm trước, bỏ qua vì đây là lỗi ít ảnh hưởng tới nội dung truyền tải. Tuy nhiên, nếu lỗi sai lặp đi lặp lại liên tục quá thì rõ ràng điểm sẽ bị ảnh hưởng ít nhiều.  


\vspace{.25cm}
\textbf{4. Phát âm (Pronunciation)}  

\vspace{.25cm}
Điểm tổng sẽ phụ thuộc phát âm của thí sinh tương đối nhiều. Đây là cách ngữ âm (accent) của thí sinh có thể khiến người chấm dễ hiểu hay không và mức độ dùng những nét đặc trưng trong ngôn ngữ của 1 người bản ngữ: sự nhấn mạnh (stress), ngữ điệu lên bổng xuống trầm trong câu nói (intonation).  

\vspace{.25cm}
Có một vài âm cụ thể khiến thí sinh khó khăn khi nói, ví dụ: phân biệt cách đọc nguyên âm ngắn hoặc dài, phụ âm /th/ khi có thể đọc là $/\theta/$, /ð/. Nếu gặp phải vấn đề này, thí sinh nên nhờ 1 giáo viên, tốt nhất là bản ngữ, sửa và tập luyện hằng ngày để phát âm chuẩn. Phát âm dựa nhiều vào thói quen. Nếu từ nhỏ được học phát âm bài bản thì người học sẽ được uốn nắn dễ dàng hơn. Nếu phát âm không được chú trọng từ nhỏ mà đến lớn mới tập trung sửa thì người học sẽ gặp rất nhiều khó khăn vì đã có khi thói quen đã hình thành từ trong tiềm thức. Nếu không có điều kiện nhờ người bản ngữ thì tốt nhất nên tra cứu từ điển điện tử có phần phát âm để học theo cách phát âm tích hợp sẵn âm thanh trên đó. Nếu không sửa được thì điểm kém là điều có thể dự báo trước.  

\vspace{.25cm}
Về trọng âm, người nói tiếng Anh bản ngữ thường nhất mạnh vào trọng âm từng câu. Nếu thí sinh nói với trọng âm sai hoặc không có trọng âm, điều này sẽ khiến người chấm khó hiểu về nghĩa của câu hay ý định truyền tải của thí sinh. Nếu có điều kiện thì thí sinh có thể nhờ người bản ngữ sửa cho bản thân mình hoặc nếu không thì có thể lại sử dụng từ điển điện tử để tra cứu cách phát âm cho đúng. Thực tế thì trên thị trường sách và thông tin hiện nay, cá nhân tác giả nhận thấy có vô vàn sách và tài liệu hướng dẫn các quy tắc phát âm. Nếu thí sinh có thể học và nhớ được thì tốt, còn không nhớ được cũng không phải chuyện gì nghiêm trọng. Quy tắc có đến cả trăm mà gần như quy tắc nào cũng có ngoại lệ, đó là điều khiến việc ghi nhớ quy tắc trở nên khó khăn. Chỉ cần dùng từ điển điện tử có phần phát âm từ vựng để biết trọng âm rơi vào đâu là phù hợp.  

\vspace{.25cm}
Về ngữ điệu, nếu thiếu ngữ điệu, bài nói sẽ đều đều và gây nhàm chán cho giám khảo. Tuy nhiên, thí sinh không thể nhấn nhá bừa bãi trong câu nói vì như vậy là không tự nhiên theo cách nói chuyện thông thường của người bản ngữ và khiến điểm số có thể bị ảnh hưởng. Ví dụ, chúng ta thường kết thúc một câu khẳng định bằng tông giọng trầm xuống (falling tone). Tông giọng sẽ cao lên (rising tone) khi đặt câu hỏi “yes/no”. Đây là một trong những điều khó nhất khi học tiếng Anh và đây cũng là một trong những lí do khiến thí sinh vượt qua được điểm 8 IELTS dù tất cả các tiêu chí trên đều tốt. Nói chung, nếu không có điều kiện tiếp xúc nhiều với người bản ngữ thì thí sinh nên tích cực xem các bộ phim bằng tiếng Anh để cải thiện ngữ điệu của bản thân mình.  

\newpage
\section*{CÁC CÁCH ĐỂ ĐẠT ĐIỂM SPEAKING TỐT}
\addcontentsline{toc}{section}{CÁC CÁCH ĐỂ ĐẠT ĐIỂM SPEAKING TỐT}
\label{sec:Strats}

\subsection*{I. Các vấn đề thường gặp và cách giải quyết hiệu quả}

\begin{multicols}{2}
\begin{qa}{Vấn đề 1: Cố gắng nói quá nhanh}
Nhìn chung, những thí sinh có khả năng nói lưu loát thì sẽ nói mượt mà chứ không nhanh. Có một nhận định khá sai lầm là: lưu loát tức là phải nói thật nhanh. Nếu ta nghe cách người bản ngữ nói chuyện, nếu không trực tiếp thì có thể qua đài, bản tin, phim ảnh… ta sẽ nhận ra rằng họ không thường xuyên nói quá nhanh và phần nào hiểu được tốc độ thế nào là vừa phải. Nói quá nhanh trong bất kỳ ngôn ngữ nào đều có thể khiến nội dung thông tin thiếu tính liên mạch vì đôi khi não bộ chúng ta không theo kịp tốc độ nói, chưa kể đến những lỗi sai về ngữ pháp, từ vựng… Điều này có thể ảnh hưởng trực tiếp đến kết quả bài kiểm tra.

\vspace{.2cm}
\textbf{Giải pháp}: Tập trung vào nói thật bình tĩnh, mượt mà. Tích cực nghe cách người bản ngữ nói chuyện để học hỏi tốc độ nói của họ. Điều này không chỉ cải thiện tính lưu loát mà còn cho phép thí sinh có thêm thời gian để nghĩ ý tưởng, tìm ra từ vựng và ngữ pháp thích hợp để truyền tải ý tưởng đó.
\end{qa}

\begin{qa}{Vấn đề 2: Cố gắng nghĩ ý tưởng}
Nhiều thí sinh than phiền rằng, họ không thể nghĩ ra ý tưởng nào thật đỉnh cao và điều này khiến sự lưu loát trong lời nói của họ bị ảnh hưởng. Suy nghĩ này có phần sai lầm do họ chưa hiểu điều gì được đánh giá qua bài nói. Nếu nhìn vào tiêu chí chấm điểm, ta sẽ thấy: không có khái niệm nào về “ý tưởng tốt” hay “ý tưởng đúng đắn” nào được đề cập cả. Chừng nào mà thí sinh đưa ra câu trả lời hợp lý, có liên quan đến câu hỏi thì sẽ đạt được yêu cầu. Đây không phải kỳ thi đánh giá tư tưởng hay trí tuệ mà chỉ đơn thuần là đánh giá kỹ năng ngôn ngữ nói. Không có câu trả lời nào hoàn toàn đúng hoặc hoàn toàn sai, chỉ đơn giản là phó diễn cho giám khảo xem khả năng nói đến đâu là đủ.

\vspace{.2cm}
\textbf{Giải pháp}: Đừng suy nghĩ quá nhiều về chất lượng các ý tưởng mà thay vào đó hãy tập trung vào chất lượng bài nói của bản thân. Hãy nói hết những gì mình suy nghĩ về câu hỏi được đưa ra. Nếu không biết hoặc biết rất ít về chủ đề đó, đừng ngần ngại báo ngay giám khảo. Với phần 1, vì các câu hỏi là về bản thân thí sinh, nhìn chung thì thí sinh sẽ không gặp quá nhiều vấn đề khi lên ý tưởng. Với phần 2, câu trả lời cần có cấu trúc cụ thể. Với phần 3, một vài câu sẽ tương đối khó. Thí sinh phải chấp nhận điều này và cố gắng trả lời tốt nhất có thể.
\end{qa}

\begin{qa}{Vấn đề 3: Thiếu cấu trúc}
Đây là vấn đề liên quan trực tiếp đến phần 2 khi thí sinh phải nói tối đa 2 phút về một chủ đề giám khảo đưa cho họ. Chủ đề được hỏi sẽ đi kèm 3–4 gợi ý (“cue card”). Tuy nhiên, vấn đề là nhiều người không đủ ý tưởng để nói trong 2 phút, nên sự lưu loát cũng ảnh hưởng phần nào. Có 2 lý do chính dẫn tới điều này. Đầu tiên là, nhiều người cho rằng nếu có thể nói trình bày bài nói dựa theo những gợi ý trong đề, tuy nhiên, các gợi ý không phải bắt buộc, chúng chỉ đóng vai trò hướng dẫn mà thôi. Thí sinh hoàn toàn có thể nói theo nội dung mình cảm thấy phù hợp, miễn sao phù hợp với chủ đề. Hai là, nhiều người không chuẩn bị thêm nội dung ngoài các gợi ý có sẵn.

\vspace{.2cm}
\textbf{Giải pháp}: Thí sinh phải nói về chủ đề chung được đưa ra nhưng vẫn có thể thoải mái nói về bất kì điều gì liên quan đến chủ đề đó, không nhất thiết chỉ dựa vào 3–4 gợi ý. Một số cách sau có thể dùng để khiến bài nói dài hơn, thú vị hơn, và quan trọng là tròn 2 phút:
\begin{itemize}
    \item Giới thiệu chủ đề
    \item Đưa ra nhận định về chủ đề
    \item Kể về quá khứ, hiện tại, tương lai của chủ đề
    \item Kể về một trải nghiệm cá nhân liên quan đến chủ đề
\end{itemize}
\end{qa}

\begin{qa}{Vấn đề 4: Tập trung quá nhiều vào ngữ pháp và từ vựng}
Ngữ pháp và từ vựng chiếm khoảng 50\% tổng điểm. Tầm quan trọng của phần này là rõ ràng nhưng nếu dành quá nhiều thời gian suy nghĩ sẽ khiến tính liên kết, mạch lạc trong câu bị ảnh hưởng. Cụ thể là, nếu luôn phải nghĩ xem dùng cấu trúc ngữ pháp nào cho đúng thì sẽ có thể dễ dàng ngập ngừng, ngắt quãng. Đây cũng có thể một phần là do giáo viên ở trường phổ thông quá chú trọng ngữ pháp, từ vựng mà quên mất đi một điều rằng: kỹ năng nói thì quan trọng nhất là truyền tải ý tưởng.

\vspace{.2cm}
\textbf{Giải pháp}: Nên ghi âm lại bài nói của bản thân và nghe lại. Với mỗi bài nói thì thí sinh có thể tập như sau: Lần 1 ghi âm có thể tập trung vào ngữ pháp, từ vựng sao cho đúng. Lần 2 ghi âm lại thì thí sinh cố gắng nói thật tự nhiên, không cần chú trọng quá đến ngữ pháp từ vựng. Sau khi ghi âm 2 lần xong thì thí sinh thử nghe lại xem lần nào tốt hơn. Tất nhiên lần 2 có nhiều sai sót hơn nhưng kết quả chung về lưu loát và dễ hiểu sẽ cải thiện hơn nhiều.
\end{qa}

\begin{qa}{Vấn đề 5: Tập luyện chưa đủ}
Nhiều thí sinh có lượng từ vựng và ngữ pháp lớn, tuy nhiên họ lại không dùng những gì đang có qua việc kể hay nói hằng ngày. Điều này khiến họ gặp nhiều vấn đề về cách phát âm và tính trôi chảy. Kỹ năng nói cũng như bất kỳ kỹ năng nào khác đều cần phải tập luyện hằng ngày. Vấn đề là nhiều khi thí sinh không tìm ra được người để trò chuyện cùng.

\vspace{.2cm}
\textbf{Giải pháp}: Hãy tìm người bản ngữ sống trong khu vực của mình hoặc là học hỏi qua các nhóm học tập online thông qua việc trao đổi ngôn ngữ. Một cách hiệu quả là “Săn Tây”. Khách du lịch cũng cần những khoảng thời gian rảnh rỗi và không ai lại từ chối muốn giúp đỡ những người hoàn toàn lạ mặt tự dựng sân số vào nói chuyện cả. Có một cách khác để tập luyện có thể hiệu quả, đó là nói chuyện với chính bản thân mình qua gương.
\end{qa}

\begin{qa}{Vấn đề 6: Có gắng không nói “Umm” “Ahh”}
Nhiều thí sinh cho rằng việc nói “umm” “ahh” “emmm” đồng nghĩa với kém lưu loát. Nếu thi thoảng ngập ngừng và nói như vậy thì không sao, liền tục mới là vấn đề. Đến ngay cả người bản ngữ khi nói chuyện, việc đôi lúc ngắc ngứ cũng là khó tránh khỏi. Nếu thí sinh nghĩ quá nhiều về việc tránh nói những từ này thì điều này đôi khi sẽ khiến bài nói trở nên kém tự nhiên.

\vspace{.2cm}
\textbf{Giải pháp}: Thí sinh hãy chấp nhận rằng những âm thanh này sẽ xảy ra và đây là điều hoàn toàn bình thường. Tốt nhất là thí sinh nên thu âm lại bài nói của mình để xem trong 2 phút tần suất xảy ra là bao nhiêu.
\end{qa}

\begin{qa}{Vấn đề 7: Căng thẳng}
Nếu bản thân quá căng thẳng trước và trong quá trình thi, tự khắc bài nói sẽ trở nên kém lưu loát hơn so với bình thường. Việc này có thể phần nào được giải thích là do quá trình tập luyện chưa được đầy đủ.

\vspace{.2cm}
\textbf{Giải pháp}: Cứ tập nói đầy đủ, khi đã quen thì tự khắc lời nói sẽ thành phản xạ mà thôi.
\end{qa}

\begin{qa}{Vấn đề 8: Vấn đề về phát âm}
Do cấu tạo phát âm của tiếng Anh và tiếng Việt hoàn toàn khác nhau nên thí sinh nhiều khi hay nói tiếng Việt như thế nào thì nói tiếng Anh y hệt như thế: trọng âm của từ hoặc câu không được nhấn vào, thiếu nối âm, nuốt từ, thiếu “s”… 

\vspace{.2cm}
\textbf{Giải pháp}: Khi tra từ trong từ điển thì thí sinh nên dùng từ điển Anh $-$ Anh có phát âm chuẩn để học cách phát âm sao cho đúng. Thí sinh cũng nên xem thêm nhiều phim hoặc bài hát bằng tiếng Anh để làm quen với cách phát âm của người bản ngữ. Bản chất của việc học ngoại ngữ là sự bắt chước, do đó bắt chước sao cho giống là được.
\end{qa}
\end{multicols}


\subsection*{II. Các chiến thuật cụ thể với bài thi SPEAKING}


Chương này giới thiệu các chiến thuật có thể dùng ở cả 3 phần trong bài thi speaking. Các chiến thuật sẽ được giới thiệu ngắn gọn trong phần này và sau đó cách dùng các chiến thuật sẽ được giải thích trong mỗi phần của bài kiểm tra.

\medskip

Ví dụ về 1 bài thi speaking không tốt:

\begin{center}
\begin{tabular}{|p{0.6\linewidth}|p{0.25\linewidth}|}
\hline
\textbf{Do you like to listen to music?} & Yes \\
\hline
\textbf{What kind?} & Pop \\
\hline
\textbf{Do you have a particular favorite?} & No \\
\hline
\textbf{Do you have a music idol?} & No \\
\hline
\end{tabular}
\end{center}

\medskip

Rõ ràng nếu chỉ trả lời ngắn gọn như vậy thì thí sinh không thể nào đạt được kết quả như ý. 
Dù trả lời được hết tất cả các câu hỏi nhưng bài nói quá ngắn khiến cho giám khảo 
không thể nào đánh giá được vốn từ vựng, ngữ pháp, độ nhuần nhuyễn, tính liên mạch trong 
việc kết nối ý tưởng của thí sinh. Do đó, các chiến thuật sau có thể được dùng để mở rộng, 
làm phong phú thêm câu trả lời và giành điểm tốt hơn.

\vspace{.5cm}
\textbf{\large 1. Mở rộng câu trả lời}  
\vspace{.25cm}

Để trả lời 1 câu hỏi ``What kind of films do you like?'', các chiến thuật sau có thể dùng:

\bigskip
\textbf{1.1. \textit{Chia câu hỏi thành 2--3 phần}}  

\vspace{.25cm}
Chiến thuật này đưa ra 2--3 lý do cho câu trả lời:  

\begin{quote}

$-$ Well, there are three kinds of films that I'm dead keen on.  

$-$ The one I like best is the action genre as I am crazy about thrilling fights in action-packed ones.  

$-$ I am also into science-fiction film, for it may inspire my creative imagination.  

$-$ I am kind of partial to comedy as well since it might lighten my mood.  
\end{quote}

\bigskip
\textbf{1.2. \textit{Từ ý chung chung đến lý do riêng}}  

\vspace{.25cm}
Chiến thuật này liên quan tới việc đưa ra 1 lý do chung cho câu trả lời và sau đó giải thích cụ thể hơn rồi đưa ra ví dụ cụ thể.  

\begin{quote}
Well, I'd have to say that I'm a big fan of drama films on TV. In particular, I'm quite passionate about watching Vietnamese films on TV every night. Such films as ``Song chung voi me chong'' (Living with the mother-in-law) and ``Ve nha di con'' (Please come back home) are appealing to me because it is so realistic that it creates a sense of belonging to me.  
\end{quote}

\begin{itemize}
    \item Ý chung = drama films (phim tâm lý).
    \item Cụ thể = Vietnamese films (phim Việt Nam). 
    \item Ví dụ = Such films as … (Những bộ phim như ``Sống chung với mẹ chồng'' hay ``Về nhà đi con'').    
\end{itemize}


\bigskip
\textbf{1.3. \textit{Tuỳ tình hình: mô tả các tình huống khác nhau}}  

\vspace{.25cm}
Chiến thuật này đưa ra lý do cho nhiều tình huống khác nhau cho câu trả lời.  

\begin{quote}
$-$ I think it depends really.  

$-$ If I'm stressed out, I prefer to watch something amusing. Comedy films such as those featuring Adam Sandler would do the trick.  

$-$ However, if I hang out with friends, I'd sooner enjoy science-fiction films, especially armed with a pair of 3D glasses to enjoy 3D movies.  
\end{quote}

\bigskip
\textbf{1.4. \textit{Nhượng bộ}}  

\vspace{.25cm}
Chiến thuật này bao gồm việc đưa ra một luận điểm chính và sau đó đưa thêm một luận điểm khác phản bác lại ngay.  

\begin{quote}
Well, as action is my favorite genres, I'm engrossed in buying tickets for whatever films are unveiled at the cinema. However, in my opinion, sometimes watching a comedy instead for a change of pace is not a bad idea.  
\end{quote}

\bigskip
\textbf{1.5. \textit{Lảng tránh}}  

\vspace{.25cm}
Chiến thuật này không trực tiếp trả lời câu hỏi ngay từ đầu mà trả lời ở sau.  

\begin{quote}
Well, frankly speaking, I am not passionate about a specific kind of movies but in case I have to choose one, then I'd go for action movies. It is partly because fight scenes are spectacular and eye-catching in my opinion.  
\end{quote}
\vspace{.5cm}
\begin{quote}
    Dưới đây là bảng thống kê một số chiến thuật có thể phù hợp một số dạng câu hỏi nhất định trong phần I kèm những cụm từ hữu ích thí sinh có thể dùng để khiến bài nói trở nên mượt mà hơn.
\end{quote}

\renewcommand{\arraystretch}{1.3}

\begin{tabularx}{\textwidth}{|>{\raggedright\arraybackslash}p{0.18\textwidth}|
                                 >{\raggedright\arraybackslash}p{0.18\textwidth}|
                                 >{\raggedright\arraybackslash}X|}
\hline
\textbf{Dạng câu hỏi} & \textbf{Chiến thuật} & \textbf{Những cụm từ hữu ích} \\
\hline

Mô tả (Description) & 
Chia thành 2--3 phần &
\textbf{Lúc bắt đầu:} ok, then, actually, alright, well, you know \newline
\textbf{Luận điểm 1:} I'd like to begin by saying...; The first thing I'd like to mention is...; The main thing you need to know is... \newline
\textbf{Luận điểm 2:} On top of that I'd like to add that...; Another interesting point...; Not only..., but also.../(as well) \newline
\textbf{Luận điểm 3:} Finally, I'd also like to mention...; And I shouldn't forget to mention that...; I guess I could also say that... \\
\hline
\end{tabularx}

\vspace{1cm}

\begin{tabularx}{\textwidth}{|>{\raggedright\arraybackslash}p{0.18\textwidth}|
                                 >{\raggedright\arraybackslash}p{0.18\textwidth}|
                                 >{\raggedright\arraybackslash}X|}
\hline
\textbf{Dạng câu hỏi} & \textbf{Chiến thuật} & \textbf{Những cụm từ hữu ích} \\
\hline

Các dạng (Types of) &
Chia thành 2--3 phần &
\textbf{Cách nói lại:} Kind(s) of, sort(s) of, a variety of, a range of... \newline
\textbf{Cách bắt đầu (Nhận định chung):} There's quite a mixed variety of...; There's a fairly wide range of...; There's quite a diverse mixture of... \newline
\textbf{Dạng 1:} I think the most common one is...; The main one is...; The first one that comes to mind is... \newline
\textbf{Dạng 2:} In addition...; Another kind is...; Not only that, but also.../…as well \newline
\textbf{Dạng 3 (nếu cần):} And I shouldn't forget to mention that...; I guess I could also say that...; Finally, I'd also like to mention... \\
\hline

Sở thích (Liking) / Sở ghét (Disliking) &
Từ chung chung đến cụ thể &
\textbf{Cách diễn đạt yêu thích:} I'm (pretty) fond of...; I'm (really) keen on...; I'm a big fan of...; I'm (extremely) passionate about...; I'm (really) into... \newline
\textbf{Cách diễn đạt chán ghét:} I really cannot stand...; I'm not really fond of...; I'm not much of a fan of...; ...simply gets on my nerves \newline
\textbf{Nhận định chung:} Well, in general I would say...; Generally speaking...; Overall, I think that... \newline
\textbf{Nhận định cụ thể hơn:} In particular.../ Particularly...; To be more specific.../ Specifically...; To be more precise... \newline
\textbf{Lý do:} I guess this is probably because...; I think this is due/owing to the fact that...; The reason for this is because... \\
\hline

Câu đóng (Yes/No) &
Tuỳ thuộc &
\textbf{Diễn đạt sự tuỳ thuộc:} Well, it is contingent upon...; I think it depends really; Well, to be honest, I would rather have to say it really depends \newline
\textbf{Ưu tiên 1:} If I'm..., I prefer to...; Usually, I prefer to...; Sometimes, I prefer to... \newline
\textbf{Ưu tiên 2:} But if I..., I would rather...; Although, sometimes I would rather...; However, if I have... \\
\hline

Câu hỏi mở (Wh-question) &
Nhượng bộ &
\textbf{Sự nhượng bộ:} Although, I suppose it could be argued that...; However, I'm quite aware that...; Nevertheless, it is worth remembering that... \\
\hline

So sánh (Comparing) &
Lảng tránh &
\textbf{Cấu trúc lảng tránh:} Well, to be honest, I don't really have...; Actually, I'm not really sure about that; Oh, I've never really thought about that; Oh, I like them both, so it's hard to choose only one \newline
\textbf{Câu trả lời cho câu hỏi:} ...but I guess that I will probably...; ...but I guess that if I had to choose one, then I'd...; ...perhaps the end result will be that... \newline
\textbf{Cấu trúc so sánh:} I slightly prefer...; I'd much rather...; I enjoy ... more than ... \\
\hline
\end{tabularx}

\textbf{2. Luyện nói ở tốc độ khác nhau}  

\vspace{.25cm}
Điểm chấm phần tiêu chí ``\textit{fluency}'' yêu cầu thí sinh phải nói ở tốc độ tốt. 
Đây là điều thí sinh phải chứng tỏ cho giám khảo thấy nhưng không có nghĩa là thí sinh phải nói nhanh toàn bộ bài. 
Tốc độ nói có thể thay đổi tùy từng ngữ cảnh.  

\textbf{Hãy nói chậm khi:}
\begin{itemize}
    \item Gặp chủ đề khó hoặc không chắc về những gì định nói;
    \item Muốn nhấn mạnh điều gì đó;
    \item Dùng những từ ngữ có độ khó cao khi muốn người chấm nhận ra.
\end{itemize}

\textbf{Hãy nói nhanh khi:}
\begin{itemize}
    \item Hiểu rõ chủ đề và những gì bản thân định nói;
    \item Đưa ra những thông tin không quan trọng.
\end{itemize}

\textbf{Sử dụng thán từ, hư ngữ:}  
Là những từ không có nghĩa xác định như ``oh'', ``well'', ``actually''. 

\vspace{.25cm}
Trong thực tế, người bản ngữ rất hay dùng những từ này nhưng người không phải bản ngữ lại hiếm khi dùng. 
Thí sinh hoàn toàn có thể sử dụng để bài nói tự nhiên hơn.  

\vspace{.25cm}
\textbf{Dùng dấu câu:}  
Khi nói, chúng ta thường dùng ngắt câu. Ngắt câu trong văn nói xảy ra khi chúng ta tạm ngưng một chút và đóng vai trò như dấu câu trong văn viết để định hướng người nghe.  

\vspace{.25cm}
\quad Dấu phẩy = 1 lần tạm ngưng ngắn \\
\quad Dấu chấm = 1 lần dài hơn, có hít thở không khí \\
\quad Đoạn mới = tạm ngưng lâu hơn nữa  

\bigskip
\textbf{3. Dùng nhiều tính từ và trạng từ}  

\vspace{.25cm}
Tính từ sẽ tăng sắc thái biểu cảm trong câu, khiến câu trở nên sinh động hơn. 
Trạng từ cũng vậy, giúp miêu tả sắc sảo hơn. Ví dụ:  
``fiercely loyal'' = người cực kỳ trung thành (thay cho ``very loyal'').  

Việc dùng đúng tính từ và trạng từ sẽ giúp tăng điểm từ vựng nếu phù hợp ngữ cảnh.  

\bigskip
\textbf{4. Nghe kỹ từ khóa trong câu hỏi}  

\vspace{.25cm}
Trong tiếng Anh, từ khóa câu hỏi thường xuất hiện ngay từ đầu. 
Nếu bỏ sót, thí sinh sẽ khó hiểu đúng ý. Ví dụ:  

\vspace{.25cm}

\begin{tabularx}{\textwidth}{|c|c|X|}
\hline
\textbf{Câu hỏi} & \textbf{Chức năng} & \textbf{Ví dụ} \\
\hline
what & Tìm kiếm thông tin về một điều gì đó & What kinds of movies do you like? \\
\hline
when & Hỏi về thời gian & When did you last go to the cinema? \\
\hline
where & Hỏi về nơi chốn & Where do you often go shopping? \\
\hline
which & Hỏi về lựa chọn & Which is more important to you? The quality or the price of the product. \\
\hline
who & Hỏi về người khác & Who can you confide in? \\
\hline
\end{tabularx}


\textbf{5. Những lưu ý trong phòng thi}  

\vspace{.25cm}
Đến lúc thi nói, thí sinh sẽ phải vào phòng thi trong đó có 1 chiếc bàn, 2 chiếc ghế và người đối diện giám khảo. 
Về lý thuyết, bài thi chưa bắt đầu cho đến khi giám khảo kiểm tra giấy tờ tùy thân của thí sinh xong xuôi. 
Thực tế thì, ngay khi thí sinh bước chân vào phòng, giám khảo đã có ấn tượng ban đầu rồi.  

\vspace{.25cm}
Người đi thi dễ cảm thấy lo lắng. Sẽ là vô nghĩa nếu nói điều này cho giám khảo vì họ đã nghe câu này quá nhiều lần rồi. 
Tốt nhất là thí sinh nên che giấu sự lo lắng đó, đừng để lộ ra bên ngoài. 
Thay vào đó, hãy cho giám khảo thấy được sự tự tin vào bản thân và sự bình tĩnh để họ hiểu rằng, thí sinh có thể nói tiếng Anh thật tự tin. 
Ví dụ, có thể nở một nụ cười nhẹ nhàng cùng giám khảo để tạo không khí vui vẻ giữa hai bên.  
Nhìn chung, thí sinh không cần quá lo lắng vì giám khảo là những người giàu kinh nghiệm và chắc chắn sẽ hướng dẫn từng bước trong suốt quá trình thi nói.  

\bigskip
\textbf{5.2. Nhờ giám khảo đọc lại câu hỏi}  

\vspace{.25cm}
Thí sinh được quyền hỏi giám khảo nhắc lại câu hỏi 1 lần trong phần I. 
Lưu ý là câu hỏi không được lặp lại lần 2 và giám khảo sẽ không diễn giải câu hỏi theo hướng khác.  
Trong lúc chờ giám khảo đọc lại, thí sinh có thể nghĩ ra thêm các phương án trả lời.  

\vspace{.25cm}
Thí sinh có thể nói như sau:  
\begin{quote}
``I'm sorry I didn’t quite catch that... Can you please repeat the question?''  
\end{quote}

Ngoài ra, nên dùng một số cụm từ ghi nhớ để bắt đầu câu trả lời, ví dụ:  
\begin{quote}
``Well, that’s a really interesting question.''  

``With regards to your question, I guess I would have to say that...''  
\end{quote}

\textbf{5.3. Thí sinh thực sự không hiểu câu hỏi}  

\vspace{.25cm}
Nếu thí sinh không nghe rõ hoặc không hiểu câu hỏi, giám khảo có thể nhắc lại mỗi câu hỏi 1 lần. 
Nếu vẫn không hiểu nghĩa của từ, thí sinh có thể nhờ giám khảo giải nghĩa.  
Tuy nhiên, giám khảo không thể giải thích hết toàn bộ câu hỏi nếu thí sinh không thể trả lời, ngoại trừ phần III.  

\vspace{.25cm}
Nếu thí sinh không trả lời câu hỏi, điều này không đồng nghĩa là trượt, nhưng không nên lặp lại nhiều lần. 
Nếu không hiểu, tốt nhất là cứ hỏi lại trực tiếp giám khảo.  
Trường hợp thí sinh im lặng sẽ gây khó xử vì giám khảo không biết bạn định trả lời hay không, nên cũng không biết cho điểm thế nào.  

\bigskip
\textbf{5.4. Không nên im lặng}  

\vspace{.25cm}
Nếu không có câu trả lời ngay lập tức, thí sinh nên nói vài câu đệm để giành thêm thời gian suy nghĩ.  
Ví dụ:  
\begin{quote}
``Well, let me think for a moment...''  

``That’s a good question. I haven’t thought about it before...''  
\end{quote}

Việc này giúp giữ sự trôi chảy, đồng thời thể hiện sự tự nhiên khi giao tiếp.

\newpage
\textbf{5.5. Tránh học thuộc lòng câu trả lời}  

\vspace{.25cm}
Giám khảo có thể dễ dàng nhận ra khi thí sinh trả lời theo kiểu học thuộc lòng.  
Điều này sẽ khiến bài thi kém tự nhiên và giảm điểm ở tiêu chí \textit{fluency and coherence}.  

Tốt nhất, thí sinh nên chuẩn bị ý tưởng và từ vựng cần dùng, nhưng trả lời bằng cách nói tự nhiên, linh hoạt thay vì đọc lại nguyên xi.  


